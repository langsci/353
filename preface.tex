%% -*- coding:utf-8 -*-
\chapter*{Preface}

This book has two purposes: firstly the comparative analysis of the syntactic properties of the
Germanic languages and secondly the introduction of a specific format for the description and
comparison of languages. The framework in which the analyses are couched is called \emph{HPSG
  light}. It is based on HPSG \citep{ps,ps2} in the specific version that is described in detail in
\citew{MuellerLehrbuch3}. However HPSG light does not contain any complicated attribute value
matrices (AVMs). If AVMs are used at all, they are reduced to the minimum containing a reduced set
of features like \argst for argument structure, \comps for complements and \spr for specifier. All
other aspects of the analyses are represented in syntactic trees, which are easier to read. The idea
behind the introduction of HPSG light is to provide some tool for linguists who want to provide a
more detailed description of a phenomenon without necessarily being forced to deal with all the
technicalities. The degree of formalization corresponds to what is common in Government and Binding
Theory, Minimalism, and the less formal variants of Construction Grammar. As for the one formal version
of Construction Grammar that is a variant of HPSG, namely Sign-Based Construction Grammar (SBCG, \citealp{Sag2012a}), HPSG
light can be regarded as a light version of SBCG as well, since the differences are neglected in the
abbreviated representations and trees that are used in this book. The work presented here differs
from non-formal work in GB/Minimalism and Construction Grammar in an important way: it is backed up
by implemented grammars that use the full version of HPSG including a semantic analysis in the
framework of Minimal Recursion Semantics (MRS, \citew*{CFPS2005a}). The detailed analyses are described in
conference proceedings, journal articles and books and the reader is invited to consult these
resources in case she or he is interested in the details. The implemented grammars are distributed
with the Grammix virtual machine and can be downloaded from the author's web-page. The appendix of
this book contains a list of sentences that were discussed in the respective chapters of this book
and which are covered by the grammars of the respective languages. Readers are invited to enter the
sentences into the TRALE system that comes with Grammix and inspect the complete AVMs.



\section*{On the way this book is published}

Teachers at schools and universities are payed by the state, that is by the public (you). Among
their duties is the creation of teaching material. There is no reason whatsoever to leave the
teaching material to profit oriented publishers. On the contrary, teaching material should be open
and adaptable to the needs of the teachers who want to use it. 

A study by the American Enterprise Institute shows that the price of college books rose by 812\,\%
from 1978 to 2012 while the general consumer prices rose a mere 250\,\%.\footnote{
\url{http://www.aei-ideas.org/2012/12/the-college-textbook-bubble-and-how-the-open-educational-resources-movement-is-going-up-against-the-textbook-cartel/}.
10.09.2014.%
} Similar figures exist for scientific books in general and for university text books. My favorite example is a thin text book
on logic \emph{Logik für Linguisten}, which is a translation of the English text book \emph{Logic for
Linguists} \citep{AAD73a}. This book has 112 pages. It was sold for 9,40e as a paperback by the Max Niemeyer
Verlag. This publisher was bought by De Gruyter and the book is now sold for \$126.00/89,95€ as an
eBook and \$133,00/94,95€ for the hardcover book\footnote{
  \url{http://www.degruyter.com/isbn/978-3-11-096350-2}. 1.09.2014.
} (see \citealp{MuellerOA} for other examples and a general discussion). Both the eBook and the printed book are unaffordable for students. The way out of this highly
problematic situation is to publish books open access. The PDF version of this book is free for
everybody and the printed copy is available for a reasonable price since the book is licenced under
a Creative Commons license and hence is not owned by a
profit"=oriented publisher and everybody can choose his or her own print on demand service in case
the default service provided by Language Science Press is more expensive.

~\medskip

\noindent
Berlin, \today\hfill Stefan Müller
