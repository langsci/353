%% -*- coding:utf-8 -*-
\chapter{Passive}
\label{sec-icelandic-quirky-subj}
\label{chap-case}

This chapter deals with the passive.\footnote{
This chapter is a rough draft. Most of the references to other literature are still missing \ldots
} Passive is usually analyzed as the suppression of the
subject. However, before we can develop an analysis we have to ask what it is that constitutes a
subject. This is a question that is the topic of edited volumes and dissertations and modest as I
am, I will try and provide an answer at least for the Germanic languages. As we will see, the
situation is rather clear in languages like Danish, English, and German, but there are exciting
facts to be discovered about Icelandic. 


\section{The phenomenon}


\subsection{Subjects and other subjects}
\label{sec-subj-properties}

The situation in languages like English, Danish, and German is rather clear. For instance, many
authors assume that non-predicative NPs in the nominative are subjects in German. So, \emph{der
  Mann} `the man' is the subject of the sentences in (\mex{1}): 
\eal
\ex 
\gll Der Mann lacht.\\
     the man laughs\\
\ex 
\gll Der Mann hilft ihr.\\
     the man  helps her\\
\ex 
\gll Der Mann gibt ihr ein Buch.\\
     the man gives her a book\\
\zl
The restriction to non-predicative NPs is needed since otherwise, we would have to assume that both
NPs in (\mex{1}) are subjects, but \emph{ein Lügner} `a lier' is a predicative phrase and only
\emph{der Mann} `the man' is the subject.
\ea
\gll Der Mann ist ein Lügner.\\
     the man is a liar\\
\glt `The man is a liar.'
\z
In addition certain clausal arguments are treated as subjects.

Genetives and datives as in (\mex{1}) are not counted among the subjects in German.
\eal
\ex 
\gll Ihrer wurde gedacht.\\
     they.\GEN{} was remembered\\
\glt `They were remembered.'
\ex 
\gll Ihm wurde geholfen.\\
     he.\DAT{}  was   helped\\
\glt `He was helped.'
\zl

Interestingly the question whether genitives and datives like those in (\mex{0}) are subjects was
answered quite differently for the SVO language Icelandic by researchers following the work of
\citet*{ZMT85a}. Although the sentences in (\mex{1}) look like those in (\mex{-1}), the genetive and
the dative element are assumed to be subjects.
\eal
\label{ex-subject-icelandic-passive-v2}
\ex 
\gll Hennar var saknað.\\
     she.\SG.\GEN{} was missed\\
\ex 
\gll Þeim       var hjalpað.\\
     they.\PL.\DAT{} was helped\\
\zl
Since Icelandic is a V2 language the constituent order in such simple sentences does not help us to
determine whether \emph{hennar} `her' and \emph{Þeim} `them' are subjects or not. These elements are
fronted and since both subjects and objects can be fronted, the sentences in (\mex{0}) do not help
us in determining the grammatical function of these arguments. However, \citet*{ZMT85a} argued that
these elements should be analyzed as subjects and provided several tests. Among the tests are more
elaborate positional tests and omissability in so-called control constructions. I will turn to these
tests now.


\subsubsection{The position of subjects in V2 and V1 sentences}


The first test that was suggested uses the position of constituents in V2 sentences in which a
non-subject is fronted \citep*[Section~2.3]{ZMT85a}. For instance, consider the following examples:

\eal
%% \ex[]{
%% \gll Refinn           skaut  Ólafur      með þessari byssu.\\
%%      den.Fuchs.\ACC{} schoss Olaf.\NOM{} mit diesem  Gewehr\\
%% }
% selbst erfunden, check
\ex[]{
\gll Meb  þessari byssu   skaut Ólafur      refinn.\\
     with this    shotgun shot  Olaf.\NOM{} the.fox.\ACC{}\\
}
\ex[*]{
\gll Meb þessari byssu  skaut refinn         Ólafur.\\
     with this  shotgun shot  the.fox.\ACC{} Olaf.\NOM{}\\
}
\zl
The nominative can appear directly after the finite verb \emph{skaut} `shot' as in (\mex{0}a) but it
cannot appear to the right of the accusative as in (\mex{0}b).

The same can be observed with \emph{w}-questions:
\eal
\ex[]{
\gll Hvenær hafði Sigga        hjfilpað Haraldi?\\
     when   has   Sigga.\NOM{} geholfen Harold.\DAT{}\\
}
\ex[*]{
\gll Hvenær hafði Haraldi       Sigga        hjfilpað?\\
     when   has   Harald.\DAT{} Sigga.\NOM{} geholfen\\
}
\zl
The object has to follow the participle \emph{hjfilpað} as in (\mex{0}a) and the subject immediately
follows the finite verb. Examples with the object before the subject as in (\mex{0}b) are
ungrammatical. The dative object can be fronted, but then it has to be realized in initial position
to the left of the finite verb, not to its right:
\ea[]{
\gll Haraldi       hafði Sigga        aldrei hjfilpað.\\
     Harald.\DAT{} has   Sigga.\NOM{} never  geholfen\\
}
\z

The same situation can be found in yes/no questions:
\eal
\ex[]{
\gll Hafði Sigga        aldrei hjfilpað Haraldi?\\
     has   Sigga.\NOM{} never  helped   Harald.\DAT\\
}
\ex[*]{
\gll Hafði Haraldi       Sigga        aldrei hjfilpað?\\
     has   Harald.\DAT{} Sigga.\NOM{} never  helped\\
}
\zl

\citet*[Section~2.3]{ZMT85a} observed that certain datives can appear in this postverbal position as well:

\eal
\ex[]{
\gll Hefur henni      alltaf þótt    Ólafur      leibinlegur?\\
     has   she.\DAT{} always thought Olaf.\NOM{} boring.\NOM{}\\
\glt `Has she always considered Olaf boring?'
}
\ex[]{
\gll Ólafur      hefur henni alltaf þótt leibinlegur.\\
     Olaf.\NOM{} has   she.\DAT{} always thought boring.\NOM{}\\
\glt `She alway considered Olaf boring.'
}
\ex[*]{
\gll Hefur Ólafur      henni      alltaf þótt    leibinlegur?\\
     has   Olaf.\NOM{} her.\DAT{} always thought boring.\NOM\\
}
\zl

The German eqivalent would be the sentence in (\mex{1}):
\ea[??]{
\gll Mich     dünkt  der Mann langweilig.\\
     I.\ACC{} thinks the.\NOM{} man boring\\
\glt `I think the man is boring.'
}
\z
However, \emph{dünkt} is archaic and is usually used with a \emph{dass} clause -- if it is used at
all. But there is a non-archaic verb that has a similar form:
\ea
\gll Mir scheint der Mann langweilig.\\
     I.\DAT{} seems the.\NOM{} man boring\\
\glt `The man seems boring to me.'
\z
The experiencer of \emph{scheinen} `to seem' is expressed with the dative, while the subject of the
embedded predicate \emph{langweilig} `boring' is in the nominative.



\subsubsection{Subjects in control constructions}
\label{sec-subject-control}

\citet*[Section~2.7]{ZMT85a} discuss control structures in which the subject of the embedded verb is
not expressed. (\mex{1}a) shows an example of normal control in which the subject of the matrix verb
\emph{vonast} `to hope' refers to the same discourse referent as the subject of the embedded verb \emph{fara} `to
go'. (\mex{1}b) is an example of so-called arbitrary control. In cases of arbitrary control there is
no element depending on the head that governs the infinitive that refers to the same discourse
referent as the subject of the infinitive. The unexpressed subject corresponds to a pronoun \emph{one} that is
used generically. In example (\mex{11}b) \emph{óvenjulegt} `unusual' does not select for an argument
that refers to the same referent as the subject of \emph{fara} `to go'. The subject of \emph{að fara
  heim snemma} `to go home early' is not expressed but is understood as the indefinite pronoun \emph{one}.
\eal
\ex
\gll Ég  vonast til að fara heim.\\
     I   hope   for to go   home\\
\glt `I hope to go home.'
\ex
\gll Að fara heim snemma er óvenjulegt.\\
     to go home   early is unusual\\
\glt `It is unusual to go home early.'
\zl

Now, it can be observed that Icelandic allows verbs that do not take a nominative in such control
constructions. An example is \emph{vantar} (`lacks'), which takes two accusatives rather than a
nominative and an accusative:
\ea
\gll Mig      vantar peninga.\\
     I.\ACC{} lack   money.\ACC\\
\z
(\mex{1}) shows that this verb can be embedded under \emph{vonast} (`to hope'):
\ea
\gll Ég  vonast til ab vanta ekki peninga.\\
     I hope  for to lack not money.\ACC\\
\glt `I hope that I do not lack money.'
\z

This should be compared with German:
\eal
\ex[]{
\gll Mir fehlt kein Geld.\\
     I.\DAT{} lacks no.a.\NOM{} money\\
\glt `I do not lack money.'
}
\ex[*]{
\gll Ich hoffe, kein         Geld  zu fehlen.\\
     I hope     not.a.\NOM{} money to lack\\
\glt Intended: `I hope that I do not lack money.'
}
\zl


The question at the beginning of this section was whether the datives and genetives in sentences
like (\ref{ex-subject-icelandic-passive-v2}) -- repeated here as (\mex{1}) -- are subjects or not.
\eal
\ex 
\gll Hennar var saknað.\\
     she.\SG.\GEN{} was missed\\
\ex 
\gll Þeim       var hjalpað.\\
     they.\PL.\DAT{} was helped\\
\zl

We are now able to use the tests to answer this question: The dative is right-adjacent to the finite
verb in the question in (\mex{1}) and hence in subject position.\todostefan{Add genetive examples}

\ea
\gll Var honum     aldrei hjfilpað af foreldrum sinum?\\
     was he.\DAT{} never  helped   by parents   his\\
\glt `Did his parents never help him?'
\z

Similarly the dative follows the finite verb in the V2 sentence in (\mex{1}):
\ea
\gll Í prófinu  var honum vist hjálpað.\\
     in the.exam was he.\DAT{} apparently helped\\
\glt `Apparently he was helped in the exam.'
\z

In addition these datives can be omitted in control constructions as the examples in (\mex{1}) show:
\eal
\ex
\gll Ég vonast til að verba hjálpað.\\
     I  hope   for to be helped\\
\ex
\gll Að vera hjálpað i prófinu er óleyfilegt.\\
     to be helped in the.exam is un.allowed\\
\glt `It is not allowed to be helped in the exam.'
\zl
This should be compared to German: While verbs like \emph{unterstützen} `to support' that govern a
nominative and an accusative can appear in such control constructions, verbs like \emph{helfen} `to
help' that take a nominative and a dative are ruled out in this construction:
\eal
\ex 
\gll dass jemand ihm hilft\\
     that somebody him.\DAT{} helps\\
\ex
\gll dass jemand ihn unterstützt\\
     that somebody him.\ACC{} supports\\
\ex
\gll  dass ihm geholfen wird\\
      that him.\DAT{} helped is\\
\ex
\gll  dass er unterstützt wird\\
      that he.\NOM{} supported is\\
\zl

\eal
\ex[]{
\gll Ich hoffe unterstützt zu werden.\\
     I   hope  supported   to be\\
}
\ex[*]{
\gll Ich hoffe geholfen zu werden.\\
     I   hope  helped   to be\\
}
\zl
The dative object cannot be omitted in such control constructions as (\mex{0}b) shows. The only way
to realize a passive below \emph{hoffen} is to use the dative passive with
\emph{erhalten}/""\emph{bekommen}/""\emph{kriegen}. The dative passive can turn a dative object into a
nominative subject:
\ea[]{
\gll Er bekommt geholfen.\\
     he.\NOM{} gets helped\\
}
\z
Since the object of \emph{helfen} is then nominative and hence undoubtfully a subject in German, it
does not come with a surprise that it can be omitted in control constructions like (\mex{1}):
\ea[]{
\gll Ich hoffe hier geholfen zu bekommen.\footnotemark\\
     I   hope  here helped   to get\\
\footnotetext{    
\url{http://www.photovoltaikforum.com/sds-allgemein-ueber-solar-log-f38/solarlog-1000-mit-wifi-anschliesen-t96371.html}. 10.01.2014
}
}
\z



\if0


\subsubection{Subjekt-Verb-Kongruenz?}

\begin{itemize}
\item Verben kongruieren mit dem Nominativelement.\\
      Wenn es keins gibt, ist das Verb dritte Person Singular (Neutrum).


\item In (\mex{1}) keine Kongruenz:
\eal
\ex 
\gll Þeim       var hjalpað.\\
     sie.\emph{\PL}.\DAT{} wurde geholfen\\
\ex 
\gll Hennar var saknað.\\
     sie.\emph{\SG}.\GEN{} wurde vermisst\\
\zl


\item Der Dativ und der Genitiv sind aber dennoch Subjekte,\\
      wie wir gleich sehen werden.

\end{itemize}

\fi

\section{The Case Principle}


\subsection{Structural and lexical case}
\label{sec-struk-lex-kas}
\label{sec-struc-lex-kas}

In order to analyze the passive it is useful to distinguish between structural and lexical
case. Structural case is case that depends on the syntactic structure in which arguments get
realized, while lexical case is case that stays constant independent of the sytactic environment. In
addition to lexical and structural case there is semantic case. This case is not assigned by a
governing head like a verb, adjective or preposition but is due to a certain function of an
adverbial. For instance time expressions like \emph{den ganzen Tag} `the whole day' in (\mex{1}) are
in the accusative in German.
\ea
\gll Er arbeitet den ganzen Tag.\\
     he.\NOM{} works the.\ACC{} whole day\\
\glt `He works the whole day.'
\z
Since this chapter is about the passive and its variation in the Germanic languages, I will ignore
semantic case here.

\subsubsection{Nominatives and accusative objects}

Until now the case that an argument gets assigned by its head was represented in the valency list of
the head. With such a representation we would need two different lexical items for the verb
\emph{lesen} (`to read'): one in which the verb takes a nominative and an accusative as in
(\mex{1}c) and one in which it takes two accusatives as in (\mex{1}d).
\eal
\ex 
\gll Er wird das Buch lesen.\\
     he.\NOM{} will  the.\ACC{} book read\\\german
\glt `He will read the book.'
\ex 
\gll Ich sah ihn das Buch lesen.\\
     I   saw him the book read\\
\glt `I saw him read the book.'
\ex \sliste{ NP[\type{nom}], NP[\type{acc}] }
\ex \sliste{ NP[\type{acc}], NP[\type{acc}] }
\zl
Rather than having these two lexical items, one can have just one lexical item and leave the actual
case assignment for later. So depending on whether the subject of \emph{lesen} is realized as the
subject of \emph{wird} or as the object of \emph{sah} `saw', it gets nominative or accusative. Such
cases are called structural cases. The distinction between structural and lexical case will play an
important role in the analysis of passive. It is this distinction that makes a unified analysis of
the so-called personal and impersonal passive possible.

(\mex{1}) provides additional examples and involves different forms of the verb (finite
vs.\ non-finite) and a nominalization:
\eal
\ex 
\gll Der Installateur kommt.\\
     the plumber      comes\\
\glt `The plumber comes.'
\ex 
\gll Der Mann läßt den Installateur kommen.\\
     the man  lets the plumber      come\\
\glt `The man lets the plumber come.'
\ex 
\gll das Kommen des Installateurs\\
     the coming of.the plumber\\
\glt `the coming of the plumber'
\zl
The example in (\mex{0}c) also shows that the subject of \emph{kommen} `to come' can be realized as
genetive. So, nominative, genitive, and accusative are structural cases in German. (The question
whether some or all datives should be treated as structural case is addressed below).

The examples in (\mex{0}) show that the case of subjects in German can change, those in (\mex{1})
show that the case of accusative objects can change as well:
\eal
\ex 
\gll Karl schlägt den Weltmeister.\\
     Karl defeats the world.champion\\
\glt `Karl defeats the world champion.'
\ex 
\gll Der Weltmeister wird geschlagen.\\
     the world.champion is beaten\\
\glt `The world champion is beaten.'
\zl

\subsubsection{Genitive objects}

The examples in (\mex{1}) show instances of lexical case: Genitive that depends on the verb is
lexical since it does not change when the verb is passivized.
\eal
\ex[]{
\gll Wir gedenken der Opfer.\\
     we.\NOM{} remember the victims.\GEN{}\\
\glt `We remember the victims.'
}
\ex[]{
\gll Der Opfer wird gedacht.\\
     the.\GEN{} victims is remembered\\
\glt `The victims are remembered.'
}
\ex[*]{
\gll Die Opfer wird / werden gedacht.\\
     the.\NOM{} victims is {} are remembered\\
}
\zl
As the example in (\mex{0}c) shows, the nominative is impossible. The genitive object remains in the
genitive in passive constructions. Passives without a subject as in (\mex{0}b) are traditionally
called impersonal passives.

\subsubsection{Dative objects}

Now, lets turn to the dative. If we look at examples like (\mex{1}), we see that the dative does not
change either in the passive:
\eal
\ex 
\gll Der Mann hat ihm geholfen.\\
     the\NOM{} man  has him.\DAT{} helped\\
\glt `The man helped him.'
\ex 
\gll Ihm wird geholfen.\\
     him.\DAT{} is helped\\
\glt `He is helped.'
\zl
So in analogy to the genitive examples above, the dative should be a lexical case.

But there are examples like those in (\mex{1}) and according to the view that structural cases are
those cases that vary according to the syntactic environment, the dative should be a structural case.
\eal
\ex 
\gll Der Mann  hat   den Ball dem Jungen geschenkt.\\
     the man   has   the ball the boy given\\
\glt `The man gave the boy the ball as a present.'
\ex 
\gll Der Junge bekam den Ball geschenkt.\\
     the boy   got   the ball given\\   
\glt `The boy got the ball as a present.'
\zl
The question whether the dative should be seen as a structural or a lexical case is a hotly debated
one. In principle there are three possibilities and all three of them were suggested in the
literature. One could assume that all datives are lexical, that some are lexical and others are
structural, or that all datives are structural.\todostefan{update references}

I follow \citet{Haider86} and treat all datives as lexical cases. Under this assumption, the
contrast in Haider's examples \citeyearpar[\page 20]{Haider86} in (\mex{1}) is explained immediately:
\eal
\ex[]{
\gll Er streichelt den Hund.\\
     he.\NOM{} strokes the.\ACC{} dog\\
}
\ex[]{
\gll Der Hund wurde gestreichelt.\\
     the.\NOM{} dog  was   strocked\\
}
\ex[]{
\gll sein Streicheln des    Hundes\\
     his  strocking of.the\.gen{} dog\\
}

\ex[]{\label{bsp-er-hilft-den-kindern}\iw{helfen}
\gll Er hilft den Kindern.\\
     he helps the.\DAT{} children\\
}
\ex[]{
\gll Den Kindern wurde geholfen.\\
     the.\DAT{} children was helped\\
}
\ex[]{
\gll das Helfen der Kinder\\
     the helping of.the children\\
}\label{das-helfen-der-Kinder}
\ex[*]{
\gll sein Helfen der Kinder\\
     his  helping the children\\
}\label{sein-helfen-der-Kinder}
\zl
The accusative object of \emph{streicheln} `to stroke' can be realized as nominative in the passive,
so it is clearly a structural case. Nominalizations allow this object to be realized in the genitive
as (\mex{0}c) shows. However, this does not work with datives. The dative object of \emph{helfen}
`to help' cannot be realized in the genitive. (\ref{das-helfen-der-Kinder}) is possible, but only
with a reading in which the children are the agents, that is, the nominalization in
(\ref{das-helfen-der-Kinder}) corresponds to (\mex{1}) rather than (\ref{bsp-er-hilft-den-kindern}):
\ea
\gll Die Kinder helfen jemandem.\\
     the.\NOM{} children help somebody\\
\z
If the agent is expressed by a prenominal possessive as in (\ref{sein-helfen-der-Kinder}) the
genetive or dative \emph{der Kinder} is ruled out.

The only way to express the dative at all is prenominally:
\ea
\gll das Den-Kindern-Helfen\\
     the the-children-helping\\
\glt `the children's helping'
\z

So, authors who assume that all datives are structural have a problem explaining the differences in
impersonal passives and nominalizations. In addition there is a problem with bivalent verbs. While
some verbs take the dative others take the accusative although there is hardly any semantic
difference or any other reason that could be made responsible.
\eal
\ex 
\gll Er hilft ihm.\\
     he helps him\\
\ex 
\gll Er unterstützt ihn.\\
     he supports him\\
\zl
The fact that \emph{helfen} takes a dative object, while \emph{unterstützen} takes an accusative is
just an idiosyncrasy of German that speakers of German have to learn when they acquire the
language. So, the information in the lexical entry for \emph{helfen} must be different from the one
for \emph{unterstützen}. Some authors acknowledge this difference and assume that the dative of
bivalent verbs is lexical, while the dative of ditransitive verbs is structural. The assumption is
that verbs assign the nominative to their first argument, the accusative to their last argument and
if there is an additional argument that is neither the first nor the last, it gets dative. The
prediction that such mixed accounts make is that the dative passive should be possible with
ditransitive verbs but impossible with bivalent verbs, since the dative is structural for the former
verbs and lexical for the latter. The empirical situation is not as clear-cut as one might
wish. Some authors accept examples like (\mex{1}). Others reject them.
\eal
\ex 
\gll Er kriegte von vielen geholfen / gratuliert / applaudiert.\\
     he got by many helped {} congratulated {} applauded\\
\ex 
\gll Man kriegt täglich gedankt.\\
     one gets   daily thanked\\
\zl
However, there are attested examples:
\eal
\ex "`Da kriege ich geholfen."'\footnote{
Frankfurter Rundschau, 26.06.1998, p.\,7.%
}
\ex
% auch nach applaudiert geholfen + bekommen und kriegen gesucht 21.09.2003
Heute morgen bekam ich sogar schon gratuliert.\footnote{%
Brief von Irene G.\ an Ernst G.\ vom 10.04.1943, Feldpost-Archive mkb-fp-0270}
%Branich IG-Vorsitzender Friedel Schönel meinte deshalb, 
\ex
"`Klärle"' hätte es wirklich mehr als verdient, auch mal zu einem "`unrunden"' Geburtstag gratuliert zu bekommen.\footnote{
Mannheimer Morgen, 28.07.1999, Lokales; "`Klärle"' feiert heute Geburtstag.%
}
\ex
Mit dem alten Titel von Elvis Presley "`I can't help falling in love"' bekam Kassier Markus Reiß zum Geburtstag gratuliert, [\ldots]\footnote{
%der dann noch viel später bekannte: "Ich hab' immer noch Gänsehaut, das war der schönste Teil meines Geburtstages." Doch auch die anderen Abteilungen des Bürgervereins können auf ein erfolgreiches Jahr 1998 zurückblicken.
Mannheimer Morgen, 21.04.1999, Lokales; Motor des gesellschaftlichen Lebens.%
}
\zl
I think that the verbs \emph{kriegen}, \emph{erhalten}, and \emph{bekommen} are on the way to become
auxiliaries. Their meaning is getting more and more bleached. Hence there are almost no selectional
restrictions left on the downstairs verb. The only requirement for the dative passive to apply is of
course that the embedded verb governs a dative.

Now, if the dative passive is possible with verbs bivalent like \emph{helfen} and if \emph{helfen}
has to govern a lexical dative (since otherwise the difference between \emph{helfen} and
\emph{unterstützen} could not be explained), it follows that the dative passive must be able to
convert a lexical dative into a structural case (realized as nominative in the examples above). This
means that one could assume that all datives are lexical, even the datives of ditransitive
verbs. This explains why these datives are not realized as nominatives or accusatives in passives
like (\mex{1}):
\eal
\ex[]{ 
\gll dass er dem Jungen den Ball gegeben hat\\
     that he.\NOM{} the.\DAT{} boy    the.\ACC{} ball given has\\
}
\ex[]{ 
\gll dass dem Jungen der Ball gegeben wurde\\
     that the.\DAT{} boy    the.\NOM{} ball given was\\
}
\ex[*]{ 
\gll dass der Junge den Ball gegeben wurde\\
     that the.\NOM{} boy   the.\ACC{} ball given was\\
}
\ex[*]{ 
\gll dass den Junge der Ball gegeben wurde\\
     that the.\ACC{} boy   the.\NOM{} ball given was\\
}
\zl
They just stay dative. The only exception is the dative passive and this has to be analyzed as an exception.




%% \subsubsection{Non-cannonical accusative objects}

%% I already showed that the accusative can be a structural case. However, there are some exceptional
%% cases like those of subjectless verbs that govern an accusative (\mex{1}a) and ditransitive
%% verbs like \emph{lehren} `to teach' that take two accusative objects rather than one accusative and a dative.
%% \eal
%% \ex 
%% \gll Ihn dürstet.\\
%%      he.\ACC{} thirsty.is\\
%% \glt `He is thirsty.'
%% \ex 
%% \gll Der Vater lehrte seinen Sohn einen neuen Tritt.\\
%%      the.\NOM{} father taught his.\ACC{} son a.\ACC{} new kick\\
%% \glt `The father taught his son a new kick.'
%% \zl
%% Verbs like \emph{lehren} are generally bad in the passive.


%% \subsubsection{Adjektivumgebungen}

%% 
%% \subsubsection{Lexikalischer Kasus in Adjektivumgebungen}

%% Kasus von Objekten von Adjektiven kann sich nicht ändern.\\
%% Adjektive können Genitiv und Dativ zuweisen:
%% \eal
%% \ex Ich war mir \emph{dessen} sicher.
%% \ex Sie ist \emph{ihm} treu.
%% \zl
%% 
%% Die Zuweisung von Akkusativ ist ebenfalls möglich:
%% \eal
%% \ex Das ist \emph{diesen Preis} nicht wert.
%% \ex Der Student ist \emph{das Leben} im Wohnheim nicht gewohnt.\iw{gewohnt}\footnote{
%%         \citep*[S.\,312]{HB72a}
%%       }
%% \ex Du bist mir \emph{eine Erkl"arung} schuldig.\footnote{
%%         \citep*[S.\,620]{HFM81}
%%       }
%% \zl
%% Akkusativ ist bei Adjektivkomplementen aber selten \citep{Haider85b}.
%% }

%% 
%% \subsubsection{Struktureller Kasus in Adjektivumgebungen}


%% Kasus der Subjekte von Adjektiven hängt von der syntaktischen
%% Umgebung ab \citep{Wunderlich84}:
%% \eal
%% \ex \emph{Der Mond} wurde kleiner.\iw{klein}
%% \ex Er sah\iw{sehen} \emph{den Mond} kleiner werden.
%% \zl

%% }


%\if 0


%% \subsubsection{Semantische Kasus}
%% \label{sec-sem-kasus}
%% \is{Kasus!semantischer|(}


%% \subsubsection{Semantische Kasus}

%% \begin{itemize}
%% \item NPen können auch als Adjunkte auf"|treten \citep{Haider85b}:
%% \eal
%% \ex Sie hörten \emph{den ganzen Tag} dieselbe Schallplatte.
%% \ex Laßt \emph{mir} den Hund in Ruhe!
%% \ex \emph{Eines Tages} erschien ein Fremder.
%% \zl

%% \item auch der Urteilsdativ (\emph{Dativ iudicantis})  \citep{Wegener85b}:

%% \eal
%% \ex Das ist \emph{mir} zu\iw{zu!Grad} schwer.
%% \ex Das ist \emph{dem Kind} zu langweilig / nicht interessant genug.\iw{genug!Grad}
%% \ex Du läufst \emph{der Oma} zu\iw{zu!Grad} schnell.
%% \ex Das Wasser ist \emph{dem Baby} warm genug.\iw{genug!Grad}
%% \zl
%% \end{itemize}

%% \subsubsection{Zuweisung semantischer Kasus durch das Verb?}

%% \begin{itemize}
%% \item
%% Haider: Zuweisung durch Verb in (\mex{1}) nicht sinnvoll:
%% \ea
%% Sie hörten \emph{den ganzen Tag} dieselbe Schallplatte.
%% \z
%% Zeitangaben kommen auch in adjektivischen und nominalen Umgebungen vor:
%% % zitiert Toman83
%% \eal
%% \ex die Ereignisse \emph{letzten Sommer}
%% \ex der Flirt \emph{vorigen Dienstag}
%% \ex die \emph{diesen Sommer} sehr günstige Witterung
%% \ex die \emph{diesen Sommer} sehr teuren Urlaubsreisen
%% \zl
%% NPen mit strukturellem Kasus müssen in Nominalumgebungen
%% Genitiv sein. $\to$\\
%% In (\mex{0}) keine Zuweisung von strukturellem Kasus.


%% \item
%% Die Kasus in (\mex{0}) werden nicht aufgrund ihres Vorkommens in einer bestimmten
%% syntaktischen Struktur zugewiesen,\\
%% sondern sind vielmehr durch die Bedeutung des Nomens bestimmt.
%% \end{itemize}



%% \subsubsection{Akkusativ und Genitiv}

%% %\citep*{ZMT85a} -> semantische Kasusmarkierung
%% Der freie Akkusativ kommt bei Maß"-angaben\is{Maßangaben} (Zeitdauer und Zeitpunkt)
%% vor (\mex{1}) und Genitiv bei Lokalangaben oder Zeitangaben (\mex{2}).
%% \eal
%% \ex Sie studierte \emph{den ganze Abend}.
%% \ex \emph{Nächsten Monat}\iw{Monat} kommen wir.
%% \zl
%% \eal
%% \ex Ein Mann kam \emph{des Weges}.\iw{Weg}
%% \ex \emph{Eines Tages}\iw{Tag} sah ich sie wieder.
%% \zl




%% \subsubsection{Kongruenzkasus}

%% 
%% \subsubsection{Kongruenzkasus}

%% \begin{itemize}
%% \item Zwei Akkusative?
%% \eal
%% \ex Er nannte \emph{ihn} \rot{einen Experten}.
%% \ex \emph{Er} wurde \rot{ein Experte} genannt.
%% \zl
%% 
%% \item Wären das zwei unabhängige Akkusative,\\
%%       würde sich bei Passivierung nur einer ändern.

%% 
%% \item Kasus von \emph{einen Experten} wird \emph{Kongruenzkasus} genannt.\\
%% Die prädikative Phrase \emph{einen Experten} stimmt mit dem
%% Element,\\ über das prädiziert wird, im Kasus überein. 
%% \end{itemize}
%% }

%% 
%% \subsubsection{Kongruenzkasus mit Präpositionen}

%% Ähnliche Effekte kann man mit den Präpositionen \emph{als} und \emph{wie}
%% beobachten.
%% \eal
%% \ex \emph{Er} gilt als \rot{großer Künstler}.\footnote{
%%         \citew[S.\,203--204]{Heringer73a}.
%%       }
%% \ex Man läßt \emph{ihn} als \rot{großen Künstler} gelten\iw{gelten als}.
%% \zl
%% 
%% \eal
%% \ex Ich sehe \emph{ihn} als \rot{meinen Freund} an.\iw{ansehen}\footnote{
%%         \citew*[S.\,154]{SS88a}.
%% }
%% \ex \emph{Er} wird als \rot{mein Freund} angesehen.
%% \zl
%% }

%% 
%% \subsubsection{Kongruenzkasus mit Adjunkten}

%% Wie bei den prädikativen Argumenten gibt es auch Kongruenzkasus bei Adjunkten:
%% \eal
%% \ex Sie verhielt\iw{verhalten} \emph{sich} wie \emph{ihr Vater}.
%% \ex Ich behandelte\iw{behandeln} \emph{ihn} wie \emph{meinen Bruder}.
%% \ex Ich half\iw{helfen} \emph{ihm} wie \emph{einem Freund}.
%% \ex Ich erinnerte\iw{erinnern} mich \emph{dessen} wie \emph{eines fernen Alptraums}.
%% \zl

%% }

%% 
%% \subsubsection{Prädikation = Kasuskongruenz?}

%% \begin{itemize}
%% \item Kongruieren prädikative Phrasen immer mit dem Element,\\
%%       über das sie prädizieren?
%% \item Dies würde sofort auch Beispiele wie das in (\mex{1}) erklären:
%% \ea
%% Er wird ein großer Linguist.
%% \z
%% 
%% \item In AcI"=Konstruktionen müßten beide NPen im Akkusativ stehen.\\
%% Das ist nicht der Fall:
%% \eal
%% %\ex Laß ihn einen großen Linguisten werden.\label{bsp-lass-ihn-einen-grossen}
%% \ex Laß\iw{lassen|(} den wüsten Kerl [\ldots] meinetwegen ihr Komplize sein.\footnote{
%%         (\ref{bsp-lass-den-wuesten-kerl}) und (\ref{bsp-lass-mich}) sind aus dem \citet*[{\S}\,6925]{Duden66}.\iaf{Duden} %\citet*[{\S}\,1473]{Duden73}.\iaf{Duden}
%%         Die Quellen finden sich dort.
%%       }\label{bsp-lass-den-wuesten-kerl}
%% \ex Laß mich dein treuer Herold sein.\label{bsp-lass-mich}
%% \ex Baby, laß\iw{lassen|)} mich dein Tanzpartner sein.\footnote{
%%         Funny van Dannen, Benno-Ohnesorg-Theater, Berlin, Volksbühne, 11.10.1995
%%         }
%% \zl
%% 
%% \item
%% $\to$ Nominativ des Nicht-Subjekts in Kopulakonstruktionen ist\\
%%       ein lexikalischer Kasus \citep[S.\,54]{Thiersch78a}.
%% \end {itemize}
%% }

%% \subsubsection{Der Kasus nicht ausgedrückter Subjekte}
%% \label{sec-kasus-nicht-realisierter-subj}

%% 
%% \subsubsection{Der Kasus nicht ausgedrückter Subjekte (I)}
%% \savespace

%% \begin{itemize}
%% \item \citet*[Kapitel~6]{Hoehle83}:\\
%% Kasus nicht an der Oberfläche auf"|tretender Elemente bestimmbar.

%% {\em ein- nach d- ander-\/} kann sich auf mehrzahlige Konstituenten beziehen. 

%% Dabei muß Kasus und Genus mit der Bezugsphrase übereinstimmen.
%% 
%% \item In (\mex{1}) Bezug auf Subjekte bzw.\ Objekte:
%% \eal
%% \ex Die Türen sind eine nach der anderen kaputtgegangen.
%% \ex Einer nach dem anderen haben wir die Burschen runtergeputzt.
%% \ex Einen nach dem anderen haben wir die Burschen runtergeputzt.
%% \ex Ich ließ die Burschen einen nach dem anderen einsteigen.
%% \ex Uns wurde einer nach der anderen der Stuhl vor die Tür gesetzt.
%% \zl
%% \end{itemize}
%% }

%% 
%% \subsubsection{Der Kasus nicht ausgedrückter Subjekte (II)}
%% \savespace

%% In (\mex{1}) Bezug auf Dativ- bzw.\ Akkusativobjekte
%% eingebetteter Infinitive:

%% \eal
%% \ex Er hat uns gedroht, die Burschen demnächst einen nach dem anderen wegzuschicken.
%% \ex Er hat angekündigt, uns dann einer nach der anderen den Stuhl vor die Tür zu setzen.
%% \ex Es ist nötig, die Fenster, sobald es geht, eins nach dem anderen auszutauschen.
%% \zl

%% }

%% 
%% \subsubsection{Der Kasus nicht ausgedrückter Subjekte (III)}
%% \savespace

%% In (\mex{1}) Bezug auf Subjekt innerhalb der Infinitiv"=VP:
%% \eal
%% \ex Ich habe den Burschen geraten, im Abstand von wenigen Tagen einer nach dem anderen
%%       zu kündigen.
%% \ex Die Türen sind viel zu wertvoll, um eine nach der anderen verheizt zu werden.
%% \ex Wir sind es leid, eine nach der anderen den Stuhl vor die Tür gesetzt zu kriegen.
%% \ex Es wäre fatal für die Sklavenjäger, unter Kannibalen zu fallen und einer nach dem
%%       anderen verspeist zu werden.
%% \zl
%% {\em ein- nach d- ander-\/} im Nominativ $\to$\\
%% Das nicht realisierte Subjekt steht ebenfalls im Nominativ.

%% }


%% 
%% \subsubsection{Der Kasus nicht ausgedrückter Subjekte (IV)}

%% Dasselbe gilt für nicht realisierte Subjekte von adjektivischen Partizipien:
%% \eal
%% \ex die eines nach dem anderen einschlafenden Kinder
%% \ex die einer nach dem anderen durchstartenden Halbstarken
%% \ex die eine nach der anderen loskichernden Frauen
%% \zl
%% }

%% 
%% \subsubsection{Der Kasus nicht ausgedrückter Subjekte (V)}

%% Man muß also sicherstellen, daß auch nicht realisierte Subjekte Kasus zugewiesen bekommen.
%% Würde man diesen Kasus unterspezifiziert lassen, würden Sätze wie (\mex{1}) falsch analysiert werden.
%% \judgewidth{\#}
%% \ea[\#]{
%% Ich habe den Burschen geraten, im Abstand von wenigen Tagen einen nach dem anderen zu kündigen.
%% }
%% \z
%% In der zulässigen Lesart von (\mex{0}) ist die Phrase 
%% \emph{einen nach dem anderen} das Objekt von \emph{kündigen} und kann
%% sich nicht auf das Subjekt des Infinitivs, das referenzidentisch
%% mit \emph{den Burschen} ist, beziehen.

%% }



After this discussion of lexical and structural case in German, I will provide the Case Principle,
which is responsible for case assignment. As was explained in Section~\ref{sec-linking}, it is assumed that all arguments of a
head are represented in one list: the \textsc{argument structure} list (\argst list\isfeat{arg-st}). 
(\mex{1}) shows the argument structure list of a ditransitive verb like \word{geben} `to give':
\ea
\sliste{ NP[\type{str}], NP[\type{ldat}], NP[\type{str}] }
\z
As was argued above, dative is treated as a lexical case. \type{ldat} is an abbreviation for lexical
dative and \type{str} stands for for structural case. The Case
Principle has the following form (adapted from \citealp{Prze99}; \citealp{Meurers99b}):

\begin{principle-break}[\hypertarget{case-p}{Case Principle}]\is{Prinzip!Kasus-}
\label{case-p}
\begin{itemize}
\item In a list that contains both the subject and the complements of a verbal head, the first
  element with structural case gets nominative unless it is raised by a higher head.
\item All other elements in this list that have structural case and are not raised get accusative.\is{case!accusative}
\item In nominal environments elements with structural case get genitive.\is{case!genetive}
\end{itemize}
\end{principle-break}
This principle is inspired by \citet*{YMJ87} and as will be demonstrated below it also works for the
complex case system in Icelandic. It differs for in not assigning case to elements that are raised
to a higher predicate. This point will be explained in more detail below.

The effect of this principle will be explained with respect to the verbs in (\mex{1}):
\ea
\begin{tabular}[t]{@{}l@{~}l@{~}l}
a. & \emph{schläft} `sleep':       & \argst \sliste{ NP[\type{str}]$_i$ }\\[2mm]
b. & \emph{unterstützt} `support': & \argst \sliste{ NP[\type{str}]$_i$, NP[\type{str}]$_j$ }\\[2mm]
c. & \emph{hilft} `help':          & \argst \sliste{ NP[\type{str}]$_i$, NP[\type{ldat}]$_j$ }\\[2mm]
d. & \emph{schenkt} `give as a present':     & \argst \sliste{ NP[\type{str}]$_i$, NP[\type{ldat}]$_j$, NP[\type{str}]$_k$ }\\
\end{tabular}
\z
The first element in these lists that has structural case gets nominative and the second one
accusative. This is exactly what one expects. The result is given in (\mex{1}). \type{snom} stands
for structural nominative.
\ea
\begin{tabular}[t]{@{}l@{~}l@{~}l}
a. & \emph{schläft} `sleep':       & \argst \sliste{ NP[\type{snom}]$_i$ }\\[2mm]
b. & \emph{unterstützt} `support': & \argst \sliste{ NP[\type{snom}]$_i$, NP[\type{sacc}]$_j$ }\\[2mm]
c. & \emph{hilft} `help':          & \argst \sliste{ NP[\type{snom}]$_i$, NP[\type{ldat}]$_j$ }\\[2mm]
d. & \emph{schenkt} `give as a present':     & \argst \sliste{ NP[\type{snom}]$_i$, NP[\type{ldat}]$_j$, NP[\type{sacc}]$_k$ }\\
\end{tabular}
\z

\subsection{Argument reduction and case assignment: the passive}
\label{sec-case-assignment-passive}

Now, with the structural/lexical case distinction the analysis of the passive is really simple and
directly corresponds to the intuition that the passive is the suppression of the subject (the most
prominent, that is, the first argument in the \argstl). If the first argument is removed from the
lists in (\mex{-1}), the following lists result:
\ea
\begin{tabular}[t]{@{}l@{~}l@{~}l}
a. & \emph{geschlafen}:  & \argst \sliste{ }\\[2mm]
b. & \emph{unterstützt}: & \argst \sliste{ NP[\type{str}]$_j$ }\\[2mm]
c. & \emph{geholfen}:    & \argst \sliste{ NP[\type{ldat}]$_j$ }\\[2mm]
d. & \emph{geschenkt}:   & \argst \sliste{ NP[\type{ldat}]$_j$, NP[\type{str}]$_k$ }\\
\end{tabular}
\z
The NPs that are in the first position in (\mex{0}) where in the second position in (\mex{-1}). The
first NP with structural case gets nominative and hence the following case assignments result:
\ea
\begin{tabular}[t]{@{}l@{~}l@{~}l}
a. & \emph{geschlafen}:  & \argst \sliste{ }\\[2mm]
b. & \emph{unterstützt}: & \argst \sliste{ NP[\type{snom}]$_j$ }\\[2mm]
c. & \emph{geholfen}:    & \argst \sliste{ NP[\type{ldat}]$_j$ }\\[2mm]
d. & \emph{geschenkt}:   & \argst \sliste{ NP[\type{ldat}]$_j$, NP[\type{snom}]$_k$ }\\
\end{tabular}
\z
Lexical case as in (\mex{0}c--d) is not affected by the case principle, it stays the way it was
specified, namely dative.

It should be noted here that this simple approach to passive accounts both for the so-called
personal and the impersonal passive. The passives of \emph{schlafen} `to sleep' and \emph{helfen}
`to help' are called impersonal passives since the respective clauses do not have a subject. 
\eal
\ex 
\gll dass geschlafen wurde\\
     that slept was\\
\glt `that there was sleeping there'
\ex
\gll dass dem Mann geholfen wurde\\
     that the.\DAT{} man helped was\\
\glt `that the man was helped'
\zl
The passives of \emph{unterstützen} `to support' and \emph{schenken} `to give as a present' do have
subjects, namely the arguments that are realized as accusative objects in the active:
\eal
\ex 
\gll dass der Mann unterstützt wurde\\
     that the.\NOM{} man supported was\\
\glt `that the man was supported'
\ex
\gll dass dem Jungen der Ball geschenkt wurde\\
     that the.\DAT{} boy the\NOM{} ball given was\\
\glt `that the ball was given to the boy as a present'
\zl
Those analyses that assign all cases lexically would have to assume that the case of the objects
(accusative) is changed into nominative in the passive. Hence there would be two variants of the
passive: The impersonal passive just suppresses the subject and the personal one suppresses the
subject and additionally changes the case of the object into nominative. The analysis using the
structural/lexical case distinction just postpones the case assignment until the point where it is
clear what the right case will be. If we have a participle and use it with the passive auxiliary it
is clear what the case of the arguments has to be.



%% \subsubsubsubsection{Dativpassiv}

%% \frame[shrink=15]{
%% \subsubsection{Dativpassiv}

%% Bei der Kombination von \emph{geholfen} und
%% \emph{bekommen} bzw.\ von \emph{geschenkt} und \emph{bekommen} wird das Dativargument von 
%% \emph{geholfen} bzw.\ von \emph{geschenkt} zum ersten Argument gemacht und der lexikalische
%% Dativ beim eingebetteten Verb wird zu einem strukturellen Kasus beim Passiv"=Hilfsverb:
%% \ea
%% \begin{tabular}[t]{@{}l@{~}l@{~}l}
%% c. & \emph{hilft}:       & \argst \sliste{ NP[\type{str}]$_j$, NP[\type{ldat}]$_k$ }\\
%% d. & \emph{schenkt}:     & \argst \sliste{ NP[\type{str}]$_j$, NP[\type{str}]$_k$, NP[\type{ldat}]$_l$ }\\
%% \end{tabular}
%% \z
%% \ea
%% \begin{tabular}[t]{@{}l@{~}l@{~}l}
%% a. & \emph{geholfen bekommt}:    & \argst \sliste{ NP[\type{str}]$_k$ }\\
%% b. & \emph{geschenkt bekommt}:   & \argst \sliste{ NP[\type{str}]$_l$, NP[\type{str}]$_k$ }\\
%% \end{tabular}
%% \z
%% Details kommen im Kapitel über Passiv.

%% Kasusvergabe: Dadurch, daß das Dativargument an erster Stelle in der Valenzliste\\
%% von \emph{geholfen bekommen} bzw.\ von \emph{geschenkt bekommen} steht, kriegt es
%% Nominativ. 

%% Bei \emph{geschenkt bekommen} bekommt das zweite Element (das direkte Objekt) Akkusativ.

%% Die Umwandlung eines lexikalischen in einen strukturellen Kasus ist unschön,\\
%% es scheint zur Zeit jedoch keine bessere Alternative zu geben. 

%% }


\subsection{Argument extension and case assignment: AcI constructions}

The case principle contains restrictions on case assignment that prohibits the assignment to
elements that are raised. These restrictions have not been explained yet. Consider the examples in (\mex{1}):
\eal
\ex
\gll Der Junge liest den Aufsatz.\\
     the.\NOM{} boy reads the.\ACC{} paper\\
\glt `The boy reads the paper.'
\ex\label{ex-the-man-lets-the-boy-read-the-book}
\gll Der Mann läßt den Jungen den Aufsatz lesen.\\
     the.\NOM{} man lets the.\ACC{} boy the.\ACC{} paper read\\
\glt `The man lets the boy read the paper.'
\zl
The example (\mex{0}a) shows that the subject of \emph{lesen} is
assigned nominative. However, the subject of \emph{lesen} gets accusative in (\mex{0}b). So, if
one would assign case on the basis of the argument structure of \emph{lesen} in (\mex{0}b), one
would assign nominative, but the AcI verb \emph{lassen} `to let' assigns accusative to its object. The
point is that the subject of \emph{lesen} is raised to the object of \emph{lassen}. The Case
Principle is set up in a way such that case is assigned only to those arguments that are not raised
to a higher head. Hence, \emph{den Jungen} does not get case from \emph{lesen}, but from \emph{lässt}.

The analysis of (\mex{0}b) is given in Figure~\vref{fig-vc-aci}.
\begin{figure}
\centerfit{
\begin{forest}
sm edges
[V\feattab{
              \sliste{ }}
        [{NP[\type{snom}]} [der Mann;the man, roof] ]
        [V\feattab{
              \sliste{ NP[\type{snom}] }}
          [{NP[\type{sacc}]} [den Jungen;the boy, roof] ]
          [V\feattab{
%              \vform \type{fin},\\
              \sliste{ NP[\type{snom}], NP[\type{sacc}]}} 
            [{NP[\type{sacc}]} [den Aufsatz;the paper, roof] ]
            [V\feattab{
%                \vform \type{fin},\\
                \sliste{ NP[\type{snom}], NP[\type{sacc}], NP[\type{sacc}]}} 
               [~~~~~~V\feattab{
%                \vform \type{bse},\\
                 \sliste{ NP[\type{sacc}], NP[\type{sacc}]}} [lesen;read] ]
               [V\feattab{
%                \vform \type{fin},\\
                  \sliste{ NP[\type{snom}], NP[\type{sacc}], NP[\type{sacc}], V }} [lässt;lets] ] ] ] ] ]
\end{forest}}
\caption{\label{fig-vc-aci}Analysis of AcI constructions as raising constructions and the verbal
  complex in German}
\end{figure}
The arguments of \emph{lesen} `to read' are taken over by \emph{lässt}. Since \emph{lässt}
contributes its own argument, the causer or the one who gives the permission, \emph{lässt} selects
for three NPs with structural case and a verb in the specific sentence depicted in
Figure~\ref{fig-vc-aci}. According to the Case Principle the first NP with structural case gets
nominative and the other NPs with structural case get accusative. This results into a list with one
NP in the nominative and two NPs in the accusative.

(\mex{1}) shows the \argstl of \emph{lässt} when it is combined with \emph{schlafen},
\emph{unterstützen}, \emph{helfen}, or \emph{schenken}, respectively.
\ea
\oneline{%
\begin{tabular}[t]{@{}l@{~}l@{~}l@{}}
a. & \emph{läßt} with \emph{schlafen}:      & \argst \sliste{ NP[\type{str}]$_l$, NP[\type{str}]$_i$, V}\\[2mm]
b. & \emph{läßt} with \emph{unterstützen} : & \argst \sliste{ NP[\type{str}]$_l$, NP[\type{str}]$_i$, NP[\type{str}]$_j$, V }\\[2mm]
c. & \emph{läßt} with \emph{helfen}:        & \argst \sliste{ NP[\type{str}]$_l$, NP[\type{str}]$_i$, NP[\type{ldat}]$_j$, V }\\[2mm]
d. & \emph{läßt} with \emph{schenken}:      & \argst \sliste{ NP[\type{str}]$_l$, NP[\type{str}]$_i$, NP[\type{ldat}]$_j$, NP[\type{str}]$_k$, V }\\
\end{tabular}
}
\z
The NP that is added has the index \type{l}. As the first NP with structural case on these lists it
gets nominative. All other elements of this list that have structural case get accusative. Hence the
subject of the embedded verb is assigned accusative, the lexical cases stay the same and the
accusative objects of the embedded verb get accusative as well, since their case is structural too.

Note that the question of whether a language has a verbal complex or not is orthogonal to issues of
case assignment. Figure~\vref{fig-the-man-lets-the-boy-read-the-book} shows the analysis of the English translation of (\ref{ex-the-man-lets-the-boy-read-the-book}).
\begin{figure}
\centerfit{
\begin{forest}
sm edges
[S
        [{NP[\type{snom}]} [the man, roof] ]
        [VP
          [\vbar [V [let] ]
                 [{NP[\type{sacc}]} [the boy] ] ]
              [VP
                [V [read] ]
                [{NP[\type{sacc}]} [the book, roof] ] ] ] ]
\end{forest}}
\caption{\label{fig-the-man-lets-the-boy-read-the-book}AcI constructions in English}
\end{figure}
\emph{let} selects for the subject, the object and a VP. The subject of \emph{read} is
simultaneously the object of \emph{let} and hence the Case Principle does not assign nominative to
the subject of the embedded verb \emph{read}, but accusative to the object of the matrix verb \emph{let}.

%% 
%% \subsubsection{Adjektivsubjekte}


%% Die Kasuszuweisungen an das Subjekt von Adjektiven funktioniert analog. Die Kopula wird mit dem Adjektiv
%% verbunden, und es entsteht eine Valenzliste, die die Argumente des Adjektivs enthält (\mex{1}a).\\
%% Wird ein solcher Komplex noch unter ein AcI"=Verb wie \emph{sehen} eingebettet,\\
%% erhält man die Liste in (\mex{1}b):
%% \ea
%% \begin{tabular}[t]{@{}l@{~}l@{~}l}
%% a. & \emph{kleiner werden}:     & \argst \sliste{ NP[\type{str}]$_j$ }\\
%% b. & \emph{kleiner werden sah}: & \argst \sliste{ NP[\type{str}]$_i$, NP[\type{str}]$_j$ }\\
%% \end{tabular}
%% \z
%% Die Kasuszuweisung funktioniert analog zu den bereits diskutierten Fällen. In den verbalen Umgebungen
%% der Kopula bzw.\ des AcI"=Verbs bekommen die NPen mit strukturellem Kasus Nominativ bzw.\ Akkusativ.%
%% }


%% \subsubsection{Semantischer Kasus}
%% 
%% \subsubsection{Semantischer Kasus (I)}


%% Der Kasus von NPen wie \emph{den ganzen Tag} in (\mex{1}) ist von der syntaktischen Umgebung unabhängig.
%% \eal
%% \ex Sie arbeiten den ganzen Tag.
%% \ex Den ganzen Tag wird gearbeitet, [\ldots].\footnote{
%%   \url{http://www.philo-forum.de/philoforum/viewtopic.html?p=146060}. \urlchecked{12}{05}{2005}.
%% }
%% \zl
%% Daß die NP im Akkusativ steht, hängt mit ihrer Funktion zusammen. 

%% Unterschiedliche Lexikoneinträge für \emph{Tag} in (\mex{0}) und (\mex{1}):
%% \eal
%% \ex Ich liebe diesen Tag.
%% \ex Dieser Tag gefällt mir.
%% \zl

%% In (\mex{0}) liegen ganz gewöhnliche Argumente vor,\\
%% in (\mex{-1}) dagegen ein Adjunkt. 
%% }

%% 
%% \subsubsection{Semantischer Kasus (II)}

%% Adjunkte unterscheiden sich von Argumenten durch ihren \modw und durch ihern \contw.

%% Für (\mex{-1}) muß es unter \cont eine Dauer-Relation geben.

%% Zusammen mit dieser Information wird im Lexikoneintrag für das modifizierende Nomen der Kasus fest kodiert. 

%% Die morphologische Komponente kann dann für diesen Lexikoneintrag nur die Akkusativform erzeugen, 
%% da alle anderen Flexionsformen mit der bereits im Lexikoneintrag angegebenen Kasusinformation inkompatibel sind. 

%% Dadurch wird sichergestellt, daß Sätze wie (\mex{1}) nicht analysiert werden:
%% \eal
%% \ex[*]{
%% Er arbeitet der ganze Tag.
%% }
%% \ex[*]{
%% weil der ganze Tag gearbeitet wurde
%% }
%% \zl

%% }



\section{Comparison German, Danish, English, Icelandic}

In the following subsections I want to compare several dimensions in which the Germanic languages
vary:
\begin{itemize}
\item Danish has a morphological passive, English, German, and Icelandic do not.

\item German, Icelandic allow for subjectless constructions, Danish and English do not.

\item Danish, German and Icelandic allow for impersonal passives, English does not.

\item Danish and Icelandic allow both objects to be promoted to subject, English and German do not.

\item German has the so-called remote passive, Danish the so-called complex passive and Danish and English
  have the so-called reportive passive.
\end{itemize}



\subsection{Morphological and Analytic Forms}

Danish allows for a morphological passive. It is formed by appending the \suffix{s} suffix and there
are present tense (\ref{ex-laeses}) and past forms (\ref{ex-laestes}):
\eal
\ex[]{\label{ex-laeseract}
\gll Peter læser avisen.\\
     Peter reads newspaper.\textsc{def}\\\danish
\glt `Peter is reading the newspaper.'}
\ex[]{\label{ex-laeses}
\gll Avisen              læses af Peter.\\
     newspaper.\textsc{def} read.\textsc{pres}.\textsc{pass} by Peter\\
\glt `The newspaper is read by Peter.'}
\ex[]{\label{ex-laestes}
\gll Avisen              læstes af Peter.\\
     newspaper.\textsc{def} read.\textsc{past}.\textsc{pass} by Peter\\
\glt `The newspaper was read by Peter.'}
\zl

Danish also allows for the analytic form with \emph{blive} `be' and participle: 
\ea
\gll Avisen              bliver læst af Peter.\\
     newspaper.\textsc{def} is     read by Peter\\
\glt `The newspaper is read by Peter.'
\z
The morphological passive may also apply to infinitives:
\ea
\gll Avisen skal læses hver dag.\\
     newspaper.def must read.\textsc{inf}.\textsc{pass} every day\\
\glt `The newspaper must be read every day.'
\z


English and German only have the analytic variant:
\eal
\ex The paper was read.
\ex 
\gll Der Aufsatz wurde gelesen.\\
     the.\NOM{} paper was read\\\german
\zl    





\subsection{Personal and impersonal Passive}

All languages under consideration allow for the promotion of an accusative object to subject. As the
following examples show, the subject can be an S or a VP:\todostefan{show that these are really
  subjects. We have V2 here, could be objects.}

\eal
\ex
\gll At regeringen træder tilbage, bliver påstået.\\
     that government.\textsc{def} resigns \textsc{part} is claimed\\
\glt `It is claimed that the government resigns.'
\ex
\gll At reparere bilen, bliver forsøgt.\\
     to repair car.\textsc{def} is tried\\
\glt `It is tried to repair the car.'
\zl

In addition to such personal passives, Danish, German, and Icelandic allow for impersonal
passives. Since German does not require a subject, impersonal passives like (\mex{1}) are expected:

\ea
\gll weil noch getanzt wurde\\
     because still danced was\\\german
\glt `because there was still dancing there'
\z

The following two examples from Icelandic show that Icelandic also knows impersonal constructions \citep[\page 264]{Thrainsson2007a-u}:
\eal
\ex 
\gll Oft var   talað      um   þennan mann.\\
     often was talked about this Mann.\ACC.\SG.\M\\\icelandic
\ex
\gll Aldrei hefur verið    sofið      í  þessu  rúmi.\\
     never    has   been slept in this bed.\DAT\\
\glt `This bed has never been slept in.'
\zl

Danish also allows for impersonal passives but it differs from the languages discussed so far in
that it requires an expletive subject:
\eal
\ex 
\gll at der bliver danset\\
     that \textsc{expl} is danced\\
\glt `that there is dancing'
\ex
\gll at der danses\\
     that \textsc{expl} dance.\textsc{pres}.\textsc{pass}\\
\glt `that there is dancing'
\ex[*]{ 
\gll Bliver danset?\\
     is danced\\
}
\ex[*]{
\gll Danses?\\
     dance.\textsc{pass}\\
}
\zl
So Danish is like English in always requiring a subject, but while this constraint results in the
impossibility of impersonal passives in English, Danish found a solution to the subject problem by
inserting an expletive.

Expletives are excluded in German impersonal constructions:
\nocite{MOe2011a}
\ea[*]{
\gll weil es noch gearbeitet wurde\\
     because it still worked was\\
\glt Intended: `because there was still working there'
}
\z



%% The examples in (\ref{ex-gearbeitet-wurde}) and (\ref{ex-bliver-arbejder}) show passives of
%% mono-valent verbs but of course bi-valent intransitive verbs like the German \emph{denken} (`think')
%% and Danish \emph{passe} (`take care of') also form impersonal passives:
%% \ea
%% \gll dass an die Männer gedacht wurde\\
%%      that \textsc{prep} the men thought was\\
%% \glt `that one thought about the men'
%% \z
%% \eal
%% \label{ex-impersonal-passive-pp}
%% \ex
%% \gll Der passes på børnene.\\
%%      \textsc{expl} take.care.of.\textsc{pres}.\textsc{pass} on children.\textsc{def}\\
%% \glt `Somebody takes care of the children.'
%% \ex
%% \gll Der bliver passet  på børnene.\\
%%      \textsc{expl} is taken.care.of on children.\textsc{def}\\
%% \glt `Somebody takes care of the children.'
%% \zl


\subsection{Promotion of the primary and secondary object}

English and German allow the promotion of one of the objects of a ditransitive verb only. (\mex{1})
shows that the accusative object can be realized as subject, but the dative cannot:
\eal
\ex[]{
\gll weil der Mann dem Jungen den Ball schenkt\\
     because the.\NOM{} man the.\DAT{} boy the.\ACC{} ball gives\\\german
\glt `because the man gives the boy a ball as a present'
}
\ex[]{
\gll weil dem Jungen der Ball geschenkt wurde\\
     because the.\DAT{} boy the.\NOM{} ball given was\\
\glt `because the ball was given to the boy'
}
\ex[*]{
\gll weil der Junge den Ball geschenkt wurde\\
     because the.\NOM{} boy the.\ACC{} ball given was\\
}
\zl

Similarly, English can realize the first object as subject, but the second object cannot be promoted
to subject:
\eal
\ex[]{
because the man gave the boy the ball
}
\ex[]{
because the boy was given the ball
}
\ex[*]{
because the ball was given the boy
}
\zl
The information structural effect can be reached with a different lexical variant of \emph{give}
though. \emph{give} can be used with an NP object and a \emph{to} PP instead of two NPs as in
(\mex{1}a). The first object of the ditransitive \emph{give} is realized as PP in (\mex{1}a) and the
second object \emph{the ball} is the first object in (\mex{1}a). This alternation is also called dative-shift\is{dative-shift}.
\eal
\ex because the man gave the ball to the boy
\ex because the ball was given to the boy
\zl
(\mex{1}b) is the passive variant of (\mex{0}a). As in (\mex{-1}b), the primary object is promoted
to subject.

Danish and Icelandic differ from English and German. In the latter languages both objects can be
promoted to subject without any previous alternation of valence frames like dative shift.
\eal
\ex\label{ex-fordi-manden-giver-drengen-bolden} 
\gll fordi manden giver drengen bolden\\ 
     because man.\textsc{def} gives boy.\textsc{def} ball.\textsc{def}\\\danish
\glt `because the man gives the boy the ball'
\ex\label{ex-boy-was-given-ball-danish}
\gll fordi drengen bliver givet bolden\\ 
     because boy.\textsc{def} is given ball.\textsc{def}\\
\glt `because the boy is given the ball'
\ex\label{ex-ball-was-given-boy-danish}
\gll fordi bolden bliver givet drengen\\ 
     because ball.\textsc{def} is given boy.\textsc{def}\\
\glt `because the ball is given to the boy'
\zl

One could assume that it is always the first object (the primary object) that is promoted to subject
and that Danish does not have an order of the objects, so that both objects are equally prominent
and can be promoted to subject. Moro is a language that is said to have such properties
\citep{AMM2013a}. However, Danish differs from Moro in that the order of the objects in sentences is
clearly fixed: While (\mex{0}a) is possible, the reverse order of the objects is ungrammatical as
(\mex{1}) shows.
\ea[*]{
fordi manden giver bolden drengen
}
\z



As far as Icelandic is concerned, \citet*[\page 460]{ZMT85a} note that apart from the possibility to
promote the accusative to subject, the dative can become a quirky subject:
\ea
\label{ex-dat-subj-passive-ditransitive-icelandic}
\gll Konunginum voru gefnar ambáttir.\\
     the.king.\DAT{} were given.\F.\PL{} maidservants.\NOM.\F.\PL\\
\glt `The king was given female slaves.'
\z
The structure of (\mex{0}) is sketched in (\mex{1}):
\ea
\label{sub-aux-v-o}
{}[S$_i$ Aux \_$_i$ V O] 
\z
Since the nominative is serialized after the participle it cannot be the subject, which implies that
the fronted dative element is the subject.

Alternatively the accusative object is promoted to subject:
\ea
\label{ex-nom-subj-passive-ditransitive-icelandic}
\gll Ambáttin var gefin konunginum.\\
     the.maidservant.\NOM.\SG{}  was given.\F.\SG{} the.king.\DAT\\
\glt `The female slave was given to the king.'
\z
This sentence also has the structure in (\ref{sub-aux-v-o}).

In order to show that the dative is really promoted to subject in (\ref{ex-dat-subj-passive-ditransitive-icelandic}) and the accusative is
promoted to subject in (\ref{ex-nom-subj-passive-ditransitive-icelandic}), \citet*[\page 460]{ZMT85a} apply a battery of tests. I only give the
V2 examples with an adjunct in initial position, the questions, and the control structures here. The
examples in (\mex{1}) and (\mex{2}) show that the
sentences above really have the structure in (\ref{sub-aux-v-o}). The first position in (\mex{1}) is
filled by an adjunct, which entails that the subject remains in subject position and hence shows
that the dative \emph{konunginum} `the king' is the subject. Similarly the nominative
\emph{ambfittin} is the subject in (\mex{0}b).
\eal
\ex
\gll Um veturinn voru konunginum gefnar ambfittir.\\
     in the.winter were the.king.\DAT{} given slaves.\NOM\\
\glt `In the winter, the king was given (female) slaves.'
\ex
\gll Um veturinn var ambfittin gefin konunginum.\\
     in the.winter was the.slave.\NOM{} given the.king.\NOM\\
\glt `In the winter, the slave was given to the king.'
\zl
The questions in (\mex{1}) are further evidence. The initial position is not filled and the dative
in (\mex{1}a) and the nominative in (\mex{1}b) is realized prenominally.
\eal
\ex\label{ex-were-the-king-given-the-slaves}
\gll Voru konunginum gefnar ambfittir?\\
     were the.king.\DAT{} given slaves.\NOM{}\\
\glt `Was the king given slaves?'
\ex\label{ex-were-the-slaves-given-the-king}
\gll Var ambfittin gefin konunginum?\\
     was the.slave.\NOM{} given the.king.\DAT\\
\glt `Was the slave given to the king?'
\zl

(\mex{1}) shows the respective control examples:
\eal
\ex
\gll Að vera gefnar ambáittir var mikill heiður.\\
     to be given slaves.\NOM{} was great honor\\
\glt `To be given slaves was a great honor.'
\ex
\gll Að vera gefin konunginum olli miklum vonbrigðum.\\
     to be given the.king.\DAT{} caused great disappointment\\
\glt `To be given to the king caused great disappointment.'

%% \ex
%% \gll Ambáttin vonast til að verða gefin konunginum.\\
%%      the.slave.\NOM{} hopes for to be given the.king.\DAT\\
%% \glt `The slave hopes to be given to the king.'
\zl
In (\mex{0}a) the dative is not expressed and in (\mex{0}b) the nominative is omitted. This shows
that both the primary and the secondary object can be promoted to subject in Icelandic, even though
one of them has structural and the other one lexical case.


\subsection{Theoretical analysis of the crosslinguistic differences}

Argument reduction and case assignment was already explained for German in
Section~\ref{sec-case-assignment-passive}. I want to get a little bit more explicit now and provide
lexical items for the passive and perfect auxiliary for German. After this I discuss the other
languages and explain how the differences can be dealt with in a worked out analysis.

\subsubsection{Designated Argument Reduction}

\citet{Haider86} suggested marking the argument of a verb that has subject properties. He calls
these special arguments \emph{designated argument}. \citet{HM94a} transferered this idea to HPSG and
\citet{Mueller2003e} modified it slightly to get certain facts with modal infinitives
right.\todostefan{check} One important use of the designated argument is to distinguish so"=called
unaccusative verbs from unergative verbs. \citet{Perlmutter78} pointed out that unaccusative verbs
have remarkable properties and argued that their subjects are not really subjects but behave more
like objects. One of their properties is that they do not allow for passives. Furthermore their
participles can be used attributively which is usually not possible:
\eal
\ex[]{ 
\gll der angekommene Zug\\
     the arrived     train\\
\glt `the arrived train'
}
\ex[*]{
\gll der geschlafene Mann\\
     the slept man\\
}
\zl
This is explained if one assumes that the subject of \emph{ankommen} `arrive' is indeed like an
object. As an object it patterns with the object of transitive verbs:
\ea
\gll der geliebte Mann\\
     the beloved  man\\
\z
\emph{Mann} `man' fills the object slot of \emph{geliebte}. If the sole argument of \emph{ankommen}
is treated as an object, the similarity to the transitive \emph{lieben} is explained
immedeately. Similarly the fact that unaccusatives do not allow for passives is explained: If
passive is the suppression of the subject and \emph{ankommen} does not have a subject in that sense,
passive cannot apply.
\eal
\ex[]{
\gll Der Zug ist angekommen.\\
     the train is arrived\\
\glt `The train arrived.'
}
\ex[*]{
\gll weil angekommen wurde\\
     because arrived was\\
}
\zl

In the HPSG analyses the authors assume that there is a list"=valued feature \textsc{designated
  argument} (\textsc{da}). This list contains the subject of transitive and unergative verbs
(intransitive verbs that are not unaccusative). The \dav of unaccusative verbs is the empty list,
since these verbs do not have an argument with subject properties.

The passive is analyzed as a lexical rule that licences a lexical item for the participle. The
\argstl of the particple is the \argstl of the verb stem that is the input to the lexical rule minus
the \dalist. Since this is not the focus of this book, I will not discuss unaccusative verbs in the
following. (\mex{1}) provides some prototypical examples for unergative and transitive verbs:

\ea
%\resizebox{\linewidth}{!}{%
\begin{tabular}[t]{@{}l@{ }l@{ }l@{ }l@{}}
  &                     & \textsc{arg-st} & \textsc{da}\\[2mm]
a.&tanzen (dance):   & \sliste{ \ibox{1}NP[\type{str}] }                                              & \sliste{ \ibox{1} }\\[2mm]
b.&lesen  (read):    & \sliste{ \ibox{1}NP[\type{str}], NP[\type{str}] }                              & \sliste{ \ibox{1} }\\[2mm]
c.&schenken (give as a present): & \sliste{ \ibox{1}NP[\type{str}], NP[\type{ldat}], NP[\type{str}] } & \sliste{ \ibox{1} }\\[2mm]
d.&helfen   (help):   & \sliste{ \ibox{1}NP[\type{str}], NP[\type{ldat}] }                            & \sliste{ \ibox{1} }\\
\end{tabular}
%}
\z
The lexical rule that forms the participle is sketched in (\ref{lr-passive-prelim}):
\ea
\label{lr-passive-prelim}
Lexical rule for the formation of the participle (preliminary):\\
\ms[stem]{
head   & \ms[verb]{ da & \ibox{1}\\
                  }\\
arg-st & \ibox{1} $\oplus$ \ibox{2} \\
} $\mapsto$
\ms[word]{
arg-st & \ibox{2} \\
}
\z
This rule splits the \argstl of the input into two lists \ibox{1} and \ibox{2}. \ibox{1} is
identical to the \dav. Therefore the designated argument is taken off the \argstl and is not present
in the lexical item that is licensed by the rule.


The \argstl of the participle that is licensed is either empty (\mex{1}a) or starts with an object of the active form:
\ea
\label{partizipien-hm}
%\resizebox{\linewidth}{!}
\z
As was explained above, the first element in the \argstl with structural case gets nominative and
hence the accusative object of \emph{lesen} in (\mex{1}a) is realized as nominative in (\mex{1}b):
\eal
\ex
\gll Er liest den Aufsatz.\\
     he.\NOM{} reads the.\ACC{} paper\\
\ex
\gll Der Aufsatz wurde gelesen.\\
     the.\NOM{} paper was read\\
\zl

English differs from German in not having a dative case at all. I am talking about morphological
markings here, not about semantics. Therefore both objects of English ditransitive verbs are
accusative objects. However, only one of the objects can be promoted to subject. This is modeled in
the analysis at hand by assuming that the secondary object bears lexical accusative (see also \citew[\page 57]{Grewendorf2002a} for the assumption of lexical accusative for the secondary object in English).\footnote{
  Admittidly this is just a restatement of the facts, since assigning lexical case means that the
  argument under consideration cannot have another case. But taken together with constraints on
  subjects in English the facts about promotion or non-promotion of arguments follow nicely.
}

\ea\label{da-repr-hm-English}
%\resizebox{\linewidth}{!}{%
\begin{tabular}[t]{@{}l@{ }l@{ }l@{ }l@{ }l@{}}
  &                     & \textsc{arg-st}\\[2mm]
b.&dance   (unerg):     & \liste{ NP[\type{str}]}\\[2mm]
%c.&auf"|fallen (unacc): & \liste{}                         & \liste{NP[\type{str}], NP[\type{ldat}]}\\[2mm]
c.&read      (trans):   & \liste{ NP[\type{str}], NP[\type{str}]}\\[2mm]
d.&give      (ditrans): & \liste{ NP[\type{str}], NP[\type{str}], NP[\type{lacc}] }\\[2mm]
e.&help      (trans):   & \liste{ NP[\type{str}], NP[\type{str}] }\\
\end{tabular}
%}
\z
German can promote the second object (accusative) and English the first one. The commonality is that
the object that is closer to the verb can be promoted. This is the accusative for German since nominative,
dative, accusative is the unmarked order and German is a OV language and the first accusative in
English since English is a VO language.

\eal
\ex 
\gll dass dem Jungen der Ball gegeben wurde\\
     that the.\DAT{} boy the.\NOM{} ball given was\\
\glt `that the ball was given to the boy'
\ex because the boy was given the ball
\zl
A further difference is the lexical item for \emph{help}. Since there is no dative in English, the
object is marked accusative as it is the case for \emph{read}. Interestingly, English allows for the
personal passive of \emph{help}, while this is not possible in German:
\eal
\ex[]{
because he was helped
}
\ex[]{
\gll weil ihm geholfen wurde\\
     because he.\DAT{} helped was\\
}
\ex[*]{
\gll weil    er geholfen wurde\\
     because he.\NOM{} helped was\\
}
\zl



%% 
%% \subsubsection{Isländisch}

%% \begin{itemize}
%% \item Dativ und Genitiv sind lexikalisch:



%% \end{itemize}

%% }



\subsubsection{Primary and secondary objects}


In this section I want to look at languages that allow both objects to be promoted. Danish is like
English in not having a dative. This is refelcted in the following \argstvs:
\ea\label{da-repr-hm-Danish}
%\resizebox{\linewidth}{!}
\z
Danish has two objects with structural case, English and German have just one object with structural
case and the other one with lexical accusative and lexical dative, respectively. Since English and
German do not allow for subjects with lexical case it is clear that the promotion to subject of the argument
that bears lexical case is excluded. Danish also disallows subjects with lexical case, but since the
two objects have structural case anyway, they both can be promoted.

Note however that the lexical rule in (\ref{lr-passive-prelim}) does not account for the promotion
of the secondary object. What it does is suppressing the subject. Under the assumption that the
first NP with structural case is the subject, the secondary object could never be realized as
subject. Note that it would not help to say any NP with structural case can be the subject, since
this would admit wrong realizations. In addition to the correct
(\ref{ex-fordi-manden-giver-drengen-bolden}), the following two sentences would be admitted: 
\eal
\ex[*]{
\gll fordi drengen giver manden bolden\\ 
     because boy.\textsc{def} gives man.\textsc{def} ball.\textsc{def}\\
}
\ex[*]{
\gll fordi   bolden         giver manden        drengen\\ 
     because ball.\textsc{def} gives man.\textsc{def} boy.\textsc{def}\\
}
\zl
(\mex{0}a) is ungrammatical with \emph{drengen} `boy' the recipient of the giving. Similarly the
transfered object \emph{bolden} cannot be realized as subject in active sentences. This means that
the promotion to subject has to be a part of the lexical  rule that licences the participle that is
used in the passive. The lexical rule in (\mex{1}) takes the \argstl and splits it into two
lists. The first list \ibox{1} is identical to the value of \da. The second list \ibox{2} is the
remainder of the \argstl. \ibox{2} is related to \ibox{3} by the relational constrint
\texttt{promote}. \ibox{3} is either eaqual to \ibox{2} or additional provides a list in which
another NP with structural case is positioned at the beginning of \ibox{3}.

\eas
\label{lr-passive-double-object}
Lexical rule for the passive for Danish, English, German, and Icelandic:\\
\ms{
head   & \ms[verb]{ da & \ibox{1}\\
                  }\\
arg-st & \ibox{1} $\oplus$ \ibox{2} \\
} $\mapsto$
\ms{
arg-st & \ibox{3} \\
} $\wedge$ promote(\ibox{2}, \ibox{3})
\zs

(\mex{1}) shows the \argstvs of our prototypical verbs:
\ea\label{da-repr-hm-Danish-participles}
%\resizebox{\linewidth}{!}{%
\begin{tabular}[t]{@{}l@{ }l@{ }l}
  &                        & \textsc{arg-st}\\[2mm]
%a.&ankomme (unacc):       & \liste{}                         & \liste{NP[\type{str}]}\\[2mm]
a.&danset/-s   (dance, unerg):     & \liste{}\\[2mm]
%c.&auf"|fallen (unacc): & \liste{}                         & \liste{NP[\type{str}], NP[\type{ldat}]}\\[2mm]
b.&læst/-s      (read, trans):   &  \liste{NP[\type{str}]$_j$ } \\[2mm]
c.&givet/-s      (give, ditrans): & \liste{NP[\type{str}]$_j$, NP[\type{str}]$_k$ } \\[2mm]
  &                         & \liste{NP[\type{str}]$_k$, NP[\type{str}]$_j$ } \\[2mm]
d.&hjulpet/-s    (help, trans):   & \liste{NP[\type{str}]$_j$ }                    \\
\end{tabular}
%}
\z
The NP[\str]$_i$ that is the first element in (\ref{da-repr-hm-Danish}) is suppressed. The effect of
\texttt{promote} is that there are two different \argstvs for the passive variants of \emph{givet}
`to give': one with an \argstl in which NP[\type{str}]$_j$ preceedes NP[\type{str}]$_k$ and another
one in which NP[\type{str}]$_j$ follows NP[\type{str}]$_k$. The first order corresponds to
(\ref{ex-boy-was-given-ball-danish}) -- repeated here as (\ref{ex-boy-was-given-ball-danish-two}) -- and
the second corresponds to (\ref{ex-ball-was-given-boy-danish}) -- repeated here as (\ref{ex-ball-was-given-boy-danish-two}):
\eal
\ex\label{ex-boy-was-given-ball-danish-two}
\gll fordi drengen bliver givet bolden\\ 
     because boy.\textsc{def} is given ball.\textsc{def}\\
\glt `because the boy is given the ball'
\ex\label{ex-ball-was-given-boy-danish-two}
\gll fordi bolden bliver givet drengen\\ 
     because ball.\textsc{def} is given boy.\textsc{def}\\
\glt `because the ball is given to the boy'
\zl
Before turning to impersonal passives in Danish in the next subsection, I discuss the passive in
double object constructions in Icelandic.


The distribution of structural/lexical case in Icelandic is basically the same as in German. The
difference is that Icelandic allows for subjects with lexical case and German does not.
(\mex{1}) shows our standard examples in Icelandic:
\ea\label{da-repr-hm-Icelandic}
%\resizebox{\linewidth}{!}{%
\begin{tabular}[t]{@{}l@{ }l@{ }l@{ }l@{ }l@{}}
  &                     & \textsc{arg-st}\\[2mm]
a.&dansa   (dance, unerg):     & \liste{ NP[\type{str}] }\\[2mm]
%c.&auf"|fallen (unacc): & \liste{}                         & \liste{NP[\type{str}], NP[\type{ldat}]}\\[2mm]
b.& lesa      (read, trans):   & \liste{ NP[\type{str}], NP[\type{str}] }\\[2mm]
c.&gefa       (give, ditrans): & \liste{ NP[\type{str}], NP[\type{ldat}], NP[\type{str}] }\\[2mm]
d.&hjálpa     (help, trans):   & \liste{ NP[\type{str}], NP[\type{ldat}] }\\
\end{tabular}
%}
\z
The lexical rule in (\ref{lr-passive-double-object}) licences the following participles:

\ea\label{da-repr-hm-Danish-two}
%\resizebox{\linewidth}{!}{%
\begin{tabular}[t]{@{}l@{ }l@{ }l@{~~~~~~}l@{~~~}l@{}}
  &                        & \textsc{arg-st}                     & \spr   & \comps\\[2mm]
%a.&ankomme (unacc):       & \liste{}                         & \liste{NP[\type{str}]}\\[2mm]
a.& dansað    (danced, unerg):     & \liste{}                        & \liste{ } & \liste{} \\[2mm]
%c.&auf"|fallen (unacc): & \liste{}                         & \liste{NP[\type{str}], NP[\type{ldat}]}\\[2mm]
b.& lesið      (read, trans):   &  \liste{NP[\type{str}]$_j$ }         & \liste{NP[\type{str}]$_j$ } & \eliste\\[2mm]
c.& gefið      (given, ditrans): & \liste{NP[\type{ldat}]$_j$, NP[\type{str}]$_k$ } & \liste{NP[\type{ldat}]$_j$ } & \liste{NP[\type{str}]$_k$ }\\[2mm]
  &                      & \liste{NP[\type{str}]$_k$, NP[\type{ldat}]$_j$ } & \liste{NP[\type{str}]$_k$ } & \liste{NP[\type{ldat}]$_j$ }\\[2mm]
d.& hjálpað    (helped, trans):   & \liste{NP[\type{ldat}]$_j$ }                  & \liste{ NP[\type{ldat}]$_j$ } & \liste{}\\
\end{tabular}
%}
\z\todostefan{Check \emph{read}}
In addition to the \argstl (\mex{0}) shows the mapping to the \spr and \comps features. Since
Icelandic allows for quirky subjects the dative argument of `to help' can be mapped to the
\sprl \citep[\page 147--148]{Wechsler95a-u}. Similarly the two orders of the \argst of `to give' result in participles with a dative
subject and a nominative subject as it is required for the analysis of (\ref{ex-were-the-king-given-the-slaves}) and (\ref{ex-were-the-slaves-given-the-king}) repeated
here as (\mex{1}):
\eal
\ex\label{ex-were-the-king-given-the-slaves-two}
\gll Voru konunginum gefnar ambfittir?\\
     were the.king.\DAT{} given slaves.\NOM{}\\
\glt `Was the king given slaves?'
\ex\label{ex-were-the-slaves-given-the-king-two}
\gll Var ambfittin gefin konunginum?\\
     was the.slave.\NOM{} given the.king.\DAT\\
\glt `Was the slave given to the king?'
\zl

The impersonal passive with `to dance' is parallel to the German impersonal passive, but the
passivization of `to help' differs since this is an instance of the personal passive in Icelandic.


\subsubsection{Impersonal passive}
\label{sec-impersonals}

As a final point in this subsection let us have a look at the impersonal passive. German and
Icelandic do not insist on subjects. So if there is no NP with structural case, the construction in
German is subjectless. Similarly Icelandic does not require a subject: If there is no NP argument,
the result is an impersonal passive. An example of the latter case is the passivization of
\emph{dansa} `to dance'. The \argstl is the empty list and therefore the \sprl and the \compsl are
empty as well. Passive participles of verbs that govern an NP and a PP object will have an \argstl
that just contains the PP argument. This PP argument will be mapped to the \compsl and hence a
subjectless construction will result.


English does not allow for impersonal passives since it requires an NP or a sentential argument that
can serve as a subject. Danish requires a subject as well, but allows for impersonal
constructions. The trick that Danish employs is the insertion of an expletive. I assume that the
expletive insertion happens during the mapping of the \argst elements to \spr and \comps. If there
is an NP/VP/CP at the beginning of the \argstl, it is mapped to \spr and all other elements are
mapped to \comps. If there is no element that can be mapped to \spr, an expletive is inserted.
\nocite{BB2007a}

(\mex{1}) shows the mappings for Danish.
\ea\label{da-repr-hm-Danish-three}
%\resizebox{\linewidth}{!}{%
\begin{tabular}[t]{@{}l@{ }l@{ }l@{ }l@{ }l@{~~~~~}l@{}}
  &                        & \textsc{arg-st}                     & \spr   & \comps\\[2mm]
%a.&ankomme (unacc):       & \liste{}                         & \liste{NP[\type{str}]}\\[2mm]
a.&danset/-s   (unerg):     & \liste{}                        & \liste{ NP$_{expl}$ } & \liste{} \\[2mm]
%c.&auf"|fallen (unacc): & \liste{}                         & \liste{NP[\type{str}], NP[\type{ldat}]}\\[2mm]
b.&læst/-s      (trans):   &  \liste{ NP[\type{str}]$_j$ }                     & \liste{ NP[\type{str}]$_j$ } & \eliste\\[2mm]
c.&givet/-s      (ditrans): & \liste{ NP[\type{str}]$_j$, NP[\type{str}]$_k$ } & \liste{ NP[\type{str}]$_j$ } & \liste{ NP[\type{str}]$_k$ }\\[2mm]
  &                         & \liste{ NP[\type{str}]$_k$, NP[\type{str}]$_j$ } & \liste{ NP[\type{str}]$_k$ } & \liste{ NP[\type{str}]$_j$ }\\[2mm]
d.&hjulpet/-s    (trans):   & \liste{ NP[\type{str}]$_j$ }                     & \liste{ NP[\type{str}]$_j$ } & \liste{ }\\
\end{tabular}
%}
\z







%\if 0
%\section{Variation and Generalizations}

\subsubsection{The passive auxiliary}
\label{sec-auxiliary}



\begin{itemize}
\item Das Passivhilfsverb ist für alle behandelten Sprachen ähnlich:
\ea
Passivhilfsverb für Dänisch, Deutsch, Englisch:
\ms{
arg-st \ibox{1} $\oplus$ \ibox{2} $\oplus$  \liste{ \ms{ vform & ppp\\
                                                                        da & \sliste{ XP$_{ref}$ }\\
                                                                                      spr   & \ibox{1}\\
                                                                                      comps & \ibox{2}\\
                                                                                    } } 
}
\z


\item \daw schließt unakkusatische Verben und Wetterverben aus


\item Deutsch bildet Verbalkomplex: Argumente des Partizips (\ibox{1} and \ibox{2}) werden vom
  Passivhilfsverb angezogen \citep{HN89a}. 


\item Verbalkomplexschema erlaubt ungesättigte Nicht-Kopftochter.


\item Funktioniert auch für Sprachen, die keine Verbalkomplexe bilden:\\
\ibox{2} ist dann die leere Liste. 

%% \item Hence, we have explained how
%% Danish and English embed a VP and German forms a verbal complex although the lexical item of the
%% auxiliary does not require a VP complement.

\end{itemize}









\subsubsection{Das morphologische Passiv}


%% We assume that the same lexical rule that accounts for the participle forms can be used for the
%% morphological passives in Danish, modulo differences in the realizations of affixes of course. For
%% the morphological passive it is assumed that the \da of the input to the lexical rule has to contain
%% a referential XP. As was discussed in the previous section, this excludes morphological passives of
%% unaccusatives and weather verbs. 

\begin{itemize}
\item Lexikonregel funktioniert auch für das morphologische Passiv. Es wird einfach ein \suffix{s} angehängt.
\end{itemize}



%% \section{Agent Expressions}

%% We follow \citet[Chapter~7]{Hoehle78a} and \citet[Section~5]{Mueller2003e} and treat the \emph{by}
%% phrases as adjuncts.


\subsubsection{Perfect}

\begin{itemize}
\item Deutsch: Nur ein Partizip für Passiv und Perfekt \citep{Haider86}. 

\item Das designated argument wird blockiert, ist aber im Lexikonelement enthalten

\item Perfekthilfsverb deblockiert es.
\eal
\ex
Der Aufsatz wurde gelesen.
\ex
Er hat den Aufsatz gelesen.
\zl

\ms{
arg-st \ibox{1} $\oplus$ \ibox{2} $\oplus$ \ibox{3} $\oplus$  \liste{ \ms{ vform & ppp\\
                                                                        da & \ibox{1}\\
                                                                        spr   & \ibox{2}\\
                                                                        comps & \ibox{3}\\
                                                                       } } 
}


\end{itemize}



%\subsubsection{Analyse als komplexes Prädikat für Dänisch und Englisch?}

\begin{itemize}
\item Bei einer Analyse mit Argumentdeblockierung müsste man Struktur in (\mex{1}a--b) annhemen:
\eal
\ex He [has given] the book to Mary.
\ex The book [was given] to Mary.
\ex He has [given the book to Mary].
\ex The book was [given to Mary].
\zl

Sonst wüssten wir zu spät vom deblockierten Subjekt, denn das Partizip würde ja nur -- wie in
(\mex{0}d) eine PP verlangen.


\end{itemize}



%% 
%% \subsubsection{Expletives}

%% \begin{itemize}
%% \item Expletives needed for passive only:

%% \eal
%% \ex[]{
%% \gll  at   der        bliver arbejdet\\
%%       that \textsc{expl} is     worked\\
%% }
%% \ex[*]{
%% \gll  at   Peter har arbejdet der\\
%%       that Peter has worked   \textsc{expl}\\
%% }
%% \ex[*]{
%% \gll  at   der        har arbejdet Peter\\
%%       that \textsc{expl} has worked   Peter\\
%% }
%% \zl

%% \end{itemize}

%% }



%% 
%% \subsubsection{A Solution that Almost Works}

%% \begin{itemize}
%% \item Complex Passive: There has to be a way to distinguish between participles that can be used in both perfect and
%%   passive:\\
%% \textsc{voice} feature. 

%% \begin{itemize}
%% \item Value is \type{passive} for those participles that cannot be used in perfect constructions.

%% 
%% \item Value is underspecified for participles that can be used in both perfect and passive

%% 
%% \item Perfect requires \textsc{voice} value to be \type{active}.
%% \end{itemize}

%% 
%% \item Expletives: Perfect attracts args from \argstl rather than \spr/\comps.
%% \begin{itemize}
%% \item Since expletives are not on \argst, they will not get into the way.
%% \end{itemize}
%% \end{itemize}


%% }



%\subsubsection{But: (Partial) Fronting}
%\subsubsection{Problem: (Partial) Fronting}


\begin{itemize}
\item \citet{Meurers99b} hat einen Trick gefunden, wie man die Kasuszuweisung in (\mex{1})
  analysieren kann:
\nocite{Meurers2000b,MdK2001a}
\eal
\ex 
Gelesen wurde der Aufsatz schon oft.
\ex 
Der Aufsatz gelesen wurde schon oft.
\ex
Den Aufsatz gelesen hat er schon oft.
\zl

\item Das funktioniert aber nicht für Dänisch/Englisch, denn hier haben wir nicht nur Kasus- sondern
  auch Positionsunterschiede:
\eal
\ex The book should have been given to Mary and\\
    {}[given to Mary] it was.
\ex He wanted to give the book to Mary and\\
    {}[given the book to Mary] he has.
\zl

Wenn sich keine ausgeklügelten Mechanismen für die Unterspezifikation verschiedenenr Mappings finden
lassen, müssen wir wohl zwei verschiedene Partizipformen annehmen.
for the participle form.
\end{itemize}










\subsubsection{The remote passive}
\label{sec-remote-passive-phen}


\begin{itemize}
\item \citet[\page 175--176]{Hoehle78a}: in bestimmten Kontexten Objekte
von \emph{zu}-Infinitiven im Nominativ.

Die folgenden Sätze sind Beispiele für das sogenannte Fernpassiv:
\eal
\ex
daß er auch von mir zu überreden versucht wurde\footnote{
        \citew*[\page 212]{Oppenrieder91a}.%
}
\ex
weil    der Wagen oft zu reparieren versucht wurde
\zl
\end{itemize}



Akkusativobjekte eingebetteter Verben können im Passiv zum Nominativ werden:
\eal
\ex Dabei darf jedoch nicht vergessen werden, daß in der Bundesrepublik, wo \emph{ein Mittelweg} \emph{zu gehen versucht wird}, 
die Situation der Neuen Musik allgemein und die Stellung der Komponistinnen im besonderen noch recht unbefriedigend ist.\footnote{
Mannheimer Morgen, 26.09.1989, Feuilleton; Ist's gut, so unter sich zu bleiben?
}

\ex Noch ist es nicht so lange her, da ertönten gerade aus dem Thurgau jeweils die lautesten Töne, 
    wenn im Wallis oder am Genfersee im Umfeld einer Schuldenpolitik mit den unglaublichsten Tricks 
    \emph{der sportliche Abstieg} \emph{zu verhindern versucht wurde}.\footnote{
St.\ Galler Tagblatt, 09.02.1999, Ressort: TB-RSP; HCT und das Prinzip Hoffnung.%
}

\ex Die Auf- und Absteigenden erzeugen ungewollt einen Ton,
        \emph{der} bewusst nicht als lästig \emph{zu eliminieren versucht wird}, 
    sondern zum Eigenklang des Hauses gehören soll, so wünschen es sich die Architekten.\footnote{
Züricher Tagesanzeiger, 01.11.1997, p.\,61.%
}
\zl



\subsubsection{Beispiele mit \word{beginnen}, \word{vergessen} und \word{wagen}}

\citet{Wurmbrand2003a}:
\eal
%% \ex
%% \emph{dieser} wurde bereits zu bauen begonnen.\footnote{
%%         \url{http://www.hollabrunn.noe.gv.at/mariathal/ortsvorsteher.html}, 28.07.2003.
%% }

\ex
\emph{der zweite Entwurf} wurde zu bauen begonnen,\footnote{
\url{http://www.waclawek.com/projekte/john/johnlang.html}, 28.07.2003.
}
\zl

\eal
\ex Anordnungen, die zu stornieren vergessen \emph{wurden}\footnote{
        \url{http://www.rlp-irma.de/Dateien/Jahresabschluss2002.pdf}, 28.07.2003.
}

\ex Aufträge [\ldots], die zu drucken vergessen worden \emph{sind}\footnote{
        \url{http://www.iitslips.de/news.html}, 28.07.2003.
}
\zl

\eal
%\ex Ist plötzlich übervoll von Emotionen und längst begrabenen Träumen, die nicht zu leben gewagt wurden\footnote{
% nicht auffindbar
\ex NUR Leere, oder doch noch Hoffnung, weil aus Nichts wieder Gefühle entstehen,
    die so vorher nicht mal zu träumen gewagt \emph{wurden}?\footnote{
        \url{http://www.ultimaquest.de/weisheiten_kapitel1.htm}, 28.07.2003.
}

\ex Dem Voodoozauber einer Verwünschung oder die gefaßte Entscheidung zu einer Trennung,
    die bis dato noch nicht auszusprechen gewagt \emph{wurden}\footnote{
        \url{http://www.wedding-no9.de/adventskalender/advent23_shawn_colvin.html}, 28.07.2003.
}
\zl
% Kasus bei PVP wie Haiders entziffern: Am leichtesten zu erklären fiel den 
% Experten dabei gestern der Kursverlust der Telekom, zu deren Schuldenproblem 
% eine neue Hiobsbotschaft kam.  (taz. 8./9. 9. 01 S. 9.)
%




%\subsubsection{Fernpassiv und Verbalkomplexbildung (I)}

\begin{itemize}
\item Objekt eines Verbs, das unter ein Passivpartizip eingebettet ist,\\
wird zum Subjekt des Satzes:
\eal
\ex weil er den Wagen oft zu reparieren versucht hat
\ex weil der Wagen oft \emph<2>{zu reparieren versucht wurde}
\zl

\item Fernpassiv nur bei Verbalkomplexbildung möglich:
\eal
\ex[]{
weil oft versucht wurde, \emph{den Wagen zu reparieren}
}
\ex[*]{
weil oft versucht wurde, der Wagen zu reparieren
}

\ex[]{
\emph{Den Wagen zu reparieren} wurde oft versucht
}
\ex[*]{
Der Wagen zu reparieren wurde oft versucht
}
\zl
\end{itemize}



%\subsubsection{Fernpassiv und Verbalkomplexbildung (II)}

\begin{itemize}

\item Erklärung: Fernpassiv = Passivierung des Prädikatskomplexes
\ea
weil    der Wagen     oft   [[zu reparieren versucht] wurde]
%
\z


\item In (\mex{1}a,c) liegen keine Verbalkomplexe vor. 
\eal
\ex[]{
weil oft versucht wurde, \emph<2>{den Wagen zu reparieren}
}
\ex[*]{
weil oft versucht wurde, der Wagen zu reparieren
}
\ex[]{
\emph<2>{Den Wagen zu reparieren} wurde oft versucht
}
\ex[*]{
Der Wagen zu reparieren wurde oft versucht
}
\zl

Objekt von \emph{zu reparieren} ist Teil der VP $\to$ bekommt Akkusativ


Die Passive in (\mex{0}a,c) sind unpersönliche Passive.

\end{itemize}


%% \centerline{\scalebox{1}{
%% %\centerfit{%
%% \begin{tikzpicture}
%% \tikzset{level 1+/.style={level distance=4\baselineskip}}
%% \tikzset{level 2+/.style={level distance=5\baselineskip}}
%% \tikzset{frontier/.style={distance from root=14\baselineskip}}
%% \Tree[.V\feattab{
%%               \vform \type{fin},\\
%%               \comps \ibox{1} } 
%%         [.{\ibox{3} V\feattab{
%%               \vform \type{ppp},\\
%%               \subj  \ibox{1},\\
%%               \comps \eliste }} 
%%            [.{\highlight{\ibox{2} V}<1>\feattab{
%%               \highlight{\vform \type{inf}}<1>,\\
%%               \subj  \highlight{\sliste{ NP[\type{str}] }}<2,3>, \\ 
%%               \comps \highlight{\ibox{1} \sliste{ NP[\type{str}] }}<2,3> }} {zu reparieren} ]
%%            [.V\feattab{
%%               \vform \type{ppp},\\
%%               \subj  \ibox{1},\\
%%               \comps \sliste{ \ibox{2} } } versucht ] ]
%%         [.V\feattab{
%%               \vform \type{fin},\\
%%               \comps \ibox{1} $\oplus$ \sliste{
%%                 \ibox{3} }} wurde ] 
%% ]
%% \end{tikzpicture}}
%% }

\centerline{\scalebox{1}{
\begin{forest}
sm edges
[V\feattab{
              \vform \type{fin},\\
              \comps \ibox{1} } 
        [{\ibox{3} V\feattab{
              \vform \type{ppp},\\
              \subj  \ibox{1},\\
              \comps \eliste }} 
           [{\ibox{2} V\feattab{
              \vform \type{inf},\\
              \subj  \sliste{ NP[\str] }, \\ 
              \comps \ibox{1} \sliste{ NP[\str] } }} [zu reparieren] ]
           [V\feattab{
              \vform \type{ppp},\\
              \subj  \ibox{1},\\
              \comps \sliste{ \ibox{2} } } [versucht] ] ]
        [V\feattab{
              \vform \type{fin},\\
              \comps \ibox{1} $\oplus$ \sliste{
                \ibox{3} }} [wurde] ] 
]
\end{forest}}}

\begin{itemize}
\item \emph{versuchen} zieht Argumente von \emph{reparieren} an: \argstw \sliste{ NP[\type{str}], NP[\type{str}], V[\type{inf}] }

\item Passiv-LR unterdrückt erstes Argument: \emph{versucht} hat
\argstw \sliste{ NP[\type{str}], V[\type{inf}] } 

\item \emph{zu reparieren versucht}: \argstw \sliste{ NP[\type{str}] } und
\emph{zu reparieren versucht wurde} auch
\end{itemize}




%\subsubsection{Fernpassiv mit Objektkontrollverben}

\begin{itemize}
\item Fernpassiv auch mit Objektkontrollverben möglich:
\eal
\ex
Keine Zeitung         wird ihr       zu lesen erlaubt.\footnote{
        Stefan Zweig. \emph{Marie Antoinette}. Leipzig: Insel-Verlag. 1932, p.\,515, 
        zitiert nach \citew[\page 309]{Bech55a}. Siehe \citet[\page 13]{Askedal88}.
}
\ex\iw{auskosten}
Der Erfolg        wurde uns      nicht auszukosten erlaubt.\footnote{
        \citew[\page 110]{Haider86c}.%
}
\zl


\item Passiv der Konstruktion ohne Verbalkomplex ist ein unpersönliches Passiv:
\eas
Uns wurde erlaubt, den Erfolg auszukosten.
\zs

\item Generalisierung: In Passivkonstruktionen, in denen ein Verbalkomplex unter das Passivhilfsverb
eingebettet ist, wird das Subjekt unterdrückt und von den verbleibenden Argumenten
wird das erste Argument mit strukturellem Kasus zum Subjekt und bekommt Nominativ.%
\end{itemize}


%\subsubsection{Fernpassiv mit Objektkontrollverben}

\ea
Keine Zeitung         wird ihr       zu lesen erlaubt.\footnote{
        Stefan Zweig. \emph{Marie Antoinette}. Leipzig: Insel-Verlag. 1932, p.\,515, 
        zitiert nach \citew[\page 309]{Bech55a}. Siehe \citet[\page 13]{Askedal88}.
}
\z

\oneline{%
\begin{tabular}{@{}l@{ }l@{}}
\emph{erlauben}: & \sliste{ NP[\type{str}]$_i$, NP[\ldat]$_j$ } $\oplus$ \ibox{1} $\oplus$ \sliste{ V[\comps \ibox{1}] }\\

\emph{zu lesen erlauben}: & \sliste{ NP[\type{str}]$_i$, NP[\ldat]$_j$, NP[\type{str}]$_k$, V[\comps \sliste{ NP[\type{str}]$_k$ }] }\\

\emph{zu lesen erlaubt wird}: & \sliste{ NP[\ldat]$_j$, NP[\type{str}]$_k$, V[\comps \sliste{ NP[\type{str}]$_k$ } ] }\\
\end{tabular}
}



Erste NP mit strukturellem Kasus ist Subjekt.





%% \subsubsection{Complex Passives}

%% \begin{itemize}
%% \item Complex Passives:

%% \ea
%% \gll at Bilen           blev forsøgt repareret\\
%%      that car.\textsc{def} was  tried   repaired\\
%% \glt `that an attempt was made to repair the car'
%% \z



%% \item Raising in passive only.



%% \item \emph{forsøgt} (`to try') does not even take a participle in the active:
%% \eal
%% \ex[]{
%% \gll at   Peter har  forsøgt \emphbf{at} \emphbf{reparere} bilen\\
%%      that Peter has  tried   to repair   car.\textsc{def}\\
%% \glt `that Peter tried to repair the car'
%% }
%% \ex[*]{
%% \gll at   Peter har  forsøgt \emphbf{repareret} bilen\\
%%      that Peter has  tried   repaired car.\textsc{def}\\
%% %\glt `that an attempt was made to repair the car'
%% }
%% \zl

%% %% \item Conclusion: We need special lexical items for passive participles.

%% %% \item analysis of the German passive and perfect can be maintained,\\
%% %% compatible with a more general analysis of the passive

%% \end{itemize}



\subsubsection{Summary}





\begin{itemize}
\item account for the Danish, Englisch and German passive 

\item LRs for morphological and analytical passives

\item first element on the \argstl is suppressed

\item \emph{promote} promotes any NP with structural case


\item languages differ in cases and the lexical/structural distinction

\item expletive is inserted in the \argst mapping in Danish.


\item SVO languages seem to require different items for the perfect/passive participles, but
  analysis for German can be maintained.


\end{itemize}







%\fi

\questions{
\begin{enumerate}
\item What tests do you know for subjecthood?
\item Do these tests work for all Germanic languages?
\item In which way is German different from Icelandic in terms of subjects?
\item What is structural case? What is lexical case?

\item What is an impersonal passive?
\item Does Icelandic have impersonal passives?

\end{enumerate}

}

\exercises{


\begin{enumerate}
\item Which NPs in (\mex{1}) do have structural and which lexical case?
\eal
\ex 
\gll Der        Junge lacht.\\
     the.\NOM{} boy   laughs\\
\ex 
\gll Mich friert.\\
     I.\ACC{} freeze\\
\glt `I am cold.'
\ex 
\gll Er zerstört das Auto.\\
     he.nom{} destroys the.\ACC{} car\\
\glt `He destroys the car.'
\ex 
\gll Das dauert ein ganzes Jahr.\\
     this.\NOM{} takes  a.acc{} whole year\\
\glt `This takes a whole year.'
\ex 
\gll Er hat nur einen Tag dafür gebraucht.\\
     he.\NOM{} has just one.\ACC{}  day there.for needed\\
\glt `He needed a day for this.'
\ex 
\gll Er denkt an den morgigen Tag.\\
     he.\NOM{} thinks at the.\ACC{} tomorrow day\\
\glt `He thinks about tomorrow.'
\zl


\item Give \argst lists for the following verbs: 
\eal
\ex show, read, meet \english
\ex zeigen, begegnen, treffen \german
\ex \danish
\ex \icelandic
\zl
If you are uncertain as far as case is concerned, you may use the
  Wiktionary\footnote{
\url{https://de.wiktionary.org/}, 2018-07-02.
}.

\item Draw the analysis tree for the following sentence:
\ea
that the box was opened
\z
Please provide valence features (\spr and \comps) and part of speech information. You may abbreviate
the NP using a triangle.

%% \item Draw the analysis tree for the following sentence:
%% \ea
%% \gll dass der Kasten geöffnet wurde\\
%%      that the box    opened was\\
%% \glt `that the box opened was'
%% \z
%% Please provide valence features (\comps only) and part of speech information. You may abbreviate
%% the NP using a triangle.

\end{enumerate}

}




%      <!-- Local IspellDict: en_US-w_accents -->
