%% -*- coding:utf-8 -*-

\chapter{The verbal complex}


SOV languages like Dutch and German form verbal complexes. There are several indicators for this
that were worked out in detail by Gunar \citet{Bech55a}. One way to analyze such verbal
complexes is to assume that the verbs in a sentence form a unit that basically behaves like a
simplex verb. This explains for instance why the arguments of the three verbs in Haider's example \citeyearpar{Haider90b} in (\mex{1}) can be scrambled:
\ea\label{ex-weil-es-ihm-jemand-zu-lesen-versprochen-hat}
\gll weil es ihm jemand zu lesen versprochen hat\\
     because it him somebody to read promised has\\
\glt `because somebody promised him to read it'
\z
\emph{es} depends on \emph{zu lesen} `to read', \emph{ihm} `him' depends on \emph{versprochen}
`promised' and \emph{jemand} is the subject and agrees with the finite verb \emph{hat} `has'
(usually it is also treated as a dependent of the auxiliary \emph{hat}).

It should be said that there is extreme variation in the German dialects as far as the serialization
of elements in the verbal complex ist concerned. The governing verb is realized to the right of the
embedded verb in Standard German: V$_3$ V$_2$ V$_1$ as in (\mex{0}), but there are examples like
(\mex{2}) taken from \citew[376]{Mueller99a}.\footnote{
Interview partner in: \emph{Insekten und andere Nachbarn -- ein Haus in Berlin}, ARD 15.11.1995.
}
\eal
\ex Ich hätte stapelweise Akten kön\-nen haben.
\ex weil ich mir das nich hab' lassen gefallen
\ex wenn se mir hier würden rausschmeißen, \ldots
\zl
The orders in (\mex{0}) correspond to the order that is most natural in Dutch. (\mex{1}) shows some
Dutch examples: 
\eal
\ex
\gll dat Jan het boek wil lezen\\
     that John the book wants read\\
\glt 'that John wants to read the book' 
\ex
\gll dat Jan Marie het boek laat lezen\\
     that John Mary the book lets read\\
\glt 'that John lets Mary read the book'
\ex 
\gll dat Jan Marie het boek wil laten lezen\\
     that John Mary the book wants let read \\
\glt 'that John wants to let Mary read the book'
\zl

SVO languages like Danish and English do not allow the arguments of embedded verbs to be scrambled
with arguments of higher verbs. All arguments stay in their VP (modulo extraction, of course).




The trick that is used to analyze the verbal complexes is called \emph{argument attraction} or
\emph{argument composition} and was developed by \citet{Geach70a} in the framework of Categorial
Grammar and adapted for HPSG by \citet{HN94a}. The analysis of \emph{lesen wird} `read will' as it occurs in
(\mex{1}) is shown in Figure~\vref{fig-lesen-wird}.
\ea
\gll dass keiner das Buch lesen wird\\
     that nobody the book read will\\
\glt `that nobody will read the book'
\z
\begin{figure}
\begin{forest}
sm edges
[V\feattab{
%              \vform \type{fin},\\
              \sliste{ NP[\type{nom}], NP[\type{acc}] } } 
        [V\feattab{
%              \vform \type{bse},\\
              \sliste{ NP[\type{nom}], NP[\type{acc}]} }, name=lesen [lesen;read] ]
        [V\feattab{
%              \vform \type{fin},\\
              \sliste{ NP[\type{nom}], NP[\type{acc}]}, V}, name=wird [wird;will] ]
]
\draw[semithick,->] (lesen)..controls +(south east:2) and +(south west:2)..(wird);
\end{forest}
\caption{\label{fig-lesen-wird}Analysis of the verbal complex formation of \emph{lesen wird} `read
  will' using argument composition}
\end{figure}
\emph{wird} `will' selects an infinitive without \emph{zu} and in addition its arguments. This
infinitive (\emph{lesen} `read') is combined with the verb and hence is not contained in the valency list
of the mother node.
The combination of \emph{lesen} and \emph{wird} behaves like a simplex verb in that it can be
combined with its arguments in any order. Figure~\vref{fig-vc-nom-acc} shows the analysis of (\mex{1}a) and
Figure~\vref{fig-vc-acc-nom} shows the analysis of (\mex{1}b).
\eal
\ex\label{ex-dass-keiner-das-buch-lesen-wird}
\gll [dass]         keiner das Buch lesen wird\\
     \spacebr{}that nobody the book read will\\
\glt `that nobody will read the book'
\ex  
\gll [dass] das Buch keiner lesen wird\\
     \spacebr{}that the book nobody read will\\
\glt `that nobody will read the book'
\zl

\begin{figure}
\centerfit{
\begin{forest}
sm edges
[V\feattab{
              \sliste{ }}
        [{NP[\type{nom}]} [keiner;nobody] ]
        [V\feattab{
              \sliste{ NP[\type{nom}] }}
          [{NP[\type{acc}]} [das Buch;the book, roof] ]
          [V\feattab{
%              \vform \type{fin},\\
              \sliste{ NP[\type{nom}], NP[\type{acc}]}} 
             [V\feattab{
%              \vform \type{bse},\\
              \sliste{ NP[\type{nom}], NP[\type{acc}]}} [lesen;read] ]
             [V\feattab{
%              \vform \type{fin},\\
                \sliste{ NP[\type{nom}], NP[\type{acc}], V }} [wird;will] ] ] ] ]
\end{forest}}
\caption{\label{fig-vc-nom-acc}Formation of a verbal complex and realization of arguments in normal order}
\end{figure}




\begin{figure}
\centerfit{
\begin{forest}
sm edges
[V\feattab{
              \sliste{ }}
        [{NP[\type{acc}]} [das Buch;the book, roof] ]
        [V\feattab{
              \sliste{ NP[\type{acc}] }}
          [{NP[\type{nom}]} [keiner;nobody] ]
          [V\feattab{
%              \vform \type{fin},\\
              \sliste{ NP[\type{nom}], NP[\type{acc}] }} 
             [V\feattab{
%              \vform \type{bse},\\
              \sliste{ NP[\type{nom}], NP[\type{acc}]}} [lesen;read] ]
             [V\feattab{
%              \vform \type{fin},\\
                \sliste{ NP[\type{nom}], NP[\type{acc}], V }} [wird;will] ] ] ] ]
\end{forest}}
\caption{\label{fig-vc-acc-nom}Formation of a verbal complex and scrambling of arguments}
\end{figure}

I follow \citet[Section~3.1.1]{Kiss95a} and represent the subject of non-finite verbs as the value of a special
feature \subj. \subj differs from \spr and \comps in that it is not a valency feature. The reason
for this special treatment is that the subject cannot be realized as a part of a non-finite verb
phrase:
\eal
\judgewidth{?*}
\ex[]{
\gll Das Buch lesen wird der Mann morgen.\\
     the book read  will the man  tomorrow\\
\glt `The man will read the book tomorrow.'
}
\ex[*]{
\gll Der Mann lesen wird das Buch morgen.\\
     the man  read  will the book tomorrow\\
}
\ex[?*]{
\gll Der Mann das Buch lesen wird morgen.\\
     the man  the book read will tomorrow\\
}
\zl
The lexical item for the non-finite form of \emph{lesen} `to read' is given in (\mex{1}):
\ea
\emph{lesen} `to read' non-finite form:\\
\ms{
subj  & \sliste{ NP[\type{nom}] }\\
comps & \sliste{ NP[\type{acc}] }\\
}
\z

The following Attribute Value Matrix (AVM)\is{Attribute Value Matrix (AVM)} is a representation of the auxiliary verb \emph{werden} `will':
\ea
\emph{werden} `will' non-finite form:\\
\ms{
subj  & \ibox{1}\\
comps & \ibox{2} $\oplus$ \sliste{ V[ \vform \type{bse}, \textsc{lex}+, \subj \ibox{1}, \comps \ibox{2} ] }\\
}
\z
\emph{werden} selects a verb that has the \type{bse} form, that is an infinitive without \emph{zu}
`to'. The embedded element has to be lexical (\textsc{lex}+), that is, a single word or a verbal
complex. All phrases that are licensed by the Head-Complement Schema and the Specifier-Head Schema
are assumed to be \textsc{lex}$-$.
The boxes with numbers are basically variables. Their values depend on the values of the
embedded verbs. Therefore this lexical item can be used with a verb like \emph{lesen} `to read',
which takes a nominative and an accusative case but also with a verb like \emph{helfen} `to help',
which takes a nominative and a dative object.

Before I turn to the details of the analysis, I have to provide the lexical items for the finite
form of auxiliaries. Since the subject of finite verbs can of course be realized it has to be
represented in one of the valency lists. As was discussed in Section~\ref{sec-intro-spr-comps},
German subjects are represented in the \compsl of finite verbs. Hence the lexical item for
\emph{wird} `will' has the following form:
\ea
\emph{wird} `will' finite form:\\
\ms{
subj  & \sliste{}\\
comps & \ibox{1} $\oplus$ \ibox{2} $\oplus$ \sliste{ V[ \vform \type{bse}, \textsc{lex}+, \subj \ibox{1}, \comps \ibox{2} ] }\\
}
\z
This basically says that the valency of \emph{wird} consists of an embedded verb and whatever the
\subjl of this verb is plus whatever the \compsl of this verb is. This is exemplified for
\emph{lesen wird} in Figure~\vref{fig-lesen-wird-details}.
\begin{figure}
\begin{forest}
sm edges
[V\feattab{
              \vform \type{fin},\\
              \comps \ibox{1} $\oplus$ \ibox{2} } 
        [{\ibox{3} V}\feattab{
              \vform \type{bse},\\
              \subj  \ibox{1} \sliste{ NP[\type{nom}] }, \\ 
              \comps \ibox{2} \sliste{ NP[\type{acc}] } } [lesen;read] ]
        [V\feattab{
              \vform \type{fin},\\
              \comps \ibox{1} $\oplus$ \ibox{2} $\oplus$ \sliste{ \ibox{3} } } [wird;will] ] ]
\end{forest}
\caption{\label{fig-lesen-wird-details}Detailed analysis of a verbal complex}
\end{figure}
The auxiliary selects an infinitive without \emph{zu} `to' \iboxb{3}. This is ensured by the value
\type{bse} for the \vformf of the selected verb: \type{bse}\istype{bse} stands for infinitive without
\emph{to}/\emph{zu}/\ldots{}, \type{inf}\istype{inf} stands for an infinitive form with marker, \type{ppp}\istype{ppp}
stands for participle and \type{fin}\istype{fin} for a finite verb. The subject of the selected infinitive
\iboxb{1} and the complements \iboxb{2} are taken over. The result is that \emph{lesen wird} has the
same arguments as \emph{liest} `reads'. 

To make all of this even more fun, we can make it more complex and look at verbal complexes with
three verbs. Figure~\vref{fig-lesen-koennen-wird} shows the analysis of the verbal complex \emph{lesen können wird} `read
can will' in sentences like (\mex{1}):
\ea
\label{ex-lesen-koennen-wird}
\gll [dass] er das Buch lesen können wird\\
     \spacebr{}that he the book read can will\\
\z



\begin{figure}
\centerfit{%
\begin{forest}
sm edges
[V\feattab{
              \vform \type{fin},\\
              \comps \ibox{1} $\oplus$ \ibox{2} } 
        [{\ibox{4} V\feattab{
              \vform \type{bse},\\
              \subj  \ibox{1},\\
              \comps \ibox{2} }} 
           [{\ibox{3} V\feattab{
              \vform \type{bse},\\
              \subj  \ibox{1} \sliste{ NP[\type{nom}] }, \\ 
              \comps \ibox{2} \sliste{ NP[\type{acc}] } }} [lesen;read] ]
           [V\feattab{
              \vform \type{bse},\\
              \subj  \ibox{1},\\
              \comps \ibox{2} $\oplus$ \sliste{
                \ibox{3} } } [können;can] ] ]
        [V\feattab{
              \vform \type{fin},\\
              \comps \ibox{1} $\oplus$ \ibox{2} $\oplus$ \sliste{
                \ibox{4} } } [wird;will] ] 
]
\end{forest}}
\caption{\label{fig-lesen-koennen-wird}Analysis of a German verbal complex with three verbs in cannonical order}
\end{figure}

One interesting aspect of the analysis is that it can explain a phenomenon that is called Auxiliary
Flip\is{Auxiliary Flip} or \emph{Oberfeldumstellung}\is{Oberfledumstellung@\emph{Oberfeldumstellung}}. German optionally allows verbs that govern a modal to be placed
to the left of the verbal complex rather than to the right of the modal. So instead of
(\ref{ex-lesen-koennen-wird}) one can also use the order in (\mex{1}):
\ea
\gll [dass] er das Buch wird lesen können\\
     \spacebr{}that he the book will read can\\
\z
\begin{figure}
\centerfit{%
\begin{forest}
sm edges
[V\feattab{
              \vform \type{fin},\\
              \comps \ibox{1} $\oplus$ \ibox{2} } 
        [V\feattab{
              \vform \type{fin},\\
              \comps \ibox{1} $\oplus$ \ibox{2} $\oplus$ \sliste{
                \ibox{4} } } [wird;will] ]
        [{\ibox{4} V\feattab{
              \vform \type{bse},\\
              \subj  \ibox{1},\\
              \comps \ibox{2} }} 
           [{\ibox{3} V\feattab{
              \vform \type{bse},\\
              \subj  \ibox{1} \sliste{ NP[\type{nom}] }, \\ 
              \comps \ibox{2} \sliste{ NP[\type{acc}] } }} [lesen;read] ]
           [V\feattab{
              \vform \type{bse},\\
              \subj  \ibox{1},\\
              \comps \ibox{2} $\oplus$ \sliste{
                \ibox{3} } } [können;can] ] ] 
]
\end{forest}}
\caption{\label{fig-wird-lesen-koennen}Analysis of a German verbal complex with three verbs with
  Auxiliary Flip}
\end{figure}


After having discussed the analysis of verbal complexes as they are known from the OV languages like
German, Dutch, and Afrikaans, I want to briefly comment on the SVO languages like Danish and English
and so on. Usually a head requires its argument to be fully saturated, that is the \sprv and the
\compsv has to be the empty list. Verbal complexes are different: Words are combined directly. The
VO languages differ from the OV languages in not allowing this. In VO languages the verb forms a
phrase with its complements and this verb phrase may be embedded under another verb. (\mex{1}a)
shows an example with auxiliary verbs, (\mex{1}b) is an example with a full verb that takes an
infinitive verb phrase with \emph{to} and an object in addition.
\eal
\ex Peter [will [have [read the book]]].
\ex Peter [promises Mary [to read the book]].
\zl
Languages like Danish and English only have the Head-Complement Schema and the Specifier-Head
Schema, while languages like Dutch and German have an additional schema that can combine unsaturated
words.\todostefan{check whether this has to be explained better} The schema for predicate complex formation is sketched in Figure~\vref{fig-pred-complex}.
\begin{figure}
\begin{forest}
[{[\comps \ibox{1}]}
  [\ibox{2} {[\textsc{lex}+]}]
  [{[\comps \ibox{1} $\oplus$ \sliste{ \ibox{2} }]}]]
\end{forest}
\caption{\label{fig-pred-complex}Sketch of the Predicate Complex Schema}
\end{figure}
This schema is very similar to the Head-Complement Schema that was given on
page~\pageref{fig-head-comp}. The difference is that the daughter in \ibox{2} has to be lexical ({\sc
  lex}+)\isfeat{lex}. Therefore words and verbal complexes are compatible with this daughter while full phrases
like \emph{das Buch lesen} `read the book' are not.

Before turning to the next phenomenon, I want to briefly discuss the alternative to the verb complex
analysis presented here. One alternative suggestion was to analyze auxiliaries in German as VP
embedding verbs \citep{Wurmbrand2003b}. Our standard example would then have the analysis in (\mex{1}):
\ea
\gll dass keiner [[das Buch lesen] wird]\\
     that nobody \hspaceThis{[[}the book read will\\
\z
The question that such analyses have to answer is how scrambling of arguments of the involved verbs
can be accounted for. The answer is often that it is assumed that the object of the embedded verb is
extracted from the VP and moved to the left periphery of the clause. This is shown in (\mex{1}):
\ea
\gll dass [das Buch]$_i$ keiner [[ \_$_i$ lesen] wird]\\
     that \spacebr{}the book nobody {} {} read will\\
\glt `that nobody will read the book'
\z
However, analyses that treat scrambling as movement are problematic since they predict additional
readings of sentences that have quantifiers in their NPs (\citealp[\page 146]{Kiss2001a}; \citealp[Abschnitt~2.6]{Fanselow2001a}).


Before I turn to the analysis of the verb position, I want to show how sentences with several verbs
in SVO languages can be analyzed. Figure~\vref{fig-vp-embedding-svo} shows the analysis of the English version of sentence
(\ref{ex-dass-keiner-das-buch-lesen-wird}).
\begin{figure}
\centerfit{
\begin{forest}
sm edges
[S
        [{NP[\type{nom}]} [nobody] ]
        [VP
          [V%% \feattab{
            %%   spr \sliste{ NP[\type{nom}] }\\
            %%   comps \sliste{ VP }} 
              [will] ]
          [VP%% \feattab{
             %%  spr \sliste{ NP[\type{nom}] }}
            [V%% \feattab{
              %% spr \sliste{ NP[\type{nom}] }\\
              %% comps \sliste{ NP[\type{acc}] }} 
              [read] ]
            [{NP[\type{acc}]} [the book, roof] ] ] ] ]
\end{forest}}
\caption{\label{fig-vp-embedding-svo}Embedding of a VP in SVO languages}
\end{figure}
The verb \emph{reads} selects a subject and an object. The verb forms a VP with the NP \emph{the
  book}. This VP is still lacking a subject. The auxiliary \emph{will} selects a VP and a subject
that is identical with the subject of \emph{read}. The combination of \emph{will} and the VP is
licensed by the Head-Complement Schema that was sketched in Figure~\vref{fig-head-comp}.

The equivalent of \emph{lesen können wird} `read can will' cannot be given here, since English modal
verbs do not have non-finite forms, but one can construct examples with modals as the highest verb:
\ea
He [must [have [seen him]]]
\z
This sentence has a structure that is similar to the one in Figure~\ref{fig-vp-embedding-svo}:
\emph{must} and \emph{have} both embedd VPs.

Finally, Figure~\vref{fig-somebody-has-promised-him-to-read-it} shows the translation of
(\ref{ex-weil-es-ihm-jemand-zu-lesen-versprochen-hat}):
\ea
Somebody has promised him to read it.
\z
\emph{promise} is a verb that takes a subject, an object, and a VP complement. 
\begin{figure}
\centerfit{
\begin{forest}
sm edges
[S
        [{NP[\type{nom}]} [somebody] ]
        [VP
          [\vbar [V [promised] ]
                 [{NP[\type{acc}]} [him] ] ]
          [VP [V [to]]
              [VP
                [V [read] ]
                [{NP[\type{acc}]} [the book, roof] ] ] ] ] ]
\end{forest}}
\caption{\label{fig-somebody-has-promised-him-to-read-it}Embedding of a VP with verbs that take an
  additional object}
\end{figure}
Like in the analysis of (\ref{ex-nobody-gives-him-the-book}) on
page~\pageref{ex-nobody-gives-him-the-book} -- which is repeated here as (\mex{1}) for convenience
-- the verb \emph{promised} is combined with its NP complement first and then with its VP
argument. The VP consists of \emph{to} and another VP with an infinitive in base form. \emph{to} is
analyzed as an auxiliary verb.\todostefan{add references} 
It is important to note that the object \emph{him} cannot appear in any other position (appart from
extraction to the left periphery). For instance, it cannot appear in the position of \emph{the book}
and the same holds for \emph{the book}: This phrase cannot appear in any other place than in the
object position.


\exercises{


\begin{enumerate}
\item Sketch the analysis of the verbal complexes in the following examples:
\eal
\ex dass er darüber lachen wird
\ex dass er darüber wird lachen müssen
\ex dass er über diesen Witz wird haben lachen müssen
\zl
\item Search for two sentences with a verbal complex in a newspaper or in corpora (COSMAS, COW) and
  analyze the verbal complexes.

\item Search for verbal complexes with more than four verbs in a corpus and document your search.
\end{enumerate}

}


%      <!-- Local IspellDict: en_US -->
