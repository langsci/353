%% -*- coding:utf-8 -*-
\chapter{Solutions}


\section{Valency, argument order and adjunct placement}

\settowidth\jamwidth{(German)}
\begin{enumerate}
\item Provide the valency lists for the following words:

\ea
\begin{tabular}[t]{@{}l@{~}ll@{\hspace{5em}}r@{}}
a. & laugh    & \sliste{ NP[\type{nom}] }\\
b. & eat      & \sliste{ NP[\type{nom}], NP[\type{acc}] }\\
c. & to douse & \sliste{ NP[\type{nom}], NP[\type{acc}] }\\
d. & bezichtigen& \sliste{ NP[\type{nom}], NP[\type{gen}] } &(German)\\
   &  accuse\\ 
e. & he  & \eliste\\
f. & the & \eliste\\
\end{tabular}
\z
If you are uncertain as far as case assignment is concerned, you may use the
  Wiktionary\footnote{
\url{https://de.wiktionary.org/}, 2018-07-02.
}.


\item Draw trees for the following examples:

\eal
\ex weil der Mann ihm ein Buch schenkt \german
\ex because the man gave a book to him
\ex Peter saw this yesterday.
\ex
\gll at Bjarne læste bogen\\
     that Bjarne read book.\textsc{def}\\\danish
\glt `that Bjarne read the book'
\zl

\begin{figure}
\centerfit{%
\begin{forest}
sm edges
[CP
  [{C[\spr \eliste, \comps \sliste{ S }]} [weil;because]]
  [{V[\spr \eliste, \comps \eliste]}
      [{NP[\type{nom}]} [der Mann;the man,roof] ]
      [{V\feattab{
            \spr \sliste{ }, \comps \sliste{ NP[\type{nom}] } }}
        [{NP[\type{dat}]} [ihm;him] ] 
        [{V\feattab{
              \spr \sliste{ }, \comps \sliste{ NP[\type{nom}], NP[\type{dat}] }}}
          [{NP[\type{acc}]} [ein Buch;a book,roof] ] 
          [V\feattab{
              \spr \sliste{  },\\
              \comps \sliste{ NP[\type{nom}], NP[\type{dat}], NP[\type{acc}]}} [schenkt;gives] ]]
] ] ]
\end{forest}}
\caption{\label{fig-weil-der-mann-ihm-ein-buch-schenkt}The analysis of \emph{weil der Mann ihm ein Buch schenkt} `because
  the man gave him a book as a present'}
\end{figure}



\begin{figure}
\centerfit{%
\begin{forest}
sm edges
[CP
  [{C[\spr \eliste, \comps \sliste{ S }]} [because]]
  [{V[\spr \eliste, \comps \eliste]}
      [{NP[\type{nom}]} [the man,roof] ]
      [{V\feattab{
            \spr \sliste{ NP[\type{nom}] }, \comps \sliste{ } }}
        [{V\feattab{
                \spr \sliste{ NP[\type{nom}] }, \comps \sliste{ PP[\type{acc}] } }}
           [V\feattab{
              \spr \sliste{ NP[\type{nom}] },\\
              \comps \sliste{ NP[\type{acc}], PP[\type{acc}]}} [gave] ]
            [{NP[\type{acc}]} [a book, roof] ] ] 
        [{PP[\type{acc}]} [to him,roof] ] ] ] ]
\end{forest}}
\caption{\label{fig-because-the-man-gave-the-book-to-him}The analysis of \emph{because the man gave a book to him}}
\end{figure}

\begin{figure}
\centerfit{%
\begin{forest}
sm edges
[{V[\spr \eliste, \comps \eliste]}
   [{NP[\type{nom}]} [Peter] ]
   [V\feattab{
      \spr \sliste{ NP[\type{nom}] }, \comps \sliste{}}
     [V\feattab{
        \spr \sliste{ NP[\type{nom}] }, \comps \sliste{}}
       [V\feattab{
           \spr \sliste{ NP[\type{nom}] },\\
           \comps \sliste{ NP[\type{acc}] }} [saw] ]
          [{NP[\type{acc}]} [this] ] ]
     [{Adv[\textsc{mod} VP]} [yesterday] ] ] ]
\end{forest}}
\caption{\label{fig-peter-saw-this-yesterday}Analysis of \emph{Peter saw this yesterday.}}
\end{figure}

\begin{figure}
\centerfit{%
\begin{forest}
sm edges
[CP
  [{C[\spr \eliste, \comps \sliste{ S }]} [at;that]]
  [{V[\spr \eliste, \comps \eliste]}
    [{NP[\type{nom}]} [Bjarne;Bjarne] ]
    [V\feattab{
       \spr \sliste{ NP[\type{nom}] }, \comps \sliste{}}
       [V\feattab{
           \spr \sliste{ NP[\type{nom}] },\\
           \comps \sliste{ NP[\type{acc}] }} [læste;read] ]
          [{NP[\type{acc}]} [bogen;book.\textsc{def}] ] ] ] ]
\end{forest}}
\caption{\label{fig-at-bjarne-laeste-bogen}Analysis of \emph{at Bjarne læste bogen} `that Bjarne
  read the book'}
\end{figure}



\end{enumerate}

\clearpage

\section{The verbal complex}

\section{Verb position: Verb first and verb second}



\begin{enumerate}
\item Classify the Germanic languages according to their basic constituent order (SVO, SOV, VSO,
  \ldots) and V2 assuming that you know that one of the following patterns exist in the language:
\eal
\label{ex-v2-task-solution}
\ex 
\label{ex-acc-aux-nom-v-dat}
NP[acc] V-Aux NP[nom] V NP[dat]   \hfill  V2 SVO 
\ex
\label{ex-acc-aux-nom-dat-v} 
NP[acc] V-Aux NP[nom] NP[dat] V   \hfill  V2 SOV
\ex 
\label{ex-acc-nom-v-acc}
NP[acc] NP[nom] V NP[acc]         \hfill -V2 SVO
\ex 
\label{ex-acc-nom-aux-v-acc}
NP[acc] NP[nom] V-Aux V NP[acc]   \hfill -V2 SVO
\ex 
\label{ex-acc-aux-nom-v-pp}
NP[acc] V-Aux NP[nom] V PP        \hfill not classifiable
\zl

The pattern in (\ref{ex-acc-aux-nom-v-dat}) cannot be Englih, since English does not have a dative. Hence it is a V2
language. The dative object follows the verb, so it must be an SVO language. An example would be Icelandic.
\ea

\z

(\ref{ex-acc-aux-nom-dat-v}) has an auxiliary and two NPs followed by a verb. Since the dative object would follow the verb
in an SVO language, it must be a SOV language. Since all Germanic SOV languages are also V2
languages, (\ref{ex-acc-aux-nom-dat-v}) must be a V2 language. German and Dutch would be examples.
\ea
\gll Den Roman hat er dem Jungen gegeben.\\
     the novel has he the boy given\\
\glt `He has given the boy the novel.'
\z

Ignoring multiple frontings in German \citep{Mueller2003b}, (\ref{ex-acc-nom-v-acc}) must be a non-V2 pattern. The language can only be
English:
\ea
This book, Peter gave Mary.
\z
For the same reason, (\ref{ex-acc-nom-aux-v-acc}) is non-V2 and SVO. The language must be English:
\ea
This book, Peter had given Mary.
\z
The pattern in (\ref{ex-acc-aux-nom-v-pp}) cannot be unambiguously classified with respect to V2 and
SOV/SVO. Since PPs can be extraposed easily, it could be an SOV langauge with extraposition (\eg
German) or it could be English with question formation (residual V2):
\eal
\ex 
\gll Wen hat Peter gesehen bei    der Demonstration.\\
     who has Peter seen    during the rally\\
\glt `Who has Peter seen during the rally.'
\ex Who did Peter see during the rally?
\zl
\item Sketch the analysis for the following examples:
\eal
\ex 
\gll dass er darüber lacht\\
     that he there.upon laughs\\\jambox{(German)}

\ex 
\gll dass er darüber lachen wird\\
     that he there.upon laugh will\\
\ex
\gll Wird er darüber lachen?\\
     will he there.upon laugh\\
\zl

\ea
\gll Arbejder Bjarne ihærdigt  på bogen?\\
     works    Bjarne seriously at book.\textsc{def}\\\jambox{(Danish)}
\glt `Does Bjarne work seriously on the book?'
\z




\begin{figure}
\begin{forest}
sm edges
[CP
  [C\feattab{\spr \sliste{},\\
             \comps \sliste{ \ibox{1} }} [dass;that]]
  [{\ibox{1} V\feattab{\spr \sliste{ },\\
                       \comps \sliste{ }}}
     [{\ibox{2} NP[nom]} [er;he]]
     [V\feattab{\spr  \sliste{ },\\
                 \comps \sliste{ \ibox{2} }}
       [\ibox{3} PP [darüber;there.about]]
       [V\feattab{\spr \sliste{ },\\
                  \comps \sliste{ \ibox{2}, \ibox{3} }} [lacht;laughs]]]]]
\end{forest}
\caption{Analysis of \emph{dass er darüber lacht} `that he laughs about this'}
\end{figure}


\begin{figure}
\begin{forest}
sm edges
[CP
  [C\feattab{\spr \sliste{},\\
             \comps \sliste{ \ibox{1} }} [dass;that]]
  [{\ibox{1} V\feattab{\spr \sliste{ },\\
                       \comps \sliste{ }}}
     [{\ibox{2} NP[nom]} [er;he]]
     [V\feattab{\spr  \sliste{ },\\
                 \comps \sliste{ \ibox{2} }}
       [\ibox{3} PP [darüber;there.about]]
       [V\feattab{\spr \sliste{ },\\
                  \comps \sliste{ \ibox{2}, \ibox{3} }}
         [\ibox{4} V\feattab{\subj \sliste{ \ibox{2} },\\
                             \spr \sliste{ },\\
                             \comps \sliste{ \ibox{3} }} [lachen;laugh]]
         [V\feattab{\spr \sliste{ },\\
                    \comps \sliste{ \ibox{2}, \ibox{3}, \ibox{4} }} [wird;will]]]
]]]
\end{forest}
\caption{Analysis of \emph{dass er darüber lachen wird} `that he will laugh about this'}
\end{figure}


\begin{figure}
\begin{forest}
sm edges
[V\feattab{\spr \sliste{},\\
           \comps \sliste{ \ibox{1} }}
  [V\feattab{\spr \sliste{},\\
             \comps \sliste{ \ibox{1} }}
    [V
      [wird;will]]]
  [{\ibox{1} V//V\feattab{\spr \sliste{ },\\
                       \comps \sliste{ }}}
     [{\ibox{2} NP[nom]} [er;he]]
     [V//V\feattab{\spr  \sliste{ },\\
                 \comps \sliste{ \ibox{2} }}
       [\ibox{3} PP [darüber;there.about]]
       [V//V\feattab{\spr \sliste{ },\\
                  \comps \sliste{ \ibox{2}, \ibox{3} }}
         [\ibox{4} V\feattab{\subj \sliste{ \ibox{2} },\\
                             \spr \sliste{ },\\
                             \comps \sliste{ \ibox{3} }} [lachen;laugh]]
         [V//V\feattab{\spr \sliste{ },\\
                    \comps \sliste{ \ibox{2}, \ibox{3}, \ibox{4} }} [\trace]]]
]]]
\end{forest}
\caption{Analysis of \emph{Wird er darüber lachen?} `Will he laugh about this?'}
\end{figure}



\end{enumerate}

\clearpage

\section{Passive}

\begin{figure}
\begin{forest}
sm edges
[CP
  [{C[\comps \sliste{ \ibox{1} }]} [that]]
  [{\ibox{1} V\feattab{\spr \sliste{ },\\
                       \comps \sliste{ }}}
     [{\ibox{2} NP[nom]} [the box, roof]]
     [V\feattab{\spr  \sliste{ \ibox{2} },\\
                 \comps \sliste{ }}
       [V\feattab{\spr  \sliste{ \ibox{2} },\\
                  \comps \sliste{ \ibox{3} }} [was]]
       [{\ibox{3} V\feattab{\spr \sliste{ \ibox{2} },\\
                            \comps \sliste{}}} [opened]]]]]
\end{forest}
\caption{Analysis of the passive sentences \emph{that the box was opened}}
\end{figure}
The transitive verb \emph{open} takes a subject and an object. The \argst list contains two NPs with
structural case. The passive lexical rule removes one argument. For the passive participle this
leaves us with one element on the \argstl. This element gets mapped to the \sprl of
\emph{opened}. The passive auxiliary takes a VP in passive form and takes over its element from
\spr. After combination of auxiliary and passive VP, we have the VP \emph{was opened} still
selecting for a specifier. The NP \emph{the box} functions as the specifier and the combination of
\emph{the box} and \emph{was opened} is a complete sentence.


%% \begin{figure}
%% \begin{forest}
%% sm edges
%% [CP
%%   [C [dass;that]]
%%   [V
%%      [\ibox{1} {NP[nom]} [der Kasten;the box, roof]]
%%      [V
%%        [{\ibox{2} V[\comps \ibox{3} \sliste{ \ibox{1} }]} [geöffnet;opened]]
%%        [{V[\comps \ibox{3} $\oplus$ \sliste{ \ibox{2} }] [wurde;was]]]]]
%% \end{forest}
%% \end{figure}

\section{Clause types and expletives}



%      <!-- Local IspellDict: en_US-w_accents -->
