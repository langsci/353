%% -*- coding:utf-8 -*-
\chapter{Solutions}


\section{Valency, argument order and adjunct placement}

\eal
\ex weil der Mann ihm ein Buch schenkt
\ex because the man gave the book to him
\ex Peter saw this yesterday.
\ex
\gll at Bjarne læste bogen\\
     that Bjarne read book.\textsc{def}\\
\glt `that Bjarne read the book'
\zl

\begin{figure}
\centerfit{%
\begin{forest}
sm edges
[CP
  [{C[\spr \eliste, \comps \sliste{ \ibox{1} }]} [weil;because]]
  [{V[\spr \eliste, \comps \eliste]}
      [{NP[\type{nom}]} [der Mann;the man] ]
      [{V\feattab{
            \spr \sliste{ }, \comps \sliste{ NP[\type{nom}] } }}
        [{NP[\type{acc}]} [ihm;him] ] 
        [{V\feattab{
              \spr \sliste{ }, \comps \sliste{ NP[\type{nom}], NP[\type{dat}] } }}
          [{NP[\type{acc}]} [ein Buch;a book] ] 
          [V\feattab{
              \spr \sliste{  },\\
              \comps \sliste{ NP[\type{nom}], NP[\type{dat}], NP[\type{acc}]}} [schenkt;gives] ]]
] ] ]
\end{forest}}
\caption{\label{fig-spr-german}The analysis of \emph{weil der Mann ihm ein Buch schenkt}}
\end{figure}



\section{Passive}

\begin{figure}
\begin{forest}
sm edges
[CP
  [{C[\comps \sliste{ \ibox{1} }]} [that]]
  [{\ibox{1} V\feattab{\spr \sliste{ },\\
                       \comps \sliste{ }}}
     [{\ibox{2} NP[nom]} [the box, roof]]
     [V\feattab{\spr  \sliste{ \ibox{2} },\\
                 \comps \sliste{ }}
       [V\feattab{\spr  \sliste{ \ibox{2} },\\
                  \comps \sliste{ \ibox{3} }} [was]]
       [{\ibox{3} V\feattab{\spr \sliste{ \ibox{2} },\\
                            \comps \sliste{}}} [opened]]]]]
\end{forest}
\caption{Analysis of the passive sentences \emph{that the box was opened}}
\end{figure}
The transitive verb \emph{open} takes a subject and an object. The \argst list contains two NPs with
structural case. The passive lexical rule removes one argument. For the passive participle this
leaves us with one element on the \argstl. This element gets mapped to the \sprl of
\emph{opened}. The passive auxiliary takes a VP in passive form and takes over its element from
\spr. After combination of auxiliary and passive VP, we have the VP \emph{was opened} still
selecting for a specifier. The NP \emph{the box} functions as the specifier and the combination of
\emph{the box} and \emph{was opened} is a complete sentence.


%% \begin{figure}
%% \begin{forest}
%% sm edges
%% [CP
%%   [C [dass;that]]
%%   [V
%%      [\ibox{1} {NP[nom]} [der Kasten;the box, roof]]
%%      [V
%%        [{\ibox{2} V[\comps \ibox{3} \sliste{ \ibox{1} }]} [geöffnet;opened]]
%%        [{V[\comps \ibox{3} $\oplus$ \sliste{ \ibox{2} }] [wurde;was]]]]]
%% \end{forest}
%% \end{figure}

