%% -*- coding:utf-8 -*-
\chapter{Valency, argument order and adjunct placement}
\label{sec-valency}

This chapter deals with the representation of valency information and sketches the basic structures that
are assumed for SVO and SOV languages. I provide an account for scrambling in those languages that
allow for it and discuss the fixed vs.\ free position of adjuncts.

\section{Valency representations}

The valency of a head is represented in its lexical entry in the form of a list with descriptions of
the elements that belong to the head's valency. (\mex{1}) provides some prototypical examples:
\ea
\begin{tabular}[t]{@{}l@{~}l@{~}l}
a. & \emph{schläft} `sleeps':        & \sliste{ NP[\type{nom}] }\\
b. & \emph{unterstützt} `supports':  & \sliste{ NP[\type{nom}], NP[\type{acc}] }\\
c. & \emph{hilft} `helps':           & \sliste{ NP[\type{nom}], NP[\type{dat}] }\\
d. & \emph{gibt} `gives':            & \sliste{ NP[\type{nom}], NP[\type{dat}], NP[\type{acc}] }\\
e. & \emph{wartet} `waits':          & \sliste{ NP[\type{nom}], PP[\type{auf}] }\\
\end{tabular}
\z
The elements in such lists come with a fixed order. The order corresponds to the order of the
elements in English and to the so-called unmarked order in German, that is, for ditransitive verbs
the order is usually nom, dat, acc (see \citew{Hoehle82a} for comments on the unmarked order). This fixed order is needed for establishing the link between
syntax and semantics\is{semantics}.\is{linking} The details can not be provided in this book but the
interested reader is referred to (\citealp{ps2}; \citealp{MuellerLehrbuch1}). 

Given such a valency representation for a verb like \emph{kennen} `know' one can assume a grammar rule
or an Immediate Dominance Schema that combines an element from the valence list with the respective head
and passes all unsaturated elements on to the result of the combination. This can be depicted as in
Figure~\vref{fig-valency-German}, which is an example analysis of (\mex{1}).
\ea
\label{ex-dass-niemand-ihn-kennt}
\gll  {}[dass] niemand ihn kennt\\
      \spacebr{}that nobody.\nom{} him.\acc{} knows\\ 
\glt `that nobody knows him'
\z
\begin{figure}
\centerfit{%
\begin{forest}
sm edges
[{S \eliste}
  [{NP[\type{nom}]} [niemand;nobody] ]
  [{V$'$\sliste{ NP[\type{nom}] } }
    [{NP[\type{acc}]} [ihn;him] ]
    [{V \sliste{ NP[\type{nom}], NP[\type{acc}]}} [kennt;knows]] ] ]
\end{forest}}
\caption{\label{fig-valency-German}Analysis of (\emph{dass}) \emph{niemand ihn kennt} `that nobody
  knows him', valency information is represented in a list}
\end{figure}

The lexical item for \emph{kennt} `knows' has a valence description containing two NPs. In a first
step \emph{kennt} is combined with its accusative object. The resulting phrase \emph{ihn kennt} `him
knows' is something whose most importent constituent is a verb. Therefore it has a V in its category
label. Since \emph{ihn kennt} is not a sentence but something intermediate, it gets the label
V$'$.\footnote{%
  These labels are abbreviations for complex categories. Their internal makeup is given in
  Table~\ref{tab-abbreviations-v-vbar-s}. The labels are similar to what is known from \xbart but
  the theory developed here is not following all the tenets of \xbart. For example, simple nouns
  like \emph{house} are N$'$ and there is no \nnull in the analysis of NPs like \emph{the house}.

  Section~\ref{sec-intro-spr-comps} explains why the abbreviation for \emph{ihn kennt} `him knows' is V$'$ rather than VP.
} The valency list of this V$'$ contains all elements that still have to be realized in order to
yield a complete sentence, that is, it contains an NP with nominative case. After the combination of
\emph{ihn kennt} with \emph{niemand} `nobody' we get the full sentence \emph{niemand ihn kennt}
`nobody him knows'. As an abbreviation for full sentences I use S. S stands for something whose most
important element is a verb and whose valency list is empty, that is, it is fully saturated. Hence,
the specification of the empty valency list in Figure~\ref{fig-valency-German} is somewhat redundant.

The nodes for V$'$ and S are licensed by a schema that combined a head with one element of its
valence list. The full schema will be given in Chapter~\ref{chap-HPSG-light}, but we will discuss a
simplified version of it in Section~\ref{sec-intro-schemata}.


\section{Scrambling}
\label{sec-scrambling}

As we already saw in the data discussion in the previous chapter, some languages allow for
scrambling of arguments. For those languages one can assume that heads can combine with any of its
arguments not necessarily beginning with the last one as it was the case in the analysis in Figure~\ref{fig-valency-German}.
Figure~\vref{fig-scrambling-German} shows the analysis of (\mex{1}).
\ea
\gll {}[dass] ihn niemand kennt\\
     \spacebr{}that him.\acc{} nobody.\nom{} knows\\
\glt `that nobody knows him'
\z
\begin{figure}
\centerfit{%
\begin{forest}
sm edges
[{S \eliste}
   [{NP[\type{acc}]} [ihn;him] ]
   [{V$'$\sliste{ NP[\type{acc}] } }
      [{NP[\type{nom}]} [niemand;nobody] ]
      [{V \sliste{ NP[\type{nom}], NP[\type{acc}]}} [kennt;knows] ] ] ]
\end{forest}}
\caption{\label{fig-scrambling-German}Analysis of (\emph{dass}) \emph{ihn niemand kennt} `that nobody
  knows him', languages that allow for scrambling permit the saturation of arguments in any order}
\end{figure}
Rather than combining the verb with the accusative argument (the object) first, it is combined with
the nominative (the subject) and the accusative (the object) is added in a later step.


\section{SVO: Languages with fixed SV order and valence features}
\label{sec-intro-schemata}
\label{sec-intro-spr-comps}

The last section demonstrated how verb-final sentences in German can be analyzed. Of course it is
easy to imagine how this extends to VSO languages: The head is initial and combines with the first
element in the valency list first and then with all the other elements. However, nothing has been
said about the SVO languages so far. In languages like Danish, English, and so on all objects are
realized after the verb as in (\mex{1}), it is just the subject that preceedes the verb.
\ea
Kim gave Sandy the book.
\z
The verb together with its objects forms a unit in a certain sense: It can be fronted (\mex{1}a). It can be
selected by dominating verbs (\mex{1}b), and it is the place where adjuncts attach to (\mex{1}c--d).
\eal
\ex John promised to read the book and read the book, he will.
\ex He will [read the book].
\ex He often [reads the book].
\ex \ldots{} often read the book slowly, he will.
\zl
This can be modeled adequately by assuming two valency lists: one for the complements (\comps short for \textsc{complements}\isfeat{comps}) and
one for the subject. The list for the subject is called \textsc{specifier} list (\spr\isfeat{spr}). The specifier list
plays a role both in the analysis of sentences and in the analysis of noun phrases. Nouns
select their determiner via \spr and all their other arguments via \comps. Figure~\vref{fig-svo}
shows the analysis of the sentence (\mex{1}) using the features \spr and \comps.
\ea
Nobody knows him.
\z
\begin{figure}
\centerfit{%
\begin{forest}
sm edges
[{V[\spr \eliste, \comps \eliste]}, name=S
   [{NP[\type{nom}]} [nobody] ]
   [V\feattab{
      \spr \sliste{ NP[\type{nom}] }, \comps \sliste{} }, name=VP
     [V\feattab{
         \spr \sliste{ NP[\type{nom}] },\\
         \comps \sliste{ NP[\type{acc}] }} [knows] ]
        [{NP[\type{acc}]} [him] ] ] ]
\node [right=4cm] at (S)
    {
        = S
    };
\node [right=4cm] at (VP)
    {
        = VP
    };
\end{forest}}
\caption{\label{fig-svo}Analysis of the SVO order with two separate valency features}
\end{figure}
The \compsl of \emph{knows} contains a description of the accusative object and the accusative
\emph{him} is combined in a first step with \emph{knows}. In addition to the accusative object
\emph{knows} selects for a subject. This selection is passed on to the mother node, the VP. Hence,
the \sprv of \emph{knows him} is identical to the \sprv of \emph{knows}. The VP \emph{knows him}
selects for a nominative NP. This NP is realized as \emph{nobody} in Figure~\ref{fig-svo}. The
result of the combination of \emph{knows him} with \emph{nobody} is \emph{nobody knows him}, which
is complete: It has both an empty \sprl and an empty \compsl. The two rules that are responsible for
the combinations in Figure~\ref{fig-svo} are called the Specifier-Head Schema and the
Head-Complement Schema. I use VP as abbreviation for something with a verbal head and an empty \compsl and at least
one element in the \sprl and S as abbreviation for something with a verbal head and empty lists for
both the \spr and the \compsv.

In Section~\ref{sec-scrambling} it was explained how scrambling can be accounted for: The rules that
combine heads with their arguments can take the arguments from the list in any order. For languages
with stricter constituent order requirements the rules are stricter: The arguments have to be taken
off the list consistently from the beginning or from the end. So for English and Danish one starts
at the beginning of the list and for head-final languages without scrambling one starts at the end
of the list.\todostefan{provide an example of head-final language}
Figure~\ref{fig-svo-ditrans} shows the analysis of a sentence with a ditransitive verb.
\begin{figure}
\centerfit{%
\begin{forest}
sm edges
[{V[\spr \eliste, \comps \eliste]}
   [{NP[\type{nom}]} [Kim] ]
   [V\feattab{
      \spr \sliste{ NP[\type{nom}] }, \comps \sliste{} }
     [V\feattab{
         \spr \sliste{ NP[\type{nom}] },\\
         \comps \sliste{ PP[\type{to}] }}
       [V\feattab{
           \spr \sliste{ NP[\type{nom}] },\\
           \comps \sliste{ NP[\type{acc}], PP[\type{to}] }} [gave] ]
         [{NP[\type{acc}]} [a book, roof] ] ]
       [{PP[\type{to}]} [to Sandy, roof] ] ] ]
\end{forest}}
\caption{\label{fig-svo-ditrans}Analysis of the SVO order with two separate valency features and two
  elements in \comps}
\end{figure}
The accusative object is the first element in the \compsl and it is combined with the verb
first. The result of the combination is a verbal projection that has the PP[\type{to}] as the sole
element in the \compsl. It is combined with an appropriate PP in the next step resulting in a verbal
projection that has an empty \compsl (a VP).


The analysis of our first German example in Figure~\ref{fig-valency-German} did not use a name
for the valency list. So the question is: How does the analysis of German relate to the analysis of
English using \spr and \comps. A lot of researchers from various frameworks argued that it is not
useful to distinguish the subjects of finite verbs from other arguments. All the tests that have
been used to show that subjects in English differ from complements do not apply to the arguments of
finite verbs in German. Hence, researchers like \citet{Pollard90a}, \citet{Haider93a}, 
\citet[\page 376]{Eisenberg94b}, and \citet{Kiss95a} argued for so-called subject as complement
analyses. Figure~\vref{fig-spr-german} shows the adapted analysis of
(\ref{ex-dass-niemand-ihn-kennt}) -- repeated here as
(\ref{ex-dass-niemand-ihn-kennt-two}):
\ea
\label{ex-dass-niemand-ihn-kennt-two}
\gll  {}[dass] niemand ihn kennt\\
      \spacebr{}that nobody.\nom{} him.\acc{} knows\\ 
\glt `that nobody knows him'
\z
\begin{figure}
\centerfit{%
\begin{forest}
sm edges
[{V[\spr \eliste, \comps \eliste]}, name=S
        [{NP[\type{nom}]} [niemand;nobody] ]
        [{V\feattab{
              \spr \sliste{ }, \comps \sliste{ NP[\type{nom}] } }}, name = Vs
          [{NP[\type{acc}]} [ihn;him] ] 
          [V\feattab{
              \spr \sliste{  },\\
              \comps \sliste{ NP[\type{nom}], NP[\type{acc}]}} [kennt;knows] ]
] ]
\node [right=4cm] at (S)
    {
        = S
    };
\node [right=4cm] at (Vs)
    {
        = V$'$
    };
\end{forest}}
\caption{\label{fig-spr-german}The analysis of a German sentence with \spr and \compsl}
\end{figure}
The difference between German and English is that German contains all arguments in the \compsl of
the finite verb and no arguments in the \sprl. Since the elements in the \compsl can be combined
with the head in any order, it is explained why all permutations of arguments are
possible. Specifiers are realized to the left of their head. This is the same for German and
English. For German this is not relevant in the verbal domain, but the Specifier-Head Schema, which
is introduced shortly, is used in the analysis of noun phrases.

Throughout the remainder of this book I use the abbreviations in Table~\vref{tab-abbreviations-v-vbar-s}.
\begin{table}
 \begin{tabular}[t]{@{}l@{ = }l}\lsptoprule
             S  & V[\spr \eliste, \comps \eliste]\\
             VP & V[\spr \sliste{ NP[\type{nom}] }, \comps \sliste{}]\\
             V$'$ & all other V projections apart from verbal complexes\\[2pt]
             NP & N[\spr \eliste, \comps \eliste]\\
             N$'$ & V[\spr \sliste{ Det }, \comps \sliste{}]\\\lspbottomrule
             \end{tabular}
\caption{\label{tab-abbreviations-v-vbar-s}Abbreviations for S, VP, and V$'$ and NP, N$'$}
\end{table}

In Section~\ref{sec-valency} I already mentioned that the non-terminal nodes in a tree, that is, the
nodes that do not directly dominate a lexical entry, are licensed by rules. Syntactic rules are
usually called schemata since they are rather abstract. The details about such schemata will be
given in Chapter~\ref{chap-HPSG-light}, but Figure~\vref{fig-spr-head} and
Figure~\vref{fig-head-comp} provide the respective tree representations.
\begin{figure}
\begin{forest}
[{H[\spr \ibox{1}]}
  [\ibox{2}]
  [{H[\spr \ibox{1} $\oplus$ \sliste{ \ibox{2} }]}]]
\end{forest}
\caption{\label{fig-spr-head}Sketch of the Specifier-Head Schema}
\end{figure}
The H stands for \emph{head}. \emph{append} ($\oplus$) is a relation that concatenates two lists. For instance the concatenation
of \sliste{ \normalfont a } and \sliste{ \normalfont b } is \sliste{ \normalfont a, b }. The concatenation of the empty list \eliste{}
with another list yields the latter list. For such a schema to apply the descriptions of the
daughters have to match the actual daughters. For instance \emph{knows him} is compatible with the
right daughter: It has an NP[\type{nom}] in its \sprl. When \emph{knows him} is realized as a
daughter of the schema in Figure~\ref{fig-spr-head}, \ibox{2} is instantiated as
NP[\type{nom}]. Therefore the left daughter has to be compatible with an NP[\type{nom}]. It can be
realized as a simple pronoun like \emph{he} or a complex NP like \emph{the man who sold the
  world}. The \sprl of \emph{knows him} is \sliste{ NP[\type{nom}] }. If this list is split up into
two lists, one containing \sliste{ NP[\type{nom}] } and another one containing the rest, the second
list is the empty list. Hence \ibox{1} is instantiated as \eliste{} and \emph{Nobody knows him.}
corresponds to a structure with an empty \sprl. See also Figure~\ref{fig-nobody-gives-him-the-book}
below.

The Specifier-Head Schema is used for subject-VP combinations in the SVO languages but it is also
used for NPs in all the Germanic languages. Figure~\ref{fig-spr-head-the-man} shows the analysis of the NP \emph{the man}.
\begin{figure}
\begin{forest}
[{N[\spr \eliste, \comps \eliste]}
  [\ibox{1} Det [the]]
  [{N[\spr \sliste{ \ibox{1} }, \comps \eliste]} [man]]]
\end{forest}
\caption{\label{fig-spr-head-the-man}Analysis of the NP \emph{the man}}
\end{figure}
\emph{man} selects for a determiner and the result of combining \emph{man} with a determiner is a
complete nominal projection, that is, an NP. There are also nouns like \emph{picture} that take a
complement:
\ea
a picture of Kim
\z
The combination of \emph{picture} and its complement \emph{of Kim} is parallel to the combination of
a verb with its object in VO languages with fixed constituent order. For such combinations we need a separate schema: the Head-Complement Schema,
which is given in Figure~\ref{fig-head-comp}.
\begin{figure}
\begin{forest}
[{H[\comps \ibox{1}]}
  [{H[\comps  \sliste{ \ibox{2} } $\oplus$ \ibox{1}  ]}]
  [\ibox{2}]]
\end{forest}
\caption{\label{fig-head-comp}Sketch of the Head-Complement Schema}
\end{figure}
The schema splits the \compsl of a head into an initial list with one element \iboxb{2}, which is
realized as the complement daughter to the right.\footnote{%
  In principle daughters are unordered in HPSG as they were in GPSG. Special linearization rules are
  used to order a head with respect to its siblings in a local tree. So a schema licensing a tree
  like the one in Figure~\ref{fig-head-comp} would also license a tree with the daughters in a
  different order unless one head linearization rules that rule this out.
}
This schema licenses all the non-terminal nodes in the VP in
Figure~\vref{fig-nobody-gives-him-the-book}, which shows the analysis of (\mex{1}).\footnote{%
  English nouns and determiners do not inflect for case. However, case is manifested at pronouns:
  \emph{he} (nominative), \emph{his} (genitive), \emph{him} (accusative). Hence, verbs in double object
  constructions select for two accusatives. 
}
\ea
\label{ex-nobody-gives-him-the-book}
Nobody gives him the book.
\z
\begin{figure}
\centerfit{%
\begin{forest}
sm edges
[{V[\spr \eliste, \comps \eliste]}
   [{NP[\type{nom}]} [nobody] ]
   [V\feattab{
      \spr \sliste{ NP[\type{nom}] }, \comps \sliste{} }
     [V\feattab{
         \spr \sliste{ NP[\type{nom}] },\\
         \comps \sliste{ NP[\type{acc}]}} 
        [V\feattab{
           \spr \sliste{ NP[\type{nom}] },\\
           \comps \sliste{ NP[\type{acc}], NP[\type{acc}]}} [gives] ]
        [{NP[\type{acc}]} [him] ] ]
     [{NP[\type{acc}]} [the book,roof ] ] ] ]
\end{forest}}
\caption{\label{fig-nobody-gives-him-the-book}Analysis of the sentences with a ditransitive verb}
\end{figure}

\section{Scrambling and free VO/OV order}

\todostefaninline{Explain Initial feature}

Now, in order to analyze languages with free constituent order, we need a more liberal variant of
the schema in Figure~\ref{fig-head-comp}. Figure~\vref{fig-head-comp-free} splits the \compsl of a
head into three parts: a list \ibox{1}, a list containing exactly one element \sliste{ \ibox{3} }
and a third list \ibox{2}. The element of the second list is realized as the complement of the head.
\begin{figure}
\begin{forest}
[{H[\comps \ibox{1} $\oplus$ \ibox{2}]}
  [\ibox{3}]
  [{H[\comps  \ibox{1} $\oplus$ \sliste{ \ibox{3} } $\oplus$ \ibox{2}  ]}]]
\end{forest}
\caption{\label{fig-head-comp-free}Sketch of the Head-Complement Schema for languages with free
  constituent order}
\end{figure}
The length of the lists \ibox{1} and \ibox{2} is not restricted. If one restricts \ibox{1} to be the
empty list, one gets grammars that saturate complements from the beginning of the list (like
English) and if one restricts \ibox{2} to be the empty list, one gets grammars that take the last
element from the \compsl for combination with a head. Scrambling languages like German allow any
complement to be combined with its head since there is neither a restriction on \ibox{1} nor one on \ibox{2}.

\todostefaninline{Yiddish VO/OV}




\section{Adjuncts}

While arguments are selected by their head, adjuncts select the head. The difference between
languages like Dutch and German on the one hand and Danish and English on the other hand can be
explained by assuming that adjuncts in the former languages are less picky as far as the element is
concerned with which they combine.\todostefan{provide examples} Dutch and German adjuncts can attach to any verbal projection (\mex{1}),
while Danish and English require a VP as in (\mex{2}) (\citealp{Wechsler2015a}).
\eal
\ex
\label{ex-m-j-b-l} 
\gll {}[dass] morgen jeder das Buch liest\\
     \spacebr{}that tomorrow everybody the book reads\\
\glt `that everybody reads the book tomorrow'
\ex
\label{ex-j-m-b-l} 
\gll {}[dass] jeder morgen das Buch liest\\
     \spacebr{}that everybody tomorrow the book reads\\ 
\ex
\label{ex-j-b-m-l}
\gll {}[dass] jeder das Buch morgen liest\\
    \spacebr{}that everybody the book tomorrow reads\\
\zl

\eal
\ex Kim will have been [promptly [removing the evidence]].
\ex Kim will have been [[removing the evidence] promptly].
\zl

For the selection of arguments the features \spr and \comps are used. In parallel there is a \modf
that is part of a lexical description of a head of a phrase that can function as an adjunct (\textsc{mod} is
an abbreviation for \emph{modified}). The value of  \textsc{mod} is a description of an appropriate head. 
Head"=adjunct structures are licencsed by the schema in
Figure~\vref{fig-head-adj}.\todostefaninline{explain why spr and comps are the empty list}
\begin{figure}
\begin{forest}
[{H[\spr \ibox{1}, \comps \ibox{2}]}
  [{[\textsc{mod} \ibox{3}, \spr \eliste, \comps \eliste]}]
  [{\ibox{3} H[\spr \ibox{1}, \comps  \ibox{2}]}]]
\end{forest}
\caption{\label{fig-head-adj}Sketch of the Head-Adjunct Schema}
\end{figure}
For instance, attributive adjectives have \nbar as their \modv, where \nbar is an abbreviation for a
nominal projection that has an empty \compsl and a \sprl that contains a determiner. The analysis of the phrase
\emph{smart woman} is shown in Figure~\vref{fig-smart-woman}.
\begin{figure}
\begin{forest}
[{\nbar}
  [{Adj[\textsc{mod} \ibox{2}]} [smart]]
  [{\ibox{2} \nbar} [woman]]]
\end{forest}
\caption{\label{fig-smart-woman}Analysis of the head-adjunct structure \emph{smart woman}}
\end{figure}
In languages like German in which the adjective agrees with the noun in gender, number, and
inflection class, the properties that the noun must have can be specified inside the \modv. For
instance, \emph{kluger} selects a male noun and \emph{kluge} selects a female one:
\eal
\ex ein kluger Mann\\
    a   smart man\\
\ex eine kluge Frau\\
    a    smart woman\\
\zl


For German adverbials the value restricts the part
of speech of the head to be verb (or rather verbal since adjectival participles can be modified as well) and the value of \textsc{initial} to be $-$. This ensures that the
adjunct attaches to verbs in final position only (Verb initial sentences are discussed in
Chapter~\ref{chap-verb-position}). The \modv of English adverbials is simply VP. This allows for a
pre- and a post-VP attachment of adjuncts.
\begin{itemize}
\item SOV (Dutch, German, \ldots): \textsc{mod} V[\textsc{ini}$-$]
\item SVO (Danish, English, \ldots): \textsc{mod} VP
\end{itemize}
The analysis of (\ref{ex-m-j-b-l}) is shown in Figure~\vref{fig-m-j-b-l}, the one of
(\ref{ex-j-m-b-l}) in Figure~\vref{fig-j-m-b-l}, and the one of (\ref{ex-j-b-m-l}) in
Figure~\vref{fig-j-b-m-l}. The only difference between the figures is the respective place of
attachment of the adverb.


\begin{figure}
\centerfit{
\begin{forest}
sm edges
[{V[\spr \eliste, \comps \eliste]}
        [Adv [morgen;tomorrow] ]
        [{V[\spr \eliste, \comps \eliste]}
          [{NP[\type{nom}]} [jeder;everybody] ]
          [V\feattab{
              \spr \sliste{ }, \comps \sliste{ NP[\type{nom}] } }
            [{NP[\type{acc}]} [das Buch;the book, roof] ] 
            [V\feattab{
              \spr \sliste{  },\\
              \comps \sliste{ NP[\type{nom}], NP[\type{acc}]}} [liest;reads] ] ]
] ]
\end{forest}}
\caption{\label{fig-m-j-b-l}Analysis of [\emph{dass}] \emph{morgen jeder das Buch liest} `that everybody will read the
  book tomorrow' with the adjunct attaching above subject and object}
\end{figure}


\begin{figure}
\centerfit{%
\begin{forest}
sm edges
[{V[\spr \eliste, \comps \eliste]}
          [{NP[\type{nom}]} [jeder;everybody] ]
          [V\feattab{
              \spr \sliste{ }, \comps \sliste{ NP[\type{nom}] } }
            [Adv [morgen;tomorrow] ]
            [V\feattab{
                \spr \sliste{ }, \comps \sliste{ NP[\type{nom}] } }
              [{NP[\type{acc}]} [das Buch;the book, roof] ] 
              [V\feattab{
                \spr \sliste{  },\\
                \comps \sliste{ NP[\type{nom}], NP[\type{acc}]}} [liest;reads] ] ]
] ]
\end{forest}}

\caption{\label{fig-j-m-b-l}Analysis of [\emph{dass}] \emph{jeder morgen das Buch liest} `that everybody will read the
  book tomorrow' with the adjunct attaching between subject and object}
\end{figure}


\begin{figure}
\centerfit{%
\begin{forest}
sm edges
[{V[\spr \eliste, \comps \eliste]}
    [{NP[\type{nom}]} [jeder;everbody] ]
      [V\feattab{
         \spr \sliste{ }, \comps \sliste{ NP[\type{nom}] } }
         [{NP[\type{acc}]} [das Buch;the book, roof] ] 
           [V\feattab{
              \spr \sliste{  },\\
              \comps \sliste{ NP[\type{nom}], NP[\type{acc}]}} 
             [Adv [morgen;tomorrow] ]
               [V\feattab{
                 \spr \sliste{  },\\
                 \comps \sliste{ NP[\type{nom}], NP[\type{acc}]}} [liest;reads] ] ] ] ]
\end{forest}}
\caption{\label{fig-j-b-m-l}Analysis of [\emph{dass}] \emph{jeder das Buch morgen liest} `that everybody will read the
  book tomorrow' with the adjunct attaching between object and verb}
\end{figure}

The Figures~\ref{fig-adj-vp} and~\ref{fig-vp-adj} show the analysis of adjunction with the adverb in
pre-VP and post-VP position respectively.
\begin{figure}
\centerfit{%
\begin{forest}
sm edges
[{V[\spr \eliste, \comps \eliste]}
          [{NP[\type{nom}]} [Peter] ]
          [V\feattab{
              \spr \sliste{ NP[\type{nom}] }, \comps \sliste{  } }
            [Adv [often] ]
            [V\feattab{
                \spr \sliste{ NP[\type{nom}] }, \comps \sliste{  } }
              [V\feattab{
                \spr \sliste{ NP[\type{nom}] },\\
                \comps \sliste{  NP[\type{acc}]}} [reads] ]
              [{NP[\type{acc}]} [books] ] ]
] ]
\end{forest}}
\caption{\label{fig-adj-vp}Analysis of adjuncts in SVO languages: The adjunct is realized left-adjacent to the VP.}
\end{figure}
\begin{figure}
\centerfit{%
\begin{forest}
sm edges
[{V[\spr \eliste, \comps \eliste]}
          [{NP[\type{nom}]} [Peter] ]
          [V\feattab{
              \spr \sliste{ NP[\type{nom}] }, \comps \sliste{  } }
            [V\feattab{
                \spr \sliste{ NP[\type{nom}] }, \comps \sliste{  } }
              [V\feattab{
                \spr \sliste{ NP[\type{nom}] },\\
                \comps \sliste{  NP[\type{acc}]}} [reads] ]
              [{NP[\type{acc}]} [books] ] 
               ]
            [Adv [often] ]
] ]
\end{forest}}
\caption{\label{fig-vp-adj}Analysis of adjuncts in SVO languages: The adjunct is realized right-adjacent to the VP.}
\end{figure}


\section{Linking between syntax and semantics}
\label{sec-linking}

HPSG assumes that all arguments of a head are contained in a list that is called \textsc{argument
  structure} (\argst).\todostefan{references} This list contains descriptions of the syntactic and semantic properties of
the selected arguments. For instance the \argstl of English \emph{give} and its German, Danish and
Dutch and Icelandic variants is given in (\mex{1}):
\ea
\sliste{ NP, NP, NP }
\z
The case systems of the involved languages vary a bit as will be explained in
Chapter~\ref{chap-case}, but nevertheless the orders of the NPs in the \argstl are the same across
languages. They correspond to nom, dat, acc in German (\mex{1}a) and subject, primary object, secondary object
in English (\mex{1}b):
\eal
\ex dass der Mann dem Jungen den Ball gibt\\
    that the man  the boy    the ball gives\\
\glt `that the man gives the boy the ball'
\ex that the man gives the boy the ball
\zl
In addition to the syntactic features we have seen so far semantic features are used to describe the
semantic contribution of linguistic objects. (\mex{1}) shows some aspects of the description of the English verb
\emph{gives}:
\ea
lexical item for \emph{gives}:\\*
\ms{
arg-st & \sliste{ NP\ind{1}, NP\ind{2}, NP\ind{3} }\\[2mm]
cont   & \ms[give]{
          agens & \ibox{1}\\
          goal  & \ibox{2}\\
          trans-obj & \ibox{3}\\
        }\\
}
\z
The lowered boxes refer to the referential indices of the NPs. One can imagine these indices as
variables that refer to the object in the real world that the NP is referring to. These indices are
identified to semantic roles of the verb \emph{give}. The representations for the other languages
mentioned above is entirely parallel. Therefore it is possible to capture crosslinguistic
generalizations. Nevertheless there are differences between the Germanic OV and VO languages. As was
explained above the VO languages map their subject to \spr and all other arguments to \comps, while
the finite verbs of OV languages have all arguments on \comps.
\todostefaninline{maybe add some examples}
%% \eal
%% \ex Linking and argument mapping in SVO languages:\\
%% \ms{
%% spr    & \sliste{ \ibox{1} }\\
%%        & 
%% arg-st & \sliste{ NP\ind{1}, NP\ind{2}, NP\ind{3} }\\
%% cont   & \ms[give]{
%%           agens & \ibox{1}\\
%%           goal  & \ibox{2}\\
%%           trans-obj & \ibox{3}\\
%%         }\\
%% }



\section{Alternatives}

\todostefaninline{Advanced stuff. Ignore if you do not dare.}

\subsection{CP/TP/VP models}

\citet{Grewendorf88a,Grewendorf93}, \citet{Lohnstein2014a} and many others assume that German has a
structure that is parallel to the one that is assumed for English. As for English the verb is
assumed to form a phrase with its objects and this VP functions as the argument of a Tense head to
form a maximal projection together with the subject of the verb, which is realized in the specifier
position of the TP. Figure~\ref{fig-cp-tp-vp} shows the analysis of (\mex{1}) with the respective
layers.
\ea
\gll dass jeder diesen Mann kennt\\
     that everbody this man knows\\
\glt `that everybody knows this man'
\z
\begin{figure}
\centering
\begin{forest}
sm edges
[CP
[C$'$
	[C [dass;that]]
	[TP
		[NP [jeder;everybody,roof]]
		[T$'$
			[VP
				[V$'$
					[NP [diesen Mann;this man, roof]]
					[V [\trace$_j$]]]]
			[T [kenn-$_j$ -t;know- -s]]]]]]
\end{forest}
\caption{\label{fig-cp-tp-vp}Sentence in the CP/TP/VP model}
\end{figure}%

The problem with such proposals is that the subject does not depend on the verb but on T. Therefore
there is no way of serializing the accusative object before the subject unless one assumes that the
object is moved to a higher position in the tree, \eg adjoined to TP as in Figure~\ref{fig-cp-tp-vp-scrambling}.
\begin{figure}
\centering
\begin{forest}
sm edges
[CP
[C$'$
	[C [dass;that]]
        [TP
          [NP$_i$ [diesen Mann;this man, roof]]
	  [TP
	    [NP [jeder;everybody,roof]]
	    [T$'$
	      [VP
		[V$'$
		  [NP [\trace$_i$]]
		  [V [\trace$_j$]]]]
	      [T [kenn-$_j$ -t;know- -s]]]]]]]
\end{forest}
\caption{\label{fig-cp-tp-vp-scrambling}Scrambling has to be movement in the CP/TP/VP model}
\end{figure}%

\todostefaninline{add scope discussion here}




\exercises{


\begin{enumerate}
\item Provide the valency lists for the following words:

\eal
\ex laugh
\ex eat
\ex to douse
\ex 
\gll bezichtigen\\
     accuse\\\jambox{(German)}
\ex he
\ex the
\zl
If you are uncertain as far as case assignment is concerned, you may use the
  Wiktionary\footnote{
\url{https://de.wiktionary.org/}, 2018-07-02.
}.

\item Draw trees for the following examples:
\eal
\ex weil der Mann ihm ein Buch schenkt
\ex because the man gave the book to him
\ex Peter saw this yesterday.
\ex
\gll at Bjarne læste bogen\\
     that Bjarne read book.\textsc{def}\\
\glt `that Bjarne read the book'
\zl
\end{enumerate}

}




%      <!-- Local IspellDict: en_US-w_accents -->
