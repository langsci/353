%% -*- coding:utf-8 -*-
\chapter{Valence, argument order and adjunct placement}
\label{chap-valence}


%\if0
This chapter deals with the representation of valence information and sketches the basic structures that
are assumed for SVO and SOV languages. I provide an account for scrambling in those languages that
allow for it and discuss the fixed vs.\ free position of adjuncts.

\section{Valence representations}
\label{sec-valence}

The word sequences in (\mex{1}) were already discussed in footnote~\ref{fn-ex-das-kind-erwartet} on
page~\pageref{fn-ex-das-kind-erwartet}.
\eal
\ex[*]{
\gll der        Delphin erwartet\\
     the.\NOM{} dolphin expects\\
}
\ex[*]{
\gll des        Kindes       der       Delphin den        Ball dem        Kind  gibt\\
     the.\GEN{} child.\GEN{} the.\NOM{} dolphin the.\ACC{} ball the.\DAT{} child gives\\
}
\zl
The problem is that there are too few (\mex{0}a) or too many NPs (\mex{0}b) present. The concept
that is needed here is valence: like in chemistry it is assumed that heads have a certain potential
to enter into stable relations with other material \citep[\page 239]{Tesniere2015a-u}. For example, the verb \emph{erwarten} `to
expect' requires an NP in the nominative and one in the accusative. \emph{geben} `to give' is the
prototypical ditransitive verb: it can be combined with an NP in the nominative, an NP in the dative
and an NP in the accusative, but as (\mex{0}b) shows, a genitive object could not be integrated into
a sentence. 

The NPs in the examples in (\mex{1}) are arguments of the respective verbs:
\eal
\ex[]{
\gll {}[dass] der        Delphin den        Menschen erwartet\\
     \that{}  the.\NOM{} dolphin den.\ACC{} human   expects\\
\glt `that the dolphin expects the human'
}
\ex[]{
\gll {}[dass] der        Delphin dem        Kind  den        Ball gibt\\
     \that{}  the.\NOM{} dolphin the.\DAT{} child the.\ACC{} ball gives\\
\glt `that the dolphin gives the child the ball'
}
\zl
Most syntactic arguments also fill a so-called \term{semantic role} in the semantic representation of the
head. For example, the dolphin is the giver, the child is the recipient, and the ball is the item
given. \citet[Chapter~48]{Tesniere2015a-u} suggested using the analogy of dramas for the explanation of valence: if we
imagine the scene of giving, what has to happen on stage to call an event that is acted out a giving
event? There have to be the three participants, a giver, a recipient and something that is
given. Without these participants, we do not have a proper giving event. 

In addition to elements like the NPs in the examples above, which are called arguments\is{argument},
there are also so-called adjuncts. \emph{schnell} `quickly' and \emph{quickly} are examples for adjuncts:
\eal
\ex
\gll \dass{} der        Delphin dem        Kind  schnell den        Ball gibt\\
     \that{} the.\NOM{} dolphin the.\DAT{} child quickly the.\ACC{} ball gives\\
\glt `that the dolphin gives the child the ball quickly'
\ex that the dolphin gives the child the ball quickly
\zl
The adverbials provide additional information about the giving event, but they do not fill a
semantic role.

To make things complicated not all arguments have to be realized in a sentence. The ditransitive verb \emph{geben} can be
realized with any subset of its arguments, provided the context fills in the missing information.
\eal
\ex 
\gll Sie gibt Geld.\\
     she gives money\\
\glt `She gives money.'
\ex 
\gll Sie gibt den Armen.\\
     she gives the poor\\
\glt `She gives to the poor.'
\ex\label{ex-sie-gibt} 
\gll Sie gibt.\\
     she gives\\
\ex 
\gll Gib!\\
     give\\
\zl
In the case of (\mex{0}a), a certain charity setting could have been established and one can either
donate food or money or contribute some voluntary work. In such a situation, (\mex{0}a) is perfectly
fine. The transfered object in (\mex{0}b) is probably money. A possible context for (\mex{0}c) and
(\mex{0}d) is the card game skat where the person who deals out is rotating among the
players. (\mex{0}d) is an imperative. Even subjects can be dropped in imperatives since the referent
of the subject is obvious: it is the addressee of the utterance. 

The examples in (\mex{0}) show that the arguments of \emph{geben} `to give' may be omitted. This is
not the case for the accusative object of \emph{erwarten} `to expect': it is obligatory. So
arguments may be optional or obligatory, but adjuncts are always optional. While the number of
argument is limited (by the number of available slots), the number of adjuncts is not: there can be
arbitrarily many adjuncts in a phrase. (\mex{1}) shows an example with two adjuncts:
\ea
\gll \dass{} der        Delphin jetzt dem        Kind  schnell den        Ball gibt\\
     \that{} the.\NOM{} dolphin now   the.\DAT{} child quickly the.\ACC{} ball gives\\
\glt `that the dolphin now gives the child the ball quickly'
\z

The analogy to chemistry and drama may be confusing since H$_2$O is a very nice and stable molecule
and it is helpful to imagine the parallel combination of a verb with its two
arguments. Figure~\vref{fig-chemistry-valence} shows H$_2$O and the parallel combination of a verb
with its arguments corresponding to (\mex{1}a). The problem is that a single H and an O do not form
a stable combination, while (\mex{1}b) is fine: 
\eal
\ex Kirby helps Sandy.
\ex Kirby helps.
\zl
\begin{figure}
\centering
\begin{forest}
[O
  [H] 
  [H] ]
\end{forest}
\hspace{5em}
\begin{forest}
[helps
 [Kirby]
 [Sandy] ]
\end{forest}
\caption{\label{fig-chemistry-valence}Combination of hydrogen and oxygen and the combination
of a verb with its arguments}
\end{figure}%
Of course one can simply assume that there is a version of \emph{helps} that has a valence different
from the two-place valence usually assumed. Here is were the parallel breaks down since we do not
have an oxygen atom with just one open slot for the hydrogen atom. The drama analogy adds to the
confusion since the helping event described in (\mex{0}b) of course involves somebody who is
helped. The solution to this problem is to distinguish between syntactic and semantic valence. The
drama analogy helps us to find the semantic valence, the chemistry analogy is more about syntactic valence.

Given that chemistry and drama have their problems, we may go for
another\label{page-shopping-analogy} analogy: food. Let's assume you want to prepare a meal with pasta, tofu and a
tomato sauce. For the tomato sauce you also need some onions. You put all the ingredients onto a
shopping list and go to the shop. Once in the shop you realize that they run out of tofu. Your meal
will work without tofu. Tofu is optional. Fortunately, the shop has plenty of pasta. You may choose
between the different types and select the pasta type and brand you prefer. Some onions, tomatoes
and your done. Wait, next to the cashier there are these gummy bears. OK, you take some of these as
well although you did not want to and they have nothing to do with your meal and your shopping
list. The gummy bears are the adjuncts.


Back to linguistics: there are two ways of ensuring that arguments are realized together with their heads. The first one
uses techniques that were introduced in Chapter~\ref{chap-psg-xbar}. If one uses flat phrase
structure rules, one can make sure that certain arguments appear together with certain heads. A
schema similar to the one in (\mex{1}) was discussed as (\ref{ditrans-schema}) on page~\pageref{ditrans-schema}.

\ea
\label{ditrans-schema-two}
\begin{tabular}[t]{@{}l@{ }l@{ }l}
S  & $\to$ & NP[\type{nom}] NP[\type{dat}] NP[\type{acc}] V[\type{ditransitive}]\\
\end{tabular}
\z

Such schemata have been used in Generalized Phrase Structure Grammar \citep*{GKPS85a,Uszkoreit87a}, but they were
abandoned later in favor of lexicalist models, that is, models assuming that information about
arguments of a head is encoded in the lexical description of the head rather than in phrase
structure rules \parencites{Jacobson87b}[Section~5.5]{MuellerGT-Eng1}{MWArgSt}. Reasons for abandoning the phrasal
approach of GPSG were problems with so-called partial verb phrase frontings \citep{Nerbonne86a,Johnson86a} and with accounting for
interactions with morphology \citep[Section~5.5.1]{MuellerGT-Eng1}.\footnote{%
  Starting with influential work by Adele \citet{Goldberg95a} in the framework of \isi{Construction
  Grammar}, the phrasal approaches had a revival \citep{GJ2004a}. Phrasal approaches are wide-spread
  and also assumed in other frameworks 
\cites{Haugereid2007a,Haugereid2009a,CJ2005a,Alsina96a,Christie2010a,% resultative constructions as phrasal constructions and 
ADT2008a,ADT2013a}.            % argue for a phrasal analysis of the (Swedish) caused motion construction. 
The problems that lead to the abandonment of GPSG are
  ignored in the literature and newly introduced ones are not properly addressed. See
  \cites{Mueller2006d,MuellerPersian,MuellerUnifying,MWArgSt,MWArgStReply,MuellerFCG,MuellerLFGphrasal,MuellerPotentialStructure,MuellerGT-Eng4,MuellerCxG}
  for some discussion. Note that there are also lexical variants of Construction
  Grammar. \citet*{SBK2012a}, introducing \sbcg, explicitly argue for a
  lexical view citing some of the references just given. I am working in the framework of
  Constructional HPSG \citep{Sag97a}, which is also underlying the proposals sketched in this book.
} 
 
In lexical approaches, the valence of a head is represented in its lexical entry in the form of a list with descriptions of
the elements that belong to the head's valence. (\mex{1}) provides some prototypical examples:
\ea
\label{valence-specifications-German}
\begin{tabular}[t]{@{}l@{~}l@{~}l}
a. & \emph{schläft} `sleeps':        & \sliste{ NP[\type{nom}] }\\
b. & \emph{kennt} `knows':           & \sliste{ NP[\type{nom}], NP[\type{acc}] }\\
%b. & \emph{unterstützt} `supports':  & \sliste{ NP[\type{nom}], NP[\type{acc}] }\\
c. & \emph{hilft} `helps':           & \sliste{ NP[\type{nom}], NP[\type{dat}] }\\
d. & \emph{gibt} `gives':            & \sliste{ NP[\type{nom}], NP[\type{dat}], NP[\type{acc}] }\\
e. & \emph{wartet} `waits':          & \sliste{ NP[\type{nom}], PP[\type{auf}] }\\
\end{tabular}
\z
The elements in such lists come with a fixed order. The order corresponds to the order of the
elements in English and to the so-called unmarked order in German, that is, for ditransitive verbs
the order is usually nom, dat, acc (see \citew{Hoehle82a} for comments on the unmarked order). This
fixed order is needed for establishing the link between syntax and
semantics\is{semantics}.\is{linking} This will be briefly discussed in Section~\ref{sec-linking}.

Given such a valence representation for a verb like \emph{kennen} `know', one can assume a schema
that combines an element from the valence list with the respective head and passes all unsaturated
elements on to the result of the combination. The first step of the analysis of (\mex{1}) is
provided in Figure~\vref{fig-ihn-kennt}.\footnote{%
  Note that this sounds as if there were an order in which things have to be combined. This is not
  the case. HPSG grammars are sets of constraints that can be applied in any order. It is for
  explanatory purposes only that analyses are explained in a bottom up fashion throughout the book.%
}
\ea
\label{ex-dass-niemand-ihn-kennt}
\gll  {}[dass] niemand ihn kennt\\
      \spacebr{}that nobody.\NOM{} him.\ACC{} knows\\ 
\glt `that nobody knows him'
\z
\begin{figure}
\centerfit{%
\begin{forest}
sm edges
[{V \sliste{ NP[\type{nom}] } }
  [{NP[\type{acc}]} [ihn;him] ]
  [{V \sliste{ NP[\type{nom}], NP[\type{acc}] }} [kennt;knows]] ] ]
\end{forest}}
\caption{\label{fig-ihn-kennt}Analysis of \emph{ihn kennt} `him knows', valence information is represented in a list}
\end{figure}
The lexical item for \emph{kennt} `knows' has a valence description containing two NPs. In a first
step \emph{kennt} is combined with its accusative object. The resulting phrase \emph{ihn kennt} `him
knows' is something whose most importent constituent is a verb. Therefore it has V as its category
label. Certain important properties of linguistic objects are called \emph{head features}\is{head feature}. Part of speech
is one of these properties. It is assumed that all head features are passed up from the head in the tree
automatically.

The element that is not yet combined with \emph{kennt} `knows' is the \npnom. It is still
represented in the valence list of \emph{ihn kennt} `him knows'. Figure~\vref{fig-niemand-ihn-kennt}
shows the next step combining \emph{ihn kennt} with the subject \emph{niemand} `nobody'.
\begin{figure}
\centerfit{%
\begin{forest}
sm edges
[V \eliste
  [{NP[\type{nom}]} [niemand;nobody] ]
  [{V \sliste{ NP[\type{nom}] } }
    [{NP[\type{acc}]} [ihn;him] ]
    [{V \sliste{ NP[\type{nom}], NP[\type{acc}] }} [kennt;knows]] ] ]
\end{forest}}
\caption{\label{fig-niemand-ihn-kennt}Analysis of (\emph{dass}) \emph{niemand ihn kennt} `that nobody
  knows him'}
\end{figure}
The result is a linguistic object of category verb with the empty list as valence representation.

As will be shown shortly, the schema that licenses structures like the V \eliste{} and V \sliste{ NP[\type{nom}] }  in
Figure~\ref{fig-niemand-ihn-kennt} is a more abstract version of the rule in (\ref{psg-binaer}) on page~\pageref{psg-binaer}.

% This can be depicted as in
% Figure~\vref{fig-valence-German}, which is an example analysis of (\mex{1}).
% \ea
% \label{ex-dass-niemand-ihn-kennt}
% \gll  {}[dass] niemand ihn kennt\\
%       \spacebr{}that nobody.\NOM{} him.\ACC{} knows\\ 
% \glt `that nobody knows him'
% \z
% \begin{figure}
% \centerfit{%
% \begin{forest}
% sm edges
% [{S \eliste}
%   [{NP[\type{nom}]} [niemand;nobody] ]
%   [{V$'$\sliste{ NP[\type{nom}] } }
%     [{NP[\type{acc}]} [ihn;him] ]
%     [{V \sliste{ NP[\type{nom}], NP[\type{acc}]}} [kennt;knows]] ] ]
% \end{forest}}
% \caption{\label{fig-valence-German}Analysis of (\emph{dass}) \emph{niemand ihn kennt} `that nobody
%   knows him', valence information is represented in a list}
% \end{figure}

% The lexical item for \emph{kennt} `knows' has a valence description containing two NPs. In a first
% step \emph{kennt} is combined with its accusative object. The resulting phrase \emph{ihn kennt} `him
% knows' is something whose most importent constituent is a verb. Therefore it has a V in its category
% label. Since \emph{ihn kennt} is not a sentence but something intermediate, it gets the label
% V$'$.\footnote{%
%   These labels are abbreviations for complex categories. Their internal makeup is given in
%   Table~\ref{tab-abbreviations-v-vbar-s} on p.\,\pageref{tab-abbreviations-v-vbar-s}. The labels are
%   similar to what is known from \xbart, but -- as was already mentioned in the previous chapter --
%   the theory developed here is not following all the tenets of \xbart. For example, simple nouns
%   like \emph{house} are N$'$ and there is no \nnull in the analysis of NPs like \emph{the house}.

%   Section~\ref{sec-intro-spr-comps} explains why the abbreviation for \emph{ihn kennt} `him knows' is V$'$ rather than VP.
% }
%
% The nodes for V$'$ and S are licensed by a schema that combined a head with one element of its
% valence list. The full schema will be given in Chapter~\ref{chap-HPSG-light}, but we will discuss a
% simplified version of it in Section~\ref{sec-intro-schemata}.

It is probably helpful to return to our meal-shopping analogy. Assume we are using an app to
organize our shopping lists. For our current meal we need pasta and tomatoes. They are listed in the
app in a certain order (tomatoes, pasta) and there are little images attached to the products. Once
we found something matching the pasta, we remove the pasta from the list and the remaining list
contains an icon reminding us of the tomatoes. Once we have those, we remove them from the list and
since nothing is left on the list, we pay. Linguistic structures are similar: we start
with a verb selecting two NPs, we combine it with one NP and then with the second one. The result is
a complete structure, something with an empty valence list.

There are various ways to deal with optional arguments. The most simple one is to assume further
lexical items selecting fewer arguments. For the example in (\ref{ex-sie-gibt}) one would assume the valence
representation in (\mex{1}a) and for sentences with \emph{warten} `to wait' without prepositional
object, one would assume (\mex{1}b) in addition to the representations in (\ref{valence-specifications-German}):
\ea
\begin{tabular}[t]{@{}l@{~}l@{~}l}
a. & \emph{gibt} `gives':            & \sliste{ NP[\type{nom}] }\\
b. & \emph{wartet} `waits':          & \sliste{ NP[\type{nom}] }\\
\end{tabular}
\z


\section{Scrambling}
\label{sec-scrambling}

As we already saw in the data discussion in Chapter~\ref{sec-phenomenon-scrambling}, some languages allow for
scrambling of arguments. For those languages one can assume that heads can combine with any of its
arguments not necessarily beginning with the last one as it was the case in the analysis in Figure~\ref{fig-niemand-ihn-kennt}.
Figure~\vref{fig-scrambling-German} shows the analysis of (\mex{1}).
\ea
\gll {}[dass] ihn niemand kennt\\
     \spacebr{}that him.\ACC{} nobody.\NOM{} knows\\
\glt `that nobody knows him'
\z
\begin{figure}
\centerfit{%
\begin{forest}
sm edges
[{V \eliste}
   [{NP[\type{acc}]} [ihn;him] ]
   [{V \sliste{ NP[\type{acc}] } }
      [{NP[\type{nom}]} [niemand;nobody] ]
      [{V \sliste{ NP[\type{nom}], NP[\type{acc}] }} [kennt;knows] ] ] ]
\end{forest}}
\caption{\label{fig-scrambling-German}Analysis of (\emph{dass}) \emph{ihn niemand kennt} `that nobody
  knows him', languages that allow for scrambling permit the saturation of arguments in any order}
\end{figure}
Rather than combining the verb with the accusative argument (the object) first, it is combined with
the nominative (the subject) and the accusative (the object) is added in a later step.


\section{SVO: Languages with fixed SV order and valence features}
\label{sec-intro-schemata}
\label{sec-intro-spr-comps}

The last section demonstrated how verb-final sentences in German can be analyzed. Of course it is
easy to imagine how this extends to VSO languages: The head is initial and combines with the first
element in the valence list first and then with all the other elements. However, nothing has been
said about the SVO languages so far. In languages like Danish, English, and so on all objects are
realized after the verb as in (\mex{1}), it is just the subject that preceedes the verb.
\ea
Kim gave Sandy the book.
\z
The verb together with its objects forms a unit in a certain sense: it can be fronted (\mex{1}a). It can be
selected by dominating verbs (\mex{1}b), it can be coordinated (\mex{1}c), and it is the place where adjuncts attach to (\mex{1}d--e).
\eal
\ex John promised to read the book and [read the book], he will.
\ex He will [read the book].
\ex Kim [[sold the car] and [bought a bicycle]]. 
\ex He often [reads the book].
%\ex He [reads the book] often.
\ex \ldots{} [often [read the book] slowly], he will.
\zl
This can be modeled adequately by assuming two valence lists: one for the complements (\comps short for \textsc{complements}\isfeat{comps}) and
one for the subject. The list for the subject is called \textsc{specifier} list
(\spr).\isfeat{spr}\footnote{%
  There are various versions of HPSG: \citet[Chapter~3.2]{ps} assumed that all arguments of a head are
  represented in one list. This list was called \subcatl. \citet{Borsley87a} argued that one should
  use several valence features (\subj, \spr, and \comps) and this was adopted in
  \citew[Chapter~9]{ps2}: subjects of verbs were selected via \subj and determiners via
  \spr. \citet*[Chapter~4.3]{SWB2003a} assume that both subjects and determiners are selected via \spr, which is
  what is assumed in the grammars developed here too. \citet[Section~3.3]{Sag2012a} presents a versions of HPSG
  called Sign-Based Construction Grammar (SBCG) which assumes one valence list for all arguments a
  it was common in 1987. This return to an abandoned approach came without any argumentation. Hence, I do not adopt this
  variant of HPSG but stick to the separation of subjects and other arguments to \spr and \comps. I
  will not use \subj as a valence feature, but it will be introduced in the analysis of verbal
  complexes in Chapter~\ref{chap-verbal-complex}.
} The specifier list plays a role both in the analysis of sentences and in the analysis of noun phrases. Nouns
select their determiner via \spr and all their other arguments via \comps. Figure~\vref{fig-svo}
shows the analysis of the sentence (\mex{1}) using the features \spr and \comps.
\ea
Nobody knows him.
\z
\begin{figure}
\centerfit{%
\begin{forest}
sm edges
[{V[\spr \eliste, \comps \eliste]}, name=S
   [{NP[\type{nom}]} [nobody] ]
   [V\feattab{
      \spr \sliste{ NP[\type{nom}] }, \comps \sliste{}}, name=VP
     [V\feattab{
         \spr \sliste{ NP[\type{nom}] },\\
         \comps \sliste{ NP[\type{acc}] }} [knows] ]
        [{NP[\type{acc}]} [him] ] ] ]
\node [right=4cm] at (S)
    {
        = S
    };
\node [right=4cm] at (VP)
    {
        = VP
    };
\end{forest}}
\caption{\label{fig-svo}Analysis of the SVO order with two separate valence features}
\end{figure}
The \compsl of \emph{knows} contains a description of the accusative object and the accusative
\emph{him} is combined in a first step with \emph{knows}. In addition to the accusative object
\emph{knows} selects for a subject. This selection is passed on to the mother node, the VP. Hence,
the \sprv of \emph{knows him} is identical to the \sprv of \emph{knows}. The VP \emph{knows him}
selects for a nominative NP. This NP is realized as \emph{nobody} in Figure~\ref{fig-svo}. The
result of the combination of \emph{knows him} with \emph{nobody} is \emph{nobody knows him}, which
is complete: It has both an empty \sprl and an empty \compsl. The two rules that are responsible for
the combinations in Figure~\ref{fig-svo} are called the Specifier-Head Schema and the
Head-Complement Schema. I use VP as abbreviation for something with a verbal head and an empty \compsl and at least
one element in the \sprl and S as abbreviation for something with a verbal head and empty lists for
both the \spr and the \compsv.

In Section~\ref{sec-scrambling} it was explained how scrambling can be accounted for: the rules that
combine heads with their arguments can take the arguments from the list in any order. For languages
with stricter constituent order requirements the rules are stricter: the arguments have to be taken
off the list consistently from the beginning or from the end. So for English and Danish one starts
at the beginning of the list and for head-final languages without scrambling one starts at the end
of the list. Figure~\ref{fig-svo-ditrans} shows the analysis of a sentence with a ditransitive verb.
\begin{figure}
\centerfit{%
\begin{forest}
sm edges
[{V[\spr \eliste, \comps \eliste]}
   [{NP[\type{nom}]} [Kim] ]
   [V\feattab{
      \spr \sliste{ NP[\type{nom}] }, \comps \sliste{}}
     [V\feattab{
         \spr \sliste{ NP[\type{nom}] },\\
         \comps \sliste{ PP[\type{to}] }}
       [V\feattab{
           \spr \sliste{ NP[\type{nom}] },\\
           \comps \sliste{ NP[\type{acc}], PP[\type{to}] }} [gave] ]
         [{NP[\type{acc}]} [a book, roof] ] ]
       [{PP[\type{to}]} [to Sandy, roof] ] ] ]
\end{forest}}
\caption{\label{fig-svo-ditrans}Analysis of the SVO order with two separate valence features and two
  elements in \comps}
\end{figure}
The accusative object is the first element in the \compsl and it is combined with the verb
first. The result of the combination is a verbal projection that has the PP[\type{to}] as the sole
element in the \compsl. It is combined with an appropriate PP in the next step resulting in a verbal
projection that has an empty \compsl (a VP).


The analysis of our first German example in Figure~\ref{fig-niemand-ihn-kennt} did not use a name
for the valence list. So the question is: How does the analysis of German relate to the analysis of
English using \spr and \comps. A lot of researchers from various frameworks argued that it is not
useful to distinguish the subjects of finite verbs from other arguments. All the tests that have
been used to show that subjects in English differ from complements do not apply to the arguments of
finite verbs in German. Hence, researchers like \citet{Pollard90a}, \citet{Haider93a}, 
\citet[\page 376]{Eisenberg94b}, and \citet{Kiss95a} argued for so-called subject as complement
analyses.\itdopt{Provide examples.} Figure~\vref{fig-spr-german} shows the adapted analysis of
(\ref{ex-dass-niemand-ihn-kennt}) -- repeated here as
(\ref{ex-dass-niemand-ihn-kennt-two}):
\ea
\label{ex-dass-niemand-ihn-kennt-two}
\gll  {}[dass] niemand ihn kennt\\
      \spacebr{}that nobody.\NOM{} him.\ACC{} knows\\ 
\glt `that nobody knows him'
\z
\begin{figure}
\centerfit{%
\begin{forest}
sm edges
[{V[\spr \eliste, \comps \eliste]}, name=S
        [{NP[\type{nom}]} [niemand;nobody] ]
        [{V\feattab{
              \spr \sliste{ }, \comps \sliste{ NP[\type{nom}] } }}, name = Vs
          [{NP[\type{acc}]} [ihn;him] ] 
          [V\feattab{
              \spr \sliste{  },\\
              \comps \sliste{ NP[\type{nom}], NP[\type{acc}]}} [kennt;knows] ]
] ]
\node [right=4cm] at (S)
    {
        = S
    };
\node [right=4cm] at (Vs)
    {
        = V$'$
    };
\end{forest}}
\caption{\label{fig-spr-german}The analysis of a German sentence with \spr and \compsl}
\end{figure}
The difference between German and English is that German contains all arguments in the \compsl of
the finite verb and no arguments in the \sprl. Since the elements in the \compsl can be combined
with the head in any order, it is explained why all permutations of arguments are
possible. Specifiers are realized to the left of their head. This is the same for German and
English. For German this is not relevant in the verbal domain, but the Specifier-Head Schema, which
is introduced shortly, is used in the analysis of noun phrases.

Throughout the remainder of this book I use the abbreviations in Table~\vref{tab-abbreviations-v-vbar-s}.
\begin{table}
 \begin{tabular}[t]{@{}l@{ = }l}\lsptoprule
             S  & V[\spr \eliste, \comps \eliste]\\
             VP & V[\spr \sliste{ NP[\type{nom}] }, \comps \sliste{}]\\
             V$'$ & all other V projections apart from verbal complexes\\[2pt]
             NP & N[\spr \eliste, \comps \eliste]\\
             N$'$ & N[\spr \sliste{ Det }, \comps \sliste{}]\\\lspbottomrule
             \end{tabular}
\caption{\label{tab-abbreviations-v-vbar-s}Abbreviations for S, VP, and V$'$ and NP, N$'$}
\end{table}

\section{Immediate dominance schemata}

In Section~\ref{sec-valence} I already mentioned that the non-terminal nodes in a tree, that is, the
nodes that are not the leaves of the tree, are licensed by schemata similar to those introduced in
Chapter~\ref{sec-PSG-Merkmale} and~\ref{sec-xbar}. In fact, the schemata are even more abstract than
\xbar schemata since they do not make any statements about linear order of the daughters. The
two schemata discussed in this section are sketched here as (\mex{1}):
\ea\label{schema-head-spr-and-head-comps-preliminary}
Specifier-Head Schema and Head-Complement Schema (preliminary)
\begin{tabular}[t]{@{}l@{ }l@{}}
H[\spr \ibox{1}]   & $\to$ H[\spr \ibox{1} $\oplus$ \sliste{ \ibox{2} }, \comps \eliste]\hspace{1em}\ibox{2}  \\
H[\comps \ibox{1}] & $\to$ H[\comps \sliste{ \ibox{2} } $\oplus$ \ibox{1}]\hspace{1em}\ibox{2} \\
\end{tabular}
\z


Syntactic rules are usually called schemata since they are rather abstract. 
%The details about such schemata will be given in Chapter~\ref{chap-HPSG-light}, 
The details about such schemata are given in more formal HPSG literature like
\citew[Chapter~4]{MuellerLehrbuch3} or \citew{AB2021a}, 
but Figure~\vref{fig-spr-head} and
Figure~\vref{fig-head-comp} provide the respective tree representations.
\begin{figure}
\begin{forest}
[H\feattab{\spr \ibox{1}}%,\\
           %\comps \eliste}
  [\ibox{2}]
  [H\feattab{\spr \ibox{1} $\oplus$ \sliste{ \ibox{2} },\\
             \comps \eliste}
  ]]
\end{forest}
\caption{\label{fig-spr-head}Sketch of the Specifier-Head Schema (preliminary)}
\end{figure}
The H stands for \emph{head}. The term \term{head daughter} is used for the daughter that either is
the head of a phrase or contains the head of the phrase (\eg the verb in a sentence or the noun in a
noun phrase). \emph{append} ($\oplus$) is a relation that concatenates two 
lists. For instance the concatenation of \sliste{ \normalfont a } and \sliste{ \normalfont b } is
\sliste{ \normalfont a, b }. The concatenation of the empty list \eliste{} with another list yields
the latter list. To give some examples that are of relevance in this chapter consider the list
\sliste{ NP[\type{nom}], NP[\type{dat}], NP[\type{acc} ] }. \emph{append} can be used to append two
lists resulting in our list in the following ways:
\eal
\ex \eliste{} $\oplus$ \sliste{ NP[\type{nom}], NP[\type{dat}], NP[\type{acc}] } = \sliste{ NP[\type{nom}], NP[\type{dat}], NP[\type{acc}] }
\ex \sliste{ NP[\type{nom}] } $\oplus$ \sliste{ NP[\type{dat}], NP[\type{acc}] } = \sliste{ NP[\type{nom}], NP[\type{dat}], NP[\type{acc}] }
\ex \sliste{ NP[\type{nom}], NP[\type{dat}] } $\oplus$ \sliste{ NP[\type{acc}] } = \sliste{ NP[\type{nom}], NP[\type{dat}], NP[\type{acc}] }
\ex \sliste{ NP[\type{nom}], NP[\type{dat}], NP[\type{acc}] } $\oplus$ \eliste{} = \sliste{ NP[\type{nom}], NP[\type{dat}], NP[\type{acc}] }
\zl
The schema in Figure~\ref{fig-spr-head} takes a list apart in such a way that a list with a
singleton element ( \sliste{ \ibox{2} } ) and a remaining list \iboxb{1} results. Assuming the
three-element list with nom, dat and acc elements, this would be the case in (\mex{0}c) and \ibox{2} would be NP[\type{acc}] and
\ibox{1} would be \sliste{ NP[\type{nom}], NP[\type{dat}] }. In this book, the \sprl has at most one element.\footnote{
  But see \citew{MOe2013b} for an analysis of \isi{object shift} in \ili{Danish} assuming multiple elements in
  the \sprl.
} It can be an NP[\type{nom}] in the case of verbs in the SVO languages or the determiner, if
the head is a noun. If one splits a list with a singleton element into a list containing one element
and a rest, the rest will always be the empty list. Hence, with the lists at the right-hand side of the equations in (\mex{1}), \ibox{1} will be the
empty list and \ibox{2} will be NP[\type{nom}] and Det, respectively. 
\eal
\ex \eliste{} $\oplus$ \sliste{ NP[\type{nom}] } = \sliste{ NP[\type{nom}] }
\ex \eliste{} $\oplus$ \sliste{ Det } = \sliste{ Det }
\zl

For a schema like the one in Figure~\ref{fig-spr-head} to apply, the descriptions of the
daughters have to match the actual daughters. For instance \emph{sleeps} is compatible with the
right daughter: it has an NP[\type{nom}] in its \sprl. When \emph{sleeps} is realized as a
daughter of the schema in Figure~\ref{fig-spr-head}, \ibox{2} is instantiated as
NP[\type{nom}]. Therefore the left daughter has to be compatible with an NP[\type{nom}]. It can be
realized as a simple pronoun like \emph{she} or a complex NP like \emph{the brown squirrel}. Two
analyses are shown in Figure~\vref{fig-she-sleeps-the-brown-squirrel-sleeps}.
\begin{figure}
\hfill
\begin{forest}
sm edges
[V\feattab{\spr \eliste,\\
           \comps \eliste}
  [{\ibox{1} NP[\type{nom}]} [she]]
  [V\feattab{\spr \sliste{ \ibox{1} NP[\type{nom}] },\\
             \comps \eliste} [sleeps]]]
\end{forest}
\hfill
\begin{forest}
sm edges
[V\feattab{\spr \eliste,\\
           \comps \eliste}
  [{\ibox{1} NP[\type{nom}]} [the brown squirrel,roof]]
  [V\feattab{\spr \sliste{ \ibox{1} NP[\type{nom}] },\\
             \comps \eliste} [sleeps]]]
\end{forest}\hfill\mbox{}
\caption{Head-Specifier phrases with a subject and an intransitive verb}\label{fig-she-sleeps-the-brown-squirrel-sleeps}
\end{figure}
The \ibox{1} in Figure~\ref{fig-she-sleeps-the-brown-squirrel-sleeps} says that whatever is in the
\sprl is identified with whatever is the other element in the tree. I wrote down \npnom following
the \ibox{1} in both the NP node and within the \sprl, but it would have been sufficient to mention
\npnom at one of the two places. The actual number in the box does not matter. What matters is where the same
number appears in the trees or structures. I usually start with \ibox{1} at the top of the tree and
use consecutive number for the following sharings.

Figure~\ref{fig-nobody-gave-the-child-a-book} on p.\,\pageref{fig-nobody-gave-the-child-a-book}
below shows an example analysis with a ditransitive verb also involving the Specifier-Head
Schema. The specification of the \compsv of the head daughter in the Specifier-Head Schema ensures
that the verb is combined with its complements before the specifier is added.

Apart from its use for the analysis of subject--VP combinations in the SVO languages, the Specifier-Head Schema is also
used for the analysis of NPs in all the Germanic languages. Figure~\ref{fig-spr-head-the-squirrel} shows the analysis of the NP \emph{the squirrel}.
\begin{figure}
\begin{forest}
[{N[\spr \eliste, \comps \eliste]}
  [\ibox{1} Det [the]]
  [{N[\spr \sliste{ \ibox{1} }, \comps \eliste]} [squirrel]]]
\end{forest}
\caption{\label{fig-spr-head-the-squirrel}Analysis of the NP \emph{the squirrel}}
\end{figure}
\emph{squirrel} selects for a determiner and the result of combining \emph{squirrel} with a determiner is a
complete nominal projection, that is, an NP. There are also nouns like \emph{picture} that take a
complement:
\ea
a picture of Kim
\z
The combination of \emph{picture} and its complement \emph{of Kim} is parallel to the combination of
a verb with its object in VO languages with fixed constituent order. For such combinations we need a separate schema: the Head-Complement Schema,
which is given in Figure~\ref{fig-head-comp}.
\begin{figure}
\begin{forest}
[{H[\comps \ibox{1}]}
  [{H[\comps  \sliste{ \ibox{2} } $\oplus$ \ibox{1}  ]}]
  [\ibox{2}]]
\end{forest}
\caption{\label{fig-head-comp}Sketch of the Head-Complement Schema (preliminary)}
\end{figure}
The schema splits the \compsl of a head into an initial list with one element \iboxb{2}, which is
realized as the complement daughter to the right.\footnote{%
  In principle daughters are unordered in HPSG as they were in GPSG \citep{GKPS85a}. Special linearization rules are
  used to order a head with respect to its siblings in a local tree. So a schema licensing a tree
  like the one in Figure~\ref{fig-head-comp} would also license a tree with the daughters in a
  different order unless one had linearization rules that rule this out. See
  \citew{MuellerOrder} for an overview of approaches to constituent order in HPSG. Linear precedence
  rules are discussed in more detail in Section~\ref{sec-lp-rules}.
}
This schema licenses both the combination of \emph{gave} and \emph{the child} and the combination of
\emph{gave the child} and \emph{a book} in Figure~\vref{fig-nobody-gave-the-child-a-book}, which shows the
analysis of (\mex{1}).\footnote{%
  English nouns and determiners do not inflect for case. However, case is manifested at pronouns:
  \emph{he} (nominative), \emph{his} (genitive), \emph{him} (accusative). Hence, verbs in double object
  constructions select for two accusatives rather than for dative and accusative as in German. 
}
\ea
\label{ex-nobody-gave-the-child-a-book}
Nobody gave the child a book.
\z
\begin{figure}
\centerfit{%
\begin{forest}
sm edges
[{V[\spr \eliste, \comps \eliste]}
   [{\ibox{1} NP[\type{nom}]} [nobody] ]
   [V\feattab{
      \spr \sliste{ \ibox{1} NP[\type{nom}] }, \comps \sliste{}}
     [V\feattab{
         \spr \sliste{ NP[\type{nom}] },\\
         \comps \sliste{ \ibox{2} NP[\type{acc}] }} 
        [V\feattab{
           \spr \sliste{ NP[\type{nom}] },\\
           \comps \sliste{ \ibox{3} NP[\type{acc}], \ibox{2} NP[\type{acc}]}} [gave] ]
        [{\ibox{3} NP[\type{acc}]} [the child,roof] ] ]
     [{\ibox{2} NP[\type{acc}]} [a book,roof ] ] ] ]
\end{forest}}
\caption{\label{fig-nobody-gave-the-child-a-book}Analysis of the sentences with a ditransitive verb}
\end{figure}

To keep things simple, the Specifier-Head Schema did not mention the \compsv of the mother. The
Head-Complement Schema did neither mention the \sprv of the head daughter nor the one of the mother. But the
respective values are important, since something has to be said about these values in structures
that are licensed by these schemata. If the \sprv in the combination of \emph{gave} and \emph{the
  child} would not be constrained by the Head-Complement Schema, an value would be possible. This
includes a \sprl containing two genitive NPs and an accusative NP. Sequences like (\mex{1}) would be
licensed:

\ea[*]{
his his him gave the child a book
}
\z
To avoid such unspecified \sprvs, the \sprv of the head daughter is identified with the \sprv of the
mother node in the schema. This is the \ibox{1} in (\mex{1}b). Similarly, the \compsv of the mother
in Specifier-Head phrases has to be specified to be identical to the \compsv of the head daughter
(\ibox{2} in (\mex{1}a)) and hence the empty list.

\ea\label{schema-head-spr-and-head-comps}
Specifier-Head Schema and Head-Complement Schema (final)
\begin{tabular}[t]{@{}l@{~}l@{ }l@{}}
a. & H[\spr \ibox{1}, \comps \ibox{2}] & $\to$ H[\spr \ibox{1} $\oplus$ \sliste{ \ibox{3} }, \comps \ibox{2} \eliste]\hspace{1em}\ibox{3}  \\
b. & H[\spr \ibox{1}, \comps \ibox{2}] & $\to$ H[\spr \ibox{1}, \comps \ibox{2} $\oplus$ \sliste{ \ibox{3} }]\hspace{1em}\ibox{3} \\
\end{tabular}
\z
Figure~\vref{fig-spr-head-head-comps-final} shows the final versions of the two schemata.
\begin{figure}
\hfill
\begin{forest}
[H\feattab{\spr \ibox{1},\\
           \comps \ibox{2}}
  [\ibox{3}]
  [H\feattab{\spr \ibox{1} $\oplus$ \sliste{ \ibox{3} },\\
              \comps \ibox{2} \eliste}]]
\end{forest}
\hfill
\begin{forest}
[H\feattab{\spr \ibox{1},\\
           \comps \ibox{2}}
  [H\feattab{\spr \ibox{1},\\
             \comps  \sliste{ \ibox{3} } $\oplus$ \ibox{2}  ]}]
  [\ibox{3}]]
\end{forest}
\hfill\mbox{}
\caption{\label{fig-spr-head-head-comps-final}Sketch of the Specifier-Head and Head-Complement Schema}
\end{figure}




\section{Scrambling and free VO/OV order}

Now, in order to analyze languages with free constituent order, we need a more liberal variant of
the schema in Figure~\ref{fig-head-comp}. Figure~\vref{fig-head-comp-free} splits the \compsl of a
head into three parts: a list \ibox{1}, a list containing exactly one element \sliste{ \ibox{3} }
and a third list \ibox{2}. The element of the second list is realized as the complement of the head.
\begin{figure}
\begin{forest}
[{H[\comps \ibox{1} $\oplus$ \ibox{2}]}
  [\ibox{3}]
  [{H[\comps  \ibox{1} $\oplus$ \sliste{ \ibox{3} } $\oplus$ \ibox{2}  ]}]]
\end{forest}
\caption{\label{fig-head-comp-free}Sketch of the Head-Complement Schema for languages with free
  constituent order}
\end{figure}
The length of the lists \ibox{1} and \ibox{2} is not restricted. For our example list containing a
nom, a dat and an acc element, there are the following possibilities to split the list:

\eal
\ex \eliste{} $\oplus$ \sliste{ NP[\type{nom}] } $\oplus$ \sliste{ NP[\type{dat}], NP[\type{acc}] } 
\ex \sliste{ NP[\type{nom}] } $\oplus$ \sliste{ NP[\type{dat}] } $\oplus$ \sliste{ NP[\type{acc}] } 
\ex \sliste{ NP[\type{nom}], NP[\type{dat}] } $\oplus$ \sliste{ NP[\type{acc}] } $\oplus$ \eliste 
\zl
So \ibox{3} in Figure~\ref{fig-head-comp-free} would be \npnom in (\mex{0}a), \npdat in (\mex{0}b) and \npacc in (\mex{0}c).

If one restricts \ibox{1} to be the
empty list, one gets grammars that saturate complements from the beginning of the list (VO languages
with fixed order like \ili{English}) and if one restricts \ibox{2} to be the empty list, one gets grammars that take the last
element from the \compsl for combination with a head (this would be an OV languages with fixed
order, if such a language would exist). Scrambling languages like German allow any
complement to be combined with its head since there is neither a restriction on \ibox{1} nor one on \ibox{2}.


\inlinetodostefan{Yiddish VO/OV}




\section{Linear precedence rules}
\label{sec-lp-rules}

The abstract schemata are similar to the schemata that we gained by abstracting over simple phrase
structure rules in Chapter~\ref{chap-psg}. They are similar to abstract \xbar rules. However, there
is an important difference: the elements at the right-hand side of a rule and the daughters in the
corresponding treelets in the figures visualizing the schemata are not ordered. This means that a
schema like the one in (\mex{1}) can be used to analyze configurations with a preceeding b and with
b preceeding a.
\ea
m $\to$ a b
\z
As will be shown shortly, this comes handy in situations in which one wants to leave the actual
order underspecified.

For the Head Complement Schema discussed above this means that actually two orders can be analyzed:
head-daughter before complement and complement before head-daughter. Hence the Head-Complement
Schema is general enough to analyze the German and English phrases in (\mex{1}):
\eal
\ex
\gll dem Kind  ein Buch gibt\\
     the child the book gives\\
\ex gives the child the book
\zl
But such a general schema without restrictions would also allow an analysis for (\mex{1}b) and (\mex{1}c):
\eal
\ex[]{
\gll \dass{} niemand dem Kind  ein Buch vorliest\\
     \that{} nobody  the child a book \partic.reads\\
\glt `that nobody reads a book to the child'
}
\ex[*]{ 
\gll \dass{} dem Kind  niemand vorliest ein Buch\\
     \that{} the child nobody  \partic.reads a book\\
}
\ex[*]{
\gll \dass{} niemand vorliest dem Kind   ein Buch\\
     \that{} nobody  \partic.reads the child a book\\
}
\zl
The structures licensed by the Head-Complement Schema without any restrictions are shown in \thefiguresref{fig-dem-kind-niemand-gab-ein-buch-head-comp}{fig-niemand-gab-dem-kind-ein-buch-head-comp}
\begin{figure}
\begin{forest}
sm edges
[{V[\comps \sliste{ }]},s sep+=1em
  [{V[\comps \sliste{ \ibox{1} }]}
    [\ibox{2} \npdat [dem Kind;the child,roof]]
    [{V[\comps \sliste{ \ibox{2}, \ibox{1} }]} [\ibox{3} \npnom [niemand;nobody]]
       [{V[\comps \sliste{ \ibox{3}, \ibox{2}, \ibox{1} }]}  [vorliest;\textsc{part}.reads]]]]
  [\ibox{1} \npacc [ein Buch;a book,roof]]]]
\end{forest}
\caption{\label{fig-dem-kind-niemand-gab-ein-buch-head-comp}Unwanted analysis using the
  Head-Complement Schema without linearization constraints}
\end{figure} 
\begin{figure}
\begin{forest}
sm edges
[{V[\comps \sliste{ }]}
  [{V[\comps \sliste{ \ibox{1} }]},s sep+=1em
    [{V[\comps \sliste{ \ibox{2}, \ibox{1} }]} [\ibox{3} \npnom [niemand;nobody]]
       [{V[\comps \sliste{ \ibox{3}, \ibox{2}, \ibox{1} }]}  [vorliest;\textsc{part}.reads]]]
    [\ibox{2} \npdat [dem Kind;the child,roof]]] 
  [\ibox{1} \npacc [ein Buch;a book,roof]]]]
\end{forest}
\caption{\label{fig-niemand-gab-dem-kind-ein-buch-head-comp}Unwanted analysis using the
  Head-Complement Schema without linearization constraints}
\end{figure} 

Now, this problem is easy to fix: what is needed is binary feature specifying whether a head is
initial or not. The feature is called \textsc{initial} (abbreviated as \textsc{ini}). All
head-daughters that are \ini{}+ are always serialized to the left of their complement and all those
that are \ini{}$-$ are serialized to the right. The linearization rules are provided in (\mex{1}):
\eal
\label{lp-regeln}
\ex HEAD [\textsc{initial}+] $<$ COMPLEMENT
\ex COMPLEMENT $<$  HEAD [\textsc{initial}$-$]
\zl
German verbs are specified to be \textsc{initial}$-$, while English verbs are
\textsc{initial}$+$. Because of this specification and the linearization rules in (\mex{0}), verbs
are always ordered after their complements in German (and other SOV languages) and before their
complements in English (and other SVO languages). Of course, there are sentences in German in which
the verb is in first or second position and there are sentences in the Germanic SVO languages in
which the object precedes the verb. These sentences will be covered in Chapter~\ref{chap-verb-position}.


\section{Adjuncts}

While arguments are selected by their head, adjuncts select the head. The difference between
languages like Dutch and German on the one hand and Danish and English on the other hand can be
explained by assuming that adjuncts in the former languages are less picky as far as the element is
concerned with which they combine. Dutch (\mex{1}) and German (\mex{2})
adjuncts can attach to any verbal projection, while Danish (\mex{3}) and English (\mex{4}) require a
VP (see also Section~\ref{sec-phenomena-position-of-adverbials}):

\eal
% todo check
\ex 
\gll [dat] onmiddellijk iedereen het boek leest\\
     \spacebr{}that promptly everybody the book reads\\\dutch
\glt `that everybody reads the book promptly'
\ex
\gll [dat] iedereen onmiddellijk het boek leest\\ 
     \spacebr{}that everybody promptly the book reads\\ 
\ex
\gll [dat] iedereen het boek onmiddellijk leest\\ 
    \spacebr{}that everybody the book promptly reads\\
\zl

\eal
\ex
\label{ex-m-j-b-l} 
\gll {}[dass] sofort jeder das Buch liest\\
     \spacebr{}that promptly everybody the book reads\\\german
\glt `that everybody reads the book promptly'
\ex
\label{ex-j-m-b-l} 
\gll {}[dass] jeder sofort das Buch liest\\
     \spacebr{}that everybody promptly the book reads\\ 
\ex
\label{ex-j-b-m-l}
\gll {}[dass] jeder das Buch sofort liest\\
    \spacebr{}that everybody the book promptly reads\\
\zl


\eal
% todo check
\ex 
\gll at hver læst bogen straks\\
     that everybody reads book.\textsc{def} promptly\\ \danish
\glt `that everybody reads the book promptly'
\ex 
\gll at hver straks læst bogen\\
     that everybody promptly reads book.\textsc{def}\\
\glt `that everybody promptly reads the book'
\zl

\eal
\ex that everybody reads the book promptly
\ex that everybody promptly reads the book
\zl

% \eal
% \ex Kim will have been [promptly [removing the evidence]].\footnote{
%   The examples are due to Stephen Wechsler (p.\,c.\, 2013).}
% %from \citew{Wechsler2015a}. in the draft but now gone.}
% \ex Kim will have been [[removing the evidence] promptly].
% \zl

For the selection of arguments the features \spr and \comps are used. In parallel there is a \modf
that is part of a lexical description of a head of a phrase that can function as an adjunct (\textsc{mod} is
an abbreviation for \emph{modified}). The value of  \textsc{mod} is a description of an appropriate head. 
Head"=adjunct structures are licencsed by the schema in
Figure~\vref{fig-head-adj}.
\begin{figure}
\begin{forest}
[{H[\spr \ibox{1}, \comps \ibox{2}]}
  [{[\textsc{mod} \ibox{3}, \spr \eliste, \comps \eliste]}]
  [{\ibox{3} H[\spr \ibox{1}, \comps  \ibox{2}]}]]
\end{forest}
\caption{\label{fig-head-adj}Sketch of the Head-Adjunct Schema}
\end{figure}
For instance, attributive adjectives have \nbar as their \modv, where \nbar is an abbreviation for a
nominal projection that has an empty \compsl and a \sprl that contains a determiner. (\mex{1}) shows
the lexical item for \emph{brown}:
\eas
Lexical item for \emph{brown}:\\
\avm{
[ phon & \phonliste{ brown }\\
  mod  & \nbar\\
  spr  & <>\\
  comps & <> ]
}
\zs
The analysis of the phrase \emph{brown squirrel} is shown in Figure~\vref{fig-brown-squirrel}.
\begin{figure}
\begin{forest}
sm edges
[{\nbar}
  [{Adj[\textsc{mod} \ibox{2}]} [brown]]
  [{\ibox{2} \nbar} [squirrel]]]
\end{forest}
\caption{\label{fig-brown-squirrel}Analysis of the head-adjunct structure \emph{brown squirrel}}
\end{figure}
In languages like German in which the adjective agrees with the noun in gender, number, and
inflection class, the properties that the noun must have can be specified inside the \modv. For
instance, \emph{kleiner} selects a male noun and \emph{kleine} selects a female one:
\eal
\ex 
\gll ein kleiner Hund\\
     a   little  dog\\
\ex 
\gll eine kleine Katze\\
     a    little cat\\
\zl


For German adverbials the value restricts the part of speech of the head to be verb (or rather
verbal since -- as (\mex{1}b) shows -- adjectival participles can be modified as
well) and the value of \textsc{initial} to be $-$. 
\eal
\ex
\gll dass es oft lacht\\
     that it often laughs\\\german
\glt `that he/she laughs often'
\ex 
\gll des oft lachende Kind\\
     the often laughing child\\
\glt `the child who laughs often'
\zl
The specification of the modified element to be \initial$-$ ensures that the adjunct attaches to
verbs in final position only (Verb initial sentences are discussed in
Chapter~\ref{chap-verb-position}). A linearization rule has to make sure that adverbials are
serialized to the left of the verb, that is, somewhere in the \mf. The \modv of English adverbials
is simply VP. Without any further restrictions, this allows for a pre- and a post-VP attachment of
adjuncts.
\begin{itemize}
\item SOV (Dutch, German, \ldots): \textsc{mod} V[\textsc{ini}$-$]
\item SVO (Danish, English, \ldots): \textsc{mod} VP
\end{itemize}

The analysis of (\ref{ex-m-j-b-l}) is shown in Figure~\vref{fig-m-j-b-l}, the one of
(\ref{ex-j-m-b-l}) in Figure~\vref{fig-j-m-b-l}, and the one of (\ref{ex-j-b-m-l}) in
Figure~\vref{fig-j-b-m-l}. The only difference between the figures is the respective place of
attachment of the adverb. I marked the parts of the tree that are licensed by the Head-Adjunct
Schema by including them into a box. All other nodes in the tree are licensed by the Head-Complement
Schema in Figure~\ref{fig-head-comp}.


\begin{figure}

\centerfit{
\begin{forest}
sm edges
[{V[\spr \eliste, \comps \eliste]}, schema
        [{Adv[\textsc{mod} \ibox{3} V]} [morgen;tomorrow] ]
        [{\ibox{3} V[\spr \eliste, \comps \eliste]}
          [{\ibox{1} NP[\type{nom}]} [Aicke;Aicke] ]
          [V\feattab{
              \spr \sliste{ }, \comps \sliste{ \ibox{1} } }
            [{\ibox{2} NP[\type{acc}]} [das Buch;the book, roof] ] 
            [V\feattab{
              \spr \sliste{  },\\
              \comps \sliste{ \ibox{1}, \ibox{2} }} [liest;reads] ] ]
] ]
\end{forest}}
\caption{\label{fig-m-j-b-l}Analysis of [\emph{dass}] \emph{morgen jeder das Buch liest} `that everybody will read the
  book tomorrow' with the adjunct attaching above subject and object}
\end{figure}


\begin{figure}
\centerfit{%
\begin{forest}
sm edges
[{V[\spr \eliste, \comps \eliste]},s sep+=1.5em
          [{\ibox{1} NP[\type{nom}]} [Aicke;Aicke] ]
          [V\feattab{
              \spr \sliste{ }, \comps \sliste{ \ibox{1} } }, schema
            [{Adv[\textsc{mod} \ibox{3} V]} [morgen;tomorrow] ]
            [\ibox{3} V\feattab{
                \spr \sliste{ }, \comps \sliste{ \ibox{1} } }
              [{\ibox{2} NP[\type{acc}]} [das Buch;the book, roof] ] 
              [V\feattab{
                \spr \sliste{  },\\
                \comps \sliste{ \ibox{1}, \ibox{2} }} [liest;reads] ] ]
] ]
\end{forest}}

\caption{\label{fig-j-m-b-l}Analysis of [\emph{dass}] \emph{jeder morgen das Buch liest} `that everybody will read the
  book tomorrow' with the adjunct attaching between subject and object}
\end{figure}


\begin{figure}
\centerfit{%
\begin{forest}
sm edges
[{V[\spr \eliste, \comps \eliste]}
    [{\ibox{1} NP[\type{nom}]} [Aicke;Aicke] ]
      [V\feattab{
         \spr \sliste{ }, \comps \sliste{ \ibox{1} } }, s sep+=1em
         [{\ibox{2} NP[\type{acc}]} [das Buch;the book, roof] ] 
           [V\feattab{
              \spr \sliste{  },\\
              \comps \sliste{ \ibox{1}, \ibox{2} }}, schema 
             [{Adv[\textsc{mod} \ibox{3} V]} [morgen;tomorrow] ]
             [\ibox{3} V\feattab{
                 \spr \sliste{  },\\
                 \comps \sliste{ \ibox{1}, \ibox{2} }} [liest;reads] ] ] ] ]
\end{forest}}
\caption{\label{fig-j-b-m-l}Analysis of [\emph{dass}] \emph{jeder das Buch morgen liest} `that everybody will read the
  book tomorrow' with the adjunct attaching between object and verb}
\end{figure}
The attentive reader will notice that there is a description following the \ibox{3} in the \modv of
the adverbials, while there is no such description in the \modvs of the English examples that
follow. Of course this is purely notational since the numbered boxes identify all values with the
same numbers, but the convention behind this is to state the description if it differs from what is
given at other places where the box occurs. In the case of German, the \modv of adverbials is just
\type{verb} without any restrictions regarding valence features. The valence features are given at
the modified node (\eg \spr \eliste, \comps \sliste{ \ibox{1}, \ibox{2} } in
Figure~\ref{fig-j-b-m-l}), but not in the \modv. Since English adverbials modify VPs and since the
modified node is a VP, the value of the \modv is not given in detail in the Figures below, but is just shared with the
properties of the modified node.


The Figures~\ref{fig-adj-vp} and~\ref{fig-vp-adj} show the analysis of adjunction with the adverb in
pre-VP and post-VP position respectively.
\begin{figure}
\centerfit{%
\begin{forest}
sm edges
[{V[\spr \eliste, \comps \eliste]}, s sep+=1.5em % puts more space between the NP[nom] and the VP,
                                    % otherwise the box would overlap
          [{\ibox{1} NP[\type{nom}]} [Kim] ]
          [V\feattab{
              \spr \sliste{ \ibox{1} }, \comps \sliste{  } }, schema
            [{Adv[\textsc{mod} \ibox{3}]} [often] ]
            [\ibox{3} V\feattab{
                \spr \sliste{ \ibox{1} }, \comps \sliste{  } }
              [V\feattab{
                \spr \sliste{ \ibox{1} },\\
                \comps \sliste{  \ibox{2} }} [reads] ]
              [{\ibox{2} NP[\type{acc}]} [books] ] ]
] ]
\end{forest}}
\caption{\label{fig-adj-vp}Analysis of adjuncts in SVO languages: the adjunct is realized left-adjacent to the VP.}
\end{figure}
\begin{figure}
\centerfit{%
\begin{forest}
sm edges
[{V[\spr \eliste, \comps \eliste]},s sep+=1em
          [{\ibox{1} NP[\type{nom}]} [Kim] ]
          [V\feattab{
              \spr \sliste{ \ibox{1} }, \comps \sliste{  } }, schema
            [\ibox{3} V\feattab{
                \spr \sliste{ \ibox{1} }, \comps \sliste{  } }
              [V\feattab{
                \spr \sliste{ \ibox{1} },\\
                \comps \sliste{ \ibox{2} }} [reads] ]
              [{\ibox{2} NP[\type{acc}]} [books] ] 
               ]
            [{Adv[\textsc{mod} \ibox{3}]} [often] ]
] ]
\end{forest}}
\caption{\label{fig-vp-adj}Analysis of adjuncts in SVO languages: the adjunct is realized right-adjacent to the VP.}
\end{figure}

The values of \spr and \comps in the schema in Figure~\ref{fig-head-adj} on
page~\pageref{fig-head-adj} have not been explained so far. First there is the sharing of the \spr
and \compsvs between mother and head-daughter. Whatever element an adjunct attaches to, the valence
requirements of the mother are always identical to the valence requirement of the
head-daughter. Nothing is added, nothing is missing. Adjuncts are additional elements that are not
selected for via valence features, hence nothing has to be discharged. This can be seen by looking
at the German examples in Figures~\ref{fig-m-j-b-l} to~\ref{fig-j-b-m-l}: \emph{morgen} `tomorrow'
attaches to a node with certain valence requirements and the dominating node has exactly the same
valence requirements. In principle the figures should have little numbered boxes in them indicating
the identity of the valence requirements of mother and head daughter in head-adjunct combinations. I
omitted these so-called structure sharings to keep things simple and readable. 

The adjunct itself has to have empty valence lists, that is, it has to be complete. Without this
requirement, sentences like the one in (\mex{1}) would be licensed:
\ea[*]{
Sandy read the book in.
}
\z
\emph{in} is a preposition that has an \npacc in its \compsl. If the Head-Adjunct Schema would not
specify the \compsl of the adjunct daughter to be empty, a preposition could function as the adjunct
daughter and a structure for ungrammatical sentence like (\mex{0}) would be licensed by the
grammar. 

The specifier specification is as important as the specification of the \compsl. If non-empty \sprls
were allowed, the contrast in (\mex{1}) could not be explained:

\eal
\ex[]{
\gll dass Aicke eine Stunde liest\\ 
     that Aicke an   hour reads\\\german
\glt `Aicke is reading for an hour.'
}
\ex[*]{
\gll dass Aicke Stunde liest\\ 
     that Aicke hour reads\\
%\glt `Aicke is reading for an hour.'
}
\zl
The analysis of (\mex{0}a) is shown in Figure~\vref{fig-dass-Aicke-eine-Stunde-liest}. The adjunct
is a full NP. The schema requires the adjunct daughter to be fully complete. If it did not have this
requirement, a noun without determiner like \emph{Stunde} `hour' in (\mex{0}b) could enter the
schema as adjunct daughter and ungrammatical sentences like (\mex{0}b) would be licensed.
\begin{figure}
\begin{forest}
sm edges
[{V[\comps \eliste]}, s sep+=.5em
  [\ibox{1} NP [Aicke;Acike]]
  [{V[\comps \sliste{ \ibox{1} }]} ,schema
    [{NP[\textsc{mod} \ibox{2} V]} [eine Stunde;one hour,roof]]
    [{\ibox{2} V[\comps \sliste{ \ibox{1} }]}  [liest;reads]]]]
\end{forest}
\caption{Analysis of an adverbial NP in \emph{dass Aicke eine Stunde liest} `that Aicke is reading
  for an hour'}\label{fig-dass-Aicke-eine-Stunde-liest}
\end{figure}

\section{Linking between syntax and semantics}
\label{sec-linking}


HPSG assumes that all arguments of a head are contained in a list that is called \textsc{argument
  structure} (\argst, \citealp*{WKD2021a}).\footnote{%
See \citealp[\page 28--29]{ps2}; \citealp{Wechsler95a-u};
\citealp{Davis2001a-u}; \citealp[Section~5.6]{MuellerLehrbuch1} for argument
linking in HPSG. \citew*{WKD2021a} is a handbook article on linking in HPSG.
} This list contains descriptions of the syntactic and semantic properties of
the selected arguments. For instance the \argstl of English \emph{give} and its German, Danish and
Dutch and Icelandic variants is given in (\mex{1}):
\ea
\sliste{ NP, NP, NP }
\z
The case systems of the involved languages vary a bit as will be explained in
Chapter~\ref{chap-case}, but nevertheless the orders of the NPs in the \argstl are the same across these
languages. They correspond to nom, dat, acc in German (\mex{1}a) and subject, primary object, secondary object
in English (\mex{1}b):
\eal
\ex 
\gll dass das Kind dem Eichhörnchen die Nuss gibt\\
    that the child  the squirrel    the nut gives\\
\glt `that the child gives the squirrel the nut'
\ex that the child gives the squirrel the nut
\zl
In addition to the syntactic features we have seen so far semantic features are used to describe the
semantic contribution of linguistic objects. (\mex{1}) shows some aspects of the description of the English verb
\emph{gives}:
\ea
lexical item for \emph{gives}:\\*
\ms{
arg-st & \sliste{ NP\ind{1}, NP\ind{2}, NP\ind{3} }\\[2mm]
cont   & \ms[give]{
          agens & \ibox{1}\\
          goal  & \ibox{2}\\
          trans-obj & \ibox{3}\\
        }\\
}
\z
The lowered boxes refer to the referential indices of the NPs. One can imagine these indices as
variables that refer to the object in the real world that the NP is referring to. These indices are
identified to semantic roles of the verb \emph{give}. Finding reasonable role names is not trivial
and some authors just use \argone, \argtwo and \argthree to avoid the problems (see \citealp{Dowty91a}
for discussion).

The representations for the other languages mentioned above is entirely parallel. Therefore it is
possible to capture crosslinguistic generalizations. Nevertheless there are differences between the
Germanic OV and VO languages. As was explained above the VO languages map their subject to \spr and
all other arguments to \comps, while the finite verbs of OV languages have all arguments on \comps. 
\inlinetodostefan{maybe add some examples}
%% \eal
%% \ex Linking and argument mapping in SVO languages:\\
%% \ms{
%% spr    & \sliste{ \ibox{1} }\\
%%        & 
%% arg-st & \sliste{ NP\ind{1}, NP\ind{2}, NP\ind{3} }\\
%% cont   & \ms[give]{
%%           agens & \ibox{1}\\
%%           goal  & \ibox{2}\\
%%           trans-obj & \ibox{3}\\
%%         }\\
%% }



\section{Alternatives}

\inlinetodostefan{Advanced stuff. Ignore if you do not dare.}

\subsection{CP/TP/VP models}
\label{sec-cp-tp-vp}
\label{sec-discussion-scope}

\citet{Grewendorf88a,Grewendorf93}, \citet{Lohnstein2014a} and many others assume that German has a
structure that is parallel to the one that is assumed for English. As for English the verb is
assumed to form a phrase with its objects and this VP functions as the argument of a Tense head to
form a maximal projection together with the subject of the verb, which is realized in the specifier
position of the TP. Figure~\ref{fig-cp-tp-vp} shows the analysis of (\mex{1}) with the respective
layers.
\ea
\gll dass jeder dieses Buch kennt\\
     that everbody this book knows\\
\glt `that everybody knows this book'
\z
\begin{figure}
\centering
\begin{forest}
sm edges
[CP
  [C$'$
    [C [dass;that]]
    [TP
      [NP [jeder;everybody,roof]]
      [T$'$
	[VP
	  [V$'$
	    [NP [dieses Buch;this book, roof]]
	    [V [\trace$_j$]]]]
	[T [kenn-$_j$ -t;know- -s]]]]]]
\end{forest}
\caption{\label{fig-cp-tp-vp}Sentence in the CP/TP/VP model}
\end{figure}%

The problem with such proposals is that the subject does not depend on the verb but on T. Therefore
there is no way of serializing the accusative object before the subject unless one assumes that the
object is moved to a higher position in the tree, \eg adjoined to TP as in Figure~\ref{fig-cp-tp-vp-scrambling}.
\begin{figure}
\centering
\begin{forest}
sm edges
[CP
[C$'$
	[C [dass;that]]
        [TP
          [NP$_i$ [dieses Buch;this book, roof]]
	  [TP
	    [NP [jeder;everybody,roof]]
	    [T$'$
	      [VP
		[V$'$
		  [NP [\trace$_i$]]
		  [V [\trace$_j$]]]]
	      [T [kenn-$_j$ -t;know- -s]]]]]]]
\end{forest}
\caption{\label{fig-cp-tp-vp-scrambling}Scrambling has to be movement in the CP/TP/VP model}
\end{figure}%

While researchers like \citet[\page 185]{Frey93a} argued that quantifier scopings are actually
evidence for movement-based approaches, the argument backfires. Let us consider Frey's examples. Frey
argues that sentences without movement have only one reading and sentences like (\mex{1}b) in which
-- according to the movement-based theory -- movement is involved have two readings: one corresponding to
the visible order and one to the order before movement, the so-called underlying order. 
\eal
\ex 
\gll Es ist nicht der Fall, daß er mindestens einem Verleger fast jedes Gedicht anbot.\\
     it is not the case that he at.least one publisher almost every poem offered\\
\glt `It is not the case that he offered at least one publisher almost every poem.'
\ex 
\gll Es ist nicht der Fall, daß er fast jedes Gedicht$_i$ mindestens einem Verleger \_$_i$ anbot.\\
	 it is not the case that he almost every poem at.least one publisher {} offered\\
\glt `It is not the case that he offered almost every poem to at least one publisher.'
\zl

\noindent
However, \citet[\page 146]{Kiss2001a} and \citet[Section~2.6]{Fanselow2001a} pointed out that such
approaches have problems with multiple moved constituents. For instance in an example such as
(\mex{1}), it should be possible to interpret \emph{mindestens einem Verleger} `at least one
publisher' at the position of \_$_i$, which would lead to a reading where \emph{fast jedes Gedicht}
`almost every poem' has scope over \emph{mindestens einem Verleger} `at least one
publisher'. However, this reading does not exist.


\ea
\gll Ich glaube, dass mindestens einem Verleger$_i$ fast jedes Gedicht$_j$ nur dieser Dichter \_$_i$ \_$_j$ angeboten hat.\\
     I believe that at.least one publisher almost every poem only this poet {} {} offered has\\
\glt `I think that only this poet offered almost every poem to at least one publisher.'
\z

This means that one needs some way to determine the deviation with respect to an unmarked order, but
movement is not the solution. See \citew[Section~3.5]{MuellerGT-Eng4} for further discussion and
\citet{Kiss2001a} for an approach to scope within the framework assumed here.

\exercises{


\begin{enumerate}
\item Provide the valence lists for the following words:

\eal
\ex laugh
\ex eat
\ex to douse
\ex 
\gll bezichtigen\\
     accuse\\\german
\ex he
\ex the
\ex 
\gll Ankunft\\
     arrival\\\german
\zl
If you are uncertain as far as case assignment is concerned, you may use the
  Wiktionary\footnote{
\url{https://de.wiktionary.org/}, 2018-07-02.
}.

\item Draw trees for the NPs that were also used in exercise~\ref{exercise-NP-PSG} on page~\pageref{exercise-NP-PSG} in Chapter~\ref{chap-psg}.
\eal
\ex 
\gll eine Stunde vor der Ankunft des Zuges\\
     one  hour   before the arrival of.the train\\
\glt `one  hour   before the arrival of the train'
\ex 
\gll kurz    nach  der Ankunft in Paris\\
     shortly after the arrival in Paris\\
\glt `shortly after the arrival in Paris'
\ex
\gll das ein Lied singende Kind aus dem Allgäu\\
     this a song  singing child from the Allgäu\\
\glt `the child from the Allgäu singing a song'
\zl


\item Draw trees for the following examples. NPs can be abbreviated.
\eal
\ex 
\gll weil    Aicke dem        Kind  ein      Buch schenkt\\
     because Aicke the.\DAT{} child a.\ACC{} book gives.as.a.present\\ \german
\glt `because Aicke gives the child a book as a present'
\ex because Kim gave a book to him
\ex Sandy saw this yesterday.
\ex
\gll at Bjarne læste bogen\\
     that Bjarne read book.\textsc{def}\\
\glt `that Bjarne read the book'
\zl
\end{enumerate}

}




%      <!-- Local IspellDict: en_US-w_accents -->
