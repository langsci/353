%% -*- coding:utf-8 -*-
\chapter{Verb position: Verb first and verb second}
\label{chap-verb-position}

This chapter deals with the analysis of the verb position in V2 langauges. I will concentrate on
Danish and German, which may serve as prototypical examples: Danish is an SVO language, while German
is SOV. I will first discuss arguments for the classification of German as an SOV language and
provide the necessary data on Danish and then explain the respective analyses.

\section{The phenomenon}

Section~\ref{sec-intro-svo} contains a discussion of the basic order of subject, object and verb in the languages
of the world and in the Germanic languages in particular. I discussed the classification provided by
the World Atlas of Language Structures, which suggested that German is a language with two dominant
constituent orders: SOV and SVO. Claiming that SVO is a basic order on the basis of pure counting is somehow strange given the
fact that most German sentences do not have the subject in first position anyway. The following text
may serve as an example:
\eanoraggedright
\gruenbf{Für selbstfahrende Autos} soll es in Deutschland nach Angaben von Bundes\-ver\-kehrs\-mi\-nis\-ter
Alexander Dobrindt (CSU) bald eine Teststrecke geben. \gruenbf{Auf der Autobahn A9 in Bayern} sei
ein Pilotprojekt „Digitales TestfeldAutobahn“ geplant, wie aus einem Papier des
Bundesverkehrsministeriums hervorgeht. \gruenbf{Mit den ersten Maßnahmen für diese Teststrecke}
solle schon in diesem Jahr begonnen werden. \gruenbf{Mit dem Projekt} soll die Effizienz von
Autobahnen generell gesteigert werden. \gruenbf{„\rotit{Die Teststrecke} soll so digitalisiert und
  technisch ausgerüstet werden, dass es dort zusätzliche Angebote der Kommunikation zwischen Straße
  und Fahrzeug wie auch von Fahrzeug zu Fahrzeug geben wird“}, sagte Dobrindt zur Frankfurter
Allgemeinen Zeitung.  \gruenbf{Auf der A9} sollten sowohl Autos mit Assistenzsystemen als auch
später vollautomatisierte Fahrzeuge fahren können. \gruenbf{Dort} soll die Kommunikation nicht nur
zwischen Testfahrzeugen, sondern auch zwi\-schen Sensoren an der Straße und den Autos möglich sein,
etwa zur Übermittlung von Daten zur Verkehrslage oder zum Wetter. \gruenbf{\rotit{Das Vorhaben}
  solle im Verkehrsministerium von einem runden Tisch mit Forschern und Industrievertretern
  begleitet werden,} sagte Dobrindt. \rotit{Dieser} solle sich unter anderem auch mit den
komplizierten Haftungsfragen beschäftigen.  Also: \rotit{Wer} zahlt eigentlich, wenn ein
automatisiertes Auto einen Unfall baut?  \gruenbf{[\gruenbf{Mithilfe der Teststrecke}] solle die
  deutsche Automobilindustrie auch beim digitalen Auto „Weltspitze sein können“,} sagte der
CSU-Minister. \rotit{Die deut\-schen Hersteller} sollten die Entwicklung nicht Konzernen wie etwa
Google überlassen.  \gruenbf{Derzeit} ist Deutschland noch an das „Wiener Über\-ein\-kom\-men für den
Straßenverkehr“ gebunden, das Autofahren ohne Fahrer nicht zu lässt. \gruenbf{Nur unter besonderen
  Auflagen} sind Tests möglich.  \rotit{Die Grünen} halten die Pläne für
unnütz. \rotit{Grünen-Verkehrsexpertin Valerie Wilms} sagte der Saarbrücker Zeitung: „\rotit{Der
  Minister} hat wichtigere Dinge zu erledigen, als sich mit selbstfahrenden Autos zu beschäftigen.“
\rotit{Die Technologie} sei im Verkehrsbereich nicht vordringlich, auch stehe sie noch ganz am
Anfang.  \gruenbf{Aus dem grün-rot regierten Baden-Württemberg – mit dem Konzernsitz von Daimler –}
kamen hingegen andere Töne. \gruenbf{\rotit{Was in Bayern funktioniere,} müsse auch in
  Baden-Württemberg möglich sein,} sagte Wirtschaftsminister Nils Schmid (SPD). \gruenbf{Von den
  topografischen Gegebenheiten} biete sich die Autobahn A81 an. (taz: 27.01.2015)
\z
The subjects are marked in red and the non-subjects in green. I also counted subjects/non-subjects
within clauses. The ratio is 10 subjects compared to 15 non-subjects. So, the question is: What does
this number tell us? Of course we could now further differentiate the grammatical functions of the
fronted material. We would find that we have 3 object clauses fronted, the rest of the fronted
constituents is adverbials. We could conclude the SVO is more common than OVS, but saying that SVO
is basic would not be appropriate. Rather Adv V S O should be regarded as a basic pattern. But would
this be helpful in any way? I guess not. The general insight is that German fronts the finite verb
to mark the sentence type and puts one constituent infornt of this verb. This fronted constituent
can be the subject, an object or any other constituent of the sentence. It may be even a dependent
of a deeply embedded element in the clause. So, the position infront of V in the V2 languages has
nothing to do with the SVO/SOV dichotomy and basically disturbs the picture. 

In the following I will provide facts that are seen as evidence for SOV as the basic order of German
(and Dutch). Before I provide an analysis in Section~\ref{sec-analysis-verb-mevement}, I discuss the
verb position in the germanic SVO languages with Danish as an example in Section~\ref{sec-danish-verb-movement}.

\subsection{German as an SOV language}

\subsubsection{The order of particle and verb and idioms}

Verb particles\is{verb!particle} form a close unit with the verb. The unit is observable in verb final sentences only,
which supports an SOV analysis \citep[\page 35]{Bierwisch63a}. 
\eal
\ex 
\gll weil er morgen anfängt\\
     because he tomorrow at.catches\\
\glt `because he starts tomorrow'
\ex 
\gll Er fängt morgen an.\\
     he catches tomorrow at\\
\glt `He starts tomorrow.'
\zl

\noindent
The particle verb in (\mex{0}) is non-transparent: its meaning is not related to the verb
\emph{fangen} `to catch'. Such particle verbs are sometimes called mini
idioms. In fact the argument above can also be made with real idioms: Many idioms do not allow
rearrangement of the idiom\is{idiom} parts:
\eal
\judgewidth{?*}
\ex[]{
\gll dass niemand dem Mann den Garaus macht\\
     that nobody  the man  the \textsc{garaus} makes\\
\glt `that nobody kills the man'
}
\ex[?*]{
\gll dass dem Mann den Garaus niemand macht\\
     that the man  the \textsc{garaus} nobody makes\\
}
\ex[]{
\gll Niemand macht ihm den Garaus.\\
     nobody makes him the \textsc{garaus}\\
\glt `Nobody kills him.'
}
\zl
This is an instance of Behaghel's law \citeyearpar{Behaghel32-u}
that things that belong together semantically tend to be realized together. The exception is the
finite verb. The finite verb can be realized in initial or final position despite the fact that this
interrupts the continuity of the idiomatic material. Since the continuity can be observed in SOV
order only, this order is considered basic.

\subsubsection{Verbs formed by back-formation}

Verbs that are derived from nouns by backformation\is{backformation} often cannot be separated and verb second
sentences therefore are excluded (see \citealt[\page 62]{Haider93a}, who refers to unpublished work by \citealt{Hoehle91b}):
\eal
\ex[]{
\gll weil sie das Stück heute uraufführen\\
     because they the play today play.for.the.first.time\\
\glt `because they premiered the play today'
}
\ex[*]{
\gll Sie uraufführen heute das Stück.\\
     they play.for.the.first.time  today the play\\
}
\ex[*]{
\gll Sie führen heute das Stück urauf.\\
     they guide today the play  \textsc{prefix}.\textsc{part}\\
}
\zl
Hence these verbs can only be used in the order that is assumed to be the base order.

\subsubsection{Constructions that only allow SOV order}

Similarly, it is sometimes impossible to realize the verb in initial position when elements like
\emph{mehr als} `more than' are present in the clause \citep{Haider97c,Meinunger2001a}: 
\eal
\ex[]{
\gll dass Hans seinen Profit letztes Jahr mehr als verdreifachte\\
     that Hans his         profit last       year more than tripled\\
\glt `that Hans increased his profit last year by a factor greater than three'
}
\ex[]{
\gll Hans hat seinen Profit letztes Jahr mehr als verdreifacht.\\
     Hans has his    profit last    year more than tripled\\
\glt `Hans increased his profit last year by a factor greater than three.'
}
\ex[*]{
\gll Hans verdreifachte seinen Profit letztes Jahr mehr als.\\
     Hans tripled       his    profit last year more than\\
}
\zl
So, it is possible to realize the adjunct together with the verb in final position, but there are
constraints regarding the placement of the finite verb in initial position.


\subsubsection{Order in subordinate and non-finite clauses}

Verbs in non-finite clauses and in subordinate finite clauses starting with a conjunction
  always appear finally, that is, in the \emph{rechte Satzklammer}. For example, \emph{zu geben} `to
  give' and \emph{gibt} `gives' appear in the \emph{rechte Satzklammer} in (\mex{1}a) and (\mex{1}b):
\eal
\ex 
\gll Der Clown versucht, Kurt-Martin die Ware zu geben.\\
     the clown tries     Kurt-Martin the goods to give\\
\glt `The clown tries to give Kurt-Martin the goods.'
\ex 
\gll dass der Clown Kurt-Martin die Ware gibt\\
     that the clown Kurt-Martin the goods gives\\
\glt `that the clown gives Kurt-Martin the goods'
\zl



\subsubsection{Scope of adverbials}

The scope of adverbials in sentences like (\ref{bsp-absichtlich-nicht-anal}) depends on their order \citep[Section~2.3]{Netter92}:
The left-most adverb scopes over the following adverb and over the verb in final
position. This was explained by assuming the following structure:
\eal
\label{bsp-absichtlich-nicht-anal}
\ex 
\gll weil er  [absichtlich [nicht lacht ]]\\
     because he \hspaceThis{[}deliberately \hspaceThis{[}not laughs\\
\glt `because he deliberately does not laugh'
\ex 
\gll weil er [nicht [absichtlich lacht]]\\
     because he \hspaceThis{[}not \hspaceThis{[}deliberately laughs\\
\glt `because he does not laugh deliberately'
\zl
An interesting fact is that the scope relations do not change when the verb position is changed. If
one assumes that the sentences have an underlying structure like in (\mex{0}), this fact is
explained automatically:
\eal
\label{bsp-absichtlich-nicht-anal-v1}
\ex 
\gll Lacht$_i$ er [absichtlich [nicht \_$_i$]]?\\
     laughs he \hspaceThis{[}deliberately \hspaceThis{[}not\\
\glt `Does he deliberately not laugh?'
\ex 
\gll Lacht$_i$ er [nicht [absichtlich \_$_i$]]?\\
     laughs he \hspaceThis{[}not \hspaceThis{[}deliberately\\
\glt `Doesn't he laugh deliberately?'
\zl
%\item Verum-Fokus
\nocite{Hoehle88a,Hoehle97a}


It has to be mentioned here, that there seem to be exceptions to the claim that modifiers scope from
left to right. \citet*[\page47]{Kasper94a} discusses the examples in (\mex{1}), which go back to \citet*[\page137]{BV72}.
\eal
\label{bsp-peter-liest-gut-wegen}
\ex 
\gll Peter liest wegen der Nachhilfestunden gut.\\
     Peter reads because.of the tutoring well\\
\glt `Peter reads well because of the tutoring.'
\ex 
\gll Peter liest gut wegen der Nachhilfestunden.\\
     Peter reads well because.of the tutoring\\
\zl
(\mex{0}a) corresponds to the expected order in which the adverbial PP \emph{wegen der
  Nachhilfestunden} outscopes the adverb \emph{gut}, but the alternative order in (\mex{0}b) is
possible as well and the sentence has the same reading as the one in (\mex{0}a).

% Kiss95b:212
  However, \citet[Section~6]{Koster75a} and \citet*[\page67]{Reis80a} showed that these examples
  are not convincing evidence since the \emph{rechte Satzklammer} is not filled and therefore the
  orders in (\mex{0}) are not necessarily variants of \emph{Mittelfeld} orders but may be due to extraposition of
  one constituent. As Koster and Reis showed, the examples become ungrammatical when the right sentence
  bracket is filled:
\eal
\ex[*]{
\gll Hans hat gut wegen der Nachhilfestunden gelesen.\\
     Hans has well because.of the tutoring read\\
}
\ex[]{
\gll Hans hat gut gelesen wegen der Nachhilfestunden.\\
     Hans has well read   because.of the tutoring\\
\glt `Peter read well because of the tutoring.'
}
\zl
The conclusion is that (\mex{-1}b) is best treated as a variant of (\mex{-1}a) in which the PP is extraposed.

While examples like (\mex{-1}) show that the matter is not trivial, the following example from \citet[\page
383]{Crysmann2004a} shows that there are examples with a filled \emph{rechte Satzklammer} that allow
for scopings in which an adjunct scopes over another adjunct that precedes it. For instance, in
(\mex{1}) \emph{niemals} `never' scopes over \emph{wegen schlechten Wetters} `because of the bad weather':
\ea
\gll Da muß es schon erhebliche Probleme mit der Ausrüstung gegeben haben, da [wegen
  schlechten  Wetters] ein Reinhold Messner [niemals] aufgäbe.\\
     there must it \textsc{part} severe problems with the equipment given have since \hspaceThis{[}because.of bad weather a Reinhold Messner \hspaceThis{[}never give.up.would\\
\glt `There must have been severe problems with the equipment, since someone like Reinhold Messner
would never give up just because of the bad weather.'
%\ex Stefan  ist wohl deshalb krank geworden, weil er äußerst hart wegen der Konferenz in Bremen gearbeitet hat.
\z

However, this does not change the fact that the sentences in (\ref{bsp-absichtlich-nicht-anal}) and
(\ref{bsp-absichtlich-nicht-anal-v1}) have the same meaning independent of the position of the
verb. The general meaning composition may be done in the way that Crysmann suggested.%
%

Another word of caution is in order here: There are SVO languages like French that also have a left
to right scoping of adjuncts \citep[\page 156--161]{BGK2004a-u}. So, the argumentation above should not be seen as the only
fact supporting the SOV status of German. In any case the analyses of German that were
worked out in various frameworks can explain the facts nicely.



\subsubsection{Position of non-finite verbs in VO and OV languages}

Before I turn to the verb position in Danish in the next subsection, I want to repeat Ørsnes'
examples containing several non-finite verbs: The example in (\mex{1}a) shows a German subordinate clause with a verbal complex consisting of
three verbs. The level of embedding is indicated by subscript numbers. As can be seen, the verbs are
added at the end of the clause. In the corresponding Danish example it is exactly the other way
around: the embedding verb preceeds the embedded verb.
\eal
\ex
\gll dass er ihn gesehen$_3$ haben$_2$ muss$_1$\\
     that he him seen        have      must\\\jambox{(German)}
\glt `that he must have seen him'
\ex
\gll at han må$_1$ have$_2$ set$_3$ ham\\
     that he must have seen him\\\jambox{(Danish)}
\zl
%
The examples in (\mex{1}) shows variants with different complexity. If we exchange the simplex verb
\emph{sah} `saw' in (\mex{1}a) by the perfect form, the auxiliary is placed after the participle as
in (\mex{1}b).
\eal
\ex
\gll dass er ihn sah\\
     that he him saw\\\jambox{(German)}
\glt `that he saw him'
\ex
\gll dass er ihn gesehen hat\\
     that he him seen    has\\
\glt `that he has seen him'
\zl 
If a modal is added to (\mex{0}b), the modal goes to the right of the embedded verbs. This order is
distorted by the placement of the finite verb in initial position, but this placement is independent
of the order of the non-finite verbs. As the examples in (\mex{1}) show, the finite verb is realized
to the left of the subject both in German (SOV) and in Danish (SVO).

\eal
\ex 
\gll Muss er ihn gesehen haben?\\
     must he him seen have\\\jambox{(German)}
\glt `Must he have seen him?'
\ex 
\gll Må han have set ham?\\
     must he have seen him\\\jambox{(Danish)}
\glt `Must he have seen him?'
\zl





\subsection{Verb position in the germanic SVO languages}
\label{sec-danish-verb-movement}

During the discussion of scope facts I already hinted at an analysis in which a trace marks the
position of the verb in final position and the verb in initial position is coindexed with this
trace. Although the SVO languages are different a similar analysis has been suggested for languages
like Danish. The evidence for this is that adverbials in SVO languages usually attach to the VP,
that is they combine with a phrase consisting of the verb and its object or objects. (\mex{1}) gives
an example:
\ea
\gll  at   Jens ikke [\sub{VP} læser bogen]\\
      that Jens not      {}        reads          book.\textsc{def}\\\jambox{(Danish)}
\glt `that Jens does not read the book'
\z

The interesting thing now is that the finite verb is placed to the left of the negation in V2 sentences:

\ea
\gll  Jens læser ikke bogen.\\
       Jens reads   not  book.\textsc{def}\\\jambox{(Danish)}
\glt `Jens is not reading the book.'
\z
This is seen as evidence for verb fronting by many:
\ea
\gll  Jens læser$_i$ ikke [\sub{VP} \_$_i$ bogen].\\
      Jens reads      not  {} {}    book.\textsc{def}\\
\glt `Jens does not read the book.'
\z
\nocite{KS2002a}

With this as a background it should be clear what the analysis of yes/no questions as the one in (\mex{1}b) is:
\eal
\ex
\gll at Jens læser bogen\\
     that Jens reads book.\textsc{def}\\
\glt `that Jens reads the book'
\ex\label{ex-laeser-jens-bogen}
\gll Læser Jens bogen?\\
     reads Jens book.\textsc{def}\\
\glt `Does Jens read the book?'
\zl
The analysis of the first sentence involves a VP as in (\mex{1}a) and the second sentence involves a
VP with a verbal trace that corresponds to the verb in initial position:
\eal
\ex
\gll at Jens [\sub{VP} læser bogen]\\
     that Jens {} reads book.\textsc{def}\\
\glt `that Jens reads the book'

\ex
\gll Læser$_i$ Jens [\sub{VP} \_$_i$ bogen]?\\
     reads     Jens {}        {}     book.\textsc{def}\\
\glt `Does Jens read the book?'
\zl

It is interesting to note that the German and the Danish question with simplex verbs have exactly
the same constituent order. Compare (\ref{ex-laeser-jens-bogen}) with (\mex{1}):
\ea
\gll Liest Jens das Buch?\\
     reads Jens the book\\ \jambox{(German)}
\glt `Does Jens read the book?'
\z
The internal structure of these sentences is quite different though. The different nature of the two
langauges is of course more obvious when non-finite verbs are involved:
\eal
\ex
\gll Har$_i$ Jens [ \_$_i$ læst bogen]?\\
     has Jens {} {} read book.\textsc{def}\\\jambox{(Danish)}
\glt `Has Jens read the book?'
\ex
\gll Hat$_i$ Jens das Buch [gelesen \_$_i$]?\\
     has Jens the book \spacebr{}read\\ \jambox{(German)}
\glt `Has Jens read the book?'
\zl
In (\mex{0}a) the finite verb is connected to a trace in initial position of the VP and in
(\mex{0}b) it is connected to a verb in final position in a verbal complex.


\subsection{Verb second}

Even languages with rather rigid constituent order sometimes allow to front elements or to position
elements at the far right, that is, extrapose them. (\mex{1}) shows English examples of fronting:
\eal
\ex I read this book yesterday.
\ex This book, I read yesterday.
\ex Yesterday, I read this book.
\zl
The object \emph{this book} and the adjunct \emph{yesterday} are fronted in (\mex{0}b) and
(\mex{0}c), respectively.

The Germanic languages (with the exception of English) place one constituent in front of the finite
verb. As the German examples in (\mex{1}) show, the fronted constituent can be of any grammatical function:
\eal
\ex 
\gll Ich habe das Buch gestern gelesen.\\
     I have the book yesterday read\\\jambox{(German)}
\glt `I have read the book yesterday.'
\ex 
\gll Das Buch habe ich gestern gelesen.\\
     the book have I yesterday read\\
\ex 
\gll Gestern habe ich das Buch gelesen.\\
     yesterday have I the book read\\
\ex 
\gll Gelesen habe ich das Buch gestern, gekauft hatte ich es aber schon vor einem Monat.\\
     read have I the book yesterday bought had I it but yet before a month\\
\glt `I read the book yesterday, but I bought it last month already.'
\ex 
\gll Das Buch gelesen habe ich gestern.\\
     the book read    have I yesterday\\
\zl
Such frontings are not clause-bound, that is the fronting may cross one or several clause boundaries
and also boundaries of other constituents. (\mex{1}) shows English examples in which the object of
\emph{saw} is extracted across one and two clause boundaries:
\eal
\ex\label{ex-chris-we-saw} Chris, David saw.
\ex\label{ex-chris-we-think-that-david-saw} Chris, we think that David saw.
\ex Chris, we think Anna claims that David saw.
\zl
In German such extractions can be found as well:
\eal
\label{ex-fernabhaengigkeit-one}
\ex
\label{ex-wen-glaubst-du-dass}
\gll Wen$_i$ glaubst du, daß ich \_$_i$ gesehen habe.\footnotemark\\
     who believes you that I {} seen have\\\jambox{(German)}
\footnotetext{
    \citew[\page84]{Scherpenisse86a}.
    }
\ex "`Wer$_i$, glaubt\iw{glauben} er, daß er \_$_i$ ist?"' erregte sich ein Politiker vom Nil.\footnote{
        Spiegel, 8/1999, S.\,18.
}
\zl
It is generally said that they are more common in Southern German variaties, but there are other
examples that show that nonlocal dependencies are involved. In (\ref{bsp-um-zwei-millionen}) the
prepositional object \emph{um zwei Millionen Mark} `around two million Deutsche Marks' depends on
\emph{betrügen} `to cheat'. It does not
depend on any of the verbs in the matrix clause. The phrase \emph{eine Versicherung zu betrügen} `an
insurance to betray' is extraposed that is it is positioned to the right of the verbal braket in the so-called \nf. The
position of \emph{um zwei Millionen Mark} cannot be accounted for by local reordering. Similarly,
\emph{gegen ihn} `against him' depends on \emph{Angriffe} `attacs', which is part of the phrase \emph{Angriffe zu
  lancieren} `attacs to launch'. Again an analysis based on local reordering of dependents of a head is impossible.
\eal
\label{bsp-Fernabhaengigkeit}
\ex\label{bsp-um-zwei-millionen}
\gll {}[Um zwei Millionen Mark]$_i$ soll er versucht haben, [eine Versicherung \_$_i$ zu betrügen].\footnotemark\\
       \spacebr{}around two million Deutsche.Marks should he tried have \spacebr{}an insurance.company {} to deceive\\
\footnotetext{
         taz, 04.05.2001, p.\,20.
}
\glt `He apparently tried to cheat an insurance company out of two million Deutsche Marks.'
\ex
\gll {}[Gegen ihn]$_i$ falle es den Republikanern hingegen schwerer, [~[~Angriffe \_$_i$] zu lancieren].\footnotemark\\
	 {}\spacebr{}against him fall it the Republicans however more.difficult \hspaceThis{[~[~}attacks {} to launch\\
\footnotetext{
  taz, 08.02.2008, p.\,9.
}
\glt `It is, however, more difficult for the Republicans to launch attacks against him.'
\zl



\section{The analysis}
\label{sec-analysis-verb-mevement}


\subsection{Verb first}

The analysis uses a mechanism that passes up information in a tree. The verbal trace contains the
information that a verb is missing locally. This information about the missing verb is passed up to
the node that dominates the verbal trace. It is represented using $\!/\!/$ (read double slash). The
respective information is head-information and therefore it is passed up the head-path along with
other information as for instance part of speech. Figure~\vref{fig-analysis-German-verb-first}
illustrates. The verbal trace is missing a V, the \vbar is missing a V and the S as well.
\begin{figure}
\centering
\begin{forest}
sm edges
[S
  [{V \sliste{ S$/\!/$V }} 
    [V [liest$_k$;reads] ] ]
       [{S$/\!/$V}
           [NP [Jens;Jens] ]
           [{V$'$$\!/\!/$V}
             [NP [das Buch;the book, roof] ]
             [{\mybox[v1]{V}$\!/\!/$\mybox[v2]{V}} [\_$_k$] ] ] ] ] ]
%\draw[semithick,<->,color=green] (v1.south)--(v2.south);
%% \draw[semithick,<->,color=green] (3.1,-3.9) ..controls +(south east:.5) and +(south west:.5)..(2.7,-3.9);
%% \draw[semithick,<->,color=green] (3.5,-3.7) ..controls +(east:.5) and +(east:.5)..(2.8,-2.5);
%% \draw[semithick,<->,color=green] (2.8,-2.3) ..controls +(east:.5) and +(east:.5)..(1.7,-1.1);
%% \draw[semithick,<->,color=green] (1.5,-0.9) ..controls +(north:.5) and +(north:.5)..(-0.8,-0.9);
%% \draw[semithick,<->,color=green] (-0.7,-1.1) ..controls +(south east:.2) and +(north
       %% east:.5)..(-1.0,-2.4);
\end{forest}
\caption{\label{fig-analysis-German-verb-first}\label{fig-liest-jens-das-buch}Analysis of verb position in German}
\end{figure}
The initial verb selects for a sentence that is lacking a V \sliste{ S$/\!/$V }. The lexical item for
the verb in initial position is licensed by a lexical rule that relates a verb to a verb that
selects for a sentence that is lacking the input verb. Since the selectional requirement of this
verb (S$/\!/$V) is identified with the sentence lacking a V (Jens das Buch \_$_k$), the information
about the original verb \emph{liest} is identified with the V in S$/\!/$V. Since the double slash
information is head information, it percolates down along the head path to the verbal trace. The
information about the initial V is identified with the syntactic and semantic information of the
verbal trace in final position and hence this verbal trace behaves exactly like the verb in inital
position that was input to the lexical rule.

Various researchers argued that the finite verb in initial position behaves like a complementizer in
subordinated clauses \citep{Hoehle97a,Weiss2005a-u,Weiss2018a-u}. This is captured by the analysis. Compare
Figure~\ref{fig-analysis-German-verb-first} with Figure~\vref{fig-analysis-German-verb-last}.
\begin{figure}
\centering
\begin{forest}
sm edges
[CP
  [{C \sliste{ S }} [dass;that] ]
  [S
           [NP [Jens;Jens] ]
           [V$'$
             [NP [das Buch;the book, roof] ]
             [V [liest;reads] ] ] ] ]
\end{forest}
\caption{\label{fig-analysis-German-verb-last}Analysis of a verb final clause with complementizer in German}
\end{figure}
The complementizer \emph{dass} `that' selects for a complete sentence, that is, a sentence that does
not have a missing verb, and the initial verb \emph{liest} `reads' in Figure~\ref{fig-analysis-German-verb-first} selects
for a sentence that is missing \emph{liest}. So apart from the overt or covert verb the structures
are identical. This fact is important when it comes to the analysis of the scope facts. 
\begin{figure}
\centering
\begin{forest}
sm edges
[S
        [{V \sliste{ S$/\!/$V }} 
          [V [lacht$_k$;laughs] ] ]
        [{S$/\!/$V}
           [NP [er;he] ]
           [{V$'$$\!/\!/$V}
             [Adv [nicht;not] ]
             [{V$'$$\!/\!/$V}
               [Adv [absichtlich;deliberately] ]
               [{V$\!/\!/$V} [\_$_k$] ] ] ] ] ]
\end{forest}
\caption{\label{fig-analysis-German-verb-initial-scope}Analysis of sentences with adverbials in German}
\end{figure}
Since the structure is completely parallel to the one we have in verb final sentences, the scope
facts follow immediately: The trace behaves like the verb in initial position, \emph{absichtlich}
`deliberately' modifies the trace and the resulting semantics is passed up in the tree (see Figure~\vref{fig-analysis-German-verb-initial-scope}). The next
step is the modification by \emph{nicht} `not'. Again the resulting semantics is passed
up. \emph{lacht} `laughs' combines with the clause and takes its semantics over. Since \emph{lacht}
is the head the semantics is passed on from there.

The analysis of Danish is completely parallel to the one of German. The only difference between
Figure~\ref{fig-analysis-German-verb-first} and Figure~\vref{fig-analysis-verb-first-Danish} is the
position of the verbal trace relative to the object: The trace follows the object in German and
it preceeds it in Danish.\todostefan{Das \textsc{def} fehlt noch in der Abbildung.}
\begin{figure}
\centering
\begin{forest}
sm edges
[S
        [{V \sliste{ S$/\!/$V }} 
          [V [læser$_k$;reads] ] ]
        [{S$/\!/$V}
           [NP [Jens;Jens] ]
           [{VP$\!/\!/$V}
             [{V$\!/\!/$V} [\_$_k$] ] 
             [NP [bogen;book.\textsc{def}] ] ] ] ]
\end{forest}
\caption{\label{fig-analysis-verb-first-Danish}Analysis of verb position in Danish}
\end{figure}%


The last thing that is explained in this chapter is the analysis of negation and verb fronting in
Danish. Figure~\vref{fig-analysis-verb-fronting-negation-Danish} shows that the negation attaches to
the VP as in verb final clauses and the verb is fronted so that it appears to the left of the negation.
\begin{figure}
\centering
\begin{forest}
sm edges
[S
        [{V \sliste{ S$/\!/$V }} 
          [V [læser$_k$;reads] ] ]
        [{S$/\!/$V}
           [NP [Jens;Jens] ]
           [{VP$\!/\!/$V}
             [Adv [ikke;not] ]
             [{VP$\!/\!/$V}
               [{V$\!/\!/$V} [\_$_k$] ] 
               [NP [bogen;book.def] ] ] ] ] ]
\end{forest}
\caption{\label{fig-analysis-verb-fronting-negation-Danish}The analysis of verb fronting and
  negation in Danish}
\end{figure}
The next chapter explains the extraction of constituents and it will then be possible to provide the
full structure for sentences like (\mex{1}a) and it will become clear why the order
of negation and verb differs in embedded and main clauses:
\eal
\ex 
\gll Jens læser ikke bogen.\\
     Jens reads not  book.\defsc\\
\glt `Jens does not read the book.'
\ex 
\gll at Jens ikke læser bogen\\
     that Jens not reads book.\defsc\\
\glt `that Jens does not read the book'
\zl 

\subsection{Verb second}



The technique that is used for the analysis of nonlocal dependencies is the same that was employed
for the analysis of the reorderings of verbs: an empty element takes the position of the fronted
constituent and the information about the missing constituent (the so-called gap) is passed up in
the tree until it is finally bound off by the fronted element, the so-called
filler. Figure~\vref{fig-chris-david-saw} illustrates the analysis of (\ref{ex-chris-we-saw}).

\begin{figure}
\begin{forest}
sm edges
[S
  [NP [Chris] ]
  [S/NP 
    [NP [David] ] 
    [VP/NP  
      [V [saw] ]
      [NP/NP [\trace] ] ] ] ]
%% \draw[connect] (NP/NP.north east) [bend right] to (VP/NP.south east);
%% \draw[connect] (VP/NP.north east) [bend right] to (S/NP.south east);
%% \draw[connect] (S/NP.north east) [bend right] to (NP);
\end{forest}
\caption{\label{fig-chris-david-saw}The analysis of extraction in English}
\end{figure}
Figure~\vref{fig-chris-we-think-that-david-saw} shows the analysis of example
(\ref{ex-chris-we-think-that-david-saw}), which really requires a nonlocal dependency. As is shown
in the figure, the information about the missing object is passed up to the sentence level (S/NP),
to the CP level (CP/NP) and up to the next higher S. There it is bound off by the filler \emph{Chris}.
\begin{figure}
\begin{forest}
sm edges
[S
  [NP [Chris] ]
  [S/NP
    [NP [we] ] 
    [VP/NP  
       [V [think] ]
       [CP/NP
         [C [that] ]
         [S/NP
            [NP [David] ] 
            [VP/NP  
               [V [saw] ]
               [NP/NP [\trace ] ] ] ] ] ] ] ]
%% \draw[connect] (NP/NP.north east)  [bend right] to (VP/NP.south east);
%% \draw[connect] (VP/NP.north east)  [bend right] to (S/NP.south east);
%% \draw[connect] (S/NP.north east)   [bend right] to (CP/NP.south east);
%% \draw[connect] (CP/NP.north east)  [bend right] to (VP/NP1.south east);
%% \draw[connect] (VP/NP1.north east) [bend right] to (S/NP1.south east);
%% \draw[connect] (S/NP1.north east)  [bend right] to (NP);
\end{forest}
\caption{\label{fig-chris-we-think-that-david-saw}Extraction crossing the clause boundary}
\end{figure}
The binding off of the missing element is licensed by a special schema, which is called the
Filler"=Head Schema. Figure~\vref{fig-filler-head} provides a sketch of this schema.
\begin{figure}
\begin{forest}
[{H}
  [\ibox{1}]
  [H/\ibox{1}]]
\end{forest}
\caption{\label{fig-filler-head}Sketch of the Head-Filler Schema}
\end{figure}



English is the only non-V2 language among the Germanic languages. In what follows I show how German
(V2+SOV) and Danish (V2+SVO) can be analyzed with the techniques that were introduced so far.
Figure~\vref{fig-das-buch-liest-jens} shows the analysis of (\mex{1}):
\ea
\gll Das Buch liest Jens.\\
     the book reads Jens\\
\glt `Jens reads the book.'
\z
\begin{figure}
\begin{forest}
sm edges
[S
  [NP$_i$ [das Buch;the book, roof] ]
  [S/NP
     [V \sliste{ S$/\!/$V } 
        [V [liest$_j$;reads] ] ]
     [S$/\!/$V/NP
        [NP/NP [\trace$_i$] ]
        [V$'$$\!/\!/$V
           [NP [Jens;Jens] ]
           [V$\!/\!/$V [\_$_j$] ] ] ] ] ] ]
%% \draw[connect] (NP/NP) [bend right] to (S//V/NP.south east);
%% \draw[connect] (S//V/NP.north east) [bend right] to (S/NP.east);
%% \draw[connect] (S/NP.north east) [bend right] to (NP);
\end{forest}
\caption{\label{fig-das-buch-liest-jens}Analysis of V2 in German (SOV)}
\end{figure}
The analysis of the German example is more complicated than the English one since verb movement is
involved. The verb is fronted as was explained with reference to
Figure~\ref{fig-liest-jens-das-buch}. In addition the object is realized by a trace and then filled
by the filler \emph{das Buch} `the book', which is realized preverbally. 

I follow \citet{Fanselow2003d} and \citet{Frey2004a}, who assume that the position of the object is
initial in the \mf. Since German allows for both nominative, accusative and accusative, nominative
order, the position of the trace for the extracted object could be initial or final as in (\mex{1}a)
and (\mex{1}b), respectively:
\eal
\ex 
\gll {}[Das Buch]$_i$ liest$_j$ \_$_i$ Jens \_$_j$.\\
       \spacebr{}the book reads {} Jens\\
\ex 
\gll {}[Das Buch]$_i$ liest$_j$ Jens \_$_i$ \_$_j$.\\
       \spacebr{}the book reads Jens\\
\zl
Fanselow and Frey refer to information structural properties that elements in the initial position
have and argue that fronted elements like \emph{das Buch} have information structural properties
that correspond to the ones that non-fronted elements in the initial \mf position have:
\ea
\gll Liest das Buch Jens?\\
     reads the book Jens\\
\glt `Does Jens read the book.'
\z
They argue that (\mex{0}) patterns with (\mex{-1}a) rather than with (\mex{-1}b).

The complete discussion will not be repeated here, since this would take us too far away, but the
interested reader may consult the references given above.

The Danish example is similar. We first have the analysis of verb-initial position that involves the
double slash mechanism and on top of that we have the fronting of the object using the slash
mechanism. Figure~\vref{fig-bogen-laeser-jens} illustrates.
\begin{figure}
\begin{forest}
sm edges
[S
   [NP$_i$ [bogen;book.\textsc{def}] ]
      [S/NP
         [V \sliste{ S$/\!/$V }
           [V [læser$_j$;reads] ] ]
           [S$/\!/$V/NP
             [NP [Jens;Jens] ]
             [VP$\!/\!/$V/NP
               [V$\!/\!/$V  [\_$_j$] ]
               [NP/NP [\trace$_i$ ] ] ] ] ] ] ] 
%% \draw[connect] (NP/NP.north east) [bend right] to (V/V.south east);
%% \draw[connect] (V/V.north east) [bend right] to (S//V/NP.south east);
%% \draw[connect] (S//V/NP.north east) [bend right] to (S/NP.east);
%% \draw[connect] (S/NP.north east) [bend right] to (NP);
\end{forest}
\caption{\label{fig-bogen-laeser-jens}Analysis of V2 in Danish (SVO)}
\end{figure}

The careful reader will ask why we use two different mechanisms to analyze verb movement and
extraction. The answer is that these movement types are different in nature: Verb movement is clause
bound while the movement of other constituents may cross clause-boundaries. This is captured by the
fact that the double slash information is passed up together with other head features as for
instance the part of speech information and the slash information is passed up separately.

Before we deal with passive in the next chapter, we can compare the three sentences in (\mex{1}):
\eal
\ex Jens reads a book.
\ex Jens læser en bog.
\ex Jens liest ein Buch.
\zl
Again the order of the elements is the same in all three languages. However, English is an SVO
non-V2 language, Danish is an SVO+V2 language and German is an SOV+V2 language. The analyses in
bracket notation are given in (\mex{1}):
\eal
\ex {}[\sub{S} Jens [\sub{VP} reads [\sub{NP} a book]]].
\ex {}[\sub{S} Jens$_i$ [\sub{S/NP} læser$_k$ [\sub{S/NP} \_$_i$ [\sub{VP}  \_$_k$ [\sub{NP} en bog]]]].
\ex {}[\sub{S} Jens$_i$ [\sub{S/NP} liest$_k$ [\sub{S/NP} \_$_i$ [\sub{\vbar} [\sub{NP} ein Buch] \_$_k$]]]].
\zl
It may be surprising that these three sentences get such radically different analyses although the
order of elements are the same. The difference in structures is the result of the assumption that
all declarative main clauses in the Germanic V2 languages follow the same pattern, namely that the
finite verb is fronted and then another constituent is fronted. This particular construction is
connected to the clause type, that is, to the meaning of the utterance (imperative, question,
assertion). The sentences in (\ref{ex-fernabhaengigkeit-one}) and (\ref{bsp-Fernabhaengigkeit}) show
that V2 involves a nonlocal dependency. Therefore the analysis
of (\mex{0}b) is more complex than (\mex{1}) and involves the fronting of the finite verb to intital
position with a successive fronting of the subject:
\ea
{}[\sub{S} Jens [\sub{VP} læser [\sub{NP} en bog]]].
\z
The reason is that now all declarative main clauses are subsumed under the same structure, namely
(\mex{-1}b). A declarative main clause in all Germanic languages is the combination of an extracted
phrase with a verb initial phrase in which the extracted element is missing. Fronting of the finite
verb is a way to mark the clause type: If just the finite verb is fronted, a yes/no question (\mex{1}a)
or imperative results (\mex{1}b).\footnote{
  Verb initial clauses may also be declarative clauses if so-called \emph{topic drop} \citep{Fries88b} is involved:
  \ea
  \gll Was macht Peter? Gibt ihm ein Buch.\\
       what does Peter  gives him a book\\
  \glt `What does Peter do? He gives him a book.'
  \z
  The subject of \emph{gibt} `gives' is dropped. The complete sentence would be a V2 sentence:
  \emph{Er gibt ihm ein Buch.}.
}
\eal
\ex 
\gll Gibt er ihm das Buch?\\
     gives he him the book\\
\glt `Does he give him the book?'
\ex 
\gll Gib mir das Buch!\\
     give me the book\\
\zl
If another constituent is fronted, a question with question word (\mex{1}a), an imperative
(\mex{1}b) or a declarative clause (\mex{1}c) results.
\eal
\ex 
\gll Wem gibt er das Buch?\\
     who gives he the book\\
\glt `Whom does he give the book to?'
\ex 
\gll Jetzt gib ihm das Buch!\\
     now give me the book\\
\glt `Give me the book now!'
\ex 
\gll Jetzt gibt er ihm das Buch.\\
     now gives he him the book\\
\glt `He gives him the book now.'
\zl
The analysis of the semantics of clause types cannot be given here but the interested reader is
referred to \citew{MuellerSatztypen,MuellerGS}.

\section{Alternatives}

\inlinetodostefan{Advanced stuff. Ignore if you do not dare.}

In the preceding section I suggested an analysis in which the basic SVO order is just that: a
subject followed by the verb and a verb followed by the objects. The verb final sentences of SOV
languages are analyzed as a verb that is preceded by its arguments. The position of the finite verb
is accounted for by fronting it via the double slash mechanism.

There are alternative proposals to SVO and SOV order and also to the placement of the finite
verb. The proposal by \citet{Kayne94a-u} suggests that all languages have an underlying
specifier-head-complement order. The orders we see in the Germanic SOV languages would then be
derived by movement. The counterproposal by \citet{Haider2000a,Haider2020a} does not suggest that all languages are
like English or Romance but instead claims that the VO languages are derived from an underlying OV
order. These two approaches are discussed in the following two subsections. As will be shown,
Kayne's proposal makes wrong predictions and Haider's proposal is not without problems either. For
both proposals it would be unclear how they should be acquired by learners of the respective
languages without the assumption of a rich Universal Grammar.

The third class of proposals to be discussed in Section~\ref{sec-aux-flat} does not assume verb movement at
all. Rather than assuming a structure with layered VPs and some sort of movement that reorders the
finite verb authors like \citet*{GKPS85a} and \citet{Sag2020a} assume that there are alternative
linearizations for finite verbs and their subjects. The pros and cons of such analyses are the topic
of Section~\ref{sec-aux-flat}.

\subsection{OV derived from VO: \citet{Kayne94a-u}}

\citet{Kayne94a-u} 


\begin{figure}
\oneline{%
\begin{forest}
where n children=0{delay=with translation}{}
[CP
	[C$^0$[weil;because, tier=word]]
	[TopP
		[DP$_j$ [diese Sonate;this sonata,tier=below,l=31\baselineskip]]
		[SubjP
			[DP$_i$ [der Mann;the man,tier=word]]
			[ModP
				[AdvP [wahrscheinlich;probably,l=20\baselineskip]]
				[ObjP
					[DP$_j$ [diese Sonate;this sonata,tier=below]]
					[NegP
						[AdvP [nicht;not,tier=word]]
						[AspP
							[AdvP [oft;often,tier=word]]
							[MannP
								[AdvP [gut;well,tier=word]]
								[AuxP
									[VP$_k$ [gespielt;played,tier=word]]
									[Aux+
										[Aux [hat;has,tier=word]]
										[vP
											[DP$_i$]
											[VP$_k$
												[V]
												[DP$_j$
                                                                                                  [,phantom,tier=word]]]]]]]]]]]]]]
\end{forest}%
}
\caption{\label{Abbildung-Remnant-Movement-Satzstruktur}Analysis of sentence structure with leftward remnant movement
  and functional heads following \citet[\page 224]{Laenzlinger2004a}}
\end{figure}%

\citet{Haider2000a} shows why Kayne's proposals do not work.

\subsection{VO derived from OV: \citet{Haider2020a}}

\citet{Haider2000a,Haider2020a}

\subsection{Analyses of verb-initial sentences in SVO languages without verb-movement}
\label{sec-aux-flat}

\citet*{GKPS85a}, \citet{Sag2020a}


\questions{

\begin{enumerate}
\item How are clause types determined in the Germanic languages?

\end{enumerate}


}

\exercises{


\begin{enumerate}
\item Classify the Germanic languages according to their basic constituent order (SVO, SOV, VSO,
  \ldots) and V2 assuming that you know that one of the following patterns exist in the language:
\eal
\ex NP[acc] V-Aux NP[nom] V NP[dat]  % V2 SVO cannot be English since English does not have a dative
\ex NP[acc] V-Aux NP[nom] NP[dat] V  % V2 SOV
\ex NP[acc] NP[nom] V NP[acc]        % -V2 SVO
\ex NP[acc] NP[nom] V-Aux V NP[acc]  % -V2 SVO
\ex NP[acc] V-Aux NP[nom] V PP       % kann man nicht sagen
\zl
Every sentence should be paired with ±V2 and one of the six permutations of S, O, and V.

If you cannot determine the order unambiguously, please say so. If you think that this pattern does
not exist in any of the Germanic languages say so. Please keep in mind that English is a so-called
residual V2 language, which means that there are some traces of V2 left in the grammar. Think about
question formation in English.

\item Sketch the analysis for the following examples:
\eal
\ex 
\gll dass er darüber lacht\\
     that he there.upon laughs\\\jambox{(German)}

\ex 
\gll dass er darüber lachen wird\\
     that he there.upon laugh will\\
\ex
\gll Wird er darüber lachen?\\
     will he there.upon laugh\\
\zl

\ea
\gll Arbejder Bjarne ihærdigt  på bogen?\\
     works    Bjarne seriously at book.\textsc{def}\\\jambox{(Danish)}
\glt `Does Bjarne work seriously on the book?'
\z

\end{enumerate}

}


%      <!-- Local IspellDict: en_US-w_accents -->
