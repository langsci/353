%% -*- coding:utf-8 -*-
\chapter{Passive}
\label{chap-case}

%\if0

This chapter deals with the passive. The passive is usually analyzed as the suppression of the
subject. However, before I can develop an analysis, I have to ask what it is that constitutes a
subject. This is a question that is the topic of edited volumes and dissertations and modest as I
am, I will try and provide an answer at least for the \ili{Germanic} languages. As we will see, the
situation is rather clear in languages like \ili{Danish}, \ili{English}, and \ili{German}, but there are exciting
facts to be discovered about \ili{Icelandic}. 


\section{The phenomenon}


\subsection{Subjects and other subjects}
\label{sec-subj-properties}
\label{sec-icelandic-quirky-subj}

The\is{subject|(} situation in languages like \ili{Danish}, \ili{English}, and \ili{German} is rather clear. For instance, many
authors assume that non-predicative NPs in the nominative are subjects in \ili{German}. So, \emph{der
  Delphin} `the dolphin' is the subject of the sentences in (\mex{1}): 
\eal
\ex 
\gll Der Delphin lacht.\\
     the.\NOM{} dolphin laughs\\\german
\ex 
\gll Der Delphin hilft dem Kind.\\
     the.\NOM{} dolphin helps the.\DAT{} child\\
\ex 
\gll Der Delphin gibt  ihr einen Ball.\\
     the.\NOM{} dolphin gives her.\DAT{} a.\ACC{} ball\\
\zl
The restriction to non-predicative NPs is needed since otherwise, we would have to assume that both
NPs in (\mex{1}) are subjects, but \emph{ein Lügner} `a liar' is a predicative phrase and only
\emph{der Mann} `the man' is the subject.
\ea
\gll Der Mann ist ein Lügner.\\
     the.\NOM{} man is a.\NOM{} liar\\\german
\glt `The man is a liar.'
\z
In addition, certain clausal arguments are treated as subjects.

Genitives and datives as in (\mex{1}) are not counted among the subjects in \ili{German}.
\eal
\ex 
\gll Ihrer       wurde gedacht.\\
     they.\GEN{} \AUX{} remembered\\\german
\glt `They were remembered.'
\ex 
\gll Ihm       wurde  geholfen.\\
     he.\DAT{} \AUX{} helped\\
\glt `He/she was helped.'
\zl

\noindent
Interestingly the question whether genitives and datives like those in (\mex{0}) are subjects was
answered quite differently for the SVO language \ili{Icelandic} by researchers following the work of
\citet*{ZMT85a}. Although the sentences in (\mex{1}) look like those in (\mex{-1}), the genitive and
the dative element in (\mex{1}a) and (\mex{1}b) are claimed to be subjects.
\eal
\label{ex-subject-icelandic-passive-v2}
\ex 
\gll Hennar var saknað.\\
     she.\SG.\GEN{} was missed\\\icelandic
\ex 
\gll Þeim            var hjálpað.\\
     they.\PL.\DAT{} was helped\\
\zl
Since \ili{Icelandic} is a V2 language, the constituent order in such simple sentences does not help us to
determine whether \emph{hennar} `her' and \emph{Þeim} `them' are subjects or not. These elements are
fronted and since both subjects and objects can be fronted, the sentences in (\mex{0}) do not help
us in determining the grammatical function of these arguments. However, \citet*{ZMT85a} argued that
these elements should be analyzed as subjects and provided a test battery. Among the tests are more
elaborate positional tests and omitability in so-called control constructions. I will turn to these
tests now.


\subsubsection{The position of subjects in V2 and V1 sentences}


The\is{order!V2|(}\is{order!V1|(} first test that was suggested uses the position of constituents in V2 sentences in which a
non-subject is fronted \citep*[Section~2.3]{ZMT85a}. For instance, consider the following examples:

\eal
%% \ex[]{
%% \gll Refinn           skaut  Ólafur      með þessari byssu.\\
%%      den.Fuchs.\ACC{} schoss Olaf.\NOM{} mit diesem  Gewehr\\
%% }
% selbst erfunden, check
\ex[]{
\gll Með  þessari byssu   skaut Ólafur      refinn.\\
     with this    shotgun shot  Olaf.\NOM{} the.fox.\ACC{}\\\icelandic
}
\ex[*]{
\gll Með þessari byssu  skaut refinn         Ólafur.\\
     with this  shotgun shot  the.fox.\ACC{} Olaf.\NOM{}\\
}
\zl
The nominative can appear directly after the finite verb \emph{skaut} `shot' as in (\mex{0}a) but it
cannot appear to the right of the accusative as in (\mex{0}b).

The second test uses \emph{w}-questions and checks the position of the subject with respect to the
object and to non-finite verbs:
\eal
\ex[]{
\gll Hvenær hafði Sigga        hjálpað Haraldi?\\
     when   has   Sigga.\NOM{} helped Harald.\DAT{}\\\icelandic
}
\ex[*]{
\gll Hvenær hafði Haraldi       Sigga        hjálpað?\\
     when   has   Harald.\DAT{} Sigga.\NOM{} helped\\
}
\zl
The object has to follow the participle \emph{hjálpað} as in (\mex{0}a) and the subject immediately
follows the finite verb. Examples with the object before the subject as in (\mex{0}b) are
ungrammatical. The dative object can be fronted, but then it has to be realized in initial position
to the left of the finite verb, not to its right:
\ea[]{
\gll Haraldi       hafði Sigga        aldrei hjálpað.\\
     Harald.\DAT{} has   Sigga.\NOM{} never  helped\\\icelandic
}
\z

\noindent
The same situation can be found in yes/no questions:
\eal
\ex[]{
\gll Hafði Sigga        aldrei hjálpað Haraldi?\\
     has   Sigga.\NOM{} never  helped   Harald.\DAT\\\icelandic
}
\ex[*]{
\gll Hafði Haraldi       Sigga        aldrei hjálpað?\\
     has   Harald.\DAT{} Sigga.\NOM{} never  helped\\
}
\zl

\noindent
\citet*[Section~2.3]{ZMT85a} observed that certain datives can appear in this postverbal position as well:

\eal
\ex[]{
\gll Hefur henni      alltaf þótt    Ólafur      leiðinlegur?\\
     has   she.\DAT{} always thought Olaf.\NOM{} boring.\NOM{}\\\icelandic
\glt `Has she always considered Olaf boring?'
}
\ex[]{
\gll Ólafur      hefur henni alltaf þótt leiðinlegur.\\
     Olaf.\NOM{} has   she.\DAT{} always thought boring.\NOM{}\\
\glt `She always considered Olaf boring.'
}
\ex[*]{
\gll Hefur Ólafur      henni      alltaf þótt    leiðinlegur?\\
     has   Olaf.\NOM{} her.\DAT{} always thought boring.\NOM\\
}
\zl

\noindent
The \ili{German} equivalent would be the sentence in (\mex{1}):
{\judgewidth{??}
\ea[??]{
\gll Mich     dünkt  der Film langweilig.\\
     I.\ACC{} thinks the.\NOM{} movie boring\\\german
\glt `I think the movie is boring.'
}
\z}

\noindent
However, \emph{dünkt} is archaic and is usually used with a \emph{dass} `that' clause -- if it is used at
all. But there is a non-archaic verb that has a similar form:
\ea
\gll Mir scheint der Mann langweilig.\\
     I.\DAT{} seems the.\NOM{} man boring\\\german
\glt `The man seems boring to me.'
\z
The experiencer of \emph{scheinen} `to seem' is expressed with the dative, while the subject of the
embedded predicate \emph{langweilig} `boring' is in the nominative.
\is{order!V2|)}\is{order!V1|)}


\subsubsection{Subjects in control constructions}
\label{sec-subject-control}

\citet*[Section~2.7]{ZMT85a}\is{control|(} discuss control structures in which the subject of the embedded verb is
not expressed. (\mex{1}a) shows an example of normal control in which the subject of the matrix verb
\emph{vonast} `to hope' refers to the same discourse referent as the subject of the embedded verb \emph{fara} `to
go'. (\mex{1}b) is an example of so-called arbitrary control. In cases of arbitrary control there is
no element depending on the head that governs the infinitive that refers to the same discourse
referent as the subject of the infinitive. The unexpressed subject corresponds to a pronoun \emph{one} that is
used generically. In example (\mex{11}b) \emph{óvenjulegt} `unusual' does not select for an argument
that refers to the same referent as the subject of \emph{fara} `to go'. The subject of \emph{að fara
  heim snemma} `to go home early' is not expressed but is understood as the indefinite pronoun \emph{one}.
\eal
\ex
\gll Ég  vonast til að fara heim.\\
     I   hope   for to go   home\\\icelandic
\glt `I hope to go home.'
\ex
\gll Að fara heim snemma er óvenjulegt.\\
     to go home   early is unusual\\
\glt `It is unusual to go home early.'
\zl

Now, it can be observed that \ili{Icelandic} allows verbs that do not take a nominative in such control
constructions. An example is \emph{vantar} (`lacks'), which takes two accusatives rather than a
nominative and an accusative:
\ea
\gll Mig      vantar peninga.\\
     I.\ACC{} lack   money.\ACC\\\icelandic
\z
(\mex{1}) shows that this verb can be embedded under \emph{vonast} (`to hope'):
\ea
\gll Ég vonast til að vanta ekki peninga.\\
     I  hope   for to lack  not  money.\ACC\\\icelandic
\glt `I hope that I do not lack money.'
\z

This should be compared with \ili{German}:
\eal
\ex[]{
\gll Mir fehlt kein Geld.\\
     I.\DAT{} lacks no.a.\NOM{} money\\\german
\glt `I do not lack money.'
}
\ex[*]{
\gll Ich hoffe, kein         Geld  zu fehlen.\\
     I hope     not.a.\NOM{} money to lack\\
\glt Intended: `I hope that I do not lack money.'
}
\zl

\noindent
The question at the beginning of this section was whether the datives and genitives in sentences
like (\ref{ex-subject-icelandic-passive-v2}), repeated here as (\mex{1}), are subjects or not.
\eal
\ex 
\gll Hennar     var saknað.\\
     she.\GEN{} was missed\\\icelandic
\glt `She was missed.'
\ex 
\gll Þeim        var hjálpað.\\
     they.\DAT{} was helped\\
\glt `They were helped.'
\zl

\noindent
We are now able to use the tests to answer this question: the dative is right-adjacent to the finite
verb in the question in (\mex{1}) and, thus, in subject position.

\eal
\ex 
\gll Var hennar saknað?\\
     was she.\GEN{} missed\\\icelandic
\glt `Was she missed?'
\ex
\gll Var þeim       hjálpað?\\
     was they.\DAT{} helped   \\
\glt `Were they helped?'
\zl


% \ea
% \gll Var honum     aldrei hjálpað af foreldrum sinum?\\
%      was he.\DAT{} never  helped   by parents   his\\
% \glt `Did his parents never help him?'
% \z

\noindent
Similarly, the dative follows the finite verb in the V2 sentence in (\mex{1}):
\ea
\gll Í prófinu  var honum vist hjálpað.\\
     in the.exam was he.\DAT{} apparently helped\\\icelandic
\glt `Apparently he was helped in the exam.'
\z

\noindent
In addition, these datives can be omitted in control constructions as the examples in (\mex{1}) show:
\eal
\ex
\gll Ég vonast til að verða hjálpað.\\
     I  hope   for to be helped\\\icelandic
\ex
\gll Að vera hjálpað i prófinu er óleyfilegt.\\
     to be helped in the.exam is un.allowed\\
\glt `It is not allowed to be helped in the exam.'
\zl
This should be compared to \ili{German}: while verbs like \emph{unterstützen} `to support' that govern a
nominative and an accusative can appear in such control constructions, verbs like \emph{helfen} `to
help' that take a nominative and a dative are ruled out in this construction:
\eal
\ex 
\gll dass jemand ihm hilft\\
     that somebody him.\DAT{} helps\\\german
\ex
\gll dass jemand ihn unterstützt\\
     that somebody him.\ACC{} supports\\
\ex
\gll  dass ihm geholfen wird\\
      that him.\DAT{} helped \AUX\\
\ex
\gll  dass er unterstützt wird\\
      that he.\NOM{} supported \AUX\\
\zl

\eal
\ex[]{
\gll Ich hoffe unterstützt zu werden.\\
     I   hope  supported   to \AUX\\\german
}
\ex[*]{
\gll Ich hoffe geholfen zu werden.\\
     I   hope  helped   to \AUX\\
}
\zl
The dative object cannot be omitted in such control constructions, as (\mex{0}b) shows. The only way
to realize a passive below \emph{hoffen} `to hope' in an infinitival clause is to use the dative passive with
\emph{erhalten}/""\emph{bekommen}/""\emph{kriegen}. The dative passive can turn a dative object into a
nominative subject:\footnote{
  Not all speakers accept such sentences. We will return to them below when discussing the examples
  (\ref{ex-helfen-with-dative-passive}) and (\ref{ex-helfen-with-dative-passive-corpus}).
}
\ea[]{
\gll Aicke        bekommt geholfen.\\
     Aicke.\NOM{} \AUX{}  helped\\\german
\glt `Aicke gets helped.'
}
\z
\largerpage[-1]
Since the object of \emph{helfen} is then nominative and, hence, undoubtedly a subject in \ili{German}, it
does not come as a surprise that it can be omitted in control constructions like (\mex{1}):
\ea[]{
\gll Ich hoffe hier geholfen zu bekommen.\footnotemark\\
     I   hope  here helped   to \AUX\\\german
\footnotetext{    
\url{http://www.photovoltaikforum.com/sds-allgemein-ueber-solar-log-f38/solarlog-1000-mit-wifi-anschliesen-t96371.html}. 10.01.2014
}
\glt `I hope to get help here.'
}
\z
\is{subject|)}\is{control|)}


\subsection{Comparison between German, Danish, English, Icelandic}

In the following subsections, I will compare several dimensions in which the \ili{Germanic} languages
vary:
\begin{itemize}
\item \ili{Danish} and \ili{Icelandic} have a morphological passive; \ili{English} and \ili{German} do not.
% ZMT85a:443 \ili{Icelandic} also has some morphologically'middle' forms in the suffix -st, someof which
% have a passive meaning, as illustrated in (3):5

\item \ili{German} and \ili{Icelandic} allow for subjectless constructions; \ili{Danish} and \ili{English} do not.

\item \ili{Danish}, \ili{German}, and \ili{Icelandic} allow for impersonal passives; \ili{English} does not.

\item \ili{Danish} and \ili{Icelandic} allow both objects to be promoted to subject; \ili{English} and \ili{German} do not.

\item \ili{German} has the remote passive, \ili{Danish} the complex passive and \ili{Danish} and \ili{English}
  have the reportive passive.
\end{itemize}



\subsubsection{Morphological and analytic forms}

\ili{Danish} has a morphological\is{morphology} passive\is{passive!morphological}. It is formed by appending the suffix \suffix{s} to the verb, and there
are forms for the present tense (\ref{ex-laeses}) and the past tense (\ref{ex-laestes}):
\eal
\ex[]{\label{ex-laeseract}
\gll Peter læser avisen.\\
     Peter reads newspaper.\textsc{def}\\\danish
\glt `Peter is reading the newspaper.'}
\ex[]{\label{ex-laeses}
\gll Avisen              læses af Peter.\\
     newspaper.\textsc{def} read.\textsc{pres}.\textsc{pass} by Peter\\
\glt `The newspaper is read by Peter.'}
\ex[]{\label{ex-laestes}
\gll Avisen              læstes af Peter.\\
     newspaper.\textsc{def} read.\textsc{past}.\textsc{pass} by Peter\\
\glt `The newspaper was read by Peter.'}
\zl
As the examples in (\mex{1}) shows, the \emph{af} phrase is not necessary:
\eal
\ex 
\gll Avisen        læses           hver   dag.\\
     newspaper.\textsc{def} reads  every day\\\danish
\glt `The newspaper is read every day.'
\ex 
\gll Avisen        læstes           hver   dag.\\
     newspaper.\textsc{def} read  every day\\\danish
\glt `The newspaper was read every day.'
\zl

\noindent
\ili{Danish} also has an analytic form with \emph{blive} `be' plus past participle: 
\ea
\gll Avisen                 bliver læst af Peter.\\
     newspaper.\textsc{def} is     read by Peter\\\danish
\glt `The newspaper is read by Peter.'
\z
The morphological passive may also apply to infinitives:
\ea
\gll Avisen skal læses hver dag.\\
     newspaper.def must read.\textsc{inf}.\textsc{pass} every day\\\danish
\glt `The newspaper must be read every day.'
\z

\noindent
\ili{English} and \ili{German} only have the analytic variant:
\eal
\ex The paper was read.
\ex 
\gll Der        Aufsatz wurde  gelesen.\\
     the.\NOM{} paper   \AUX{} read\\\german
\zl    





\subsubsection{Personal and impersonal passive}
\label{sec-impersonal-passive-phen}

All\is{passive!personal|(}\is{passive!impersonal|(} languages under consideration allow for the promotion of an accusative object to subject, an
example of which is given in (\mex{1}).
\eal
\ex
\longexampleandlanguage{
\gll Angehörige haben den Verdächtigen zuletzt am Montag gesehen.\\
     relatives.\NOM{}  have  the.\ACC{} suspect      lastly  at.the Monday seen\\}{German}
\glt `Relatives have seen the suspect for the last time on Monday.'
\ex 
\gll Der        Verdächtige wurde  zuletzt am Montag gesehen.\\
     the.\NOM{} suspect     \AUX{} lastly at.the Monday seen\\
\glt `The suspect was seen for the last time on Monday.'
\zl

\noindent
As the following examples show, the subject can be an S or a VP:\is{subject clause}
% \todostefan{show that these are really
%   subjects. We have V2 here, could be objects.}

\eal
\ex
\gll At regeringen træder tilbage, bliver påstået.\\
     that government.\textsc{def} resigns \textsc{part} is claimed\\\danish
\glt `It is claimed that the government resigns.'
\ex
\gll At reparere bilen, bliver forsøgt.\\
     to repair car.\textsc{def} is tried\\
\glt `It is tried to repair the car.'
\zl

\noindent
In addition to such personal passives, \ili{Danish}, \ili{German}, and \ili{Icelandic} have impersonal
passives.\footnote{
The labels ``personal'' and ``impersonal passive'' are misnomers, since both passives share the property of
demoting the subject. So-called personal passives can have animate subjects or inanimate subjects:
\ea 
\gll Der        Diamant wurde  zuletzt am Montag gesehen.\\
     the.\NOM{} diamond \AUX{} lastly at.the Monday seen\\\hfill(\ili{German})
\glt `The diamond was seen for the last time on Monday.'
\z
The big \ili{German} grammar of the Institut für Deutsche Sprache tried to establish the new terms \emph{Zweitakt-Passiv} `two-phase passive' and
\emph{Eintakt-Passiv} `one-phase passive' \citep[\page 1793]{Zifonun97b}. The first phase being the suppression of the subject and
the second phase the promotion of the accusative object to subject for those verbs that govern an
accusative. While these terms are more appropriate in principle, I will not use them here since the
analysis suggested in what follows deals with both passives in a unified way: it just suppresses the
subject. Personal and impersonal passives are analyzed the same way. The difference is due to
differences in case assignment. Due to the lack of better terms, I continue to use the terms personal and
impersonal passive.
} Since \ili{German} does not require a subject, impersonal passives like (\mex{1}) are expected:

\ea
\gll weil    noch  getanzt wurde\\
     because still danced  \AUX\\\german
\glt `because there was still dancing there'
\z

\noindent
The following two examples from \ili{Icelandic} show that \ili{Icelandic} also has impersonal constructions \citep[\page 264]{Thrainsson2007a-u}:
\eal
\ex 
\gll Oft var   talað      um   þennan mann.\\
     often was talked about this Mann.\ACC.\SG.\M\\\icelandic
\ex
\gll Aldrei hefur verið    sofið      í  þessu  rúmi.\\
     never    has   been slept in this bed.\DAT\\
\glt `This bed has never been slept in.'
\zl

\noindent
\ili{Danish} also allows for impersonal passives, but it differs from the languages discussed so far in
that it requires an expletive\is{pronoun!expletive} subject:
\eal
\ex 
\gll at der bliver danset\\
     that \textsc{expl} is danced\\\danish
\glt `that there is dancing'
\ex
\gll at der danses\\
     that \textsc{expl} dance.\textsc{pres}.\textsc{pass}\\
\glt `that there is dancing'
\ex[*]{ 
\gll Bliver danset?\\
     is danced\\
}
\ex[*]{
\gll Danses?\\
     dance.\textsc{pass}\\
}
\zl
Thus, \ili{Danish} is like \ili{English} in always requiring a subject, but, while this constraint results in the
impossibility of impersonal passives in \ili{English}, \ili{Danish} found a solution to the subject problem by
inserting an expletive.

Expletives\is{pronoun!expletive} are excluded in \ili{German} impersonal constructions:
\nocite{MOe2011a}
\ea[*]{
\gll weil    es noch  gearbeitet wurde\\
     because it still worked     \AUX\\\german
\glt Intended: `because there was still working there'
}
\z
\is{passive!personal|)}\is{passive!impersonal|)}


%% The examples in (\ref{ex-gearbeitet-wurde}) and (\ref{ex-bliver-arbejder}) show passives of
%% mono-valent verbs but of course bi-valent intransitive verbs like the \ili{German} \emph{denken} (`think')
%% and \ili{Danish} \emph{passe} (`take care of') also form impersonal passives:
%% \ea
%% \gll dass an die Männer gedacht wurde\\
%%      that \textsc{prep} the men thought was\\
%% \glt `that one thought about the men'
%% \z
%% \eal
%% \label{ex-impersonal-passive-pp}
%% \ex
%% \gll Der passes på børnene.\\
%%      \textsc{expl} take.care.of.\textsc{pres}.\textsc{pass} on children.\textsc{def}\\
%% \glt `Somebody takes care of the children.'
%% \ex
%% \gll Der bliver passet  på børnene.\\
%%      \textsc{expl} is taken.care.of on children.\textsc{def}\\
%% \glt `Somebody takes care of the children.'
%% \zl


\subsubsection{Promotion of the primary and secondary object}

\largerpage
\ili{English} and \ili{German} allow the promotion of one of the objects of a ditransitive verb only. (\mex{1})
shows that the accusative object can be realized as subject, but the dative object cannot:
\eal
\ex[]{
\longexampleandlanguage{
\gll weil    der        Mann dem Jungen den Ball schenkt\\
     because the.\NOM{} man the.\DAT{} boy the.\ACC{} ball gives\\}{German}
\glt `because the man gives the boy a ball as a present'
}
\ex[]{
\gll weil    dem        Jungen der        Ball geschenkt wurde\\
     because the.\DAT{} boy    the.\NOM{} ball given     \AUX\\
\glt `because the ball was given to the boy'
}
\ex[*]{
\gll weil der Junge den Ball geschenkt wurde\\
     because the.\NOM{} boy the.\ACC{} ball given \AUX\\
}
\zl

\noindent
Similarly, \ili{English} can realize the first object as subject, but the second object cannot be promoted
to subject:
\eal
\ex[]{
because the man gave the child the ball
}
\ex[]{
because the child was given the ball
}
\ex[*]{
because the ball was given the child
}
\zl
The information structural effect can be reached with a different lexical variant of \emph{give}
though. \emph{give} can be used with an NP object and a \emph{to} PP instead of two NPs as in
(\mex{1}a). The first object of the ditransitive \emph{give} is realized as PP in (\mex{1}a) and the
second object \emph{the ball} is the first object in (\mex{1}a). This alternation is also called dative-shift\is{dative-shift}.
\eal
\ex because the man gave the ball to the child
\ex because the ball was given to the child
\zl
(\mex{0}b) is the passive variant of (\mex{0}a). As in (\mex{-1}b), the primary object is promoted
to subject.

\ili{Danish} and \ili{Icelandic} differ from \ili{English} and \ili{German}. In the former languages, both objects can be
promoted to subject without any previous alternation of valence frames like dative shift.
\eal
\ex\label{ex-fordi-manden-giver-barnet-bolden} 
\gll fordi manden giver barnet bolden\\ 
     because man.\textsc{def} gives child.\textsc{def} ball.\textsc{def}\\\danish
\glt `because the man gives the child the ball'
\ex\label{ex-child-was-given-ball-danish}
\gll fordi barnet bliver givet bolden\\ 
     because child.\textsc{def} is given ball.\textsc{def}\\
\glt `because the child is given the ball'
\ex\label{ex-ball-was-given-child-danish}
\gll fordi bolden bliver givet barnet\\ 
     because ball.\textsc{def} is given child.\textsc{def}\\
\glt `because the ball is given to the child'
\zl

\noindent
One could assume that it is always the first object (the primary object) that is promoted to subject
and that \ili{Danish} does not have an order of the objects, so that both objects are equally prominent
and can be promoted to subject. \ili{Moro} is a language that is said to have such properties
\citep{AMM2017a-u}. However, \ili{Danish} differs from Moro in that the order of the objects in sentences is
clearly fixed: while (\mex{0}a) is possible, the reverse order of the objects is ungrammatical, as
(\mex{1}) shows.
\ea[*]{
\gll fordi   manden           giver bolden            barnet \\
     because man.\textsc{def} gives ball.\textsc{def} child.\textsc{def}\\
}
\z

\largerpage
\noindent
As far as \ili{Icelandic} is concerned, \citet*[\page 460]{ZMT85a} note that, apart from the possibility to
promote the accusative to nominative subject, the dative can become a quirky subject:
\ea
\label{ex-dat-subj-passive-ditransitive-icelandic}
\gll Konunginum voru gefnar ambáttir.\\
     the.king.\DAT{} were given.\F.\PL{} maidservants.\NOM.\F.\PL\\\icelandic
\glt `The king was given female slaves.'
\z
The structure of (\mex{0}) is sketched in (\mex{1}):
\ea
\label{sub-aux-v-o}
{}[S$_i$ Aux \_$_i$ V O] 
\z
Since the nominative is serialized after the participle, it cannot be the subject, which implies that
the fronted dative element is the subject.

Alternatively, the accusative object is promoted to nominative subject:
\ea
\label{ex-nom-subj-passive-ditransitive-icelandic}
\gll Ambáttin var gefin konunginum.\\
     the.maidservant.\NOM.\SG{}  \AUX{} given.\F.\SG{} the.king.\DAT\\\icelandic
\glt `The female slave was given to the king.'
\z
This sentence, too, has the structure in (\ref{sub-aux-v-o}).

In order to show that the dative is really promoted to subject in
(\ref{ex-dat-subj-passive-ditransitive-icelandic}) and that the accusative is promoted to subject in
(\ref{ex-nom-subj-passive-ditransitive-icelandic}), \citet*[\page 460]{ZMT85a} apply a battery of
tests. I only give the V2 examples with an adjunct in initial position, the questions, and the
control structures here. The examples in (\mex{1}) and (\mex{2}) show that the sentences above
really have the structure in (\ref{sub-aux-v-o}). The first position in (\mex{1}) is filled by an
adjunct, which entails that the subject remains in subject position and hence shows that the dative
\emph{konunginum} `the king' is the subject. Similarly, the nominative \emph{ambáttin} `the female
slave' is the subject in (\mex{1}b).

\eal
\ex
\gll Um veturinn voru konunginum gefnar ambáttir.\\
     in the.winter \AUX{} the.king.\DAT{} given slaves.\NOM\\\icelandic
\glt `In the winter, the king was given (female) slaves.'
\ex
\gll Um veturinn var ambáttin gefin konunginum.\\
     in the.winter \AUX{} the.slave.\NOM{} given the.king.\NOM\\
\glt `In the winter, the slave was given to the king.'
\zl
The questions in (\mex{1}) are further evidence. The initial position is not filled, and the dative
in (\mex{1}a) and the nominative in (\mex{1}b) are realized immediately following the finite verb.
\eal
\ex\label{ex-were-the-king-given-the-slaves}
\gll Voru konunginum gefnar ambáttir?\\
     \AUX{} the.king.\DAT{} given slaves.\NOM{}\\\icelandic
\glt `Was the king given slaves?'
\ex\label{ex-were-the-slaves-given-the-king}
\gll Var ambáttin gefin konunginum?\\
     \AUX{} the.slave.\NOM{} given the.king.\DAT\\
\glt `Was the slave given to the king?'
\zl

\noindent
(\mex{1}) shows the corresponding control examples:
\eal
\ex
\gll Að vera gefnar ambáttir var mikill heiður.\\
     to \AUX{} given slaves.\NOM{} was great honor\\\icelandic
\glt `To be given slaves was a great honor.'
\ex
\gll Að vera gefin konunginum olli miklum vonbrigðum.\\
     to \AUX{} given the.king.\DAT{} caused great disappointment\\
\glt `To be given to the king caused great disappointment.'

%% \ex
%% \gll Ambáttin vonast til að verða gefin konunginum.\\
%%      the.slave.\NOM{} hopes for to be given the.king.\DAT\\
%% \glt `The slave hopes to be given to the king.'
\zl
In (\mex{0}a) the dative is not expressed and in (\mex{0}b) the nominative is omitted. This shows
that both the primary and the secondary object can be promoted to subject in \ili{Icelandic}, even though
the primary object is in the dative and the case of the NP does not change to nominative in passive examples.




% \if0


% \subsubection{Subjekt-Verb-Kongruenz?}

% \begin{itemize}
% \item Verben kongruieren mit dem Nominativelement.\\
%       Wenn es keins gibt, ist das Verb dritte Person Singular (Neutrum).


% \item In (\mex{1}) keine Kongruenz:
% \eal
% \ex 
% \gll Þeim       var hjalpað.\\
%      sie.\emph{\PL}.\DAT{} wurde geholfen\\
% \ex 
% \gll Hennar var saknað.\\
%      sie.\emph{\SG}.\GEN{} wurde vermisst\\
% \zl


% \item Der Dativ und der Genitiv sind aber dennoch Subjekte,\\
%       wie wir gleich sehen werden.

% \end{itemize}

% \fi

\section{The analysis}

\subsection{Structural and lexical case and the Case Principle}
\label{sec-struk-lex-kas}
\label{sec-struc-lex-kas}

For\is{case|(}\is{case!structural|(}\is{case!lexical|(} the analysis of the passive, it is useful to distinguish between structural and lexical
case. Structural case is case that depends on the syntactic structure in which arguments get
realized, while lexical case is case that stays constant independent of the syntactic environment. In
addition to lexical and structural case, there is semantic case. This case is not assigned by a
governing head like a verb, adjective, or preposition but is due to a certain function of an
adverbial. For instance, time expressions like \emph{den ganzen Tag} `the whole day' in (\mex{1}) are
in the accusative in \ili{German}.
\ea
\gll Er arbeitet den ganzen Tag.\\
     he.\NOM{} works the.\ACC{} whole day\\\german
\glt `He works the whole day.'
\z
Since this chapter is about the passive and its variation in the \ili{Germanic} languages, I will ignore
semantic case here.

\subsubsection{Nominatives and accusative objects}

Up to now, the case that an argument gets assigned by its head has been represented in the valence list of
the head. With such a representation, we would need two different lexical items for the verb
\emph{lesen} `to read': one in which the verb takes a nominative and an accusative as in
(\mex{1}c), and one in which it takes two accusatives as in (\mex{1}d). (\mex{1}c) would be used in
the analysis of (\mex{1}a), and (\mex{1}d) in the analysis of (\mex{1}b).
\eal
\ex 
\gll Er        wird das        Buch lesen.\\
     he.\NOM{} will the.\ACC{} book read\\\german
\glt `He will read the book.'
\ex 
\gll Ich sah ihn das Buch lesen.\\
     I   saw him the book read\\
\glt `I saw him read the book.'
\ex \sliste{ NP[\type{nom}], NP[\type{acc}] }
\ex \sliste{ NP[\type{acc}], NP[\type{acc}] }
\zl
Rather than having two distinct, yet homophonous forms in the lexicon, one can propose just one lexical item and leave the actual
case assignment to be resolved later when the syntactic context provides sufficient information. So,
depending on whether the subject of \emph{lesen} is realized as the 
subject of \emph{wird} `will' or as the object of \emph{sah} `saw', it gets nominative or accusative. Such
cases are called structural cases. The distinction between structural and lexical case will play an
important role in the analysis of the passive. It is this distinction that makes a unified analysis of
the personal and impersonal passive possible.

(\mex{1}) provides additional examples and involves different forms of the verb (finite
vs.\ non-finite) and a nominalization:
\eal
\ex 
\gll Der Installateur kommt.\\
     the.\NOM{} plumber      comes\\
\glt `The plumber comes.'
\ex 
\gll Der Mann lässt den Installateur kommen.\\
     the man  lets the.\ACC{} plumber      come\\
\glt `The man lets the plumber come.'
\ex 
\gll das Kommen des Installateurs\\
     the coming of.the.\GEN{} plumber\\
\glt `the coming of the plumber'
\zl
The example in (\mex{0}c) also shows that the subject of \emph{kommen} `to come' can be realized as
genitive. Thus, nominative, genitive, and accusative are structural cases in \ili{German}. (The question
whether some or all datives should be treated as structural case is addressed below in Section~\ref{sec-dative-objects-tructurl-lexical}).

The examples in (\mex{0}) show that the case of subjects in \ili{German} can change, those in (\mex{1})
show that the case of accusative objects can change as well:
\eal
\ex 
\gll Judit schlägt den Weltmeister.\\
     Judit defeats the.\ACC{} world.champion\\
\glt `Judit defeats the world champion.'
\ex 
\gll Der        Weltmeister    wird   geschlagen.\\
     the.\NOM{} world.champion \AUX{} beaten\\
\glt `The world champion is beaten.'
\zl

\subsubsection{Genitive objects}

The examples in (\mex{1}) show instances of lexical case: genitive that depends on the verb is
lexical since it does not change when the verb is passivized.
\eal
\ex[]{
\gll Wir gedenken der Opfer.\\
     we.\NOM{} remember the victims.\GEN{}\\
\glt `We remember the victims.'
}
\ex[]{
\gll Der        Opfer   wird   gedacht.\\
     the.\GEN{} victims \AUX{} remembered\\
\glt `The victims are remembered.'
}
\ex[*]{
\gll Die Opfer wird / werden gedacht.\\
     the.\NOM{} victims \AUX.3\SG{} {} \AUX.3\PL{} remembered\\
}
\zl
As the example in (\mex{0}c) shows, the nominative is impossible. The genitive object remains in the
genitive in passive constructions. As was explained in Section~\ref{sec-impersonal-passive-phen},
passives without a subject as in (\mex{0}b) are traditionally called ``impersonal passives''.\is{passive!impersonal}

\subsubsection{Dative objects}
\label{sec-dative-objects-tructurl-lexical}

Let\is{case!dative|(} us now turn to the dative. If we consider examples like (\mex{1}), we see that the dative does not
change either in the passive:
\eal
\ex 
\gll Der Mann hat ihm geholfen.\\
     the\NOM{} man  has him.\DAT{} helped\\
\glt `The man helped him.'
\ex 
\gll Ihm        wird geholfen.\\
     him.\DAT{} \AUX{} helped\\
\glt `He is helped.'
\zl
So in analogy to the genitive examples above, the dative should be a lexical case.

But there are examples like those in (\mex{1}) and, according to the view that structural cases are
those cases that vary according to the syntactic environment, the dative should be a structural case.
\eal
\ex 
\gll Der Mann  hat   den Ball dem Jungen geschenkt.\\
     the man   has   the ball the boy given\\
\glt `The man gave the boy the ball as a present.'
\ex 
\gll Der Junge bekam den Ball geschenkt.\\
     the boy   got   the ball given\\   
\glt `The boy got the ball as a present.'
\zl
%\largerpage
The question whether the dative should be seen as a structural or a lexical case is a hotly debated
one. In principle, there are three possibilities and all three of them have been suggested in the
literature. One could assume that all datives are lexical 
\parencites{Haider85b}%
[\page 9]{Haider86}[\page 207, 217, 228]{HM94a}%
{Mueller99a,Mueller2001a}[\page 289]{Mueller2003e}%
[\page 97]{Scherpenisse86a}%
[\page 277, 291]{Pollard94a}%
[]{Meurers99b}[]{VS98a}[]{Abraham95a-u}%
[\page 187]{McIntyre2006a}% 
{Woolford2006a},
%-----------------------------------------------------------------------
that some are lexical and others are structural 
\parencites
{Wegener85a}%
{Wegener90}% ganzes Papier 74--75, 79 100 Abschnitt 10 zu inheränten Dativen
% Im Japanischen wird angehobenes Subjekt zum Dativ, wenn Verbalkomplexbildung
                      % zu NP, NP, NP führt.
[\page 26]{denBesten85}% verbs like helfen, which assign dative case
[page 55--56]{denBesten85b}% Für helfen sagt er in Fn, dass die auch mit V' kombiniert werden.
[\page 161]{Fanselow87a}[\page 178, 205]{Fanselow2000b}[\page 181, 182, 206]{Fanselow2003b}% 181 lexical datives
[\page 286--287]{Czepluch88}%
[\page 77, 80]{Sternefeld95a}% für bekommen-Passiv mit helfen ist es dann doch struktureller Kasus
                             % und Er wird geholfen ist immer noch ausgeschlossen, weil er das mit
                             % einem Dativ-pro macht.
[\page 102--103]{Stechow96a}%
[\page 48, 51]{Wunderlich97a}%
%[\page 107]{Wunderlich97c}% dative passive possible -> structural lexical datives are not mentioned
%                          % but follow implicitely
[\page 553]{Molnarfi98a}% inheränte sind in V inkorporiert
%-----------------------------------------------------------------------
, or that all datives are structural 
\parencites[\page 80]{Sternefeld95a}%
[\page 203, 205--206]{Ryu97a}% SUBJ finit = nom, COMPS und INTARG = acc, COMPS und -INTARG = dat
[\page 96--97]{Gunkel2003b}.% two subtypes for structural case = nom v acc und nom v dat
% nom v acc müsste aber nom v acc v gen sein. Kasuszuweisung müsste dann an Objekte acc v dat sein.
% Zwei Merkmale, die GF auszeichnen: EA, IA. Passiv macht IA zu EA (IA kann auch leer sein).
% EA ist Nominativ, alle anderen sind -nom (und nicht Genitiv)
% Das funktioniert nicht, wenn das Isländische auch Dativpassive hat.
%\todostefan{update references}
%Molnarfi98a} muss noch kopiert werden

I follow \citet{Haider86} and treat all datives as lexical cases. Under this assumption, the
contrast in Haider's examples \citeyearpar[\page 20]{Haider86} in (\mex{1}) is explained immediately:
\eal
\ex[]{
\gll Er streichelt den Hund.\\
     he.\NOM{} strokes the.\ACC{} dog\\
}
\ex[]{
\gll Der        Hund wurde  gestreichelt.\\
     the.\NOM{} dog  \AUX{} stroked\\
}
\ex[]{
\gll sein Streicheln des    Hundes\\
     his  stroking of.the.\GEN{} dog\\
}

\ex[]{\label{bsp-er-hilft-den-kindern}\iw{helfen}
\gll Er hilft den Kindern.\\
     he helps the.\DAT{} children\\
}
\ex[]{
\gll Den        Kindern  wurde  geholfen.\\
     the.\DAT{} children \AUX{} helped\\
}
\ex[]{
\gll das Helfen der Kinder\\
     the helping of.the.\GEN{} children\\
}\label{das-helfen-der-Kinder}
\ex[*]{
\gll sein Helfen der Kinder\\
     his  helping the children\\
}\label{sein-helfen-der-Kinder}
\zl
The accusative object of \emph{streicheln} `to stroke' can be realized as nominative in the passive,
so it is clearly a structural case. Nominalizations allow this object to be realized in the genitive
as (\mex{0}c) shows. However, this does not work with datives. The dative object of \emph{helfen}
`to help' cannot be realized in the genitive. (\ref{das-helfen-der-Kinder}) is possible, but only
with a reading in which the children are the agents, that is, the nominalization in
(\ref{das-helfen-der-Kinder}) corresponds to (\mex{1}) rather than (\ref{bsp-er-hilft-den-kindern}):
\ea
\gll Die Kinder helfen jemandem.\\
     the.\NOM{} children help somebody.\DAT{}\\
\z
If the agent is expressed by a prenominal possessive as in (\ref{sein-helfen-der-Kinder}) the
genitive or dative \emph{der Kinder} is ruled out.

The only way to express the dative at all is prenominally:
\ea
\gll das Den-Kindern-Helfen\\
     the the-children-helping\\
\glt `the children's helping'
\z

%\largerpage[2]
\noindent
Thus, authors who assume that all datives are structural have a problem explaining the differences in
impersonal passives and nominalizations.\footnote{\label{fn-structural-dative}
  This problem can be solved by assuming more fine-grained distinctions among the structural cases
  \citep[\page 96]{Gunkel2003b} and/or additional features singling out accusative objects
  \citep[\page 208]{Ryu97a}. Gunkel's approach is equivalent to saying that primary objects must be nominative
  or dative and secondary objects must be nominative or accusative. For case assignment in verbal
  environments, he states that objects must be non-nominative (p.\,112). However, he does not include the
  genitives in nominalizations. If this were included, the secondary object would be compatible to
  nominative, genitive, and accusative. Demanding non-nominative for objects would not be
  sufficient, since this would leave the option of realizing secondary objects as genitives in
  verbal environments, which is ungrammatical. Gunkel would have to state that objects have to
  be in the dative or accusative, which would result in a rather complex and unattractive
  account restating the facts at various places in the grammar. Ryu assumes features for singling
  out the subject and the secondary object and because of this he can distinguish between primary
  and secondary object in the principle responsible for case assignment (pp.\,205--206). For
  detailed criticism of \citew{Ryu97a} and \citew{Gunkel2003b} see \citew[Section~3.4]{Mueller2003e}
  and \citew[Chapter~14.3.1]{MuellerLehrbuch3}, respectively.  
} In addition, there is a problem with bivalent verbs. While
some verbs take the dative, others take the accusative, though there is hardly any semantic
difference or any other reason that could be made responsible.
\eal
\ex 
\gll Er hilft ihm.\\
     he helps him\\
\ex 
\gll Er unterstützt ihn.\\
     he supports him\\
\zl
The fact that \emph{helfen} takes a dative object while \emph{unterstützen} `to support' takes an accusative is
just an idiosyncrasy of \ili{German} that speakers of \ili{German} have to learn when they acquire the
language. Thus, the information in the lexical entry for \emph{helfen} `to help' must be different from the one
for \emph{unterstützen}. Some authors acknowledge this difference and assume that the dative of
bivalent verbs is lexical, while the dative of ditransitive verbs is structural \citep[\page 48, 51]{Wunderlich97a}. The assumption is
that verbs assign the nominative to their first argument, the accusative to their last argument and
if there is an additional argument that is neither the first nor the last, it gets dative. The
prediction that such mixed accounts make is that the dative passive should be possible with
ditransitive verbs but impossible with bivalent verbs, since the dative is structural for the former
verbs and lexical for the latter. The empirical situation is not as clear-cut as one might
wish. Some authors accept examples like (\mex{1}); others reject them.
\eal
\label{ex-helfen-with-dative-passive}
\ex 
\gll Er kriegte von vielen geholfen / gratuliert / applaudiert.\\
     he got by many helped {} congratulated {} applauded\\
\ex 
\gll Man kriegt täglich gedankt.\\
     one gets   daily thanked\\
\zl
%\largerpage[2]
However, there are attested examples:\footnote{
  These examples were first discussed in \citew[\page 134--135]{Mueller2002b} and \citew[\page 293]{MuellerLehrbuch1}.
}
\eal
\label{ex-helfen-with-dative-passive-corpus}
\ex "`Da kriege ich geholfen."'\footnotemark\\
     \quotespace{}there get  I  helped\\
\footnotetext{
Frankfurter Rundschau, 26.06.1998, p.\,7.%
}
\glt `Somebody helps me there.'
\ex
% auch nach applaudiert geholfen + bekommen und kriegen gesucht 21.09.2003
\gll Heute morgen bekam ich sogar schon gratuliert.\footnotemark\\
     today morning \AUX{}  I even already congratulated\\
\footnotetext{%
Brief von Irene G.\ an Ernst G.\ vom 10.04.1943, Feldpost-Archive mkb-fp-0270.}
\glt `Somebody even wished me a happy birthday this morning already.'
\ex
%Branich IG-Vorsitzender Friedel Schönel meinte deshalb, 
\gll "`Klärle"' hätte es wirklich mehr als verdient, auch mal zu einem "`unrunden"' Geburtstag gratuliert zu bekommen.\footnotemark\\
     \hphantom{"`}Klärle had it really more than deserved also once to a \hphantom{"`}insignificant birthday congratulated to \AUX\\
\footnotetext{
Mannheimer Morgen, 28.07.1999, Lokales; "`Klärle"' feiert heute Geburtstag.%
}
\glt `Klärle would have more than deserved to be wished a happy birthday, even an insignificant birthday.'
\ex 
\gll Mit dem alten Titel von Elvis Presley "`I can't help falling in love"' bekam Kassier Markus Reiß zum Geburtstag gratuliert, [\ldots]\footnotemark\\
%der dann noch viel später bekannte: "Ich hab' immer noch Gänsehaut, das war der schönste Teil meines Geburtstages." Doch auch die anderen Abteilungen des Bürgervereins können auf ein erfolgreiches Jahr 1998 zurückblicken.
     with the old song   by  Elvis Presley \hphantom{"`}I can't help falling in love got cashier Markus Reiß to.the birthday congratulated\\
\footnotetext{
Mannheimer Morgen, 21.04.1999, Lokales; Motor des gesellschaftlichen Lebens.%
}
\glt `The cashier Markus Reiß was wished a happy birthday with the old Elvis Presley song ``I can't help falling in love with you''.'
\zl
It appears that the verbs \emph{kriegen}, \emph{erhalten}, and \emph{bekommen} are on the way to become
auxiliaries. Their meaning is getting more and more bleached. Hence, there are almost no selectional
restrictions left on the downstairs verb. The only requirement for the dative passive to apply is of
course that the embedded verb governs a dative.

%\largerpage[2]
Now, if the dative passive is possible with bivalent verbs like \emph{helfen} and if \emph{helfen}
has to govern a lexical dative (since otherwise the difference between \emph{helfen} and
\emph{unterstützen} could not be explained)\footnote{%
  But see footnote~\ref{fn-structural-dative}. Additional features could be assumed or subtypes of
  structural case.}, 
it follows that the dative passive must be able to
convert a lexical dative into a structural case (realized as nominative in the examples above). This
means that one could assume that all datives are lexical, even the datives of ditransitive
verbs. This explains why these datives are not realized as nominatives or accusatives in passives
like (\mex{1}):
\eal
\ex[]{ 
\gll dass er dem Jungen den Ball gegeben hat\\
     that he.\NOM{} the.\DAT{} boy    the.\ACC{} ball given has\\
}
\ex[]{ 
\gll dass dem Jungen der Ball gegeben wurde\\
     that the.\DAT{} boy    the.\NOM{} ball given \AUX\\
}
\ex[*]{ 
\gll dass der Junge den Ball gegeben wurde\\
     that the.\NOM{} boy   the.\ACC{} ball given \AUX\\
}
\ex[*]{ 
\gll dass den Junge der Ball gegeben wurde\\
     that the.\ACC{} boy   the.\NOM{} ball given \AUX\\
}
\zl
They simply remain in the dative. The only exception is the dative passive, and this has to be regarded as an exception.%
\is{case!dative|)}%


%% \subsubsubsection{Non-canonical accusative objects}

%% I already showed that the accusative can be a structural case. However, there are some exceptional
%% cases like those of subjectless verbs that govern an accusative (\mex{1}a) and ditransitive
%% verbs like \emph{lehren} `to teach' that take two accusative objects rather than one accusative and a dative.
%% \eal
%% \ex 
%% \gll Ihn dürstet.\\
%%      he.\ACC{} thirsty.is\\
%% \glt `He is thirsty.'
%% \ex 
%% \gll Der Vater lehrte seinen Sohn einen neuen Tritt.\\
%%      the.\NOM{} father taught his.\ACC{} son a.\ACC{} new kick\\
%% \glt `The father taught his son a new kick.'
%% \zl
%% Verbs like \emph{lehren} are generally bad in the passive.


%% \subsubsubsection{Adjektivumgebungen}

%% 
%% \subsubsubsection{Lexikalischer Kasus in Adjektivumgebungen}

%% Kasus von Objekten von Adjektiven kann sich nicht ändern.\\
%% Adjektive können Genitiv und Dativ zuweisen:
%% \eal
%% \ex Ich war mir \emph{dessen} sicher.
%% \ex Sie ist \emph{ihm} treu.
%% \zl
%% 
%% Die Zuweisung von Akkusativ ist ebenfalls möglich:
%% \eal
%% \ex Das ist \emph{diesen Preis} nicht wert.
%% \ex Der Student ist \emph{das Leben} im Wohnheim nicht gewohnt.\iw{gewohnt}\footnote{
%%         \citep*[S.\,312]{HB72a}
%%       }
%% \ex Du bist mir \emph{eine Erkl"arung} schuldig.\footnote{
%%         \citep*[S.\,620]{HFM81}
%%       }
%% \zl
%% Akkusativ ist bei Adjektivkomplementen aber selten \citep{Haider85b}.
%% }

%% 
%% \subsubsubsection{Struktureller Kasus in Adjektivumgebungen}


%% Kasus der Subjekte von Adjektiven hängt von der syntaktischen
%% Umgebung ab \citep{Wunderlich84}:
%% \eal
%% \ex \emph{Der Mond} wurde kleiner.\iw{klein}
%% \ex Er sah\iw{sehen} \emph{den Mond} kleiner werden.
%% \zl

%% }


%\if 0


%% \subsubsubsection{Semantische Kasus}
%% \label{sec-sem-kasus}
%% \is{Kasus!semantischer|(}


%% \subsubsubsection{Semantische Kasus}

%% \begin{itemize}
%% \item NPen können auch als Adjunkte auf"|treten \citep{Haider85b}:
%% \eal
%% \ex Sie hörten \emph{den ganzen Tag} dieselbe Schallplatte.
%% \ex Laßt \emph{mir} den Hund in Ruhe!
%% \ex \emph{Eines Tages} erschien ein Fremder.
%% \zl

%% \item auch der Urteilsdativ (\emph{Dativ iudicantis})  \citep{Wegener85b}:

%% \eal
%% \ex Das ist \emph{mir} zu\iw{zu!Grad} schwer.
%% \ex Das ist \emph{dem Kind} zu langweilig / nicht interessant genug.\iw{genug!Grad}
%% \ex Du läufst \emph{der Oma} zu\iw{zu!Grad} schnell.
%% \ex Das Wasser ist \emph{dem Baby} warm genug.\iw{genug!Grad}
%% \zl
%% \end{itemize}

%% \subsubsubsection{Zuweisung semantischer Kasus durch das Verb?}

%% \begin{itemize}
%% \item
%% Haider: Zuweisung durch Verb in (\mex{1}) nicht sinnvoll:
%% \ea
%% Sie hörten \emph{den ganzen Tag} dieselbe Schallplatte.
%% \z
%% Zeitangaben kommen auch in adjektivischen und nominalen Umgebungen vor:
%% % zitiert Toman83
%% \eal
%% \ex die Ereignisse \emph{letzten Sommer}
%% \ex der Flirt \emph{vorigen Dienstag}
%% \ex die \emph{diesen Sommer} sehr günstige Witterung
%% \ex die \emph{diesen Sommer} sehr teuren Urlaubsreisen
%% \zl
%% NPen mit strukturellem Kasus müssen in Nominalumgebungen
%% Genitiv sein. $\to$\\
%% In (\mex{0}) keine Zuweisung von strukturellem Kasus.


%% \item
%% Die Kasus in (\mex{0}) werden nicht aufgrund ihres Vorkommens in einer bestimmten
%% syntaktischen Struktur zugewiesen,\\
%% sondern sind vielmehr durch die Bedeutung des Nomens bestimmt.
%% \end{itemize}



%% \subsubsubsection{Akkusativ und Genitiv}

%% %\citep*{ZMT85a} -> semantische Kasusmarkierung
%% Der freie Akkusativ kommt bei Maß"-angaben\is{Maßangaben} (Zeitdauer und Zeitpunkt)
%% vor (\mex{1}) und Genitiv bei Lokalangaben oder Zeitangaben (\mex{2}).
%% \eal
%% \ex Sie studierte \emph{den ganze Abend}.
%% \ex \emph{Nächsten Monat}\iw{Monat} kommen wir.
%% \zl
%% \eal
%% \ex Ein Mann kam \emph{des Weges}.\iw{Weg}
%% \ex \emph{Eines Tages}\iw{Tag} sah ich sie wieder.
%% \zl




%% \subsubsubsection{Kongruenzkasus}

%% 
%% \subsubsubsection{Kongruenzkasus}

%% \begin{itemize}
%% \item Zwei Akkusative?
%% \eal
%% \ex Er nannte \emph{ihn} \rot{einen Experten}.
%% \ex \emph{Er} wurde \rot{ein Experte} genannt.
%% \zl
%% 
%% \item Wären das zwei unabhängige Akkusative,\\
%%       würde sich bei Passivierung nur einer ändern.

%% 
%% \item Kasus von \emph{einen Experten} wird \emph{Kongruenzkasus} genannt.\\
%% Die prädikative Phrase \emph{einen Experten} stimmt mit dem
%% Element,\\ über das prädiziert wird, im Kasus überein. 
%% \end{itemize}
%% }

%% 
%% \subsubsubsection{Kongruenzkasus mit Präpositionen}

%% Ähnliche Effekte kann man mit den Präpositionen \emph{als} und \emph{wie}
%% beobachten.
%% \eal
%% \ex \emph{Er} gilt als \rot{großer Künstler}.\footnote{
%%         \citew[S.\,203--204]{Heringer73a}.
%%       }
%% \ex Man lässt \emph{ihn} als \rot{großen Künstler} gelten\iw{gelten als}.
%% \zl
%% 
%% \eal
%% \ex Ich sehe \emph{ihn} als \rot{meinen Freund} an.\iw{ansehen}\footnote{
%%         \citew*[S.\,154]{SS88a}.
%% }
%% \ex \emph{Er} wird als \rot{mein Freund} angesehen.
%% \zl
%% }

%% 
%% \subsubsubsection{Kongruenzkasus mit Adjunkten}

%% Wie bei den prädikativen Argumenten gibt es auch Kongruenzkasus bei Adjunkten:
%% \eal
%% \ex Sie verhielt\iw{verhalten} \emph{sich} wie \emph{ihr Vater}.
%% \ex Ich behandelte\iw{behandeln} \emph{ihn} wie \emph{meinen Bruder}.
%% \ex Ich half\iw{helfen} \emph{ihm} wie \emph{einem Freund}.
%% \ex Ich erinnerte\iw{erinnern} mich \emph{dessen} wie \emph{eines fernen Alptraums}.
%% \zl

%% }

%% 
%% \subsubsubsection{Prädikation = Kasuskongruenz?}

%% \begin{itemize}
%% \item Kongruieren prädikative Phrasen immer mit dem Element,\\
%%       über das sie prädizieren?
%% \item Dies würde sofort auch Beispiele wie das in (\mex{1}) erklären:
%% \ea
%% Er wird ein großer Linguist.
%% \z
%% 
%% \item In AcI"=Konstruktionen müßten beide NPen im Akkusativ stehen.\\
%% Das ist nicht der Fall:
%% \eal
%% %\ex Laß ihn einen großen Linguisten werden.\label{bsp-lass-ihn-einen-grossen}
%% \ex Laß\iw{lassen|(} den wüsten Kerl [\ldots] meinetwegen ihr Komplize sein.\footnote{
%%         (\ref{bsp-lass-den-wuesten-kerl}) und (\ref{bsp-lass-mich}) sind aus dem \citet*[{\S}\,6925]{Duden66}.\iaf{Duden} %\citet*[{\S}\,1473]{Duden73}.\iaf{Duden}
%%         Die Quellen finden sich dort.
%%       }\label{bsp-lass-den-wuesten-kerl}
%% \ex Laß mich dein treuer Herold sein.\label{bsp-lass-mich}
%% \ex Baby, laß\iw{lassen|)} mich dein Tanzpartner sein.\footnote{
%%         Funny van Dannen, Benno-Ohnesorg-Theater, Berlin, Volksbühne, 11.10.1995
%%         }
%% \zl
%% 
%% \item
%% $\to$ Nominativ des Nicht-Subjekts in Kopulakonstruktionen ist\\
%%       ein lexikalischer Kasus \citep[S.\,54]{Thiersch78a}.
%% \end {itemize}
%% }

%% \subsubsubsection{Der Kasus nicht ausgedrückter Subjekte}
%% \label{sec-kasus-nicht-realisierter-subj}

%% 
%% \subsubsubsection{Der Kasus nicht ausgedrückter Subjekte (I)}
%% \savespace

%% \begin{itemize}
%% \item \citet*[Kapitel~6]{Hoehle83}:\\
%% Kasus nicht an der Oberfläche auf"|tretender Elemente bestimmbar.

%% {\em ein- nach d- ander-\/} kann sich auf mehrzahlige Konstituenten beziehen. 

%% Dabei muß Kasus und Genus mit der Bezugsphrase übereinstimmen.
%% 
%% \item In (\mex{1}) Bezug auf Subjekte bzw.\ Objekte:
%% \eal
%% \ex Die Türen sind eine nach der anderen kaputtgegangen.
%% \ex Einer nach dem anderen haben wir die Burschen runtergeputzt.
%% \ex Einen nach dem anderen haben wir die Burschen runtergeputzt.
%% \ex Ich ließ die Burschen einen nach dem anderen einsteigen.
%% \ex Uns wurde einer nach der anderen der Stuhl vor die Tür gesetzt.
%% \zl
%% \end{itemize}
%% }

%% 
%% \subsubsubsection{Der Kasus nicht ausgedrückter Subjekte (II)}
%% \savespace

%% In (\mex{1}) Bezug auf Dativ- bzw.\ Akkusativobjekte
%% eingebetteter Infinitive:

%% \eal
%% \ex Er hat uns gedroht, die Burschen demnächst einen nach dem anderen wegzuschicken.
%% \ex Er hat angekündigt, uns dann einer nach der anderen den Stuhl vor die Tür zu setzen.
%% \ex Es ist nötig, die Fenster, sobald es geht, eins nach dem anderen auszutauschen.
%% \zl

%% }

%% 
%% \subsubsubsection{Der Kasus nicht ausgedrückter Subjekte (III)}
%% \savespace

%% In (\mex{1}) Bezug auf Subjekt innerhalb der Infinitiv"=VP:
%% \eal
%% \ex Ich habe den Burschen geraten, im Abstand von wenigen Tagen einer nach dem anderen
%%       zu kündigen.
%% \ex Die Türen sind viel zu wertvoll, um eine nach der anderen verheizt zu werden.
%% \ex Wir sind es leid, eine nach der anderen den Stuhl vor die Tür gesetzt zu kriegen.
%% \ex Es wäre fatal für die Sklavenjäger, unter Kannibalen zu fallen und einer nach dem
%%       anderen verspeist zu werden.
%% \zl
%% {\em ein- nach d- ander-\/} im Nominativ $\to$\\
%% Das nicht realisierte Subjekt steht ebenfalls im Nominativ.

%% }


%% 
%% \subsubsubsection{Der Kasus nicht ausgedrückter Subjekte (IV)}

%% Dasselbe gilt für nicht realisierte Subjekte von adjektivischen Partizipien:
%% \eal
%% \ex die eines nach dem anderen einschlafenden Kinder
%% \ex die einer nach dem anderen durchstartenden Halbstarken
%% \ex die eine nach der anderen loskichernden Frauen
%% \zl
%% }

%% 
%% \subsubsubsection{Der Kasus nicht ausgedrückter Subjekte (V)}

%% Man muß also sicherstellen, daß auch nicht realisierte Subjekte Kasus zugewiesen bekommen.
%% Würde man diesen Kasus unterspezifiziert lassen, würden Sätze wie (\mex{1}) falsch analysiert werden.
%% \judgewidth{\#}
%% \ea[\#]{
%% Ich habe den Burschen geraten, im Abstand von wenigen Tagen einen nach dem anderen zu kündigen.
%% }
%% \z
%% In der zulässigen Lesart von (\mex{0}) ist die Phrase 
%% \emph{einen nach dem anderen} das Objekt von \emph{kündigen} und kann
%% sich nicht auf das Subjekt des Infinitivs, das referenzidentisch
%% mit \emph{den Burschen} ist, beziehen.

%% }



After this discussion of lexical and structural case in \ili{German}, let us now move on to the Case Principle,
which is responsible for case assignment. As was explained in Section~\ref{sec-linking}, it is assumed that all arguments of a
head are represented in one list: the \textsc{argument structure} list (\argst list\isfeat{arg-st}). 
(\mex{1}) shows the argument structure list of a ditransitive verb like \word{geben} `to give':
\ea
\sliste{ NP[\type{str}], NP[\type{ldat}], NP[\type{str}] }
\z
As was argued above, dative is treated as a lexical case. \type{ldat} is an abbreviation for lexical
dative and \type{str} stands for for structural case. The Case
Principle has the following form (adapted from \citealp{Prze99}; \citealp{Meurers99b}):

\begin{principle-break}[\hypertarget{case-p}{Case Principle}]\is{Principle!Case}
\label{case-p}
\begin{itemize}
\item In a list that contains both the subject and the complements of a verbal head, the first
  element with structural case gets nominative unless it is raised by a higher head.
\item All other elements in this list that have structural case and are not raised get accusative.\is{case!accusative}
\item In nominal environments elements with structural case get genitive.\is{case!genitive}
\end{itemize}
\end{principle-break}
This principle is inspired by \citet*{YMJ87} and, as will be demonstrated below, it works for all of
the languages analyzed here, in particular, for the complex case system of \ili{Icelandic}. The case
system assumed here differs in not assigning case to elements that are raised
to a higher predicate. This point will be explained in more detail below.

\largerpage
The effect of this principle will be explained with respect to the verbs in (\mex{1}):
\ea
\begin{tabular}[t]{@{}l@{~}l@{~}l@{}}
a. & \emph{schläft} `sleep':       & \argst \sliste{ NP[\type{str}]$_i$ }\\[2mm]
b. & \emph{unterstützt} `support': & \argst \sliste{ NP[\type{str}]$_i$, NP[\type{str}]$_j$ }\\[2mm]
c. & \emph{hilft} `help':          & \argst \sliste{ NP[\type{str}]$_i$, NP[\type{ldat}]$_j$ }\\[2mm]
d. & \emph{schenkt} `give as a present':     & \argst \sliste{ NP[\type{str}]$_i$, NP[\type{ldat}]$_j$, NP[\type{str}]$_k$ }\\
\end{tabular}
\z
The first element in these lists that has structural case gets nominative and the second one
accusative. This is exactly what one expects. The result is given in (\mex{1}). \type{snom} stands
for structural nominative, and \type{sacc} for structural accusative.
\ea
%\oneline{%
\scalebox{.99}{%
\begin{tabular}[t]{@{}l@{~}l@{~}l@{}}
a. & \emph{schläft} `sleep':       & \argst \sliste{ NP[\type{snom}]$_i$ }\\[2mm]
b. & \emph{unterstützt} `support': & \argst \sliste{ NP[\type{snom}]$_i$, NP[\type{sacc}]$_j$ }\\[2mm]
c. & \emph{hilft} `help':          & \argst \sliste{ NP[\type{snom}]$_i$, NP[\type{ldat}]$_j$ }\\[2mm]
d. & \emph{schenkt} `give as a present':     & \argst \sliste{ NP[\type{snom}]$_i$, NP[\type{ldat}]$_j$, NP[\type{sacc}]$_k$ }\\
\end{tabular}
}
\z
\is{case!structural|)}\is{case!lexical|)}

\subsection{Argument reduction and case assignment: The passive}
\label{sec-case-assignment-passive}

Given the distinction between structural and lexical case, the analysis of the passive is really simple and
directly corresponds to the intuition that the passive is the suppression of the subject (the most
prominent, that is, the first argument in the \argstl).\footnote{
  Some authors assume a demotion of the subject, that is, the subject is turned into a complement \emph{by}-PP
  or a \emph{von}-PP \citep[\page 216]{ps}. I follow \citet[\page161]{Hoehle78a}, \citet{Sadzinski87a}, \citet[\page174]{Stechow90a}, 
\citet[\page255]{Zifonun92a}, 
%\NOTE{\citet[\page86]{Leiss92a},}
\citet[\page181]{Lieb92a},
\citet[\page740]{Wunderlich93a}, 
%\NOTE{\citet[\page150]{Primus99a},}
%cannot follow myself \citet{Mueller2003e}, 
\citet[\page 65]{Gunkel2003b} and others, in assuming that the PPs for expressing the agent are adjuncts. See
  \citew[Section~5]{Mueller2003e} for details.
} If the first argument is removed from the
lists in (\mex{-1}), the following lists result:
\ea
\begin{tabular}[t]{@{}l@{~}l@{~}l}
a. & \emph{geschlafen}:  & \argst \sliste{ }\\[2mm]
b. & \emph{unterstützt}: & \argst \sliste{ NP[\type{str}]$_j$ }\\[2mm]
c. & \emph{geholfen}:    & \argst \sliste{ NP[\type{ldat}]$_j$ }\\[2mm]
d. & \emph{geschenkt}:   & \argst \sliste{ NP[\type{ldat}]$_j$, NP[\type{str}]$_k$ }\\
\end{tabular}
\z
The NPs that are in the first position in (\mex{0}) where in the second position in (\mex{-1}). The
first NP with structural case gets nominative and hence the following case assignments result:
\ea
\begin{tabular}[t]{@{}l@{~}l@{~}l}
a. & \emph{geschlafen}:  & \argst \sliste{ }\\[2mm]
b. & \emph{unterstützt}: & \argst \sliste{ NP[\type{snom}]$_j$ }\\[2mm]
c. & \emph{geholfen}:    & \argst \sliste{ NP[\type{ldat}]$_j$ }\\[2mm]
d. & \emph{geschenkt}:   & \argst \sliste{ NP[\type{ldat}]$_j$, NP[\type{snom}]$_k$ }\\
\end{tabular}
\z
As lexical case as in (\mex{0}c--d) is not affected by the case principle, it stays the way it was
specified, namely dative.

\largerpage
It should be noted here that this simple approach to the passive accounts both for the so-called
personal and the impersonal passive. The passives of \emph{schlafen} `to sleep' and \emph{helfen}
`to help' are called impersonal passives, as the respective clauses do not have a subject. 
\eal
\ex 
\gll dass geschlafen wurde\\
     that slept      \AUX\\
\glt `that there was sleeping there'
\ex
\gll dass dem Mann geholfen wurde\\
     that the.\DAT{} man helped \AUX\\
\glt `that the man was helped'
\zl
The passives of \emph{unterstützen} `to support' and \emph{schenken} `to give as a present' do have
subjects, namely the arguments that are realized as accusative objects in the active:
\eal
\ex 
\gll dass der Mann unterstützt wurde\\
     that the.\NOM{} man supported \AUX\\
\glt `that the man was supported'
\ex
\gll dass dem Jungen der Ball geschenkt wurde\\
     that the.\DAT{} boy the.\NOM{} ball given \AUX\\
\glt `that the ball was given to the boy as a present'
\zl
Those analyses that assign all cases lexically would have to assume that the case of the objects
(accusative) is changed into nominative in the passive. Hence, there would be two variants of the
passive: The impersonal passive just suppresses the subject and the personal passive suppresses the
subject and additionally changes the case of the object into nominative. The analysis using the
structural/lexical case distinction just postpones the case assignment until the point where it is
clear what the right case will be. If we have a participle and use it with the passive auxiliary it
is clear what the case of the arguments has to be.



%% \subsubsubsubsection{Dativpassiv}

%% \frame[shrink=15]{
%% \subsubsection{Dativpassiv}

%% Bei der Kombination von \emph{geholfen} und
%% \emph{bekommen} bzw.\ von \emph{geschenkt} und \emph{bekommen} wird das Dativargument von 
%% \emph{geholfen} bzw.\ von \emph{geschenkt} zum ersten Argument gemacht und der lexikalische
%% Dativ beim eingebetteten Verb wird zu einem strukturellen Kasus beim Passiv"=Hilfsverb:
%% \ea
%% \begin{tabular}[t]{@{}l@{~}l@{~}l}
%% c. & \emph{hilft}:       & \argst \sliste{ NP[\type{str}]$_j$, NP[\type{ldat}]$_k$ }\\
%% d. & \emph{schenkt}:     & \argst \sliste{ NP[\type{str}]$_j$, NP[\type{str}]$_k$, NP[\type{ldat}]$_l$ }\\
%% \end{tabular}
%% \z
%% \ea
%% \begin{tabular}[t]{@{}l@{~}l@{~}l}
%% a. & \emph{geholfen bekommt}:    & \argst \sliste{ NP[\type{str}]$_k$ }\\
%% b. & \emph{geschenkt bekommt}:   & \argst \sliste{ NP[\type{str}]$_l$, NP[\type{str}]$_k$ }\\
%% \end{tabular}
%% \z
%% Details kommen im Kapitel über Passiv.

%% Kasusvergabe: Dadurch, daß das Dativargument an erster Stelle in der Valenzliste\\
%% von \emph{geholfen bekommen} bzw.\ von \emph{geschenkt bekommen} steht, kriegt es
%% Nominativ. 

%% Bei \emph{geschenkt bekommen} bekommt das zweite Element (das direkte Objekt) Akkusativ.

%% Die Umwandlung eines lexikalischen in einen strukturellen Kasus ist unschön,\\
%% es scheint zur Zeit jedoch keine bessere Alternative zu geben. 

%% }


\subsection{Argument extension and case assignment: AcI constructions}

The case principle contains restrictions on case assignment that prohibit the assignment to
elements that are raised. These restrictions have not been explained yet. Consider the examples in (\mex{1}):
\eal
\ex
\gll Der Junge liest den Aufsatz.\\
     the.\NOM{} boy reads the.\ACC{} paper\\
\glt `The boy reads the paper.'
\ex\label{ex-the-man-lets-the-boy-read-the-book}
\gll Der Mann lässt den Jungen den Aufsatz lesen.\\
     the.\NOM{} man lets the.\ACC{} boy the.\ACC{} paper read\\
\glt `The man lets the boy read the paper.'
\zl
The example (\mex{0}a) shows that the subject of \emph{lesen} is
assigned nominative. However, the subject of \emph{lesen} gets accusative in (\mex{0}b). So, if
one were to assign case on the basis of the argument structure of \emph{lesen} in (\mex{0}b), one
would assign nominative, but the AcI verb\is{verb!AcI}\is{verb!causative} \emph{lassen} `to let' assigns accusative to its object. The
point is that the subject of \emph{lesen} is raised to the object of \emph{lassen}. The Case
Principle is set up in a way such that case is assigned only to those arguments that are not raised
to a higher head. Hence, \emph{den Jungen} does not get case from \emph{lesen}, but from \emph{lässt}.

The analysis of (\mex{0}b) is given in Figure~\vref{fig-vc-aci}.
\begin{figure}
\centerfit{
\begin{forest}
sm edges
[V\feattab{
              \sliste{ }}
        [{NP[\type{snom}]} [der Mann;the man, roof] ]
        [V\feattab{
              \sliste{ NP[\type{snom}] }}
          [{NP[\type{sacc}]} [den Jungen;the boy, roof] ]
          [V\feattab{
%              \vform \type{fin},\\
              \sliste{ NP[\type{snom}], NP[\type{sacc}]}} 
            [{NP[\type{sacc}]} [den Aufsatz;the paper, roof] ]
            [V\feattab{
%                \vform \type{fin},\\
                \sliste{ NP[\type{snom}], NP[\type{sacc}], NP[\type{sacc}]}} 
               [~~~~~~V\feattab{
%                \vform \type{bse},\\
                 \sliste{ NP[\type{sacc}], NP[\type{sacc}]}} [lesen;read] ]
               [V\feattab{
%                \vform \type{fin},\\
                  \sliste{ NP[\type{snom}], NP[\type{sacc}], NP[\type{sacc}], V }} [lässt;lets] ] ] ] ] ]
\end{forest}}
\caption{\label{fig-vc-aci}Analysis of AcI constructions as raising constructions and the verbal
  complex in German}
\end{figure}
The arguments of \emph{lesen} `to read' are taken over by \emph{lässt}. Since \emph{lässt}
contributes its own argument, the causer or the one who gives the permission, \emph{lässt} selects
for three NPs with structural case and a verb in the specific sentence depicted in
Figure~\ref{fig-vc-aci}. According to the Case Principle, the first NP with structural case gets
nominative and the other NPs with structural case get accusative. This results in a list with one
NP in the nominative and two NPs in the accusative.

(\mex{1}) shows the \argstl of \emph{lässt} when it is combined with \emph{schlafen},
\emph{unterstützen}, \emph{helfen}, or \emph{schenken}, respectively.
\ea
\onelineea{%
\begin{tabular}[t]{@{}l@{~}l@{~}l@{}}
a. & \emph{lässt} + \emph{schlafen}:      & \argst \sliste{ NP[\type{str}]$_l$, NP[\type{str}]$_i$, V}\\[2mm]
b. & \emph{lässt} + \emph{unterstützen}: & \argst \sliste{ NP[\type{str}]$_l$, NP[\type{str}]$_i$, NP[\type{str}]$_j$, V }\\[2mm]
c. & \emph{lässt} + \emph{helfen}:        & \argst \sliste{ NP[\type{str}]$_l$, NP[\type{str}]$_i$, NP[\type{ldat}]$_j$, V }\\[2mm]
d. & \emph{lässt} + \emph{schenken}:      & \argst \sliste{ NP[\type{str}]$_l$, NP[\type{str}]$_i$, NP[\type{ldat}]$_j$, NP[\type{str}]$_k$, V }\\
\end{tabular}
}
\z
\largerpage[-1]
The NP that is added has the index \type{l}. As the first NP with structural case on these lists it
gets nominative. All other elements of this list that have structural case get accusative. Hence the
subject of the embedded verb is assigned accusative, the lexical cases stay the same and the
accusative objects of the embedded verb get accusative as well, since their case is structural too.

%\largerpage
Note that the question of whether a language has a verbal complex or not is orthogonal to issues of
case assignment. Figure~\vref{fig-the-man-lets-the-boy-read-the-book} shows the analysis of the \ili{English} translation of (\ref{ex-the-man-lets-the-boy-read-the-book}).
\begin{figure}
\centerfit{
\begin{forest}
sm edges
[S
        [{NP[\type{snom}]} [the man, roof] ]
        [VP
          [\vbar [V [lets] ]
                 [{NP[\type{sacc}]} [the boy,roof] ] ]
              [VP
                [V [read] ]
                [{NP[\type{sacc}]} [the book, roof] ] ] ] ]
\end{forest}}
\caption{\label{fig-the-man-lets-the-boy-read-the-book}AcI constructions in English}
\end{figure}
\emph{let} selects for the subject, the object and a VP. The subject of \emph{read} is
simultaneously the object of \emph{let} and hence the Case Principle does not assign nominative to
the subject of the embedded verb \emph{read}, but accusative to the object of the matrix verb \emph{let}.%
\is{case|)}

%% 
%% \subsubsubsection{Adjektivsubjekte}


%% Die Kasuszuweisungen an das Subjekt von Adjektiven funktioniert analog. Die Kopula wird mit dem Adjektiv
%% verbunden, und es entsteht eine Valenzliste, die die Argumente des Adjektivs enthält (\mex{1}a).\\
%% Wird ein solcher Komplex noch unter ein AcI"=Verb wie \emph{sehen} eingebettet,\\
%% erhält man die Liste in (\mex{1}b):
%% \ea
%% \begin{tabular}[t]{@{}l@{~}l@{~}l}
%% a. & \emph{kleiner werden}:     & \argst \sliste{ NP[\type{str}]$_j$ }\\
%% b. & \emph{kleiner werden sah}: & \argst \sliste{ NP[\type{str}]$_i$, NP[\type{str}]$_j$ }\\
%% \end{tabular}
%% \z
%% Die Kasuszuweisung funktioniert analog zu den bereits diskutierten Fällen. In den verbalen Umgebungen
%% der Kopula bzw.\ des AcI"=Verbs bekommen die NPen mit strukturellem Kasus Nominativ bzw.\ Akkusativ.%
%% }


%% \subsubsubsection{Semantischer Kasus}
%% 
%% \subsubsubsection{Semantischer Kasus (I)}


%% Der Kasus von NPen wie \emph{den ganzen Tag} in (\mex{1}) ist von der syntaktischen Umgebung unabhängig.
%% \eal
%% \ex Sie arbeiten den ganzen Tag.
%% \ex Den ganzen Tag wird gearbeitet, [\ldots].\footnote{
%%   \url{http://www.philo-forum.de/philoforum/viewtopic.html?p=146060}. \urlchecked{12}{05}{2005}.
%% }
%% \zl
%% Daß die NP im Akkusativ steht, hängt mit ihrer Funktion zusammen. 

%% Unterschiedliche Lexikoneinträge für \emph{Tag} in (\mex{0}) und (\mex{1}):
%% \eal
%% \ex Ich liebe diesen Tag.
%% \ex Dieser Tag gefällt mir.
%% \zl

%% In (\mex{0}) liegen ganz gewöhnliche Argumente vor,\\
%% in (\mex{-1}) dagegen ein Adjunkt. 
%% }

%% 
%% \subsubsubsection{Semantischer Kasus (II)}

%% Adjunkte unterscheiden sich von Argumenten durch ihren \modw und durch ihern \contw.

%% Für (\mex{-1}) muß es unter \cont eine Dauer-Relation geben.

%% Zusammen mit dieser Information wird im Lexikoneintrag für das modifizierende Nomen der Kasus fest kodiert. 

%% Die morphologische Komponente kann dann für diesen Lexikoneintrag nur die Akkusativform erzeugen, 
%% da alle anderen Flexionsformen mit der bereits im Lexikoneintrag angegebenen Kasusinformation inkompatibel sind. 

%% Dadurch wird sichergestellt, daß Sätze wie (\mex{1}) nicht analysiert werden:
%% \eal
%% \ex[*]{
%% Er arbeitet der ganze Tag.
%% }
%% \ex[*]{
%% weil der ganze Tag gearbeitet wurde
%% }
%% \zl

%% }




\subsection{Accounting for the crosslinguistic differences}

As argument reduction and case assignment was already explained for \ili{German} in
Section~\ref{sec-case-assignment-passive}, I would like now to speak more directly on and provide 
lexical items for the passive and perfect auxiliary for \ili{German}. After this I discuss the other
languages (\eg \ili{Danish}, \ili{English}, and \ili{Icelandic}) and explain how the differences can be accounted for analytically.

\subsubsection{Designated argument reduction}

\largerpage
\citet[\page 10]{Haider86} suggested marking the argument of a verb that has subject properties. He calls
these special arguments \emph{designated argument}. \citet{HM94a} transferred this idea to HPSG and
\citet{Mueller2003e} modified it slightly to get certain facts with modal infinitives
right. One important use of the designated argument is to distinguish so"=called
unaccusative verbs\is{verb!unaccusative} from unergative\is{verb!unergative} verbs. \citet{Perlmutter78} pointed out that unaccusative verbs
have remarkable properties and argued that their subjects are not really subjects but behave more
like objects. One of their properties is that they do not allow for passives. Furthermore, their
participles can be used attributively, which is not possible with unergative verbs:

\eal
\ex[]{ 
\gll der angekommene Zug\\
     the arrived     train\\\hfill(unaccusative)
\glt `the arrived train'
}
\ex[*]{
\gll der geschlafene Mann\\
     the slept man\\\hfill(unergative)
}
\zl

\noindent
This is explained if one assumes that the subject of \emph{ankommen} `arrive' has object-like
properties and hence patterns with the object of transitive verbs:
\ea
\gll der geliebte Mann\\
     the beloved  man\\
\z
%\largerpage
\emph{Mann} `man' fills the object slot of \emph{geliebte}. If the sole argument of \emph{ankommen}
is treated as an object, the similarity to the transitive \emph{lieben} is explained
immediately. Similarly, the fact that unaccusatives\is{verb!unaccusative} do not allow for passives is explained: If
passive is the suppression of the \isi{subject} and \emph{ankommen} does not have a subject in that sense,
passive cannot apply.

\eal
\ex[]{
\gll Der Zug ist angekommen.\\
     the train is arrived\\
\glt `The train arrived.'
}
\ex[*]{
\gll weil angekommen wurde\\
     because arrived \AUX\\
}
\zl

\noindent
In the HPSG analyses the authors assume that there is a list"=valued feature \textsc{designated
  argument} (\textsc{da}). This list contains the subject of transitive and unergative verbs
(intransitive verbs that are not unaccusative). The \dav of unaccusative\is{verb!unaccusative} verbs is the empty list,
since these verbs do not have an argument with subject properties.

The passive is analyzed as a lexical rule that licenses a lexical item for the participle. The
\argstl of the participle is the \argstl of the verb stem that is the input to the lexical rule minus
the \dalist. Since this is not the focus of this book, I will not discuss unaccusative verbs in the
following. (\mex{1}) provides some prototypical examples for unergative and transitive verbs:

%\largerpage[-1]
\ea
%\resizebox{\linewidth}{!}{%
\begin{tabular}[t]{@{}l@{ }l@{ }l@{ }l@{}}
  &                     & \textsc{arg-st} & \textsc{da}\\[2mm]
a.&tanzen (dance):   & \sliste{ \ibox{1}NP[\type{str}] }                                              & \sliste{ \ibox{1} }\\[2mm]
b.&lesen  (read):    & \sliste{ \ibox{1}NP[\type{str}], NP[\type{str}] }                              & \sliste{ \ibox{1} }\\[2mm]
c.&schenken (give as a present): & \sliste{ \ibox{1}NP[\type{str}], NP[\type{ldat}], NP[\type{str}] } & \sliste{ \ibox{1} }\\[2mm]
d.&helfen   (help):   & \sliste{ \ibox{1}NP[\type{str}], NP[\type{ldat}] }                            & \sliste{ \ibox{1} }\\
\end{tabular}
%}
\z
The lexical\is{$\mapsto$} rule that forms the participle is sketched in (\ref{lr-passive-prelim}):

\ea
\label{lr-passive-prelim}
Lexical rule\is{lexical rule!passive} for the formation of the participle (preliminary):\\
\ms[stem]{
head   & \ms[verb]{ da & \ibox{1}\\
                  }\\
arg-st & \ibox{1} $\oplus$ \ibox{2} \\
} $\mapsto$
\ms[word]{
arg-st & \ibox{2} \\
}
\z
This rule splits the \argstl of the input into two lists \ibox{1} and \ibox{2}. \ibox{1} is
identical to the \dav. Therefore the designated argument is taken off the \argstl and is not present
in the lexical item that is licensed by the rule.


The \argstl of the participle that is licensed is either empty (\mex{1}a) or starts with an object of the active form:
\ea
\label{partizipien-hm}
%\resizebox{\linewidth}{!}
\z
As was explained above, the first element in the \argstl with structural case gets nominative and
hence the accusative object of \emph{lesen} in (\mex{1}a) is realized as nominative in (\mex{1}b):
\eal
\ex
\gll Er liest den Aufsatz.\\
     he.\NOM{} reads the.\ACC{} paper\\\german
\ex
\gll Der        Aufsatz wurde  gelesen.\\
     the.\NOM{} paper   \AUX{} read\\
\zl

\ili{English} differs from \ili{German} in not having a dative case at all. I am talking about morphological
markings here, not about semantics. Therefore, both objects of \ili{English} ditransitive verbs are
accusative objects. However, only one of the objects can be promoted to subject. This is modeled in
the analysis at hand by assuming that the secondary object bears lexical accusative (see also \citew[\page 57]{Grewendorf2002a} for the assumption of lexical accusative for the secondary object in \ili{English}).\footnote{
  Admittedly this is just a restatement of the facts, since assigning lexical case means that the
  argument under consideration cannot have another case. But taken together with constraints on
  subjects in \ili{English} the facts about promotion or non-promotion of arguments follow nicely.
}

\ea\label{da-repr-hm-English}
%\resizebox{\linewidth}{!}{%
\begin{tabular}[t]{@{}l@{ }l@{ }l@{ }l@{ }l@{}}
  &                     & \textsc{arg-st}\\[2mm]
b.&dance   (unerg):     & \liste{ NP[\type{str}]}\\[2mm]
%c.&auf"|fallen (unacc): & \liste{}                         & \liste{NP[\type{str}], NP[\type{ldat}]}\\[2mm]
c.&read      (trans):   & \liste{ NP[\type{str}], NP[\type{str}]}\\[2mm]
d.&give      (ditrans): & \liste{ NP[\type{str}], NP[\type{str}], NP[\type{lacc}] }\\[2mm]
e.&help      (trans):   & \liste{ NP[\type{str}], NP[\type{str}] }\\
\end{tabular}
%}
\z
\ili{German} can promote the second object (accusative); \ili{English} the first object. The commonality is that
the object closer to the verb can be promoted. This is the accusative for \ili{German} since
nominative"=dative"=accusative is the unmarked order and \ili{German} is a OV language, but the first
accusative in \ili{English}, as \ili{English} is a VO language.

\eal
\ex 
\gll dass dem Kind der Ball gegeben wurde\\
     that the.\DAT{} child the.\NOM{} ball given \AUX\\\german
\glt `that the ball was given to the child'
\ex because the child was given the ball
\zl
A further difference is the lexical item for \emph{help}. Since there is no dative in \ili{English}, the
object is marked accusative as it is the case for \emph{read}. As expected, \ili{English} allows for the
personal passive of \emph{help}, while this is not possible in \ili{German}:
\eal
\ex[]{
because he was helped
}
\ex[]{
\gll weil ihm geholfen wurde\\
     because he.\DAT{} helped \AUX\\\german
}
\ex[*]{
\gll weil    er geholfen wurde\\
     because he.\NOM{} helped \AUX\\
}
\zl



%% 
%% \subsubsubsection{Isländisch}

%% \begin{itemize}
%% \item Dativ und Genitiv sind lexikalisch:



%% \end{itemize}

%% }



\subsubsection{Primary and secondary objects}


In this section I want to look at languages that allow both objects to be promoted. \ili{Danish} is like
\ili{English} in not having a dative. This is reflected in the following \argstvs:
\ea\label{da-repr-hm-Danish}
%\resizebox{\linewidth}{!}
\z
\ili{Danish} thus has two objects with structural case, while \ili{English} and \ili{German} have just one object with structural
case and the other one with lexical accusative and lexical dative, respectively. Since \ili{English} and
\ili{German} do not allow for subjects with lexical case, it is clear that the promotion to subject of the argument
that bears lexical case is excluded. \ili{Danish} also disallows subjects with lexical case, but since the
two objects have structural case anyway, they can both be promoted.

Note however that the lexical rule in (\ref{lr-passive-prelim}) does not account for the promotion
of the secondary object. What it does is suppress the subject. Under the assumption that the
first NP with structural case is the subject, the secondary object could never be realized as
subject. Note that it would not help to say that any NP with structural case can be the subject, since
this would admit wrong realizations. In addition to the correct
(\ref{ex-fordi-manden-giver-barnet-bolden}), the following two sentences would be admitted: 
\eal
\ex[*]{
\gll fordi barnet giver manden bolden\\ 
     because child.\textsc{def} gives man.\textsc{def} ball.\textsc{def}\\
}
\ex[*]{
\gll fordi   bolden         giver manden        barnet\\ 
     because ball.\textsc{def} gives man.\textsc{def} child.\textsc{def}\\
}
\zl
(\mex{0}a) is ungrammatical with \emph{barnet} `child' as the recipient of the giving. Similarly, the
transferred object \emph{bolden} cannot be realized as subject in active sentences. This means that
the promotion to subject has to be a part of the lexical  rule that licenses the participle that is
used in the passive. The lexical rule in (\mex{1}) takes the \argstl in the input of the lexical rule and splits it into two
lists: \ibox{1} and \ibox{2}. The first list \ibox{1} is identical to the value of \da. The second list \ibox{2} is the
remainder of the \argstl. \ibox{2} is related to \ibox{3}, the \argstv of the output of the lexical rule, by the relational constraint
\texttt{promote}. \ibox{3} is either equal to \ibox{2} or it is a list in which another NP with
structural case is positioned at the beginning of the list.

\eas
\label{lr-passive-double-object}
Lexical rule\is{lexical rule!passive} for the passive for \ili{Danish}, \ili{English}, \ili{German}, and \ili{Icelandic}:\\
\ms{
head   & \ms[verb]{ da & \ibox{1}\\
                  }\\
arg-st & \ibox{1} $\oplus$ \ibox{2} \\
} $\mapsto$
\ms{
arg-st & \ibox{3} \\
} $\wedge$ \texttt{promote}(\ibox{2}, \ibox{3})
\zs

%\largerpage
\noindent
(\mex{1}) shows the \argstvs of our prototypical verbs:
\ea\label{da-repr-hm-Danish-participles}
%\resizebox{\linewidth}{!}{%
\begin{tabular}[t]{@{}l@{ }l@{ }l}
  &                        & \textsc{arg-st}\\[2mm]
%a.&ankomme (unacc):       & \liste{}                         & \liste{NP[\type{str}]}\\[2mm]
a.&danset/-s   (dance, unerg):     & \liste{}\\[2mm]
%c.&auf"|fallen (unacc): & \liste{}                         & \liste{NP[\type{str}], NP[\type{ldat}]}\\[2mm]
b.&læst/-s      (read, trans):   &  \liste{NP[\type{str}]$_j$ } \\[2mm]
c.&givet/-s      (give, ditrans): & \liste{NP[\type{str}]$_j$, NP[\type{str}]$_k$ } \\[2mm]
  &                         & \liste{NP[\type{str}]$_k$, NP[\type{str}]$_j$ } \\[2mm]
d.&hjulpet/-s    (help, trans):   & \liste{NP[\type{str}]$_j$ }                    \\
\end{tabular}
%}
\z
The NP[\str]$_i$ that is the first element in (\ref{da-repr-hm-Danish}) is suppressed. The effect of
\texttt{promote} is that there are two different \argstvs for the passive variants of \emph{givet}
`to give': one with an \argstl in which NP[\type{str}]$_j$ precedes NP[\type{str}]$_k$ and another
one in which NP[\type{str}]$_j$ follows NP[\type{str}]$_k$. The first order corresponds to
(\ref{ex-child-was-given-ball-danish}) -- repeated here as (\ref{ex-child-was-given-ball-danish-two}) -- and
the second corresponds to (\ref{ex-ball-was-given-child-danish}) -- repeated here as (\ref{ex-ball-was-given-child-danish-two}):
\eal
\ex\label{ex-child-was-given-ball-danish-two}
\gll fordi barnet bliver givet bolden\\ 
     because child.\textsc{def} is given ball.\textsc{def}\\
\glt `because the child is given the ball'
\ex\label{ex-ball-was-given-child-danish-two}
\gll fordi bolden bliver givet barnet\\ 
     because ball.\textsc{def} is given child.\textsc{def}\\
\glt `because the ball is given to the child'
\zl
Before turning to impersonal passives in \ili{Danish} in the next subsection, I discuss the passive in
double object constructions in \ili{Icelandic}.


The distribution of structural/lexical case in \ili{Icelandic} is basically the same as in \ili{German}. The
difference is that \ili{Icelandic} allows for subjects with lexical case and \ili{German} does not.
(\mex{1}) shows our standard examples in \ili{Icelandic}:
\ea\label{da-repr-hm-Icelandic}
%\resizebox{\linewidth}{!}{%
\begin{tabular}[t]{@{}l@{ }l@{ }l@{ }l@{ }l@{}}
  &                     & \textsc{arg-st}\\[2mm]
a.&dansa   (dance, unerg):     & \liste{ NP[\type{str}] }\\[2mm]
%c.&auf"|fallen (unacc): & \liste{}                         & \liste{NP[\type{str}], NP[\type{ldat}]}\\[2mm]
b.& lesa      (read, trans):   & \liste{ NP[\type{str}], NP[\type{str}] }\\[2mm]
c.&gefa       (give, ditrans): & \liste{ NP[\type{str}], NP[\type{ldat}], NP[\type{str}] }\\[2mm]
d.&hjálpa     (help, trans):   & \liste{ NP[\type{str}], NP[\type{ldat}] }\\
\end{tabular}
%}
\z
The lexical rule in (\ref{lr-passive-double-object}) licenses the following participles:

%\largerpage
\ea\label{da-repr-hm-Icelandic-two}%
\scalebox{.98}{%
\begin{tabular}[t]{@{}l@{~}l@{~}l@{~~}l@{~~}l@{}}
  &                        & \textsc{arg-st}                     & \spr   & \comps\\[2mm]
%a.&ankomme (unacc):       & \liste{}                         & \liste{NP[\type{str}]}\\[2mm]
a.& dansað    (danced, unerg):     & \liste{}                        & \liste{ } & \liste{} \\[2mm]
%c.&auf"|fallen (unacc): & \liste{}                         & \liste{NP[\type{str}], NP[\type{ldat}]}\\[2mm]
b.& lesið      (read, trans):   &  \liste{NP[\type{str}]$_j$ }         & \liste{NP[\type{str}]$_j$ } & \eliste\\[2mm]
c.& gefið      (given, ditrans): & \liste{NP[\type{ldat}]$_j$, NP[\type{str}]$_k$ } & \liste{NP[\type{ldat}]$_j$ } & \liste{NP[\type{str}]$_k$ }\\[2mm]
  &                      & \liste{NP[\type{str}]$_k$, NP[\type{ldat}]$_j$ } & \liste{NP[\type{str}]$_k$ } & \liste{NP[\type{ldat}]$_j$ }\\[2mm]
d.& hjálpað    (helped, trans):   & \liste{NP[\type{ldat}]$_j$ }                  & \liste{ NP[\type{ldat}]$_j$ } & \liste{}\\
\end{tabular}%
}
\z
In addition to the \argstl, (\mex{0}) shows the mapping to the \spr and \comps features. Since
\ili{Icelandic} allows for quirky subjects the dative argument of `to help' can be mapped to the
\sprl \citep[\page 147--148]{Wechsler95a-u}. Similarly, the two orders of the \argst of `to give' result in participles with a dative
subject and a nominative subject as it is required for the analysis of (\ref{ex-were-the-king-given-the-slaves}) and (\ref{ex-were-the-slaves-given-the-king}) repeated
here as (\mex{1}):
\eal
\ex\label{ex-were-the-king-given-the-slaves-two}
\gll Voru konunginum gefnar ambáttir?\\
     were the.king.\DAT{} given slaves.\NOM{}\\
\glt `Was the king given slaves?'
\ex\label{ex-were-the-slaves-given-the-king-two}
\gll Var ambáttin gefin konunginum?\\
     was the.slave.\NOM{} given the.king.\DAT\\
\glt `Was the slave given to the king?'
\zl

\noindent
The impersonal passive with `to dance' is parallel to the \ili{German} impersonal passive, but the
passivization of `to help' differs since this is an instance of the personal passive in \ili{Icelandic}.


\subsubsection{Impersonal passive}
\label{sec-impersonals}

\largerpage
As a final point in this subsection, let us have a look at the impersonal passive. \ili{German} and
\ili{Icelandic} do not require subjects. So if there is no NP with structural case, the construction in
\ili{German} is subjectless. Similarly, \ili{Icelandic} does not require a subject: If there is no NP argument,
the result is an impersonal passive. An example of the latter case is the passivization of
\emph{dansa} `to dance'. The \argstl is the empty list and therefore the \sprl and the \compsl are
empty as well. Passive participles of verbs that govern an NP and a PP object will have an \argstl
that just contains the PP argument. This PP argument will be mapped to the \compsl and hence a
subjectless construction will result.


\ili{English} does not allow for impersonal passives as it requires an NP or a sentential argument that
can serve as a subject. \ili{Danish} requires a subject as well, but allows for impersonal
constructions. The trick that \ili{Danish} employs is the insertion of an expletive\is{pronoun!expletive}. I assume that the
expletive insertion happens during the mapping of the \argst elements to \spr and \comps. If there
is an NP/VP/CP at the beginning of the \argstl, it is mapped to \spr and all other elements are
mapped to \comps. If there is no element that can be mapped to \spr, an expletive is inserted.%
\nocite{BB2007a}

%\largerpage
%\enlargethispage{5pt}
(\mex{1}) shows the mappings for \ili{Danish}.
\ea\label{da-repr-hm-Danish-three}
%\resizebox{\linewidth}{!}{%
\begin{tabular}[t]{@{}l@{ }l@{ }l@{ }l@{ }l@{~~~~~}l@{}}
  &                        & \textsc{arg-st}                     & \spr   & \comps\\[2mm]
%a.&ankomme (unacc):       & \liste{}                         & \liste{NP[\type{str}]}\\[2mm]
%a.&danset/-s   (unerg):     & \liste{}                        & \liste{ NP$_\typeSM{expl}$ } &
%\liste{} \\[2mm]
a.&danset/-s   (unerg):     & \liste{}                        & \liste{ NP\textsubscript{\normalfont\itshape expl\/} } & \liste{} \\[2mm]
%c.&auf"|fallen (unacc): & \liste{}                         & \liste{NP[\type{str}], NP[\type{ldat}]}\\[2mm]
b.&læst/-s      (trans):   &  \liste{ NP[\type{str}]$_j$ }                     & \liste{ NP[\type{str}]$_j$ } & \eliste\\[2mm]
c.&givet/-s      (ditrans): & \liste{ NP[\type{str}]$_j$, NP[\type{str}]$_k$ } & \liste{ NP[\type{str}]$_j$ } & \liste{ NP[\type{str}]$_k$ }\\[2mm]
  &                         & \liste{ NP[\type{str}]$_k$, NP[\type{str}]$_j$ } & \liste{ NP[\type{str}]$_k$ } & \liste{ NP[\type{str}]$_j$ }\\[2mm]
d.&hjulpet/-s    (trans):   & \liste{ NP[\type{str}]$_j$ }                     & \liste{ NP[\type{str}]$_j$ } & \liste{ }\\
\end{tabular}
%}
\z







%\if 0
%\subsection{Variation and Generalizations}

\subsubsection{The passive auxiliary}
\label{sec-auxiliary}

\largerpage
We now saw what the participle forms of the languages under considerations look like and how they are
licensed from lexical entries for stems via lexical rules. What is missing is the lexical items for
the auxiliary verbs. Chapter~\ref{chap-verbal-complex} dealt with the analysis of verbal complexes in SOV languages like
\ili{German}, \ili{Dutch}, and \ili{Afrikaans}, and it was pointed out that SVO languages like \ili{English} and the
\ili{Scandinavian} languages do not form a verbal complex. With this background, it may come as a surprise
that it is possible to formulate one general constraint covering all the passive auxiliaries in the
\ili{Germanic} languages. The following AVM shows a constraint holding on all \argst lists for passive
auxiliary in \ili{Germanic} languages:\footnote{%
The lexical item of the passive auxiliary used by \citet[\page 147]{Mueller2002b} and \citet[\page 149]{MOe2013a}
specifies the \dav of the embedded participle to be a referential NP. This excludes the
passivization of unaccusative\is{verb!unaccusative} verbs, which have the empty list as the \dav.
}
 
\ea
\label{le-passive-aux-arg-st}
Passive auxiliary for \ili{Germanic} languages (\emph{be}, \emph{werden}, \emph{believe}, etc.):\\
\avm{
[ arg-st \1 \+ \2 \+  < [ vform & ppp\\
%                          da    & < XP$_{ref}$ >\\
                          spr   & \1\\
                          comps & \2  ] > ] 
}
\z
%
% The specification of the \dav excludes unaccusative verbs from forming a passive since unaccusative verbs have the empty
% list as the value of their \textsc{da} feature (the \textsc{da} feature has as its value a list
% containing the argument with subject properties). Weather verbs have an element in their \textsc{da}
% list but it is non-referential. Specifying the \textsc{da} element to be referential excludes the
% passive of weather verbs like \emph{regnen} `to rain', which otherwise could enter an impersonal
% passive:
% \ea[*]{
% \gll weil dann doch geregnet wurde\\
%      because then nevertheless rained was\\
% \glt Intended: `because there was raining nevertheless'
% }
% \z

%\largerpage
%\enlargethispage{5pt}
\noindent
When the passive auxiliary in (\ref{le-passive-aux-arg-st}) is used in a \ili{German} grammar, the
arguments of the participle (\ibox{1} and \ibox{2}) are attracted by the passive auxiliary
\citep{HN89a,HN94a}. 
%The Verbal Complex Schema allows the combination of the auxiliary with an unsaturated
%non-head daughter. The SVO languages do not have the verbal complex schema. They 
Auxiliaries in the SVO languages embed a VP. This means that the \compsv of the embedded verb has to
be the empty list. Therefore \ibox{2} is the empty list and only the specifier \iboxb{1} is taken
over from the embedded verb. 

%% \item Hence, we have explained how
%% \ili{Danish} and \ili{English} embed a VP and \ili{German} forms a verbal complex although the lexical item of the
%% auxiliary does not require a VP complement.
%\fi

%\largerpage
With the lexical item for the auxiliary in place, we can have a look at some example analyses of
passive sentences. Let's start with an \ili{English} example. The analysis of (\mex{1}) is shown in
Figure~\ref{fig-the-child-was-given-a-novel}.
\ea
The child was given a novel.
\z
\begin{figure}
\scalebox{1}{%
\begin{forest}
sm edges
[V\feattab{
            \spr    \sliste{ },\\
            \comps  \sliste{ }}
   [\ibox{1} \npnom [the child,roof]]
   [V\feattab{
            \spr    \sliste{ \ibox{1} },\\
            \comps  \sliste{ }}
      [V\feattab{
%            \da     \sliste{ \ibox{2} },\\
            \spr    \sliste{ \ibox{1} },\\
            \comps  \sliste{ \ibox{2} },\\
            \argst \sliste{ \ibox{1}, \ibox{2} }}  [was]]
      [\ibox{2} V\feattab{
%           \da     \sliste{ \ibox{3} },\\
            \spr    \sliste{ \ibox{1} },\\
            \comps  \sliste{ }}, s sep+=1em
        [V\feattab{
%            \da     \sliste{ \ibox{3} },\\
            \spr    \sliste{ \ibox{1} },\\
            \comps  \sliste{ \ibox{3} },\\
            \argst \sliste{ \ibox{1}, \ibox{3} }} 
          [V\feattab{
            \da     \sliste{ \ibox{4} },\\
            \argst \sliste{ \ibox{4}, \ibox{1}, \ibox{3} }} [{give-}]]]
        [\ibox{3} \npacc [a novel,roof]]]]]
\end{forest}}
\caption{\label{fig-the-child-was-given-a-novel}Analysis of \emph{The child was given a novel.}}
\end{figure}

\noindent
The lexical item for the stem \stem{give} is input to the passive lexical rule. The passive lexical
rule licenses the participle \emph{given}. The \argst of \stem{give} is shortened by the element in
the \da list of \stem{give} \iboxb{4}. The result is an \argstl containing the two NPs that would be the
objects in active sentences. The first one \iboxb{1} is mapped to \spr and the second one \iboxb{3}
to \comps. The combination of \emph{given} and \emph{a novel} forms a VP (something with an empty
\compsl and an element in the \sprl). The passive auxiliary \emph{was} selects the VP \emph{given a
  novel}. 
%It requires the selected VP to have a referential NP as designated argument, which the VP
%has \iboxb{4}. 
The specifier of the selected VP \iboxb{1} is attracted. As the first argument of
\emph{was} with structural case, this NP gets nominative. Finally, the VP \emph{was given a novel}
is combined with \emph{the child} and we have a complete sentence.

The analysis of the parallel \ili{German} sentence in (\mex{1}) is shown in
Figure~\vref{fig-dem-Kind-ein-Roman-gegeben-wurde}.
%\largerpage
\ea
\gll dass dem Kind ein Roman gegeben wurde\\
     that the.\DAT{} child a.\NOM{} novel  given    \AUX\\\german
\glt `that the child was given a novel'
\z
% Without this, the footnote floats to the next page. Hm.
%\pagebreak

\begin{figure}
\scalebox{.95}{%
\begin{forest}
sm edges
[V\feattab{
            \spr    \sliste{ },\\
            \comps  \sliste{ }}
   [\ibox{1} \npdat [dem Kind;the child,roof]]
   [V\feattab{
            \spr    \sliste{  },\\
            \comps  \sliste{ \ibox{1} }}, s sep+=2em
      [\ibox{2} \npnom [ein Roman;a novel,roof]]
      [V\feattab{
%            \da     \sliste{ \ibox{3} },\\
            \spr    \sliste{ },\\
            \comps  \sliste{ \ibox{1}, \ibox{2} }}
          [\ibox{3} V\feattab{                                  % gegeben
%              \da    \sliste{ \ibox{3} },\\
              \spr   \sliste{ },\\
              \comps \sliste{ \ibox{1}, \ibox{2} },\\
              \argst \sliste{ \ibox{1}, \ibox{2} }} 
            [V\feattab{
              \da    \sliste{ \ibox{4} },\\
              \argst \sliste{ \ibox{4}, \ibox{1}, \ibox{2} }} [{geb-};{give-}]]]
          [V\feattab{
%            \da    \sliste{ \ibox{2} },\\
             \spr   \sliste{  },\\
             \comps \sliste{ \ibox{1}, \ibox{2}, \ibox{3} },\\
             \argst \sliste{ \ibox{1}, \ibox{2}, \ibox{3} }}  [wurde;\textsc{aux}]]]]]
\end{forest}}
\caption{\label{fig-dem-Kind-ein-Roman-gegeben-wurde}Analysis of \emph{dass dem Kind ein Roman gegeben
    wurde} `that the child was given a novel'}
\end{figure}

\noindent
The analysis is similar to the one of the \ili{English} example but it differs in that the auxiliary and
the participle is forming a verbal complex. First, the passive lexical rule applies to a verb stem
\stem{geb} and licenses the participle form \emph{gegeben}. \emph{gegeben} \iboxb{3} is combined with
\emph{wurde} and \emph{wurde} takes over the elements of the \compsl of \emph{gegeben} \sliste{
  \ibox{1}, \ibox{2} }. The result of the combination of \emph{gegeben} and \emph{werden} has the
\compsl \sliste{ \ibox{1}, \ibox{2} }.


\subsubsection{The morphological passive}

%\largerpage
A lexical rule similar to the one accounting for the participle forms can be used for the
morphological passives in \ili{Danish}. One difference is, of course, the affixal material that is added by
the rule. Furthermore, it is assumed for the morphological passive that the \da of the input to the lexical rule has to contain
a referential XP. As was discussed in the previous section, this excludes morphological passives of
unaccusatives\is{verb!unaccusative} and weather verbs\is{verb!weather}. 


%% \subsection{Agent Expressions}

%% We follow \citet[Chapter~7]{Hoehle78a} and \citet[Section~5]{Mueller2003e} and treat the \emph{by}
%% phrases as adjuncts.


\subsubsection{Perfect}

\citegen{Haider86} analysis\is{perfect|(} of the passive is brilliant since it is sufficient to have one lexical
item for the participle. The participle has a blocked designated argument, and the designated
argument remains blocked in the passive, while the perfect auxiliary deblocks the designated 
argument. 
\eal
\ex
\gll dass der        Aufsatz gelesen wurde\\
     that the.\NOM{} paper   read    \AUX\\
\glt `that the paper was read'
\ex
\gll dass Kirby den Aufsatz gelesen hat\\
     that Kirby the.\ACC{} paper read \AUX\\
\glt `that Kirby has read the paper'
\zl

\largerpage
\noindent
Deblocking of the designated argument is possible since the designated argument is encoded not just in the stem of a verb
but also in the lexical item for the participle.   

\ea
Lexical item for perfect auxiliaries in SOV languages like \ili{Dutch} and \ili{German}:\\
\avm{
[ arg-st \1 \+ \2 \+ \3 \+  < [ vform & ppp\\
                                da    & \1\\
                                spr   & \2\\
                                comps & \3 ] > ]
}
\z



%\subsubsection{Analyse als komplexes Prädikat für Dänisch und Englisch?}

\noindent
Unfortunately \citeauthor{Haider86}'s approach does not work for SVO languages. If we wanted to use
the argument blocking/deblocking approach, we would have to assume the structures in (\mex{1}a--b):
\eal
\ex He [has given] the book to Mary.
\ex The book [was given] to Mary.
\ex He has [given the book to Mary].
\ex The book was [given to Mary].
\zl
If we assume that auxiliaries embed VPs as was argued on page~\pageref{page-English-Aux-VPs}, we run
into problems since the subject of the participle is blocked and the only VP we can form with the
participle is the one in (\mex{0}d), but for the perfect we also need a VP containing the object as
in (\mex{0}c).%
\is{perfect|)}




%% 
%% \subsubsection{A Solution that Almost Works}

%% \begin{itemize}
%% \item Complex Passive: There has to be a way to distinguish between participles that can be used in both perfect and
%%   passive:\\
%% \textsc{voice} feature. 

%% \begin{itemize}
%% \item Value is \type{passive} for those participles that cannot be used in perfect constructions.

%% 
%% \item Value is underspecified for participles that can be used in both perfect and passive

%% 
%% \item Perfect requires \textsc{voice} value to be \type{active}.
%% \end{itemize}

%% 
%% \item Expletives: Perfect attracts args from \argstl rather than \spr/\comps.
%% \begin{itemize}
%% \item Since expletives are not on \argst, they will not get into the way.
%% \end{itemize}
%% \end{itemize}


%% }



%\subsubsection{But: (Partial) Fronting}
%\subsubsection{Problem: (Partial) Fronting}


% \citet{Meurers99b} hat einen Trick gefunden, wie man die Kasuszuweisung in (\mex{1})
%   analysieren kann:
% \nocite{Meurers2000b,MdK2001a}
% \eal
% \ex 
% Gelesen wurde der Aufsatz schon oft.
% \ex 
% Der Aufsatz gelesen wurde schon oft.
% \ex
% Den Aufsatz gelesen hat er schon oft.
% \zl

% Das funktioniert aber nicht für Dänisch/Englisch, denn hier haben wir nicht nur Kasus- sondern
%   auch Positionsunterschiede:
% \eal
% \ex The book should have been given to Mary and\\
%     {}[given to Mary] it was.
% \ex He wanted to give the book to Mary and\\
%     {}[given the book to Mary] he has.
% \zl

% Wenn sich keine ausgeklügelten Mechanismen für die Unterspezifikation verschiedenenr Mappings finden
% lassen, müssen wir wohl zwei verschiedene Partizipformen annehmen.
% for the participle form.







\subsubsection{The remote passive}
\label{sec-remote-passive-phen}

\largerpage
The so-called \emph{remote passive}\is{passive!remote} is a highlight of \ili{German} syntax since several
phenomena interact in a non-trivial way. It was first discussed by \citet[\page
175--176]{Hoehle78a}. Höhle observed that objects of \ili{German} infinitives with \emph{zu} appear in the
nominative in certain contexts. (\mex{1}) provides some constructed examples from the literature:
\eal
\ex\iw{versuchen|(}
\gll daß  er        auch von  mir zu überreden versucht wurde\footnotemark\\
     that he.\NOM{} also from me  to persuade  tried    \AUX\\\german
\footnotetext{
        \citew*[\page 212]{Oppenrieder91a}.%
}
\glt `that an attempt to persuade him was also made by me'
\ex 
\gll weil    der        Wagen oft   zu reparieren versucht wurde\\
     because the.\NOM{} car   often to repair     tried    \AUX\\
\glt `because many attempts were made to repair the car'\label{bsp-zu-reparieren-versucht-wurde}
\zl
The examples in (\mex{1}) are attested data collected from the COSMAS corpus by \citet[\page 136--137]{Mueller2002b}:
\eal
\ex Dabei darf jedoch nicht vergessen werden, daß in der Bundesrepublik, wo \emph{ein Mittelweg zu gehen versucht wird}, 
die Situation der Neuen Musik allgemein und die Stellung der Komponistinnen im besonderen noch recht unbefriedigend ist.\footnotemark
%But should however not forgotten get that in the BRD where a middle.way to go tried gets the situation of.the 
%new music generally and the position of.the composers in particular still quite unsatisfactory is 
\footnotetext{
Mannheimer Morgen, 26.09.1989, Feuilleton; Ist's gut, so unter sich zu bleiben?
}%
\glt `One should not forget that the situation of the New Music in general and the position of female composers 
in particular is rather unsatisfying in the Bundesrepublik, where one tries to follow a middle course.'
\ex Noch ist es nicht so lange her, da ertönten gerade aus dem Thurgau jeweils die lautesten Töne, 
    wenn im Wallis oder am Genfersee im Umfeld einer Schuldenpolitik mit den unglaublichsten Tricks 
    \emph{der sportliche Abstieg zu verhindern versucht wurde}.\footnotemark
%still is it not so long ago there sounded just out of.the Thurgau at.the.time the 
%loudest sounds when in.the Valais or at.the Lake.Geneva in.the sphere of.a debt.policy 
%with the most.unbelievable tricks the sporty relegation to prevent tried got
\footnotetext{
St.\ Galler Tagblatt, 09.02.1999, Ressort: TB-RSP; HCT und das Prinzip Hoffnung.%
}%
\glt `It still is not too long ago that the loudest protests were heard in the Thurgau itself 
when the most unbelievable tricks in the sphere of debt policies were applied to prevent relegation in the Valais or at Lake Geneva.' 
\ex Die Auf- und Absteigenden erzeugen ungewollt einen Ton, \emph{der bewusst nicht als lästig zu eliminieren versucht wird}, 
    sondern zum Eigenklang des Hauses gehören soll, so wünschen es sich die Architekten.\footnotemark
\footnotetext{
Züricher Tagesanzeiger, 01.11.1997, p.\,61.%
}
% Philippa
\glt `The people who go up and down produce a tune without intention which is not consciously sought to
be eliminated but which, rather, belongs to the individual sound of the building, as the
architects intended.'
% Uta
%the up and downclimbers create involuntarily a tone that consciously not as annoying 
%to eliminate tried gets but to.the own.sound of.the house belong should so wish it themselves the architects
%`That no attempt is made to eliminate the involuntary noise caused by people ascending and descending 
%is a conscious decision; this is not considered to be a nuisance, but as an aspect of the house's own sound, 
%at least according to the architects.'
\zl
Höhle's examples and other examples from the literature involved the verb \emph{versuchen} `to try',
but \citet{Wurmbrand2003a} showed that other verbs allow for the remote passive as well. (\mex{1})
and (\mex{2}) show some of her examples with \emph{beginnen} `to start', \emph{vergessen} `to forget', and
\emph{wagen} `to dare':
\ea
%% \ex
%% \emph{dieser} wurde bereits zu bauen begonnen.\footnote{
%%         \url{http://www.hollabrunn.noe.gv.at/mariathal/ortsvorsteher.html}, 28.07.2003.
%% }
\gll \emph{der} \emph{zweite} \emph{Entwurf} wurde  zu bauen begonnen,\footnotemark\\
     the.\NOM{} second        plan           \AUX{} to build started\\\german
\footnotetext{
\url{http://www.waclawek.com/projekte/john/johnlang.html}, 28.07.2003.
}
\glt `It was begun to build the second plan.' 
\z

\noindent
While the case of \emph{der zweite Entwurf} `the second plan' is unambiguously nominative, this is
not the case for the examples in (\mex{1}), since the respective elements are in the
plural and hence could be nominative or accusative. But due to \isi{agreement} with the finite verb, it is
clear that the relative pronoun are in the nominative.
\eal
\ex 
\gll Anordnungen, die zu stornieren vergessen \emph{wurden}\footnotemark\\
     orders       that to cancel forgotten were\\\german
\footnotetext{
        \url{http://www.rlp-irma.de/Dateien/Jahresabschluss2002.pdf}, 28.07.2003.
}
\glt `orders that were forgotten to cancel'

\ex Aufträge [\ldots], die zu drucken vergessen worden \emph{sind}\/\footnotemark\\
    orders   {}        that to print  forgot    were   are\\
\footnotetext{
        \url{http://www.iitslips.de/news.html}, 28.07.2003.
}
\glt `orders that somebody forgot to print'
%\ex Ist plötzlich übervoll von Emotionen und längst begrabenen Träumen, die nicht zu leben gewagt wurden\footnote{
% nicht auffindbar
\ex NUR Leere, oder doch noch Hoffnung, weil aus Nichts wieder Gefühle entstehen,\\
\gll die so vorher nicht mal zu träumen gewagt \emph{wurden}?\footnotemark\\
     that this.way before not even to dream dared were\\
\footnotetext{
        \url{http://www.ultimaquest.de/weisheiten_kapitel1.htm}, 28.07.2003.
}
\glt `that were not even dared to be dreamed of in this way before'

\ex Dem Voodoozauber einer Verwünschung oder die gefaßte Entscheidung zu einer Trennung,\\
\gll die bis dato noch nicht auszusprechen gewagt \emph{wurden}\footnotemark\\
     which until now not express dared were\\
\footnotetext{
        \url{http://www.wedding-no9.de/adventskalender/advent23_shawn_colvin.html}, 28.07.2003.
}
\glt `which until now have not been dared to express'
\zl
% Kasus bei PVP wie Haiders entziffern: Am leichtesten zu erklären fiel den 
% Experten dabei gestern der Kursverlust der Telekom, zu deren Schuldenproblem 
% eine neue Hiobsbotschaft kam.  (taz. 8./9. 9. 01 S. 9.)
%

\noindent
The object of a verb that is embedded under a passive participle is promoted to subject of the sentence:
\eal
\ex 
\longexampleandlanguage{
\gll weil    Aicke den        Wagen oft   zu reparieren versucht hat\\
     because Aicke the.\ACC{} car   often to repair     tried    has\\}{German}
\ex 
\gll weil    der        Wagen oft   \emph{zu} \emph{reparieren} \emph{versucht} \emph{wurde}\\
     because the.\NOM{} car   often to repair     tried    was\\
\glt `because many attempts were made to repair the car'\label{bsp-zu-reparieren-versucht-wurde-two}
\zl

\noindent
The remote passive is possible in verbal complexes only. If no verbal complex is formed as in
(\mex{1}a,c), the object of \emph{reparieren} has to appear in the accusative:
\eal
\ex[]{
\gll weil    oft   versucht wurde, den        Wagen zu reparieren.\\
     because often tried    \AUX{} the.\ACC{} car   to repair\\
\glt `because many attempts were made to repair the car.'
}
\ex[*]{
\gll weil    oft   versucht wurde, der        Wagen zu reparieren.\\
     because often tried    \AUX{} the.\NOM{} car   to repair\\
}
\ex[]{
\gll Den        Wagen zu reparieren wurde  oft   versucht.\\
     the.\ACC{} car   to repair     \AUX{} often tried\\
}
\ex[*]{
\gll Der        Wagen zu reparieren wurde  oft   versucht.\\
     the.\NOM{} car   to repair     \AUX{} often tried\\
}
\zl
The difference between (\ref{bsp-zu-reparieren-versucht-wurde-two}) and (\mex{0}a,c) are explained by an analysis that treats the remote passive as a passivization
of a predicate complex, \ie by an analysis that assigns the structure (\mex{1}) to \pref{bsp-zu-reparieren-versucht-wurde-two}.
\ea
\gll weil    der        Wagen oft   [[zu reparieren versucht] wurde].\\
     because the.\NOM{} car   often \hphantom{[[}to repair   tried     \AUX\\
\glt `because many attempts were made to repair the car.'\label{bsp-zu-reparieren-versucht-wurde-three}
%
\z
In (\mex{-1}a,c) we do not have predicate complexes. The object of \emph{zu reparieren} is part of the VP
and therefore it gets accusative. The passives in (\mex{-1}a,c) are impersonal passives\is{passive!impersonal}.


The verb \emph{versuchen} `to try' selects a subject, an infinitive with \emph{zu} `to' and the
complements of the embedded verb.
\ea
\stem{versuch} `to try':\\
\avm{
[ arg-st & < !NP[\type{str}]$_i$! > \+ \1 \+ < V[\type{inf}, \subj < !NP[\type{str}]$_i$! >, \comps \1 ] > ]
}
\z

In our example, the embedded verb is \emph{reparieren} and has one complement. (\mex{1}) shows the
\argstv of \stem{versuch}. The first NP is the subject of \emph{versuchen} and the second NP is the
attracted object of \emph{zu reparieren}.
\ea
\argstv of \stem{versuch} with embedding of a strictly transitive verb:\\
\sliste{ NP[\type{str}]$_i$, NP[\type{str}]$_j$, V[\type{inf}] }
\z
%\largerpage
When the passive lexical rule applies to this verb stem and licenses the respective participle, the
resulting lexical item for \emph{versucht} `tried' will have the following \argstv:
\ea
\argst of participle \emph{versucht} with embedding of strictly transitive verb:\\
\sliste{ NP[\type{str}]$_j$, V[\type{inf}] } 
\z

\largerpage
\noindent
The first NP is mapped to \subj and V[\type{inf}] is mapped to \comps. After combination with
\emph{zu reparieren}, we have a complex with NP[\type{str}]$_j$ in \subj and nothing in
\comps. \emph{wurde} selects a verb or verbal complex with the verb form \type{ppp} and the
arguments of the embedded verb. Hence, NP[\type{str}]$_j$ ends up as the first element of the
\argstl of \emph{werden} where it gets nominative. Since \emph{werden} is finite all \argst elements
are mapped to \comps and have to be realized in the sentence. The analysis of the verbal complex in
(\ref{bsp-zu-reparieren-versucht-wurde-three}) is shown in Figure~\vref{fig-zu-reparieren-versucht-wurde}.


\begin{figure}
\centering
\begin{forest}
sm edges
[V\feattab{
              \vform \type{fin},\\
              \comps \ibox{1} } 
        [{\ibox{2} V\feattab{
              \vform \type{ppp},\\
              \subj  \ibox{1},\\
              \comps \eliste }} 
           [{\ibox{3} V\feattab{
              \vform \type{inf},\\
              \subj  \sliste{ NP[\str]$_i$ }, \\ 
              \comps \ibox{1} \sliste{ NP[\str]$_j$ } }} [zu reparieren;to repair] ]
           [V\feattab{
              \vform \type{ppp},\\
              \subj  \ibox{1},\\
              \comps \sliste{ \ibox{3} } } [versucht;tried] ] ]
        [V\feattab{
              \vform \type{fin},\\
              \comps \ibox{1} $\oplus$ \sliste{
                \ibox{2} }} [wurde;\textsc{aux}] ] 
]
\end{forest}
\caption{\label{fig-zu-reparieren-versucht-wurde}The analysis of the remote passive as passivization of a complex forming verb}
\end{figure}


The remote passive is also possible with object control verbs, that is, verbs taking a subject and
an object and a verbal projection the subject of which is coreferential with the object. An example
is \emph{erlauben} `to permit'. (\mex{1}a) shows the verb in the active and without verbal complex
formation. The object of \emph{erlauben} `to permit' \emph{uns} `us' is coreferential with the
subject of \emph{den Erfolg auszukosten} `to enjoy the success'. (\mex{1}b,c) show that the object
of \emph{auszukosten} can be realized in the nominative, if the verbs are forming a verbal complex:
\eal
\label{bsp-auskosten-fernpassiv}
\ex 
\gll Sie erlauben uns nicht, den Erfolg auszukosten.\\
     they permitted us.\DAT{} not the.\ACC{} success to.enjoy\\
\glt `They did not permit us to enjoy the success.'
\ex\iw{erlauben}
\gll Keine Zeitung          wird   ihr       zu lesen erlaubt.\footnotemark\\
     no    newspaper.\NOM{} \AUX{} her.\DAT{} to read  allowed\\
\footnotetext{
        Stefan Zweig. \emph{Marie Antoinette}. Leipzig: Insel-Verlag. 1932, p.\,515, 
        quoted from \citew[\page 309]{Bech55a}. 
        That this is an instance of the remote passive was noted by \citet[\page 13]{Askedal88}.
}
\glt `She is not allowed to read any newspapers.'%
\ex\iw{auskosten}
\gll Der Erfolg         wurde  uns       nicht auszukosten erlaubt.\footnotemark\\
     the success.\NOM{} \AUX{} us.\DAT{} not   to.enjoy    permitted\\
\footnotetext{
        \citew[\page 110]{Haider86c}%
}
\glt `We were not permitted to enjoy our success.'%
\label{bsp-auskosten-fernpassiv-haider}
\zl

\largerpage
\noindent
The passive of the construction without verbal complex is an impersonal passive:

\ea
\gll Uns       wurde  erlaubt, den        Erfolg  auszukosten.\\
     us.\DAT{} \AUX{} allowed  the.\ACC{} success to.enjoy\\
\z

\noindent
(\mex{1}) shows the \argstv of \stem{erlaub}: \emph{erlauben} takes a subject and a dative
object. The dative object is coindexed with the subject of the embedded verb, that is, the two NPs
have the same index, namely $j$.
\ea
\stem{erlaub} `to permit':\\
\oneline{%
\avm{
[ arg-st & < !NP[\type{str}]$_i$, NP[\ldat]$_j$!  > \+ \1 \+ < V[\type{inf}, \subj < !NP[\type{str}]$_j$! >, \comps \1 ] > ]
}}
\z
The complements of the embedded verb \iboxb{1} are taken over by the embedding verb. (\mex{1}) shows
the \argstv of the respective highest verb:
\ea
\label{ex-arg-st-fernpassiv-object-control}
\oneline{%
\begin{tabular}[t]{@{}l@{~}l@{ }l@{}}
a. & \emph{zu lesen erlauben}: & \sliste{ NP[\type{str}]$_i$, NP[\ldat]$_j$, NP[\type{str}]$_k$, V[\comps \sliste{ NP[\type{str}]$_k$ }] }\\[2mm]

b. & \emph{zu lesen erlaubt wird}: & \sliste{ NP[\ldat]$_j$, NP[\type{str}]$_k$, V[\comps \sliste{ NP[\type{str}]$_k$ }] }\\
\end{tabular}
}
\z
(\mex{0}a) shows how the object of \emph{lesen} is attracted so that the combined \argst contains
three NPs. (\mex{0}b) shows the passive variant in which the subject of \emph{erlauben} is
suppressed. The result is a \argstl starting with a dative NP, an NP with lexical case. Since the
first NP with structural case gets nominative and agrees\is{agreement} with the finite verb, the theory makes the
right predictions even in situations as complex as the remote passive with object control verbs. The
first NP with structural case is the subject in \ili{German}.


%% \subsubsection{Complex passives}

%% \begin{itemize}
%% \item Complex Passives:

%% \ea
%% \gll at Bilen           blev forsøgt repareret\\
%%      that car.\textsc{def} was  tried   repaired\\
%% \glt `that an attempt was made to repair the car'
%% \z



%% \item Raising in passive only.



%% \item \emph{forsøgt} (`to try') does not even take a participle in the active:
%% \eal
%% \ex[]{
%% \gll at   Peter har  forsøgt \emphbf{at} \emphbf{reparere} bilen\\
%%      that Peter has  tried   to repair   car.\textsc{def}\\
%% \glt `that Peter tried to repair the car'
%% }
%% \ex[*]{
%% \gll at   Peter har  forsøgt \emphbf{repareret} bilen\\
%%      that Peter has  tried   repaired car.\textsc{def}\\
%% %\glt `that an attempt was made to repair the car'
%% }
%% \zl

%% %% \item Conclusion: We need special lexical items for passive participles.

%% %% \item analysis of the \ili{German} passive and perfect can be maintained,\\
%% %% compatible with a more general analysis of the passive

%% \end{itemize}

\section{Alternatives}

As with the sections about alternatives in previous chapters, this section is for advanced readers
only. It is not necessary to read it in order to understand the rest of the book.


\subsection{Government \& Binding analyses}
\label{sec-passive-GB}

The analysis adopted here was developed out of proposals by Hubert Haider
\citeyearpar{Haider86}. Haider developed analyses within the framework of Government \& Binding
\citep{Chomsky81a} and his analyses of various phenomena -- not just passive -- are to a large extent compatible with HPSG views and are adopted
by many researchers working in \ili{German} within the framework of HPSG. The most common analysis of the
passive in GB is different though.  \cites[155--157]{Grewendorf88a}[\page 1311]{Grewendorf93} adapted Chomsky's analysis of the passive in
\ili{English} \citep[\page 124]{Chomsky81a} to \ili{German} (see also \citealt[\page 180]{Lohnstein2014a} for a
more recent suggestion along these lines in a textbook). This analysis is based on the CP/TP/VP system (see Section~\ref{sec-cp-tp-vp} for a discussion of
scrambling in this system). The discussion of this analysis of the passive is based on \citew[Section~3.4]{MuellerGT-Eng5}.

\largerpage[2]
GB's passive analysis is similar to the analysis suggested here in that it is a lexical analysis:
the lexical item for the participle is special in not assigning case to the accusative object. There
is a Case Filter\is{case!filter} requiring that every NP in a sentence must have case. Since the verb does not
assign case, the NP that would have accusative in the active has to get case elsewhere. There are two
ways to get case: the subject receives case from (finite) T\is{category!functional!T} and
the case of the remaining arguments comes from V (\citealp[\page 50]{Chomsky81a}; \citealp[\page
26]{Haider84b}; \citealp[\page 71--73]{FF87a}). This is stated as the Case Principle:
\begin{principle-break}[Case Principle]\label{Kasusprinzip-GB}
\begin{itemize}
% hier stand `seinem', aber in den Abbildungen gibt es zwei Komplemente
\item V assigns objective case (accusative) to its complement if it bears structural case.
\item When finite, Tense assigns case to the subject.
\end{itemize}
\end{principle-break}

\noindent
Figure~\ref{Abb-GB-Aktiv} shows the Case Principle in action with the example in 
(\mex{1}a).\footnote{\label{fn-semantic-role-phrase-boundary}%
The figure does not correspond to \xbar theory in its classic form, since \emph{der Frau} `the woman' 
is a complement which is combined with V$'$.  In classical \xbar theory, all complements have to be combined
with \vnull. This leads to a problem in ditransitive\is{verb!ditransitive} structures since the structures have to be binary (see \citew{Larson88a} for a treatment of double object constructions).
Furthermore, in the following figures the verb has been left in \vnull for reasons of clarity. In order
to create a well"=formed S"=structure, the verb would have to move to its affix in \inull. Note also
that the assignment of the subject theta-role by the verb crosses a phrase boundary. This problem
can be solved by assuming that the subject is generated within the VP, gets a theta role there and
then moves to SpecIP. An alternative suggestion was to assume that the VP assigns a semantic role to
SpecIP \parencites[\page 104--105]{Chomsky81a}[\page 229]{AS83a}.%
}
\eal
\ex 
\gll {}[dass] der Delphin dem Kind den Ball gibt\\
     \spacebr{}that the dolphin the.\DAT{} child the.\ACC{} ball gives\\
\glt `that the dolphin gives the child the ball'
\ex 
\gll{}[dass] der Ball dem Kind gegeben wird\\
      \spacebr{}that the ball.\NOM{} the.\DAT{} child given \AUX\\
\glt `that the ball is given to the child'
\zl
\largerpage[2]
%\fi
\begin{figure}
\hfill
\begin{forest}
sm edges
[TP
  [{NP[nom]}, name=subject [der Delphin;the dolphin, roof]]
  [T\rlap{$'$}
	[VP
		[V\rlap{$'$}
			[{NP[dat]}, name=dobject [dem Kind;the child, roof]]
			[V\rlap{$'$}
				[{NP[acc]},   name=aobject [den Ball;the ball, roof]]
				[V ,name=verb    [gib-;give-]]]]]
	[T , name=Infl [-t;-s]]]]
\draw[->,dotted] (Infl.north) .. controls (3.5,-0.1) and (-1.5,0.4)  .. ($(subject.north)+(-.1,.1)$);
\draw[->]        (verb.north) .. controls (2.8,-2.9) and (-.4,.5)   .. ($(subject.north)+(0,.1)$);
\draw[->,dashed] (verb.north) .. controls (2.8,-3.0) and (0.2,-3.05)   .. ($(dobject.north)+(0,.1)$);
%\DrawControl{(2.8,-3)}{1}; \DrawControl{(0.2,-3.05)}{2};
\draw[->,dashed] (verb.north) .. controls (2.3,-4.2) and (1.8,-4.3) .. ($(aobject.north)+(0,.1)$);
%\DrawControl{(2.3,-4.2)}{1}; \DrawControl{(1.8,-4.3)}{2};
%
%\draw (-4,-7) to[grid with coordinates] (4,0.5);
\end{forest}\hfill
\begin{tabular}[b]{ll@{}}
\tikz[baseline]\draw[dotted](0,1ex)--(1,1ex);&just case\\
\tikz[baseline]\draw(0,1ex)--(1,1ex);&just theta"=role\\
\tikz[baseline]\draw[dashed](0,1ex)--(1,1ex);&case and theta"=role
\\
\\
\end{tabular}
\caption{\label{Abb-GB-Aktiv}Case and theta-role assignment in active clauses}
\end{figure}%
%
The passive morphology of \emph{gegeben} `given' blocks the subject and absorbs the structural
accusative. In GB, it is assumed that semantic roles are assigned to certain tree positions. These
positions are determined in the so-called base"=configuration, which is the configuration before any
movement and reorganization of trees takes place. The object that would get accusative in the active
receives only a semantic role in its base position in the passive, but it does not get the
absorbed case. Therefore, it has to move to a position where case can be assigned to it 
\citep[\page 124]{Chomsky81a}. Figure~\ref{Abb-GB-Passiv} shows how this works for example
(\mex{0}b).
\begin{figure}%[t]
\scalebox{1}{%
\begin{forest}
sm edges
[TP
[{NP[nom]}, name=subject [der Ball$_i$;the ball,roof]]
[T\rlap{$'$}
	[VP
		[V\rlap{$'$}
			[{NP[dat]}, name=dobject [dem Kind;the child, roof]]
			[V\rlap{$'$}
				[NP,   name=aobject [\_$_i$]]
				[V ,name=verb [gegeben wir-;given \textsc{aux}, roof]]]]]
	[T  ,name=Infl [-\/d]]]]
\draw[->,dotted] (Infl.north) .. controls (2.4,.6)   and (-1.5,-.05) .. ($(subject.north)+(0,.1)$);
%\DrawControl{(2.4,.6)}{1}; \DrawControl{(-1.5,-.05)}{2};
\draw[->,dashed] (verb.north) .. controls (2.2,-3.2)  and (0,-3.0)    .. ($(dobject.north)+(0,.1)$);
%\DrawControl{(2.2,-3.2)}{1}; \DrawControl{(0,-3.0)}{2};
\draw[->]        (verb.north) .. controls (2.0,-4.25) and (1.2,-4.25) .. ($(aobject.north)+(0,.1)$);
%\DrawControl{(2.0,-4.25)}{1}; \DrawControl{(1.2,-4.25)}{2};
%\draw (-3,-7) to[grid with coordinates] (3.6,0.5);
\end{forest}}\hfill
\begin{tabular}[b]{ll@{}}
\tikz[baseline]\draw[dotted](0,1ex)--(1,1ex);&just case\\
\tikz[baseline]\draw(0,1ex)--(1,1ex);&just theta"=role\\
\tikz[baseline]\draw[dashed](0,1ex)--(1,1ex);&case and theta"=role
\\
\\
\end{tabular}
\caption{\label{Abb-GB-Passiv}Case and theta-role assignment in passive clauses}
\end{figure}%
%\if 0

This movement"=based analysis works well for \ili{English}\il{English} since the underlying object always has to move:

\eal
\ex[]{
The dolphin gave [the child] [a ball].
}
\ex[]{
{}[The child] was given [a ball] (by the dolphin).
}
\ex[*]{
It was given [the child] [a ball].
}
\zl
%
(\mex{0}c) shows that filling the subject position with an expletive is not possible, and hence,
since \ili{English} requires a subject, the object really has to move. However, \citet[Section~4.4.3]{Lenerz77} showed that such a movement is not
obligatory in \ili{German}. (\mex{1}) illustrates:

\eal
\label{ex-passive-German-no-movement}
\ex 
\gll weil der Delphin dem Kind den Ball gab\\
     because the.\NOM{} dolphin the.\DAT{} child the.\ACC{} ball gave\\
\glt `because the dolphin gave the ball to the child'
\ex 
\gll weil    dem        Kind  der        Ball gegeben wurde\\
     because the.\DAT{} child the.\NOM{} ball given \AUX\\
\glt `because the ball was given to the child'
\ex 
\gll weil    der        Ball dem        Kind  gegeben wurde\\
     because the.\NOM{} ball the.\DAT{} child given     \AUX\\
\zl
%\addlines[1]
%\largerpage
In comparison to (\mex{0}c), (\mex{0}b) is the unmarked order. \emph{der Ball} `the ball' in (\mex{0}b) occurs
in the same position as \emph{den Ball} in (\mex{0}a), that is, no movement is necessary. Only the case differs.
(\mex{0}c) is, however, somewhat marked in comparison to (\mex{0}b). So, if one assumed (\mex{0}c) to
be the normal order for passives and (\mex{0}b) is derived from this by movement of \emph{dem
  Kind} `the child', (\mex{0}b) should be more marked than (\mex{0}c), contrary to the
facts. To solve this problem, an analysis involving abstract movement has been proposed for
cases such as (\mex{0}b): the elements stay in their positions, but are connected to
the subject position and receive their case information from there. \textcites[155--157]{Grewendorf88a}[\page 1311]{Grewendorf93}
assumes that there is an empty expletive pronoun\is{empty element}\is{pronoun!expletive}
% Fanselow81a:152 Infl weist Kasus in die VP zu. Adjazens nicht nötig.
in the subject position of sentences such as (\mex{0}b) and in the subject position of sentences with an
impersonal passive\is{passive!impersonal} such as (\mex{1}):\footnote{%
	See \citew[\page 11--12]{Koster86a} for a parallel analysis for \ili{Dutch}\il{Dutch} as well as 
	\citew[\page 180]{Lohnstein2014a} for a movement"=based account of the passive that also involves an
        empty expletive for the analysis of the impersonal passive.
}
\ea
\gll weil    heute nicht gearbeitet wird\\
     because today not   worked     \AUX\\
\glt `because there will be no work done today'
\z
A silent expletive pronoun is something that one cannot see or hear and that does not carry any
meaning. Such empty elements are rejected by many researchers, since it is unclear how their
existence is to be acquired by language learners. It seems to be necessary to assume rich innate
linguistic knowledge for this, something that not even Chomsky assumes nowadays \citep*{HCF2002a}. For discussion of 
this kind of empty element, see \citew[Section~13.1.3 and Chapter~19]{MuellerGT-Eng5} and \citew[Section~7]{MuellerHeadless}.

\citet[\page 12]{Koster86a} has pointed out that the passive in \ili{English}\il{English} cannot be derived by Case
Theory, that is, lack of case and movement to the specifier position of TP due to the Case Filter, since if one allowed empty expletive subjects for \ili{English} as well as \ili{German} and \ili{Dutch}\il{Dutch}, then it would be possible
to have analyses such as the following in (\mex{1}) where np is an empty expletive:
\ea
np was read the book.
\z
The object would not have to move to subject position and could just stay there, contrary to the facts.
Koster rather assumes that subjects in \ili{English} are either bound by other elements (that is, non"=expletive) or lexically filled, that
is, filled by visible material.
Therefore, the structure in (\mex{0}) would be ruled out and it would be ensured that \emph{the book} would have to be placed in front
of the finite verb so that the subject position is filled.
\is{passive|)}

Concluding, one can say that passive should not be explained by movement. Chomsky's analysis works,
but this is only due to the fact that \ili{English} requires a subject. Two phenomena are mixed that
should be treated separately. In the analysis suggested here, passive is the suppression of the
subject in the argument structure list. This works for all examined languages. It depends on the
language under consideration whether there has to be a subject and whether it has to be realized in
a certain position. In \ili{English}, we have to have something in the specifier position, in \ili{German}, all
arguments are listed on the \compsl. No movement is involved. Note that the equivalent to the
base-order for semantic role assignment is the \argstl. The \argstl has a certain fixed order. This fact
can be used for linking, since it is clear on which position of the list which argument is
represented. Passive sentences involve a passivized verb. The passivized verb is related to a verb
stem with an \argstl corresponding to the active form. So this \argstl is somehow accessible for
semantic role assignment but the analysis of the passive clause does not involve the analysis of an
active clause and some movement. For all what is known about human language processing up to now, this is
the right approach from a psycholinguistic point of view \citep[Section~3.2]{Wasow2021a}. 




\section{Summary}

In conclusion, it can be said that this chapter provides a unified account of the passive in \ili{Danish},
\ili{English}, \ili{German}, and \ili{Icelandic}. The lexical rule accounts both for the morphological and the
analytical passives. The first element on the \argstl is suppressed and a relational constraint
\texttt{promote} promotes any NP with structural case. The languages differ in what cases they use and
which cases are structural/lexical. \ili{Danish} inserts expletives to allow for impersonal passives and
fulfilling the need of a subject. This expletive insertion is done in the \argst mapping when
arguments are mapped from \argst to the valence lists.

The SVO languages seem to require different items for the perfect/passive participles, but \citegen{Haider86}
passive analysis for \ili{German} using just one participle form for both perfect and passive can be maintained.



\questions{
\begin{enumerate}
\item What tests do you know for subjecthood?
\item Do these tests work for all \ili{Germanic} languages?
\item In which way is \ili{German} different from \ili{Icelandic} in terms of subjects?
\item What is structural case? What is lexical case?

\item What is an impersonal passive?
\item Does \ili{Icelandic} have impersonal passives?

\end{enumerate}

}

\largerpage
\exercises{
\begin{enumerate}
\item Which NPs in (\mex{1}) have structural and which lexical case?
\eal
\ex 
\gll Der        Junge lacht.\\
     the.\NOM{} boy   laughs\\
\ex 
\gll Mich friert.\\
     I.\ACC{} freeze\\
\glt `I am cold.'
\ex 
\gll Er zerstört die Umwelt.\\
     he.\NOM{} destroys the.\ACC{} environment\\
\glt `He destroys the environment.'
\ex 
\gll Das dauert ein ganzes Jahr.\\
     this.\NOM{} takes  a.\ACC{} whole year\\
\glt `This takes a whole year.'
\ex 
\gll Er hat nur einen Tag dafür gebraucht.\\
     he.\NOM{} has just one.\ACC{}  day there.for needed\\
\glt `He needed a day for this.'
\ex 
\gll Er denkt an den morgigen Tag.\\
     he.\NOM{} thinks at the.\ACC{} tomorrow day\\
\glt `He thinks about tomorrow.'
\zl


\item Give \argst lists for the following verbs: 
\eal
\ex show, eat, meet \english
\ex zeigen `show', essen `eat', begegnen `meet', treffen `meet' \german
%\ex vise `show', spise `eat', møde `meet' \danish
%\ex sýna `show', eta `eat', mæta `meet', hitta `meet' \icelandic
\zl
If you are uncertain as far as case is concerned, you may use the
  Wiktionary: \url{https://de.wiktionary.org/}.

\item Draw the analysis tree for the following clause:
\ea
that the box was opened
\z
Please provide valence features (\spr and \comps) and part of speech information. You may abbreviate
the NP using a triangle.

%% \item Draw the analysis tree for the following sentence:
%% \ea
%% \gll dass der Kasten geöffnet wurde\\
%%      that the box    opened was\\
%% \glt `that the box opened was'
%% \z
%% Please provide valence features (\comps only) and part of speech information. You may abbreviate
%% the NP using a triangle.

\end{enumerate}

}




%      <!-- Local IspellDict: en_US-w_accents -->
