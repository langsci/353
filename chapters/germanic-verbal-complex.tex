%% -*- coding:utf-8 -*-

\chapter{The verbal complex}
\label{chap-verbal-complex}


\section{The phenomenon}

SOV languages like Dutch and German form verbal complexes, that is, combinations of verbs excluding
the non"=verbal arguments of the verbs. For example, it is assumed that \emph{zu lesen} `to read',
\emph{versprochen} `promised' und \emph{hat} `has' form a constituent in Haider's example (\citeyear[\page
  110]{Haider86c}; \citeyear[\page 128]{Haider90b}) in (\mex{1}) to the exclusion of the arguments of the verbs
\emph{es} `it', \emph{ihr} `her' and \emph{jemand} `somebody'.
\ea\label{ex-weil-es-ihr-jemand-zu-lesen-versprochen-hat}
\gll weil es ihr jemand zu lesen versprochen hat\\
     because it her somebody to read promised has\\
\glt `because somebody promised her to read it'
\z
There are several indicators for verbal complex formation
that were worked out in detail by Gunar \citet{Bech55a}. As indicated above, one way to analyze such verbal
complexes is to assume that the verbs in a sentence form a unit that basically behaves like a
simplex verb. This explains for instance why the arguments of the three verbs in (\mex{0}) can be scrambled:
\emph{es} depends on \emph{zu lesen} `to read', \emph{ihr} `her' depends on \emph{versprochen}
`promised' and \emph{jemand} is the subject and agrees with the finite verb \emph{hat} `has'
(usually it is also treated as a dependent of the auxiliary \emph{hat}).

It should be said that there is extreme variation in the German dialects as far as the serialization
of elements in the verbal complex is concerned. The governing\footnote{
  The term \emph{govern}\is{government} is used equivalently to \emph{select}. When a verb requires an accusative
  object it is said to govern the accusative/an accusative object. Verbs can also govern other verbs
  and determine the form the governed verb has to take, \eg, participle perfect/passive, infinitive
  with or without \emph{zu} `to'. 
} verb is realized to the right of the
embedded verb in Standard German: V$_3$ V$_2$ V$_1$ as in (\mex{0}), but there are examples like
(\mex{1}) taken from \citew[376]{Mueller99a}.\footnote{
Interview partner in: \emph{Insekten und andere Nachbarn -- ein Haus in Berlin}, ARD 1995-11-15.
}
\eal
\ex 
\gll Ich hätte stapelweise Akten kön\-nen haben.\\
     I   had   by.the.pile files can      have\\\hfill (German, Berlin dialect)
\glt `I could have had files by the pile.'
\ex 
\gll weil ich mir das  nich hab' lassen gefallen\\
     because I me that not  have let    please\\
\glt `because I did not put up with it'
\ex 
\gll wenn se   mir hier würden rausschmeißen, \ldots\\
     if   they me  here would  out.throw\\
\glt `if they would kick me out here'
\zl
The orders in (\mex{0}) correspond to the order that is most natural in Dutch. (\mex{1}) shows some
Dutch examples: 
\eal
\ex
\gll dat   Kim het boek wil lezen\\
     that  Kim the book wants read\\
\glt 'that Kim wants to read the book' 
\ex
\gll dat  Kim Sandy het boek laat lezen\\
     that Kim Sandy the book lets read\\
\glt 'that Kim lets Sandy read the book'
\ex 
\gll dat Kim Sandy het boek wil laten lezen\\
     that Kim Sandy the book wants let read \\
\glt 'that Kim wants to let Sandy read the book'
\zl

SVO languages like Danish and English do not allow the arguments of embedded verbs to be scrambled
with arguments of higher verbs. All arguments stay in their VP (modulo extraction, of course).


\section{The analyis}


The technique that is used to analyze the verbal complexes is called \emph{argument attraction} or
\emph{argument composition} and was developed by \citet{Geach70a} in the framework of Categorial
Grammar and adapted for HPSG by \citet{HN94a}. The analysis of \emph{lesen wird} `read will' as it occurs in
(\mex{1}) is shown in Figure~\vref{fig-lesen-wird}.
\ea
\gll dass keiner das Buch lesen wird\\
     that nobody the book read will\\
\glt `that nobody will read the book'
\z
\begin{figure}
\begin{forest}
sm edges
[V\feattab{
%              \vform \type{fin},\\
              \sliste{ NP[\type{nom}], NP[\type{acc}] } } 
        [V\feattab{
%              \vform \type{bse},\\
              \sliste{ NP[\type{nom}], NP[\type{acc}]} }, name=lesen [lesen;read] ]
        [V\feattab{
%              \vform \type{fin},\\
              \sliste{ NP[\type{nom}], NP[\type{acc}], V }}, name=wird [wird;will] ]
]
\draw[semithick,->] (lesen)..controls +(south east:2) and +(south west:2)..(wird);
\end{forest}
\caption{\label{fig-lesen-wird}Analysis of the verbal complex formation of \emph{lesen wird} `read
  will' using argument composition (preliminary version)}
\end{figure}
\emph{wird} `will' selects an infinitive without \emph{zu} and in addition its arguments. This
infinitive (\emph{lesen} `read') is combined with the verb and hence is not contained in the valence list
of the mother node.

Returning to our meal-shopping analogy from p.\,\pageref{page-shopping-analogy}, the verbal complex
formation can be envisaged by imagining a young and helpful auxiliary verb helping out a person of
the high risk group in the middle of a pandemic. Since high risk persons are not supposed to do
shopping, the helpful person takes over their shopping list and does the shopping for them. In the
case of auxiliary verbs the auxiliary verb just selects the main verb and apart from this does not
require any further arguments apart from the ones taken over from the embedded verb. This means the
auxiliary verb just does the shopping for the main verb. A very altruistic verb it is. Later we will
have a look at verbs like \emph{try} and \emph{let} that do require their own arguments in addition
to the ones of the embedded verb. This will be parallel to a shopping event where the helping person
buys its own goods in addition to buying goods for somebody else.

%Now, back to linguistics and to some details of the analysis: 
The combination of \emph{lesen} and \emph{wird} behaves like a simplex verb in that it can be
combined with its arguments in any order. Figure~\vref{fig-vc-nom-acc} shows the analysis of (\mex{1}a) and
Figure~\vref{fig-vc-acc-nom} shows the analysis of (\mex{1}b).
\eal
\ex\label{ex-dass-keiner-das-buch-lesen-wird}
\gll [dass]         keiner das Buch lesen wird\\
     \spacebr{}that nobody the book read will\\
\glt `that nobody will read the book'
\ex  
\gll [dass] das Buch keiner lesen wird\\
     \spacebr{}that the book nobody read will\\
\glt `that nobody will read the book'
\zl

\begin{figure}
\centerfit{
\begin{forest}
sm edges
[V\feattab{
              \sliste{ }}
        [{NP[\type{nom}]} [keiner;nobody] ]
        [V\feattab{
              \sliste{ NP[\type{nom}] }}, s sep+=1em
          [{NP[\type{acc}]} [das Buch;the book, roof] ]
          [V\feattab{
%              \vform \type{fin},\\
              \sliste{ NP[\type{nom}], NP[\type{acc}]}}
             [V\feattab{
%              \vform \type{bse},\\
              \sliste{ NP[\type{nom}], NP[\type{acc}]}} [lesen;read] ]
             [V\feattab{
%              \vform \type{fin},\\
                \sliste{ NP[\type{nom}], NP[\type{acc}], V }} [wird;will] ] ] ] ]
\end{forest}}
\caption{\label{fig-vc-nom-acc}Formation of a verbal complex and realization of arguments in normal
  order (preliminary version)}
\end{figure}




\begin{figure}
\centerfit{
\begin{forest}
sm edges
[V\feattab{
              \sliste{ }}
        [{NP[\type{acc}]} [das Buch;the book, roof] ]
        [V\feattab{
              \sliste{ NP[\type{acc}] }}
          [{NP[\type{nom}]} [keiner;nobody] ]
          [V\feattab{
%              \vform \type{fin},\\
              \sliste{ NP[\type{nom}], NP[\type{acc}] }} 
             [V\feattab{
%              \vform \type{bse},\\
              \sliste{ NP[\type{nom}], NP[\type{acc}]}} [lesen;read] ]
             [V\feattab{
%              \vform \type{fin},\\
                \sliste{ NP[\type{nom}], NP[\type{acc}], V }} [wird;will] ] ] ] ]
\end{forest}}
\caption{\label{fig-vc-acc-nom}Formation of a verbal complex and scrambling of arguments (preliminary version)}
\end{figure}

I follow
\citet[Section~3.1.1]{Kiss95a} and represent the subject of non-finite verbs as the value of a
special feature \subj. \subj differs from \spr and \comps in that it is not a valence feature. The
reason for this special treatment is that the subject cannot be realized as a part of a non-finite
verb phrase. This is especially clear for infinitives with \emph{zu}:
\eal
\ex[]{
\gll Aicke hat Conny versprochen, [das Buch zu lesen].\\
     Aicke has Conny promised     \spacebr{}the book to read\\
\glt `Aicke promised Conny, to read the book.'
}
\ex[*]{
\gll Aicke hat Conny versprochen, [sie das Buch zu lesen].\\
     Aicke has Conny promised     \spacebr{}the book to read\\
\glt Intended: `Aicke promised Conny that she will read the book.'
}
\zl
Other non-finite verbs (bare infinitives and participles) cannot be placed into the \nf, but they
can be fronted. (\mex{1}) shows that the subject cannot be realized together with other arguments
of the verb in the \vf.
\eal
\judgewidth{?*}
\ex[]{
\gll [Das Buch lesen] wird Aicke morgen.\\
     \spacebr{}the book read  will Aicke  tomorrow\\
\glt `Aicke will read the book tomorrow.'
}
\ex[*]{
\gll [Aicke lesen] wird das Buch morgen.\\
     \spacebr{}Aicke  read  will the book tomorrow\\
}
\ex[?*]{
\gll [Aicke das Buch lesen] wird morgen.\\
     \spacebr{}Aicke  the book read will tomorrow\\
}
\zl
The lexical item for the non-finite form of \emph{lesen} `to read' is given in (\mex{1}):
\ea
\emph{lesen} `to read' non-finite form:\\
\ms{
subj  & \sliste{ NP[\type{nom}] }\\
comps & \sliste{ NP[\type{acc}] }\\
}
\z

\noindent
The following Attribute Value Matrix (AVM)\is{Attribute Value Matrix (AVM)} is a representation of the auxiliary verb \emph{werden} `will':
\eas
\emph{werden} `will' non-finite form:\\
\ms{
subj  & \ibox{1}\\
comps & \ibox{2} $\oplus$ \sliste{ V[\vform \type{bse}, \textsc{lex}+, \subj \ibox{1}, \comps \ibox{2}] }\\
}
\zs
\emph{werden} selects a verb that has the \type{bse} form, that is an infinitive without \emph{zu}
`to'. The embedded element has to be lexical (\textsc{lex}+), that is, a single word or a verbal
complex. All phrases that are licensed by the Head-Complement Schema and the Specifier-Head Schema
are assumed to be \textsc{lex}$-$.
The boxes with numbers are basically variables. Their values depend on the values of the
embedded verbs. Therefore this lexical item can be used with a verb like \emph{lesen} `to read',
which takes a nominative and an accusative case but also with a verb like \emph{helfen} `to help',
which takes a nominative and a dative object.

Before I turn to the details of the analysis, I have to provide the lexical items for the finite
form of auxiliaries. Since the subject of finite verbs can of course be realized, it has to be
represented in one of the valence lists. As was discussed in Section~\ref{sec-intro-spr-comps},
German subjects are represented in the \compsl of finite verbs. Hence the lexical item for
\emph{wird} `will' has the following form:
\eas
\emph{wird} `will' finite form:\\
\ms{
subj  & \sliste{}\\
comps & \ibox{1} $\oplus$ \ibox{2} $\oplus$ \sliste{ V[\vform \type{bse}, \textsc{lex}+, \subj \ibox{1}, \comps \ibox{2}] }\\
}
\zs
This basically says that the valence of \emph{wird} consists of an embedded verb and whatever the
\subjl of this verb is plus whatever the \compsl of this verb is. This is exemplified for
\emph{lesen wird} in Figure~\vref{fig-lesen-wird-details}.\footnote{%
The lexical items of complex forming predicates require their verbal argument to be \lex$+$. So in
Figure~\ref{fig-lesen-wird-details}, \emph{lesen} `read' is \lex$+$ as well. Since this information
is not relevant for the discussion of argument attraction it is omitted in
Figure~\ref{fig-lesen-wird-details} and the following figures. \emph{lesen} and \emph{lesen können}
are required to be \lex$+$ in the verbal complexes depicted in the
Figures~\ref{fig-lesen-koennen-wird} and \ref{fig-wird-lesen-koennen}. Of course, lexical items like
\emph{wird} `will' and \emph{können} `can' are \lex$+$ as well, due to the fact that they are
words. Displaying this information in the trees would be confusing rather than adding to the
explanation and hence, I decided to omit the \lex information.
}
\begin{figure}
\begin{forest}
sm edges
[V\feattab{
              \vform \type{fin},\\
              \comps \ibox{1} $\oplus$ \ibox{2} } 
        [{\ibox{3} V}\feattab{
              \vform \type{bse},\\
              \subj  \ibox{1} \sliste{ NP[\type{nom}] }, \\ 
              \comps \ibox{2} \sliste{ NP[\type{acc}] } } [lesen;read] ]
        [V\feattab{
              \vform \type{fin},\\
              \comps \ibox{1} $\oplus$ \ibox{2} $\oplus$ \sliste{ \ibox{3} } } [wird;will] ] ]
\end{forest}
\caption{\label{fig-lesen-wird-details}Detailed analysis of a verbal complex}
\end{figure}
The auxiliary selects an infinitive without \emph{zu} `to' \iboxb{3}. This is ensured by the value
\type{bse} for the \vformf of the selected verb: \type{bse}\istype{bse} stands for infinitive without
\emph{to}/\emph{zu}/\ldots{}, \type{inf}\istype{inf} stands for an infinitive form with marker, \type{ppp}\istype{ppp}
stands for participle and \type{fin}\istype{fin} for a finite verb. The subject of the selected infinitive
\iboxb{1} and the complements \iboxb{2} are taken over. The result is that \emph{lesen wird} has the
same arguments as \emph{liest} `reads'. 

To make all of this even more fun, we can make it more complex and look at verbal complexes with
three verbs. Figure~\vref{fig-lesen-koennen-wird} shows the analysis of the verbal complex \emph{lesen können wird} `read
can will' in sentences like (\mex{1}):
\ea
\label{ex-lesen-koennen-wird}
\gll [dass] sie das Buch lesen können wird\\
     \spacebr{}that she the book read can will\\
\z



\begin{figure}
\centerfit{%
\begin{forest}
sm edges
[V\feattab{
              \vform \type{fin},\\
              \comps \ibox{1} $\oplus$ \ibox{2} } 
        [{\ibox{4} V\feattab{
              \vform \type{bse},\\
              \subj  \ibox{1},\\
              \comps \ibox{2} }} 
           [{\ibox{3} V\feattab{
              \vform \type{bse},\\
              \subj  \ibox{1} \sliste{ NP[\type{nom}] }, \\ 
              \comps \ibox{2} \sliste{ NP[\type{acc}] } }} [lesen;read] ]
           [V\feattab{
              \vform \type{bse},\\
              \subj  \ibox{1},\\
              \comps \ibox{2} $\oplus$ \sliste{
                \ibox{3} } } [können;can] ] ]
        [V\feattab{
              \vform \type{fin},\\
              \comps \ibox{1} $\oplus$ \ibox{2} $\oplus$ \sliste{
                \ibox{4} } } [wird;will] ] 
]
\end{forest}}
\caption{\label{fig-lesen-koennen-wird}Analysis of a German verbal complex with three verbs in cannonical order}
\end{figure}

One interesting aspect of the analysis is that it can explain a phenomenon that is called Auxiliary
Flip\is{Auxiliary Flip} or \emph{Oberfeldumstellung}\is{Oberfledumstellung@\emph{Oberfeldumstellung}}. German optionally allows verbs that govern a modal to be placed
to the left of the verbal complex rather than to the right of the modal. So instead of
(\ref{ex-lesen-koennen-wird}) one can also use the order in (\mex{1}):
\ea
\gll [dass] sie das Buch wird lesen können\\
     \spacebr{}that she the book will read can\\
\z
\begin{figure}
\centerfit{%
\begin{forest}
sm edges
[V\feattab{
              \vform \type{fin},\\
              \comps \ibox{1} $\oplus$ \ibox{2} } 
        [V\feattab{
              \vform \type{fin},\\
              \comps \ibox{1} $\oplus$ \ibox{2} $\oplus$ \sliste{
                \ibox{4} } } [wird;will] ]
        [{\ibox{4} V\feattab{
              \vform \type{bse},\\
              \subj  \ibox{1},\\
              \comps \ibox{2} }} 
           [{\ibox{3} V\feattab{
              \vform \type{bse},\\
              \subj  \ibox{1} \sliste{ NP[\type{nom}] }, \\ 
              \comps \ibox{2} \sliste{ NP[\type{acc}] } }} [lesen;read] ]
           [V\feattab{
              \vform \type{bse},\\
              \subj  \ibox{1},\\
              \comps \ibox{2} $\oplus$ \sliste{
                \ibox{3} } } [können;can] ] ] 
]
\end{forest}}
\caption{\label{fig-wird-lesen-koennen}Analysis of a German verbal complex with three verbs with
  Auxiliary Flip}
\end{figure}


After having discussed the analysis of verbal complexes as they are known from the OV languages like
German, Dutch, and Afrikaans, I want to briefly comment on the SVO languages like Danish and English
and so on. Usually a head requires its argument to be saturated, that is, the
\compsv has to be the empty list for NP, PPs, APs, CPs and sentential and VP arguments. Verbal complexes are different: words are combined directly. The
VO languages differ from the OV languages in not allowing this. In VO languages the verb forms a
phrase with its complements and this verb phrase may be embedded under another verb. (\mex{1}a)
shows an example with auxiliary verbs, (\mex{1}b) is an example with a full verb that takes an
infinitive verb phrase with \emph{to} and an object in addition.
\eal
\ex Kim [will [have [read the book]]].
\ex Somebody [promised her [to read it]].
\zl
Languages like Danish and English only have the Head-Complement Schema and the Specifier-Head
Schema, while languages like Dutch and German have an additional schema that can combine unsaturated
words.\todostefan{check whether this has to be explained better} The schema for predicate complex formation is sketched in Figure~\vref{fig-pred-complex}.
\begin{figure}
\begin{forest}
[{[\comps \ibox{1}]}
  [\ibox{2} ]
  [{[\comps \ibox{1} $\oplus$ \sliste{ \ibox{2} }]}]]
\end{forest}
\caption{\label{fig-pred-complex}Sketch of the Predicate Complex Schema}
\end{figure}
This schema is very similar to the Head-Complement Schema that was given on
page~\pageref{fig-head-comp}. The difference is that this schema does not license \lex$-$ elements
as the Specifier-Head and Head-Compelement Schema do. Therefore, the combination of two verbs is
compatible with \lex+ requirements by governing verbs and an embedding in even more complex verbal
complexes is possible. Figure~\ref{fig-lesen-koennen-wird} is an example: the combination of
\emph{lesen} `read' and \emph{können} `can' is compatible with the \lex+ requirement of \emph{wird}
`will'. The Predicate Complex Schema also differs from the Head-Complement Schema in German in that
it combines the head with the last element of the valence list. This is the embedded verb and it ha
to be combined with the head before any other argument since one would not know what the other
arguments are because they are taken over from the embedded verb.

Before turning to the next phenomenon, I want to briefly discuss the alternative to the verb complex
analysis presented here. One alternative suggestion was to analyze auxiliaries in German as VP
embedding verbs \citep{Wurmbrand2003b}. Our standard example would then have the analysis in (\mex{1}):
\ea
\gll dass keiner [[das Buch lesen] wird]\\
     that nobody \hphantom{[[}the book read will\\
\z
The question that such analyses have to answer is how scrambling of arguments of the involved verbs
can be accounted for. The answer is often that it is assumed that the object of the embedded verb is
extracted from the VP and moved to the left periphery of the clause. This is shown in (\mex{1}):
\ea
\gll dass [das Buch]$_i$ keiner [[ \_$_i$ lesen] wird]\\
     that \spacebr{}the book nobody {} {} read will\\
\glt `that nobody will read the book'
\z
However, analyses that treat scrambling as movement are problematic since they predict additional
readings of sentences that have quantifiers in their NPs (\citealp[\page 146]{Kiss2001a}; \citealp[Section~2.6]{Fanselow2001a}).


Before I turn to the analysis of the verb position, I want to show how sentences with several verbs
\label{page-English-Aux-VPs}
in SVO languages can be analyzed. Figure~\vref{fig-vp-embedding-svo} shows the analysis of the English version of sentence
(\ref{ex-dass-keiner-das-buch-lesen-wird}).
\begin{figure}
\centerfit{
\begin{forest}
sm edges
[S
        [{NP[\type{nom}]} [nobody] ]
        [VP
          [V%% \feattab{
            %%   spr \sliste{ NP[\type{nom}] }\\
            %%   comps \sliste{ VP }} 
              [will] ]
          [VP%% \feattab{
             %%  spr \sliste{ NP[\type{nom}] }}
            [V%% \feattab{
              %% spr \sliste{ NP[\type{nom}] }\\
              %% comps \sliste{ NP[\type{acc}] }} 
              [read] ]
            [{NP[\type{acc}]} [the book, roof] ] ] ] ]
\end{forest}}
\caption{\label{fig-vp-embedding-svo}Embedding of a VP in SVO languages}
\end{figure}
The verb \emph{reads} selects a subject and an object. The verb forms a VP with the NP \emph{the
  book}. This VP is still lacking a subject. The auxiliary \emph{will} selects a VP and a subject
that is identical with the subject of \emph{read}. The combination of \emph{will} and the VP is
licensed by the Head-Complement Schema that was sketched in Figure~\vref{fig-head-comp}.

The equivalent of \emph{lesen können wird} `read can will' cannot be given here, since English modal
verbs do not have non-finite forms, but one can construct examples with modals as the highest verb:
\ea
She [must [have [seen it]]].
\z
This sentence has a structure that is similar to the one in Figure~\ref{fig-vp-embedding-svo}:
\emph{must} and \emph{have} both embedd VPs.

Finally, Figure~\vref{fig-somebody-has-promised-him-to-read-it} shows the translation of
(\ref{ex-weil-es-ihr-jemand-zu-lesen-versprochen-hat}):
\ea
Somebody has promised her to read it.
\z
\emph{promise} is a verb that takes a subject, an object, and a VP complement. 
\begin{figure}
\centerfit{
\begin{forest}
sm edges
[S
        [{NP[\type{nom}]} [somebody] ]
        [VP
            [V [has]]
            [VP 
              [\vbar [V [promised] ]
                     [{NP[\type{acc}]} [her] ] ]
              [VP [V [to]]
                [VP
                  [V [read] ]
                  [{NP[\type{acc}]} [the book, roof] ] ] ] ] ] ]
\end{forest}}
\caption{\label{fig-somebody-has-promised-him-to-read-it}Embedding of a VP with verbs that take an
  additional object}
\end{figure}
Like in the analysis of (\ref{ex-nobody-gave-the-child-a-book}) on
page~\pageref{ex-nobody-gave-the-child-a-book} -- which is repeated here as (\mex{1}) for convenience
-- the verb \emph{promised} is combined with its NP complement first and then with its VP
argument.
\ea
\label{ex-nobody-gave-the-child-a-book-three}
Nobody gave the child a book.
\z 
The VP argument of \emph{promised} in Figure~\ref{fig-somebody-has-promised-him-to-read-it} consists of \emph{to} and another VP with an infinitive in base form. \emph{to} is
analyzed as an auxiliary verb \parencites[\page 600]{GPS82a-u}[\page 147]{Sag2020a}.
It is important to note that the object \emph{him} cannot appear in any other position (appart from
extraction to the left periphery). For instance, it cannot appear in the position of \emph{the book}
and the same holds for \emph{the book}: This phrase cannot appear in any other place than in the
object position.


\exercises{


\begin{enumerate}
\item Sketch the analysis of the verbal complexes in the following examples:
\eal
\ex
\gll dass sie darüber lachen muss\\
     that she there.about laugh must\\\german
\glt `that she has to laugh about it' 
\ex 
\gll dass sie darüber hat lachen müssen\\
     that she there.about has laugh  must\\
\glt `that she had to laugh about it' 
\ex 
\gll dass sie darüber     wird haben lachen müssen\\
     that she there.about will have  laugh  must\\
%\glt `that there will be a state in the future where she had to laugh about it'
\zl
You may omit the \spr values, since they are the empty list for all German verbs anyway.


\item Search for two sentences with a verbal complex in a newspaper or in corpora (for example the
  COSMAS corpus\footnote{%
\url{https://cosmas2.ids-mannheim.de}, 2020-05-11.}) and
  analyze the verbal complexes.

\item Search for verbal complexes with more than four verbs in a corpus and document your search.
\end{enumerate}

}

\furtherreading{The analysis of verbal complexes in HPSG was first developed by \citet{HN89b,HN89a}.
\citew{HN94a} is the first peer reviewed publication on this topic by Hinrichs \&
Nakazawa. \citew{Kiss95a} is a monograph dealing with verbal complexes. \citet{Meurers2000b} also
deals with verbal complexes including difficult cases like the so-called \emph{Zwischenstellung},
which is no treated here. \citet{Mueller2002b} treats not just verbal complex but also other types
of complex predicates like adjective verb complexes, resultative constructions and particle verbs.

\citet{GS2021a} provide an overview about analyses of complex predicates in HPSG.}



%      <!-- Local IspellDict: en_US -->
