%% -*- coding:utf-8 -*-
\chapter{HPSG Light}
\label{chap-HPSG-light}


The theory behind what was explained in this book is Head-Driven Phrase Structure Grammar (HPSG)
\parencites{ps}{ps2}{MuellerLehrbuch3}{HPSGHandbook}. However, HPSG is a rather complex theory and
going into all the relevant details would be too much for an introduction into syntax at BA
level. Hence, I simplyfied wherever possible. For example, I have been using sketchy schemata like
the one in Figure~\ref{fig-spr-head-head-comps-final-rep} for Specifier-Head structures and Head-Complement structures. 
\begin{figure}
\hfill
\begin{forest}
[H\feattab{\spr \ibox{1},\\
           \comps \ibox{2}}
  [\ibox{3}]
  [H\feattab{\spr \ibox{1} $\oplus$ \sliste{ \ibox{3} },\\
              \comps \ibox{2} \eliste}]]
\end{forest}
\hfill
\begin{forest}
[H\feattab{\spr \ibox{1},\\
           \comps \ibox{2}}
  [H\feattab{\spr \ibox{1},\\
             \comps  \sliste{ \ibox{3} } $\oplus$ \ibox{2}  ]}]
  [\ibox{3}]]
\end{forest}
\hfill\mbox{}
\caption{\label{fig-spr-head-head-comps-final-rep}Sketch of the Specifier-Head and Head-Complement Schema}
\end{figure}
While attribute-value matrices (AVMs) in HPSG are complex nested representations, I used rather
simple lists of feature-value pairs throughout the book so far. The intention was to keep things
simple and hence I used this way to depict schemata as little partial trees. However, not much is
needed to extend what you have seen so far to what is standard in the fully worked out theory of
Head-Driven Phrase Structure Grammar. 

In what follows, I first show how feature-value pairs can be used to encode constituent structure (Section~\ref{sec-HPSG-constituent-structure}) and
then talk about the general makeup of linguistic descriptions in HPSG.

\section{Constituent structure}


\section{Structure of linguistic objects}

\ea
Lexical item for the word \emph{Grammatik} `grammar':\\
\onems[word]{
phonology    \phonliste{ Grammatik } \\[1mm]
syntax-semantics  \ldots \ms[local]{ category  & \ms[category]{ head & \ms[noun]{ case & \ibox{1}
                                                                                               }\\[3mm]
                                                                               spr & \sliste{ Det[\textsc{case}~\ibox{1}] }\\
                                                                               \ldots\\
                                                                             } \\[10mm]
                                          content & \ldots \ms[grammatik]{ inst & X 
                                                                                    }
            }
}
\z


\section{Some details}


Nonlocal depenendecies.










%      <!-- Local IspellDict: en_US-w_accents -->
