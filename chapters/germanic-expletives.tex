%% -*- coding:utf-8 -*-
\chapter{Clause types and expletives}
\label{chap-expletives}

%Yiddish besseres Beispiel nehmen, das wirklich V2 zeigt.

Chapter~\ref{chap-verb-position} discussed the analysis of verb-initial and verb-second clauses. This chapter deals
with embedded clauses with complementizers (\eg \emph{that} in English) and embedded interrogative
clauses. 


\section{The phenomena}


This first section introduces the phenomenon and consists of three parts: Section~\ref{sec-complementizer} deals with embedded
clauses introduced by a complementizer, Section~\ref{sec-phen-interrogatives} describes the
structure of interrogative clauses and Section~\ref{sec-phen-use-of-expletives} deals with expletives in V2 and interrogative clauses.


\subsection{Embedded clauses introduced by a complementizer}
\label{sec-complementizer}

As was already mentioned, Afrikaans, Dutch, German are SOV languages and this is also shown in embedded
clauses that are introduced by a complementizer. (\mex{1}) is an example:

\ea
\gll Ich weiß, dass Aicke das Buch heute gelesen hat.\\
     I know that Aicke the book today read has\\\german
\glt `I know that Aicke read the book today.'
\z


%% Stellung der anderen Konstituenten ist frei:
%% \eal
%% \ex Ich weiß, dass das Buch Max heute gelesen hat.
%% \ex Ich weiß, dass das Buch heute Max gelesen hat.
%% \zl

English, being an SVO non-V2 language, allows for SVO order only.
\ea[]{
I  know that Kim has read the book yesterday. \jambox{(English)}
}
\z
However, elements can be fronted in \emph{that} clauses:
\eal
\ex[]{
I know that yesterday Peter came.
}
\ex[?]{
I know that bagels, he likes.
}
\zl


Interestingly, Danish, also an SVO language, allows both SVO order (\mex{1}) and V2 order (\mex{2}) in clauses
preceded by a complementizer:
\ea[]{
\label{ex-at-max-ikke-har-3}
\gll Jeg  ved, at   Gert ikke  har læst   bogen          {i dag}.\\
     I know that Gert not has read book.\textsc{def} today\\\jambox{(Danish)}
\glt `I know that Gert did not read the book today.'
}
\z
%(Negation hilft, Verbstellung zu bestimmen)

\eal
\ex[]{
\gll Jeg  ved, at   {i dag} har Gert ikke læst bogen.\\
     I    know that today   has Gert not read book.\textsc{def}\\\jambox{(Danish)}
}
\ex[]{
\gll Jeg  ved, at   bogen          har Gert ikke  læst {i dag}.\\
     I know that book.\textsc{def} has Gert not read today\\
}
\zl
The example in (\ref{ex-at-max-ikke-har-3}) includes the negation in order to show that we indeed deal
with the SVO order here. Without the negation it is not clear whether non-V2 clauses are allowed in
clauses that are introduced by a complementizer since (\mex{1}a) has the finite verb in second
position. With the negation present, it is clear that we have a V2 clause if the negation follows
the finite verb as in (\mex{1}b) and that we do not have a V2 clause if the finite verb follows the negation as in
(\mex{-1}) and hence is in third position.
\eal
\settowidth\jamwidth{(V2 or SVO)}
\ex 
\gll at Gert har læst bogen\\
     that Gert has read book.\textsc{def}\\\jambox{(V2 or SVO)}
\glt `that Gert has read the book'
\ex 
\gll at   Gert har ikke læst bogen\\
     that Gert has not  read book.\textsc{def}\\\jambox{(V2)}
\zl 
For complementizerless clauses, the V2 order is the only order that is possible:
\eal
\settowidth\jamwidth{(V2 or SVO)}
\ex[]{ 
\gll Gert har ikke læst bogen\\
     Gert has not read book.\textsc{def}\\\jambox{(V2)}
}
\ex[*]{ 
\gll Gert ikke har læst bogen\\
     Gert not  has read book.\textsc{def}\\\jambox{(SVO)}
}
\zl 


Yiddish and Icelandic are SVO languages as well. The clauses that are combined with a
complementizer are V2:
\eal
\ex
\gll Ikh meyn  az   haynt hot Max geleyent dos bukh.\footnotemark\\
     I think that today has Max read the book\\\yiddish
\footnotetext{\citew[\page 58]{Diesing90a}}
\glt `I think that Max read the book today.'

\ex% check!
\gll Ikh meyn  az   dos bukh hot Max geleyent.\\
     I think that the book has Max read\\

\zl
\ea 
\longexampleandlanguage{
\gll Engum         datt í hug,  að   vert  væri að reyna til     að kynnast honum.\footnotemark\\
     no.one.\DAT{} fell to mind that worth was  to try   \PREP{} to know    him\\}{Icelandic}
\footnotetext{\citew[\page 75]{Maling90a-u}}
\glt `It didn't occur to anyone that it was worth trying to get to know him.'
\z

\subsection{Interrogative clauses}
\label{sec-phen-interrogatives}

The OV languages form subordinated interrogative clauses by preposing a phrase containing an
interrogative pronoun\footnote{
Most interrogative pronouns start with \emph{w} in German and \emph{wh} in English. Phrases
containing an interrogative pronoun are called \emph{w}"=phrases or \emph{wh}"=phrases,
respectively. Interrogative clauses are sometimes called \emph{w}"=clauses or \emph{wh}"=clauses.
} from an
otherwise SOV clause. (\mex{1}) shows a German example:
\eal
\ex 
\gll Ich weiß, wer heute das Buch gelesen hat.\\
     I know    who today the book read has\\\jambox{(German)}
\glt `I know who read the book today.'
\ex 
\gll Ich weiß, was Aicke heute gelesen hat.\\
     I know    what Aicke today read has\\
\glt `I know what Aicke has read today.'
\zl
Since languages like German allow for scrambling, sentences like those in (\mex{0}) could just be due
to the permutation of arguments of a head. However, the generalization about these \emph{w}"=clauses
is that an arbitrary \emph{w}"=element can be fronted. (\mex{1}) gives a German example involving a nonlocal dependency:
\ea
\label{ex-wissen-Vortrag-halen-nonlocal}
\longexampleandlanguage{
\gll Ich weiß nicht, [über welches Thema]$_i$ sie versprochen hat, [[einen Vortrag \_$_i$] zu halten].\\
     I know not      \spacebr about which topic she promised has \hphantom{[[}a talk to  hold\\}{German}
\glt `I do not know about which topic she promised to give a talk.'
\z
Here, the phrase \emph{über welches Thema} `about which topic' is an argument of \emph{Vortrag},
which is embedded in the VP containing \emph{zu halten} `to hold', which is in turn embedded under
\emph{versprochen hat} `promised has'. The generalization about interrogative clauses is that an
interrogative clause consists of a interrogative phrase (\emph{über welches Thema} `about which
topic') and a clause in which this interrogative phrase is missing somewhere (\emph{sie versprochen
  hat, einen Vortrag zu halten} `she promised to give a talk').

In German the order of the other constituents is free as in assertive main clauses and embedded
clauses with a complementizer that were discussed earlier. 
\eal
\ex
\gll Ich weiß, was keiner diesem Eichhörnchen geben würde.\\
     I know    what nobody this squirrel give would\\\jambox{(German)}
\glt `I know what nobody would give this squirrel.'
\ex 
\gll Ich weiß, was diesem Eichhörnchen keiner geben würde.\\
     I know what this squirrel nobody give would\\
\zl
This follows from what we said so far, since interrogatives are just SOV clauses with one
constituent extracted. The possibility of scrambling constituents is not affected by extracting a phrase.

In Danish and English, the interrogative clauses consist of an interrogative phrase and an SVO clause
in which it is missing:
\eal
\ex 
\gll Gert har givet ham bogen.\\
     Gert has given him book.\textsc{def}\\\jambox{(Danish)}
\glt `Gert gave him the book.'
\ex
\gll Jeg ved, hvad$_i$ [Gert har givet ham \_$_i$].\\
     I know what \spacebr{}Gert has given him\\
\glt `I know what Gert gave him.'
\ex
\gll Jeg ved, hvem$_i$ [Gert har givet \_$_i$   bogen].\\
     I know who        \spacebr{}Gert has given {} book.\textsc{def}\\
\glt `I know who Gert has given the book.'
\zl
(\mex{0}a) shows the clause with SVO order and (\mex{0}b) is an example with the secondary object as
interrogative pronoun and (\mex{0}c) is an example with the primary object as interrogative
pronoun. The position that the respective objects have in non-interrogative clauses like (\mex{0}a)
is marked with \_$_i$.

Yiddish is special in that it has V2 order in interrogative clauses as well \citep[Sections~4.1, 4.2]{Diesing90a}: interrogatives
consist of an interrogative phrase that is extracted from a V2 clause:

\ea
\gll Ir veyst efsher [avu            do    voynt Roznblat   der goldshmid]?\footnotemark\\
     you know maybe  \spacebr{}where there lives Roznblat the goldsmith\\
\glt `Do you perhaps know where Roznblat the goldsmith lives?' 
% The gollsing an translation differes in the original Rozenblatt
\footnotetext{
\citew[\page 65]{Diesing90a}. Quoted from Olsvanger, \emph{Royte Pomerantsn}, 1949.
}
\z
%
%% \eal 
%% \ex
%% \label{vosmaks}
%% \gll Ikh veys nit   vos$_i$ [Max hot gegesn \_$_i$].\footnotemark\\
%%      I know not     what \hphantom{[}Max has eaten\\
%% \footnotetext{\citew[\page 68]{Diesing90a}.}
%% \glt `I do not know what Max has eaten.'

%% \ex%check
%% \gll Ikh veys nit   [vos                    hot Max gegesn].\footnotemark\\
%%      ich weiß nicht \hphantom{[}was heute hat Max gegessen\\
%% \footnotetext{\citew[p.\,68]{Diesing90a}.}
%% \glt `Ich weiß nicht, was Max heute gegessen hat.'
%\zl
So the variation is \emph{w}-phrase + SOV, \emph{w}-phrase + SVO, and \emph{w}-phrase + V2.

In addition to the question regarding the order within the embedded clause (SOV, SVO, V2), there is
variation in finiteness of the embedded clause:
\eal
\label{ex-what-to-read}
\ex[]{
I wonder what to read.
}
\ex[]{
I wonder what I should read.
}
\ex[*]{
\gll Ich frage mich, was zu lesen.\\
     I   ask   myself what to read\\\german
}
\ex[]{
\gll Ich frage mich, was   ich lesen soll / kann.\\
     I   ask   myself what I   read  shall {} can\\\german
\glt `I wonder what I shall/can read.' or `I wonder what to read.'
}
\zl
English allows for infinitives with \emph{to}. In comparison to finite interrogative clauses, the
infinitival form adds a modal meaning. German does not allow for non-finite interrogatives, as
(\mex{0}c) illustrates.



\subsection{The use of expletives to mark the clause type}
\label{sec-phen-use-of-expletives}

The Germanic languages use constituent order to code the clause type: V2 main clauses can be
assertions or questions, depending on the content of the preverbal material and
intonation. Similarly, embedded interrogative clauses consist of a \emph{w}"=phrase and an SVO, SOV,
or V2 clause. The fronting of a constituent in a V2 clause comes with certain information"=structural
effects: something is the topic or the focus of an utterance. For embedded clauses, it is important
in some languages that the structure is transparent, that is, that we have \emph{w} + SVO or
\emph{w} + V2 order. There are situations in which it is inappropriate to front an element, and in
such situations the Germanic languages use expletives, that is, pronouns that do not make a semantic
contribution, to maintain a certain order.

German uses the expletive \emph{es} to fill the position in front of the finite verb, if no other
constituent is to be fronted.
\eal
\ex 
\gll Drei Reiter ritten zum Tor hinaus.\\
     three riders rode  to.the gate out\\\german
\glt `Three riders rode out of the gate.'
\ex 
\gll Es ritten drei Reiter zum Tor hinaus.\\
     \expl{} rode   three riders to.the gate out\\
\zl

Danish uses the expletive \emph{der} to make it clear that an extraction of a constituent took place
\citep[\page 169]{MOe2011a}:\footnote{
  Examples marked with DK are extracted from KorpusDK, a corpus of 56 Million words documenting
  contemporary Danish (\url{http://ordnet.dk/korpusdk}).
}
\eal
\label{ex-danish-interrogative-expletive}
\ex[]{
\longexampleandlanguage{
\gll Politiet            ved   ikke,  hvem der           havde placeret bomben.\footnotemark\\
     police.\textsc{def} knows not    who  \textsc{expl} has placed bomb.\textsc{def}\\}{Danish}
\footnotetext{DK}%
\glt `The police does not know who placed the bomb.'
}
\ex[*]{
\gll Politiet ved ikke,  hvem havde placeret bomben.\\
     police.\textsc{def} knows not who has placed bomb.\textsc{def}\\
}
\zl
Without the expletive, one would have a pattern like the one in (\mex{0}b). In (\mex{0}b) we have the
normal SVO order and it is not obvious to the hearer that the pattern consists of an extracted
element (the subject) and an SVO clause from which it is missing. This is more transparent if an
expletive is inserted into the subject position as in (\mex{0}a). (\mex{1}) shows this using the
analysis that will be suggested in Section~\ref{sec-analysis-expletives}: (\mex{1}a) shows the hypothetical
structure that would result if one assumed that the subject \emph{hvem} `who' is
extracted. So-called \emph{string-vacuous movement} would result: the subject is moved to a place
right next to it. In (\mex{1}b), on the other hand, the subject position is taken by the expletive and
hence it is clear that the embedded clause has a special structure. The expletive \emph{der} is an overt marker for
the hearer or reader of the clause marking it as an embedded interrogative clause.
\eal
\ex[*]{ 
\gll [hvem$_i$     [\trace$_i$ havde placeret bomben]]\\
     \spacebr{}who {}          has placed bomb.\textsc{def}\\\danish
}
\ex[]{
\gll [hvem$_i$     [der                    havde \trace$_i$ placeret bomben]]\\
     \spacebr{}who \spacebr{}\textsc{expl} has   {}         placed bomb.\textsc{def}\\
}
\zl

Similarly, Yiddish uses an expletive in embedded interrogatives (\emph{w} + V2) if 
%the subject is extracted and 
there is no other element that is information"=structurally appropriate for the preverbal position.
%% \ea
%% \label{vosmaks}
%% \gll Ikh veys nit   [vos Max hot gegesn].\footnotemark\\
%%      ich weiß nicht \hphantom{[}was Max hat gegessen\\
%% \footnotetext{\citew[p.\,68]{Diesing90a}.}
%% \glt `Ich weiß nicht, was Max gegessen hat.'
%% \z
%% 
%% \item 
(\mex{1}) shows examples from \citet[\page 403--404]{Prince89a}:

\eal
\label{ex-Yiddish-interrogatives-expletive}
\ex[]{
\gll ikh hob  zi  gefregt ver es         iz beser  far ir\\
     I   have her asked   who \textsc{expl} is better for her\\\yiddish
\glt `I have asked her who is better for her.'}
\ex[]{
\gll ikh hob  im  gefregt vemen es        kenen ale dayne khaverim\\
     I   have him asked   whom \textsc{expl} know  all your friends\\
\glt `I asked him whom all your friends know.'}
\zl
(\mex{0}a) is an example involving an interrogative pronoun that is the subject, and (\mex{0}b) is an
example in which the preverbal position is not filled by an argument of \emph{kenen} `know' but by
an expletive. The subject \emph{ale dayne khaverim} `all your friends' stays behind and the object
\emph{vemen} `whom' is extracted as it is the interrogative pronoun.


\if0
\subsection{Dependent clauses introduced by a complementizer}


The SOV languages (Afrikaans, Dutch, German, \ldots) combine a complementizer with a verb-last clause.
\ea
\gll Ich weiß, dass Aicke das Buch heute gelesen hat.\\
     I   know  that Aicke the book today read    has\\
\glt `I know that Aicke has read the book today.'
\z

As in V1 or V2 clauses the order of the constituents in the \mf is free in languages like German:
\eal
\ex 
\gll Ich weiß, dass das Buch Aicke heute gelesen hat.\\
     I   know  that the book Aicke today read has\\
\ex 
\gll Ich weiß, dass das Buch heute Aicke gelesen hat.
     I   know  that the book today Aicke read has\\
\zl

In English, the complementizer is combined with a clause in SVO order.
\ea
I  know that Kim has read the book yesterday.
\z

%% 
%% Andere Stellungen sind nicht möglich:
%% \eal
%% \ex[*]{
%% I know that has Max read the book yesterday.
%% }
%% \ex[*]{
%% I know that yesterday Max has read the book.
%% }
%% \zl

Danish is interesting since it is SVO like English but it has two options for complementizer
phrases: \emph{at} `that' can combine with a clause in SVO order or with a V2 clause:
\ea[]{
\label{ex-at-max-ikke-har-2}
\gll Jeg  ved, at   Gert ikke  har læst   bogen          {i dag}.\\
     I know that Gert not has read book.\textsc{def} today\\\jambox{(Danish)}
\glt `I know that Gert did not read the book today.'
}
\z

The negation \emph{ikke} can be used to determine the verb position. Since we assumed that adverbial
material adjoins to VPs the negation in (\mex{0}) is assumed to combine with \emph{har læst bogen i
  dag} `has read the book today'. The finite verb is to the left of the negation in (\mex{1}). This
fits well with the analysis of verb placement introduced in Chapter~\ref{chap-verb-position}: the
finite verb \emph{har} `has' forms a constituent with \emph{læst  bogen i dag} to which the negation can attach, but
due to verb movement the verb is put into initial position and then the subject \emph{Gert} is fronted.
\ea[]{
\gll Jeg  ved, at   Gert har ikke  læst  bogen          {i dag}.\\
     I    know that Gert has not   read  book.\textsc{def} today\\\jambox{(V2)}
}
\z

If one is not ready to believe that data like (\mex{0}) are evidence for embedded V2 clauses, the
following examples may be even more convincing:
\eal
\ex[]{
\gll Jeg  ved, at   {i dag} har Gert ikke læst bogen.\\
     I    know that today   has Gert not read book.\textsc{def}\\\jambox{(Danish)}
}
\ex[]{
\gll Jeg  ved, at   bogen          har Gert ikke  læst {i dag}.\\
     I know that book.\textsc{def} has Gert not read today\\
}
\zl
(\mex{0}a) shows a complementizer + V2 clause with \emph{i dag} `today' fronted and (\mex{0}b) shows
a complementizer with a V2 clause with \emph{bogen} `the book' fronted. \emph{har} `has' is clearly
in second position and the first position is occupied by something that is not the subject.


%% \ex[*]{
%% \gll Jeg  ved, at   {i dag} Gert har læst bogen.\\
%%      ich weiß dass heute   Gert hat gelesen Buch.{\sc def}\\
%% }
%%
%% \ex[*]{
%% \gll Jeg  ved, at   bogen          Gert har læst {i dag}.\\
%%      ich weiß dass Buch.{\sc def} Gert hat gelesen heute\\
%% }
%% \zl
%%


% Isländisch: Wikipedia (en)

\begin{itemize}
\item Yiddish: eingebettete Sätze sind V2 \citep[]{Diesing90a}:
\eal
\ex
\gll Ikh meyn  az   haynt hot Max geleyent dos bukh.\footnotemark\\
     ich   denke dass heute hat Max gelesen   das Buch\\
\footnotetext{\citew[\page 58]{Diesing90a}}
\glt `Ich denke, dass Max heute das Buch gelesen hat.'

\ex% check!
\gll Ikh meyn  az   dos bukh hot Max geleyent.\\
     ich denke dass das Buch hat Max gelesen\\

\zl


Isländisch:
\ea 
\gll Engum         datt í hug,  að   vert  væri að reyna til     að kynnast honum.\footnotemark\\
     no.one.\DAT{} fell to mind that worth was  to try   \PREP{} to know    him\\%\icelandic
\footnotetext{\citew[\page 75]{Maling90a-u}}
\glt `It didn't occur to anyone that it was worth trying to get to know him.'
\z



\end{itemize}





\section{Interrogative Clauses}



\begin{itemize}
\item Deutsch, Niederländisch, \ldots: \emph{w} + V-letzt:

\eal
\ex Ich weiß, wer heute das Buch gelesen hat.
\ex Ich weiß, was Aicke heute gelesen hat.
\zl

Interrogativnebensätze beginnen mit einer \emph{w}-Phrase.


\item Die \emph{w}-Phrase kann von weit her kommen:
\ea
Ich weiß nicht, [\gruen{über welches Thema}]$_i$ sie versprochen hat,\\
{}[[einen Vortrag \_$_i$] zu halten].
\z


\item Stellung der anderen Konstituenten ist frei:
\eal
\ex Ich weiß, was keiner diesem Eichhörnchen geben würde.
\ex Ich weiß, was diesem Eichhörnchen keiner geben würde.
\zl


\end{itemize}



\begin{itemize}
\item Dänisch: Interrogativnebensätze sind \emph{w} + SVO

\eal
\ex
\gll Jeg ved, hvad Gert har givet ham.\\
     ich weiß was Gert  hat gegeben ihm\\
\glt `Ich weiß, was Gert ihm gegeben hat.'
\ex
\gll Jeg ved, hvem Gert har givet   bogen.\\
     ich weiß wem  Gert hat gegeben Buch.{\sc def}\\
\glt `Ich weiß, wem Gert das Buch gegeben hat.'
\zl

\end{itemize}


\begin{itemize}
\item Jiddish: Interrogativnebensätze \emph{w} + V2 \citep[Abschnitte~4.1, 4.2]{Diesing90a}

%% \ea
%% %\ex
%% \label{vosmaks}
%% \gll Ikh veys nit   [vos Max hot gegesn].\footnotemark\\
%%      ich weiß nicht \hphantom{[}was Max hat gegessen\\
%% \footnotetext{\citew[p.\,68]{Diesing90a}.}
%% \glt `Ich weiß nicht, was Max gegessen hat.'

%% %% \ex%check
%% %% \gll Ikh veys nit   [vos              hot Max gegesn].\footnotemark\\
%% %%      ich weiß nicht \hphantom{[}was heute hat Max gegessen\\
%% %% \footnotetext{\citew[p.\,68]{Diesing90a}.}
%% %% \glt `Ich weiß nicht, was Max heute gegessen hat.'


\ea
\gll Ir veyst efsher [avu            do    voynt Roznblat   der goldshmid]?\footnotemark\\
     Sie wissen vielleicht  \spacebr{}wo da wohnt Roznblat der Goldschmied\\
\glt `Wissen Sie vielleicht, wo Roznblat der Goldschmied wohnt?' 
\footnotetext{
\citew[\page 65]{Diesing90a}. Quoted from Olsvanger, \emph{Royte Pomerantsn}, 1949.
}
\z
\end{itemize}

\fi

\section{The analysis}

This section will first deal with clauses introduced by a complementizer
(Subsection~\ref{sec-embedded-cp}) and the discuss embedded interrogative clauses in
Subscetion~\ref{sec-interrogatives}. Subsection~\ref{sec-analysis-expletives} discusses a lexical rule for
the introduction of expletives that play a role in various languages in marking the clause type.

\subsection{Embedded clauses introduced by a complementizer}
\label{sec-embedded-cp}


The analysis of Afrikaans, Dutch, German and English complementizer phrases is straightforward: the
complementizer is combined with an uninverted verbal projection. For the first three languages, this is a
verb"=final clause (SOV), and for English it is an SVO clause. The respective analyses are given in
Figure~\ref{fig-german-cp} and~\ref{fig-english-cp}.
\begin{figure}
\begin{forest}
sm edges
[CP
       [C [dass;that] ]
       [S
        [{NP[\type{nom}]} [niemand;nobody] ]
        [V$'$
          [{NP[\type{acc}]} [ihn;him] ]
          [V [kennt;knows] ]
           ] ] ]
\end{forest}
\caption{Analysis of German complementizer phrase as C + SOV}\label{fig-german-cp}
\end{figure}
\begin{figure}
\begin{forest}
sm edges
[CP
       [C [that] ]
       [S
        [{NP[\type{nom}]} [nobody] ]
        [VP
          [V  [knows] ]
          [{NP[\type{acc}]} [him] ] ] ] ]
\end{forest}
\caption{Analysis of English complementizer phrase as C + SVO}\label{fig-english-cp}
\end{figure}

Yiddish complementizers select a V2 clause. The analysis of the example in (\mex{1}) is shown in Figure~\ref{fig-yiddish-cp}.
\ea
\gll Ikh meyn  az   haynt hot Max geleyent dos bukh.\footnotemark\\
     I think that today has Max read the book\\\yiddish
\footnotetext{\citew[\page 58]{Diesing90a}}
\glt `I think that Max read the book today.'
\z
\begin{figure}
\begin{forest}
sm edges
[CP
        [C [az;that] ]
        [S
          [{Adv$_i$} [haynt;today] ]
          [{S/Adv}
            [{V \sliste{ S/$\!$/V }} 
              [V [hot$_k$;has] ] ]
            [{S$/\!/$V/Adv}
              [NP [Max;Max] ]
              [{VP$\!/\!/$V/Adv}
                [{V$\!/\!/$V}  [\_$_k$] ]
                [VP/Adv 
                  [VP
                    [V [geleyent;read] ]
                    [NP [dos bukh;the book,roof] ] ]
                  [Adv/Adv [\_$_i$] ] ] ] ] ] ] ]
%% \draw[connect] (Adv/Adv.north east)   [bend right] to (VP/Adv.south east);
%% \draw[connect] (VP/Adv.north east)    [bend right] to (V/V/Adv.south east);
%% \draw[connect] (V/V/Adv.north east)   [bend right] to (S//V/Adv.south east);
%% \draw[connect] (S//V/Adv.north east)  [bend right] to (S/Adv.east);
%% \draw[connect] (S/Adv.north east)     [bend right] to (Adv);
\end{forest}
\caption{Analysis of the Yiddish complementizer phrase as C + V2}\label{fig-yiddish-cp}
\end{figure}
The analysis looks complicated, but it is really just the combination of a complementizer with a V2
clause. The V2 clause has \emph{haynt} `today' extracted and the finite auxiliary \emph{hot} `has'
is moved to V1 position.

The differences between languages can be accounted for by letting the complementizer select Ses with
different feature–value combinations. While complementizers in SOV languages select verb-final
projections (\textsc{initial}$-$), Yiddish selects V2$+$ clauses (the result of applying the
Filler-Head Schema) and Danish does not specify any of such features on the selected clause. Since
nothing is specified in Danish, the embedded clause can have the form that the rest of the Danish
grammar permits: it can be SVO or V2.



\subsection{Interrogative clauses}
\label{sec-interrogatives}

As the data discussion showed, the phrase containing the interrogative pronoun is extracted from the
remaining clause. The fronting of the \emph{w}"=phrase is like fronting in V2 clauses. In the
topological fields model, the fronted phrase in German relative clauses and interrogative clauses is
assigned to the \vf \citep[\page 47--49]{MuellerGT-Eng4}. The difference between interrogative clauses and V2 clauses is the position of
the verb: V2 clauses have the verb in initial position while it is in final position in
interrogatives and relatives. Figure~\ref{fig-interrog-clause-simple} shows the analysis of the
interrogative clause in (\mex{1}). The pronominal adverb \emph{worüber} `about what' is extracted
from the rest of the clause.

\ea
\gll Ich weiß, [worüber]$_i$ [ \trace$_i$ sie spricht].\\
     I know    \spacebr{}where.about {} {} she speaks\\
\glt `I know what she speaks about.'
\z
\begin{figure}
\begin{forest}
sm edges
[S
  [PP [worüber;where.about]]
  [S/PP
    [PP/PP [\trace]]
    [V\rlap{$'$}
      [NP [sie;she]]
      [V [spricht;speaks]]]]]
\end{forest}
\caption{Analysis of simple interrogative clause}\label{fig-interrog-clause-simple}
\end{figure}

Figure~\ref{fig-interrog-clause-simple} uses the slash notation that I have been using so far. In
order to account for more complex \emph{w}"=phrases I will use the same trick as for other nonlocal
dependencies and pass information about \emph{w}"=pronouns on to mother nodes. For modeling this I
will use list"=valued features like \spr and \comps. Figure~\ref{fig-interrog-clause-simple-slash} shows the same clause as
Figure~\ref{fig-interrog-clause-simple} but with two features that are traditionally used in the
analysis of nonlocal dependencies \citep[Chapter~4 and 5]{ps2}: \que and \slasch. By convention, the boxed numbers
are put in front of XPs if the XP is an argument and they follow the XP if the XP is involved in a
nonlocal dependency. The reason for this is that different parts of information are shared. A full explanation
of the difference requires some deeper understanding of the mechanisms and cannot be given here. The
interested reader is referred to \citew[Chapter~10]{MuellerLehrbuch3} or to \citew{Borsley:Crysman:2021a}.
%has to wait until
%Chapter~\ref{chap-HPSG-light}. 
PP\ibox{1}[\slasch \sliste{ \ibox{1} }] means that the relevant information
about the PP is put into \slasch, that is, into the list that is percolated upwards until a matching
filler is found (a PP whose relevant properties can be identified with the element in the
\slashl). Once a filler has been found no \slasch element is passed upwards. The \slashl of the
top-most node is the empty list.

\begin{figure}
\begin{forest}
sm edges
[{S[\slasch \sliste{ }]}
  [PP\ibox{1} [worüber;where.about]]
  [{S[\slasch \sliste{ \ibox{1} }]}
    [{PP\ibox{1}[\slasch \sliste{ \ibox{1} }]} [\trace]]
    [V\rlap{$'$}
      [NP [sie;she]]
      [V [spricht;speaks]]]]]
\end{forest}
\caption{Analysis of simple interrogative clause using the \slasch feature}\label{fig-interrog-clause-simple-slash}
\end{figure}

Now, this machinery can be extended to cover nonlocal dependencies for interrogative
pronouns. Figure~\ref{fig-interrog-clause-simple-que-slash} shows how the information about the
interrogative within the complex \emph{w}"=phrase can be passed upwards in a tree by using the \que feature.
\ea
\gll Ich weiß, [über welches Thema]$_i$ [ \trace$_i$ sie spricht].\\
     I know    \spacebr{}about which topic {} {} she speaks\\
\glt `I know which topic she speaks about.'
\z
%
Figure~\ref{fig-interrog-clause-simple-que-slash} is completely parallel to
Figure~\ref{fig-interrog-clause-simple} except that information about the \emph{w}"=word is
added. The content of \que is not provided here but the \que list of \emph{w}"=words contains information
that is needed for semantics: the \emph{w}"=word indicates what is asked for and this information is
passed up to the level of the complete clause (see \citealt{GSag2000a-u} on interrogatives in general and on their
semantics in particular).

\begin{figure}
\centerline{\begin{forest}
sm edges
[{S[\textsc{que} \sliste{ }, \slasch \sliste{ }]}
  [{PP\ibox{1}[\textsc{que} \sliste{ \ibox{2} }]} 
     [P [über;about]]
     [{NP[\textsc{que} \sliste{ \ibox{2} }]}
       [{Det[\textsc{que} \sliste{ \ibox{2} }]} [welches;which]]
       [N [Thema;topic]]]]
  [{S[\slasch \sliste{ \ibox{1} }]}
    [{PP\ibox{1}[\slasch \sliste{ \ibox{1} }]} [\trace]]
    [V\rlap{$'$}
      [NP [sie;she]]
      [V [spricht;speaks]]]]]
\end{forest}}
\caption{Analysis of simple interrogative clause using \slasch and \que}\label{fig-interrog-clause-simple-que-slash}
\end{figure}

Interrogative clauses are licensed by a special variant of the Filler-Head Schema, namely a schema
that requires the initial daughter (the filler) to have something in its \que list (Figure~\ref{fig-interrogative-clause-schema}). This entails
that the filler has to contain a \emph{w}"=word.

\begin{figure}
\centerline{
\begin{forest}
[{H[\textsc{que} \sliste{ }, \slasch \sliste{ }]}
  [\ibox{1}{[\textsc{que} \sliste{ \etag{} }]}]
  [H{[\slasch \sliste{ \ibox{1} }]} ]]
\end{forest}
}
\caption{Interrogative Clause Schema}\label{fig-interrogative-clause-schema}
\end{figure}

The languages differ as far as the order of the verb and its arguments are concerned, so further
specifications have to be added to what is given in
Figure~\ref{fig-interrogative-clause-schema}. For example, German interrogative clauses are verb
final, while Yiddish interrogatives involve an extraction out of a V2 clause. In addition,
constraints regarding the verb form have to be specified. German allows for finite verbs only, while
English allows for finite verbs and infinitives with \emph{to} (see (\ref{ex-what-to-read}) on p.\,\pageref{ex-what-to-read}).

Up to now, we have looked at German examples with prepositional objects
fronted. Figure~\ref{fig-wer-das-buch-liest} shows the analysis of (\mex{1}) with a subject as a
\emph{w}"=phrase: 
\ea
\gll wer das Buch liest\\
     who the books reads\\\german
\glt `who reads the book'
\z
\begin{figure}
\centerline{\begin{forest}
sm edges
[S
       [{NP[\snom]} [wer;who] ]
       [{S/NP[\snom]} 
         [{NP[\snom]/NP[\snom]} [\trace] ]
         [V$'$
           [{NP[\sacc]}  [das Buch;the book,roof] ]
           [V [liest;reads] ] ] ] ]
\end{forest}}
\caption{Analysis of German interrogative clause with the subject as \emph{w}"=word}\label{fig-wer-das-buch-liest}
\end{figure}

So, with what we have so far, we can analyze interrogatives in German and other SOV languages but
there are still open questions in languages like Danish where expletive insertion in subject
position is required when a subject is questioned (see
(\ref{ex-danish-interrogative-expletive})). Similarly, Yiddish may insert expletives in the
preverbal position in interrogatives (see (\ref{ex-Yiddish-interrogatives-expletive})). So, to
account for this data, we have to deal with expletives. Expletives are the topic of the next section.



\subsection{A lexical rule for the introduction of expletives}
\label{sec-analysis-expletives}

The various types of expletives introduced in Section~\ref{sec-phen-use-of-expletives} can -- maybe somewhat surprisingly -- be
accounted for by a simple lexical rule that adds an expletive to the \argstl of lexical items
\citep[\page 180]{MOe2011a}:
\ea
\label{positional-expl-lr}
Expletive Insertion Lexical Rule:\\
\ms{
head & \ms[verb]{
       vform & fin \\
       }\\
arg-st & \ibox{1}\\
} $\mapsto$
\ms{
arg-st & \sliste{ NP[\type{lnom}]\upshape \sub{\type{expl}} } $\oplus$ \ibox{1}\\
}
\z
The application of the lexical rule is restricted to finite verbs since positional expletives occur
in V2 clauses and these are always finite. Expletives in interrogative clauses are used to fill
the subject position to mark extraction and of course this is something that is necessary in finite
clauses only.

The case of the expletive pronoun is specified as lexical nominative, which means that it is
invisible to case assignment principles. Nominative is assigned to the first NP with structural case
(see p.\,\pageref{case-p}) and since the expletive has lexical case, nothing changes. The same is
true for structural accusatives: the first NP with structural gets nominative and all others
accusative. The expletive does not interfere with this.

Similarly, the theory of agreement entertained so far is not affected: the verb agrees with the
first NP with structural case. This makes the right predictions for agreement in Icelandic, where
the verb agrees with objects in the nominative \citep[\page 460]{ZMT85a}.
\eal
\ex
\gll Hefur henni      alltaf þótt    Ólafur      leibinlegur?\footnotemark\\
     has   she.\DAT{} always thought Olaf.\NOM{} boring.\NOM{}\\\icelandic
\footnotetext{
\citew*[\page 451]{ZMT85a}
}
\glt `Has she always considered Olaf boring?'
\ex
\label{ex-dat-subj-passive-ditransitive-icelandic-two}
\gll Konunginum voru gefnar ambáttir.\footnotemark\\
     the.king.\DAT{} were given.\F.\PL{} maidservants.\NOM.\F.\PL\\
\footnotetext{
\citew*[\page 460]{ZMT85a}
}
\glt `The king was given female slaves.'
\zl

And the approach to agreement also works for cases of remote passive in German, where the subject is not the first element
in an \argstl. One of the examples in (\ref{bsp-auskosten-fernpassiv}) on
p.\,\pageref{bsp-auskosten-fernpassiv} is repeated below:
\eal
\ex 
\gll Sie erlauben uns nicht, den Erfolg auszukosten.\\
     they permitted us.\DAT{} not the.\ACC{} success to.enjoy\\
\glt `They did not permit us to enjoy the success.'
\ex\iw{auskosten}
\gll Der Erfolg         wurde uns      nicht auszukosten erlaubt.\footnotemark\\
     the success.\NOM{} was   us.\DAT{} not   to.enjoy                  permitted\\
\footnotetext{
        \citew[\page 110]{Haider86c}%
}
\glt `We were not permitted to enjoy our success.'%
\label{bsp-auskosten-fernpassiv-haider-zwei}
\zl
See p.\,\pageref{ex-arg-st-fernpassiv-object-control} for the respective \argst lists.

As far as the position in the clause is concerned, the expletive is a subject in Danish. This is
exactly what we want and what follows from the general mapping from \argst to \spr and \comps in SVO
languages. The analysis of (\mex{1}) is shown in
Figure~\ref{fig-interrogative-Danish-subject-extraction}.
\ea
\gll hvem der læser bogen\\
     who  \expl{} reads book.\textsc{def}\\\danish
\glt `who reads the book'
\z
The lexical item for \emph{læser} `to read' is given in (\mex{1}):
\ea
Lexical item for \emph{læser} `to read' with expletive subject:
\ms{
spr    & \sliste{ NP[\type{lnom}]\upshape \sub{\type{expl}} }\smallskip\\
comps  & \sliste{ NP[\type{str}], NP[\type{str}] } \smallskip\\
arg-st & \sliste{ NP[\type{lnom}]\upshape \sub{\type{expl}}, NP[\type{str}], NP[\type{str}] } 
}
\z
The case assignment principles assign nominative to the first NP with structural case and accusative
to the second. As Figure~\ref{fig-interrogative-Danish-subject-extraction} shows, the expletive
subject is realized as specifier in the subject position and the nominative and accusative on the
\compsl are realized as objects. The ``nominative object'' is extracted and realized as the
interrogative pronoun.

\begin{figure}
\centerline{\begin{forest}
sm edges
[S
       [{NP[\snom]} [hvem;who] ]
       [{S/NP[\snom]}
         [{NP[\lnom]} [der;\textsc{expl}] ]
         [{VP/NP[\snom]}
           [{V$'$/NP[\snom]}
             [V [læser;reads] ]
             [{NP[\snom]/NP[\snom]} [\trace] ] ]
           [{NP[\sacc]} [bogen;book.\textsc{def} ] ] ] ] ]
\end{forest}}
\caption{Analysis of interrogative clauses in Danish with subject extraction}\label{fig-interrogative-Danish-subject-extraction}
\end{figure}

Similarly, the analysis of interrogatives in Yiddish may involve an initial expletive in the V2
clause if the speaker finds the subject or any other element inappropriate for this position for
information structural reasons. Figure~\ref{fig-Yiddish-interrogative-expletive} shows the analysis of (\mex{1}):
\ea
\gll ver es leyent dos bukh\\
     who \expl{} reads the book\\\yiddish
\glt `who reads the book' 
\z

\begin{figure}
\centerline{{\begin{forest}
sm edges
[S
       [{NP[\snom]} [ver$_i$;who] ]
       [{S/NP[\snom]}
         [{NP[\lnom]} [es$_j$;\textsc{expl}] ]
         [{S/NP[\snom]/NP[\lnom]}
           [{V \sliste{S//V}}
             [V [leyent$_k$;reads] ] ]
           [{S//V/NP[\snom]/NP[\lnom]}
             [{NP[\lnom]/NP[\lnom]} [\trace$_j$] ] 
             [{VP//V/NP[\snom]}, s sep+=1em
               [{V$'$//V} 
                 [V//V [\trace$_k$] ]
                 [{NP[\snom]/NP[\snom]} [\trace$_i$] ] ]
               [{NP[\sacc]} [dos bukh;the book,roof] ] ] ] ] ] ]
\end{forest}}}
\caption{Analysis of a Yiddish interrogative clause involving a fronted expletive pronoun}\label{fig-Yiddish-interrogative-expletive}
\end{figure}
The analysis is more complex than the Danish one, but this is due to the fact that Yiddish has V2
clauses in interrogatives involving verb movement in addition to extraction. Interrogatives have
two extracted elements: one for V2, the expletive in the example, and another one which is the
interrogative phrase (\emph{ver} `who' in the example). The figure shows two elements after a /. In
the notation using the \slasch feature, there would be a list with two elements.

This completes the analysis of finite interrogative clauses with and without expletives,
but there is more to be said about expletives in general. I stipulated a lexical rule adding an
expletive element above and this accounts for expletives in Yiddish interrogatives. It not just
works for interrogatives but for V2 in general. Yiddish declarative V2 clauses can have initial
expletives as well, as can German V2 clauses. (\mex{1}a) is a German example. As (\mex{1}b) shows,
the expletive is not allowed to appear in the \mf.

\eal
\ex[]{
\gll Es lachen drei Kinder.\\
     \expl{} laugh three children\\
\glt `Three children laugh.'
}
\ex[*]{
\gll dass es      drei Kinder lachen\\
     that \expl{} three children laugh\\
}
\zl
In German grammars, this expletive is called ``positional \emph{es}'' and it is emphasized that it is
not the subject and not an argument of the verb \parencites[\page 129, 177, 371]{Eisenberg2004a}[§1263]{Duden2005-Authors}. The fact that the \emph{es} cannot appear in the
\mf is seen as support for the non-argumenthood of it. However, we have seen that the expletive is
realized in the subject position in Danish, so there is some appeal to the idea to treat it
uniformly as the initial element of the \argstl across the Germanic languages. Nevertheless it is
undeniable that the \vf is the only place in which this expletive can appear in German and Yiddish.\itdopt{check Yiddish}  
The problem can be solved by adding the following constraint to the Expletive Insertion Lexical Rule
in the grammars of German and Yiddish:
\ea
Constraint on the output of the Expletive Insertion Lexical Rule for German and Yiddish:\\
\ms{
arg-st & \sliste{ NP\ibox{1}[\slasch \sliste{ \ibox{1} }] } $\oplus$ \etag\\
}
\z
This constraint says that the first element in the \argstl (the expletive) has to have a \slasch
element with the relevant properties of the expletive. Since the expletive pronoun does not have
anything in \slasch, it cannot be combined directly with the respective lexical items. The trace has
something in \slasch, this is its very nature. So a trace can combine with the lexical item for
\emph{lachen} `to laugh' and then the expletive can function as the filler.

One problem remains: extraction of expletives must be clause-bound:
\ea[*]{
\gll Es$_i$ glaube ich, dass \_$_i$ drei Kinder lachen.\\
     \expl{} believe I that {} three children laugh\\\german
\glt Intended: `I believe that three children laugh.'
}
\z
While extraction may cross clause boundaries in principle (see
(\ref{ex-wissen-Vortrag-halen-nonlocal}) on p.\,\pageref{ex-wissen-Vortrag-halen-nonlocal}), this is excluded in
(\mex{0}). However, this is not a particular problem of the analysis of the positional \emph{es} at
hand but it is a general property of expletive elements. (\mex{1}) shows an example with the weather
\emph{es}, which clearly is an argument of the verb \emph{regnen} `to rain':
\ea[*]{
\gll Es$_i$ glaube ich, dass \_$_i$ regnet.\\
     \expl{} believe I that {} rains\\
\glt Intended: `I believe that it rains.'
}
\z
So, whatever rules out examples like (\mex{0}) also accounts for (\mex{-1}).


Finally, there is one problem left: I provided a lexical rule that licenses lexical item for interrogatives with an expletive
in subject position, but what is still missing is a constraint that rules out clauses without the
expletive. There is nothing in the grammar so far that does this. It is possible to formulate
something like this but the formal tools have not been introduced in this book. The reader is
referred to \citew[\page 185]{MOe2011a} for details.



\section{Summary}

This chapter provided an analysis of dependent clauses introduced by a complementizer and of
interrogative clauses. Together with the V1 and V2 clauses dealt with in Chapter~\ref{chap-verb-position} this covers
the main clause types in the Germanic languages. The variation in these subordinated clauses is
connected to what we saw before: the SOV languages have SOV order in embedded languages and some SVO
languages have SVO order, some allow for both SVO and V2 and some allow for V2 only. Interrogative
clauses involve a clause with a gap and the filler is the interrogative phrase containing a
\emph{wh}-word in English and a corresponding word in the other Germanic languages. The \emph{wh}
phrase may consist of a single interrogative pronoun or may be internally complex. The information
about the interrogative pronoun has to be present at the top-most node of the interrogative phrase
for semantic and syntactic reasons. The syntactic reason is of course that one has to make sure that
the fronted phrase contains an interrogative pronoun at all. The information is passed up from the
interrogative pronoun by the same mechanism that is also used for extraction: like \slasch, \que is used
to pass the information up. Danish and Yiddish use expletive pronouns in interrogatives. To
account for this, \citew{MOe2011a} suggested a lexical rule that introduced the expletive into the
\argstl. This expletive can function as subject in Danish and as positional expletive in German and
Yiddish.




\questions{
\begin{enumerate}
\item Where are expletives used in Germanic languages for clause type marking?
\item Is the clause following the complementizer in (\mex{1}) a SVO or V2 clause?
\ea 
\gll at   Gert har ikke læst bogen\\
     that Gert has not  read book.\textsc{def}\\\danish
\z
\end{enumerate}

}

\exercises{

\begin{enumerate}
\item Analyze the interrogative clauses in (\mex{1}):
\eal
\ex
\gll Ich weiß, wen Kim kennt.\\
     I   know  who.\ACC{} Kim knows\\\german
\glt `I know who Kim knows.'
\ex
\gll Jeg ved, hvem der kende Kim\\
     I know who  \expl{} knows Kim\\\danish
\glt `I know who knows Kim'

\zl 

\item Analyze the clause in (\mex{1}). Use triangles for the NP and the PP.
\ea
\gll Es schwammen zwei Delphine neben dem Boot.\\
     \expl{} swam two dolphins  next.to the boat\\
\glt `Two dolphins were swimming next to the boat.'
\z
\end{enumerate}

}

%      <!-- Local IspellDict: en_US-w_accents -->
