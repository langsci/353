%% -*- coding:utf-8 -*-
\section*{Foreword of the second edition}


% Gereon Haider is not OV -> VO but rather OVO -> VOO

The first edition of the book claimed that Hubert Haider derives the English clause structure from a
SOV base configuration, but as Gereon Müller pointed out to me, this is not the case. The English
SVOO order is derived from SOVO by movement of the verb. Section~\ref{sec-vo-derived-from-ov} was
adapted accordingly.

% 7.2024
The concept of lexical rules is explained in more detail (p.\,\pageref{page-lexical-rule-explanation-start}).%--\pageref{page-lexical-rule-explanation-end}).
Apart from this, I added the discussion of expletive insertion in Danish intransitives (example (\ref{ex-at-der-ikke-læser-han-en-bog}) on
p.\,\pageref{ex-at-der-ikke-læser-han-en-bog}), which can be accounted for in the same way as
impersonal passives in Danish. I added an explanation of the mapping from \argst to \comps for
perfect and passive participles in German (Section~\ref{sec-Mapping in the Germanic OV
  languages}). The figure for the analysis of the remote passive contained a mistake, which was
fixed for the new edition. 

% 27.07.2024
I added an exercise and the solution for German V2 including a valence feature.

I thank Elisabeth Eberle for pointing out problems with this figure. Elisabeth used the book in the tutorial of our
Germanic syntax lecture. Thanks for comments, questions, and discussion.

~\medskip

\noindent
Berlin, \today\hfill Stefan Müller

%      <!-- Local IspellDict: en_US-w_accents -->
