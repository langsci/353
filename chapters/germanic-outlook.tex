%% -*- coding:utf-8 -*-

\chapter{Outlook}
\label{chap-outlook}

This book has sketched fragments of grammars of several Germanic languages. The theory lurking in the
background is Head-Driven Phrase Structure Grammar (HPSG)
\parencites{ps}{ps2}{MuellerLehrbuch3}{HPSGHandbook}. We had a look at valence and how it is
represented in valence lists like \spr and \comps. We also looked at adjuncts, which are not
represented in lists: adjuncts select the heads they modify. HPSG assumes that there are schemata
for the combinaiton of linguistic material. We dealt with the Specifier-Head Schema, the
Head-Complements Schema, the Head-Adjunct Schema and also the Predicate Complex Schema. Verbal
complexes in the Germanic OV languages have been analyzed as predicate complex formation.

The Germanic languages vary as far as their basic order is concerned (VO or OV). Apart from English
all Germanic languages are V2 languages. V2 sentences are analyzed via head movement: there is an
empty verb in final position which is related to the fronted verb.

All analyses are implemented in computer-processable grammar fragments. They are fully formalized –
otherwise they would not be processable – but they have been given here in simplified and sketchy
form. I briefly talked about the connection between syntax and semantics in Section~\ref{sec-linking}, but of
course all implementations come with semantic representations.

Due to space limitations, it is not possible to carefully introduce all concepts of HPSG, but the
interested reader is invited to have a look at the HPSG monographs
\parencites{ps}{ps2}{GSag2000a-u}{MuellerLehrbuch3}, overview articles
\parencites{LM2006a}{PK2006a-u}{Bildhauer2014a-u}{MuellerHPSGHandbook,MuellerCurrentApproaches}, Chapter~9 in
the Grammatical Theory textbook \citep{MuellerGT-Eng4} or the handbook on HPSG
\citep{HPSGHandbook}. Especially the latter volume is an up-to-date book with more than 1600 pages
dealing in 32 chapters with almost every aspect one could be interested in.







%      <!-- Local IspellDict: en_US-w_accents -->
