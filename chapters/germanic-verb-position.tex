%% -*- coding:utf-8 -*-
\chapter{Verb position: Verb-first and verb-second}
\label{chap-verb-position}

This chapter deals with the analysis of the position of the finite verb in V2 languages. I will concentrate on
Danish and German, which may serve as prototypical examples: Danish is an SVO language, while German
is SOV. I will first discuss arguments for the classification of German as an SOV language and
provide the necessary data on Danish and then explain the respective analyses.

\section{The phenomena}

Section~\ref{sec-intro-svo} contains a discussion of the basic order of subject, object and verb in the languages
of the world, and in the Germanic languages in particular. I discussed the classification provided by
the World Atlas of Language Structures \citep{Dryer2013a}, which suggested that German is a language with no dominant
constituent order, but two orders that can be observed frequently: SOV (subordinate clauses and
clauses containing an auxiliary) and SVO (in main clauses lacking an auxiliary). 
% Textbooks used in
% German schools claim that German is a SVO language.\footnote{
% ``Das Subjekt steht normalerweise am Anfang des Satzes. [\ldots] Das Prädikat steht fast immer an
% zweiter Stelle des Satzes. Es kann aus einem Wort oder aus mehreren Wörtern bestehen:
% \emph{kränkelt} -- \emph{langweilt sich} -- \emph{kann spielen}.'' Textbook for the 5th class by
% Westermann Verlag, 2018.
% } 
In the following subsection, I discuss this
assumption in more detail. Section~\ref{sec-German-as-SOV} explains why researchers working in
Mainstream Generative Grammar\footnote{
  The term \emph{Mainstream Generative Grammar} is used for work in the tradition of
  \citegen{Chomsky81a} Lectures on Government \& Binding and \citegen{Chomsky95a-u} Minimalist Program. 
} and also most of the syntacticians working in other frameworks assume
that German is an SOV language. Section~\ref{sec-Germanic-SVO-verb-position} deals with Danish as an
instance of the Germanic SVO languages and explains how verb"=initial clauses in these languages
can be best described. Section~\ref{sce-verb-second} is dedicated to fronting in English and
verb-second clauses in German (and the other Germanic languages).

%\subsection{SVO or SOV?}


\subsection{German as SVO language?}

Claiming that SVO is a basic order on the basis of pure counting is somehow strange given the
fact that most German clauses do not have the subject in first position anyway. The following text
may serve as an example:
\eanoraggedright
\gruenbf{Für selbstfahrende Autos} soll es in Deutschland nach Angaben von Bundes\-ver\-kehrs\-mi\-nis\-ter
Alexander Dobrindt (CSU) bald eine Teststrecke geben. \gruenbf{Auf der Autobahn A9 in Bayern} \rotit{sei
ein Pilotprojekt „Digitales Testfeld Autobahn“ geplant}, wie aus einem Papier des
Bundesverkehrsministeriums hervorgeht. \gruenbf{Mit den ersten Maßnahmen für diese Teststrecke}
solle schon in diesem Jahr begonnen werden. \gruenbf{Mit dem Projekt} soll die Effizienz von
Autobahnen generell gesteigert werden. \gruenbf{„\rotit{Die Teststrecke} soll so digitalisiert und
  technisch ausgerüstet werden, dass es dort zusätzliche Angebote der Kommunikation zwischen Straße
  und Fahrzeug wie auch von Fahrzeug zu Fahrzeug geben wird“}, sagte Dobrindt zur Frankfurter
Allgemeinen Zeitung.  \gruenbf{Auf der A9} sollten sowohl Autos mit Assistenzsystemen als auch
später vollautomatisierte Fahrzeuge fahren können. \gruenbf{Dort} soll die Kommunikation nicht nur
zwischen Testfahrzeugen, sondern auch zwi\-schen Sensoren an der Straße und den Autos möglich sein,
etwa zur Übermittlung von Daten zur Verkehrslage oder zum Wetter. \gruenbf{\rotit{Das Vorhaben}
  solle im Verkehrsministerium von einem runden Tisch mit Forschern und Industrievertretern
  begleitet werden,} sagte Dobrindt. \rotit{Dieser} solle sich unter anderem auch mit den
komplizierten Haftungsfragen beschäftigen.  Also: \rotit{Wer} zahlt eigentlich, wenn ein
automatisiertes Auto einen Unfall baut?  \gruenbf{[\gruenbf{Mithilfe der Teststrecke}] solle die
  deutsche Automobilindustrie auch beim digitalen Auto „Weltspitze sein können“,} sagte der
CSU-Minister. \rotit{Die deut\-schen Hersteller} sollten die Entwicklung nicht Konzernen wie etwa
Google überlassen.  \gruenbf{Derzeit} ist Deutschland noch an das „Wiener Über\-ein\-kom\-men für den
Straßenverkehr“ gebunden, das Autofahren ohne Fahrer nicht zulässt. \gruenbf{Nur unter besonderen
  Auflagen} sind Tests möglich.  \rotit{Die Grünen} halten die Pläne für
unnütz. \rotit{Grünen-Verkehrsexpertin Valerie Wilms} sagte der Saarbrücker Zeitung: „\rotit{Der
  Minister} hat wichtigere Dinge zu erledigen, als sich mit selbstfahrenden Autos zu beschäftigen.“
\rotit{Die Technologie} sei im Verkehrsbereich nicht vordringlich, \gruenbf{auch} stehe sie noch ganz am
Anfang.  \gruenbf{Aus dem grün-rot regierten Baden-Württemberg – mit dem Konzernsitz von Daimler –}
kamen hingegen andere Töne. \gruenbf{\rotit{Was in Bayern funktioniere,} müsse auch in
  Baden-Württemberg möglich sein,} sagte Wirtschaftsminister Nils Schmid (SPD). \gruenbf{Von den
  topografischen Gegebenheiten} biete sich die Autobahn A81 an.\footnote{\emph{Selbstfahrende Autos: A9 soll Teststrecke werden}, taz, 2015-01-27}
\z
The subjects are marked in red and the non-subjects in green. I also counted subjects/non-subjects
within embedded clauses. The ratio is 11 subjects (including one subject sentence) compared to 16
non-subjects (\emph{ein automatisiertes Auto} and \emph{Autofahren ohne Fahrer} in the SOV sentences were not counted, but these are
of course also a counterexamples to the SVO claim). So, the question is: What does
this number tell us? Of course we could now further differentiate the grammatical functions of the
fronted material. We would find that we have 3 object clauses fronted; the rest of the fronted
constituents is adverbials. We could conclude that SVO is more common than OVS, but saying that SVO
is basic would not be appropriate. Rather, AdvVSO should be regarded as a basic pattern if we
assume this little text as our empirical basis. Of course, assuming this text as the basis of
scientific claims is not sufficient. 
% \citet[\page 162]{Hoberg81a} counted elements in the \vf of
% 4717 sentences from different text types and found that 63.15\,\% of the \vf constituents were
% subjects. However, \citet[\page 163]{Hoberg81a} also pointed out that if one looks at subjects and
% where they appear it is only about half of the subjects that appear in the \vf, the other half are
% placed in the \mf. See also \citew[\page 1584]{Hoberg97a}. 
\citet[Section~4]{HK2005a} examined the \vf
constituents in the TüBa-D/S and Z corpora. The S corpus contains spoken German from the machine
translation project \verbmobil and the Z corpus sentences from the German newspaper taz. Both
corpora are annotated for grammatical function. The TüBa-D/S consisted of a total of 38,342 trees
and the TüBa-D/Z treebank had 22,087 trees when the paper was written in 2005. The two corpora had
subjects in the \vf in 50.3\,\% and 52.1\,\% of the sentences with a \vf, respectively. 
% Unfortunately, the number
% of subjects in the \mf was not reported. But assuming effects similar to the ones Hoberg observed,
% the number of subjects in the \mf may be larger than the number of fronted subjects. 
So, assuming
SVO as the basic order would not be helpful, since in about 50\,\% of the clauses and may be even
more, one would have to deal with an order in which the subject is not in initial position. On top of
this, there would be the problem of subordinated sentences, which clearly do have a SOV
order.\footnote{
The clauses in (i) are in SOV order. (id) is non-finite and does not have a subject.
\eal
\ex 
\gll dass es dort zusätzliche Angebote der Kommunikation zwischen Straße und Fahrzeug wie auch von Fahrzeug zu Fahrzeug geben wird\\
     that it there additional offers   of.the communication between street and vehicle as also from vehicle to vehicle give will\\
\glt `that there will be additional offers for communication between street and vehicle and also
between vehicle and vehicle'
\ex 
\gll wenn ein automatisiertes Auto einen Unfall baut\\
     when a   automatic       car  an accident  builds\\
\glt `when an automatic car causes an accisdent'
\ex 
\gll das Autofahren ohne Fahrer nicht zulässt\\
     that car.driving without driver not permits\\
\glt `that does not permit car driving without driver'
\ex
\gll als sich mit selbstfahrenden Autos zu beschäftigen\\
     rather.than self with self.driving cars to deal.with\\
\glt `rather than dealing with autonomous cars'
\zllast
}
 Therefore syntacticians of various different frameworks (see \citealp{MuellerGT-Eng4} for
approaches in GB, Minimalism, LFG, Categorial Grammar, and HPSG) assume that SOV is the base order of
German. The finite verb is fronted to mark the sentence type and one constituent is put in front of
this verb. This fronted constituent can be the subject, an object or any other constituent of the
sentence. It may be even a dependent of a deeply embedded element in the clause. So, the position
in front of V in the V2 languages has nothing to do with the SVO/SOV dichotomy and basically disturbs
the picture and makes the counting approach pursued in the WALS (see Section~\ref{sec-intro-svo})
non-applicable.\footnote{
  Martin Haspelmath (p.c.\ 2022) pointed out to that the counting approach serves comparative
  purposes rather than descriptive ones. \citet{Haspelmath2010a} -- distinguishing between comparative concepts and descriptive
  categories -- writes that comparative concepts cannot be right or wrong (p.\,665), they can just be more or
  less suited for certain purposes. I agree that the counting approach is useful for comparing the
  languages of the world, but my point was that it is not suited for Germanic languages since here
  the V2 phenomenon disturbs the picture. For further discussion of comparative concepts vs.\ descriptive categories
  see \citew{Newmeyer2010a}, \citew{Haspelmath2010b} and \citew[\page 43--44]{MuellerCoreGram}.
}
% \itdopt{M: The counting approach serves comparative purposes, while the Vorfeld
%   fronting approach serves descriptive purposes -- both are justified. Comparative concepts are not
%   appropriate as descriptive devices \citep{Haspelmath2010a}.}

In the following I will provide facts that are seen as evidence for SOV as the basic order of German
(and other Germanic languages, \eg \ili{Dutch}, \ili{Frisian}, \ili{Afrikaans} and their regional
variants). Before I provide an analysis in Section~\ref{sec-analysis-verb-mevement}, I discuss the 
verb position in the Germanic SVO languages with Danish as an example in Section~\ref{sec-danish-verb-movement}.

\subsection{German as an SOV language}
\label{sec-German-as-SOV}

\subsubsection{The order of particle and verb and idioms}

Verb particles\is{verb!particle} form a close unit with the verb. The unit is observable in verb"=final sentences only,
which supports an SOV analysis \citep[\page 35]{Bierwisch63a}. 
\eal
\ex 
\gll weil er morgen anfängt\\
     because he tomorrow at.catches\\
\glt `because he starts tomorrow'
\ex 
\gll Er fängt morgen an.\\
     he catches tomorrow at\\
\glt `He starts tomorrow.'
\zl

\noindent
The particle verb in (\mex{0}) is non-transparent: its meaning is not related to the verb
\emph{fangen} `to catch'. Such particle verbs are sometimes called mini
idioms. In fact the argument above can also be made with idioms not involving particle verbs: Many idioms do not allow
rearrangement of the idiom\is{idiom} parts in the \mf:\footnote{
  As the example (\ref{ex-niemand-macht}) shows, the verb may be used in initial position. \citet[\page
  203]{MuellerLehrbuch3} discusses examples in which the phrase \emph{den Garaus} is in the \vf.
}
\eal
\judgewidth{?*}
\ex[]{
\gll dass niemand dem Mann den Garaus macht\\
     that nobody  the man  the \textsc{garaus} makes\\
\glt `that nobody kills the man'
}
\ex[?*]{
\gll dass dem Mann den Garaus niemand macht\\
     that the man  the \textsc{garaus} nobody makes\\
}
\ex[]{
\label{ex-niemand-macht}
\gll Niemand macht ihm den Garaus.\\
     nobody makes him the \textsc{garaus}\\
\glt `Nobody kills him.'
}
\zl
This is an instance of Behaghel's law \citeyearpar{Behaghel32-u}
that things that belong together semantically tend to be realized together. The exception is the
finite verb. The finite verb can be realized in initial or final position despite the fact that this
interrupts the continuity of the idiomatic material. Since the continuity can be observed in SOV
order only, this order is considered basic.

\subsubsection{Verbs formed by back-formation}

Verbs that are derived from nouns by backformation\is{backformation} often cannot be separated and verb"=second
sentences therefore are excluded (see \citealt[\page 62]{Haider93a}, who refers to unpublished work
by \citealt{Hoehle91b} now published in a collection of Höhle's work by Language Science Press. The
examples are on page 370--371):
\eal
\ex[]{
\gll weil sie das Stück heute uraufführen\\
     because they the play today play.for.the.first.time\\
\glt `because they premiered the play today'
}
\ex[*]{
\gll Sie uraufführen heute das Stück.\\
     they play.for.the.first.time  today the play\\
}
\ex[*]{
\gll Sie führen heute das Stück urauf.\\
     they guide today the play  \textsc{prefix}.\textsc{part}\\
}
\zl
Hence these verbs can only be used in the order that is assumed to be the base order.

\subsubsection{Double particle verbs}

The examples involving backformation have been criticized for being special, since they involve
backformation. So maybe there is a certain ill-understood aspect responsible for their properties. But
\citet[\page 63]{Haider93a},
\citet[\iaddpages]{Vikner2001a-u}, \citet{Fortmann2007a} and \citet[\page 59--60]{Haider2010a} found a similar
class of verbs resisting movement: double particle verbs. Verbs like \emph{vorankündigen} `preannounce'
consists of the combination of \emph{an} `on' and \emph{kündigen} `announce' with the addition of another prefix \emph{vor} `pre'.
\ea 
\label{ex-dass-sie-es-vorankündigt}
\gll dass sie es vor-an-kündigt\\
     that she it pre-on-announces\\
\glt `that she preannounces it'
\z
Now, the interesting thing about these verbs is that they cannot be fronted. The verb stem has to be
adjacent to the particle:
\eal
\ex[*]{
\gll Sie kündigt$_i$ es vor-an \_$_i$.\\
     she announces   it pre-on\\
}
\ex[*]{
\gll Sie an-kündigt$_i$ es vor \_$_i$.\\
     she on-announced   it pre\\
}
\ex[*]{
\gll Sie vor-an-kündigt$_i$ es  \_$_i$.\\
     she pre-on-announces it\\
}
\zl
The examples show that double particle verbs are possible in OV order but impossible in VO order.


\subsubsection{Constructions that only allow SOV order}

Similarly, it is impossible to realize the verb in initial position when elements like
\emph{mehr als} `more than' are present in the clause \parencites[Section~3.1]{Haider97c}[\page 732]{Meinunger2001a}: 
\eal
\ex[]{
\gll dass Hans seinen Profit letztes Jahr mehr als verdreifachte\\
     that Hans his         profit last       year more than tripled\\
\glt `that Hans increased his profit last year by a factor greater than three'
}
\ex[]{
\gll Hans hat seinen Profit letztes Jahr mehr als verdreifacht.\\
     Hans has his    profit last    year more than tripled\\
\glt `Hans increased his profit last year by a factor greater than three.'
}
\ex[*]{
\gll Hans verdreifachte seinen Profit letztes Jahr mehr als.\\
     Hans tripled       his    profit last year more than\\
}
\zl
So, it is possible to realize the adjunct together with the verb in final position, but there are
constraints regarding the placement of the finite verb in initial position.


\subsubsection{Order in subordinate and non-finite clauses}

Verbs in non-finite clauses and in subordinate finite clauses starting with a conjunction
  always appear finally, that is, in the right sentence bracket. For example, \emph{zu geben} `to
  give' and \emph{gibt} `gives' appear in the right sentence bracket in (\mex{1}a) and (\mex{1}b):
\eal
\ex 
\gll Der Clown versucht, Kurt-Martin die Ware zu geben.\\
     the clown tries     Kurt-Martin the goods to give\\
\glt `The clown tries to give Kurt-Martin the goods.'
\ex 
\gll dass der Clown Kurt-Martin die Ware gibt\\
     that the clown Kurt-Martin the goods gives\\
\glt `that the clown gives Kurt-Martin the goods'
\zl



\subsubsection{Scope of adverbials}

The scope of adverbials in sentences like (\ref{bsp-absichtlich-nicht-anal}) depends on their order \citep[Section~2.3]{Netter92}:
The left-most adverb scopes over the following adverb and over the verb in final
position. This was explained by assuming the following structure:
\eal
\label{bsp-absichtlich-nicht-anal}
\ex 
\gll weil er  [absichtlich [nicht lacht]]\\
     because he \hphantom{[}deliberately \hphantom{[}not laughs\\
\glt `because he deliberately does not laugh'
\ex 
\gll weil er [nicht [absichtlich lacht]]\\
     because he \hphantom{[}not \hphantom{[}deliberately laughs\\
\glt `because he does not laugh deliberately'
\zl
An interesting fact is that the scope relations do not change when the verb position is changed. If
one assumes that the sentences have an underlying structure like in (\mex{0}) and that scope is
determined with reference to this structure, this fact is explained automatically:
\eal
\label{bsp-absichtlich-nicht-anal-v1}
\ex 
\gll Lacht$_i$ er [absichtlich [nicht \_$_i$]]?\\
     laughs he \hphantom{[}deliberately \hphantom{[}not\\
\glt `Does he deliberately not laugh?'
\ex 
\gll Lacht$_i$ er [nicht [absichtlich \_$_i$]]?\\
     laughs he \hphantom{[}not \hphantom{[}deliberately\\
\glt `Doesn't he laugh deliberately?'
\zl
%\item Verum-Fokus
\nocite{Hoehle88a,Hoehle97a}


It has to be mentioned here, that there seem to be exceptions to the claim that modifiers scope from
left to right. \citet*[\page47]{Kasper94a} discusses the examples in (\mex{1}), which go back to \citet*[\page137]{BV72}.
\eal
\label{bsp-peter-liest-gut-wegen}
\ex 
\gll Peter liest wegen der Nachhilfestunden gut.\\
     Peter reads because.of the tutoring well\\
\glt `Peter reads well because of the tutoring.'
\ex 
\gll Peter liest gut wegen der Nachhilfestunden.\\
     Peter reads well because.of the tutoring\\
\zl
(\mex{0}a) corresponds to the expected order in which the adverbial PP \emph{wegen der
  Nachhilfestunden} `because of the tutoring' outscopes the adverb \emph{gut} `well', but the alternative order in (\mex{0}b) is
possible as well and the sentence has the same reading as the one in (\mex{0}a).

% Kiss95b:212
  However, \citet[Section~6]{Koster75a} and \citet*[\page67]{Reis80a} showed that these examples
  are not convincing evidence since the right sentence bracket is not filled and therefore the
  orders in (\mex{0}) are not necessarily variants of \emph{Mittelfeld} orders but may be due to extraposition of
  one constituent. As Koster and Reis showed, the examples become ungrammatical when the right sentence
  bracket is filled:
\eal
\ex[*]{
\gll Hans hat gut wegen der Nachhilfestunden gelesen.\\
     Hans has well because.of the tutoring read\\
}
\ex[]{
\gll Hans hat gut gelesen wegen der Nachhilfestunden.\\
     Hans has well read   because.of the tutoring\\
\glt `Peter read well because of the tutoring.'
}
\zl
The conclusion is that (\mex{-1}b) is best treated as a variant of (\mex{-1}a) in which the PP is
extraposed and scope is determined at the position at which the PP would occur if it were not extraposed.

While examples like (\mex{-1}) show that the matter is not trivial, the following example from \citet[\page
383]{Crysmann2004a} shows that there are examples with a filled right sentence bracket that allow
for scopings in which an adjunct scopes over another adjunct that precedes it. For instance, in
(\mex{1}) \emph{niemals} `never' scopes over \emph{wegen schlechten Wetters} `because of the bad weather':
\ea
\gll Da muß es schon erhebliche Probleme mit der Ausrüstung gegeben haben, da [wegen
  schlechten  Wetters] ein Reinhold Messner [niemals] aufgäbe.\\
     there must it \textsc{part} severe problems with the equipment given have since \hphantom{[}because.of bad weather a Reinhold Messner \hphantom{[}never give.up.would\\
\glt `There must have been severe problems with the equipment, since someone like Reinhold Messner
would never give up just because of the bad weather.'
%\ex Stefan  ist wohl deshalb krank geworden, weil er äußerst hart wegen der Konferenz in Bremen gearbeitet hat.
\z

However, this does not change the fact that the sentences in (\ref{bsp-absichtlich-nicht-anal}) and
(\ref{bsp-absichtlich-nicht-anal-v1}) have the same meaning independent of the position of the
verb. The general meaning composition may be done in the way that Crysmann suggested.%
\itdopt{T: i-scrambling}
%

Another word of caution is in order here: there are SVO languages like French that also have a left
to right scoping of adjuncts \citep[\page 156--161]{BGK2004a-u}. So, the argumentation above should not be seen as the only
fact supporting the SOV status of German. In any case, the analyses of German that were
worked out in various frameworks can explain the facts nicely.



\subsubsection{Position of non-finite verbs in VO and OV languages}

Before I turn to the verb position in Danish in the next subsection, I want to repeat Ørsnes'
examples containing several non-finite verbs (see (\ref{ex-Danish-embedding}) on p.\,\pageref{ex-Danish-embedding}): the example in (\mex{1}a) shows a German subordinate clause with a verbal complex consisting of
three verbs. The level of embedding is indicated by subscript numbers. As can be seen, the verbs are
added at the end of the clause. In the corresponding Danish example which was adapted from \citet[\page 146]{Oersnes2009b}, it is exactly the other way
around: the embedding verb precedes the embedded verb.
\eal
\ex
\gll dass er ihn gesehen$_3$ haben$_2$ muss$_1$\\
     that he him seen        have      must\\\german
\glt `that he must have seen him'
\ex
\gll at hun må$_1$ have$_2$ set$_3$ ham\\
     that she must have seen him\\\danish
\zl
%
The examples in (\mex{1}) are variants with different complexity. If we replace the simplex verb
\emph{sah} `saw' in (\mex{1}a) by the perfect form, the auxiliary is placed after the participle as
in (\mex{1}b).
\eal
\ex
\gll dass er ihn sah\\
     that he him saw\\\german
\glt `that he saw him'
\ex
\gll dass er ihn gesehen hat\\
     that he him seen    has\\
\glt `that he has seen him'
\zl 
If a modal is added to (\mex{0}b), the modal goes to the right of the embedded verbs. This order is
distorted by the placement of the finite verb in initial position, but this placement is independent
of the order of the non-finite verbs. As the examples in (\mex{1}) show, the finite verb is realized
to the left of the subject both in German (SOV) and in Danish (SVO).

\eal
\ex 
\gll Muss er ihn gesehen haben?\\
     must he him seen have\\\german
\glt `Must he have seen him?'
\ex 
\gll Må han have set ham?\\
     must he have seen him\\\danish
\glt `Must he have seen him?'
\zl





\subsection{Verb position in the Germanic SVO languages}
\label{sec-danish-verb-movement}\label{sec-Germanic-SVO-verb-position}

During the discussion of scope facts, I already hinted at an analysis in which a trace marks the
position of the verb in final position and the verb in initial position is coindexed with this
trace. Although the SVO languages are different, a similar analysis has been suggested for languages
like Danish. The evidence for this is that adverbials in SVO languages usually attach to the VP,
that is, they combine with a phrase consisting of the verb and its object or objects. (\mex{1}) is
an example:
\ea
\gll  at   Conny ikke [\sub{VP} læser bogen]\\
      that Conny not      {}        reads          book.\textsc{def}\\\danish
\glt `that Conny does not read the book'
\z

The interesting thing now is that the finite verb is placed to the left of the negation in V2 sentences:

\ea
\gll  Conny læser ikke bogen.\\
       Conny reads   not  book.\textsc{def}\\\danish
\glt `Conny is not reading the book.'
\z
This is seen as evidence for verb fronting by many:
\ea
\gll  Conny læser$_i$ ikke [\sub{VP} \_$_i$ bogen].\\
      Conny reads      not  {} {}    book.\textsc{def}\\\danish
\glt `Conny does not read the book.'
\z
\nocite{KS2002a}

With this as a background, it should be clear what the analysis of yes/no questions as in (\mex{1}b) is:
\eal
\ex
\gll at Conny læser bogen\\
     that Conny reads book.\textsc{def}\\\danish
\glt `that Conny reads the book'
\ex\label{ex-laeser-jens-bogen}
\gll Læser Conny bogen?\\
     reads Conny book.\textsc{def}\\
\glt `Does Conny read the book?'
\zl
The analysis of the first sentence involves a VP as in (\mex{1}a), and the second sentence involves a
VP with a verbal trace that corresponds to the verb in initial position:
\eal
\ex
\gll at Conny [\sub{VP} læser bogen]\\
     that Conny {} reads book.\textsc{def}\\\danish
\glt `that Conny reads the book'

\ex
\gll Læser$_i$ Conny [\sub{VP} \_$_i$ bogen]?\\
     reads     Conny {}        {}     book.\textsc{def}\\
\glt `Does Conny read the book?'
\zl

It is interesting to note that the German and the Danish question with simplex verbs have exactly
the same constituent order. Compare (\ref{ex-laeser-jens-bogen}) with (\mex{1}):
\ea
\gll Liest Conny das Buch?\\
     reads Conny the book\\ \german
\glt `Does Conny read the book?'
\z
The internal structure of these sentences is quite different though. The different nature of the two
languages is of course more obvious when non-finite verbs are involved:
\eal
\ex
\gll Har$_i$ Conny [ \_$_i$ læst bogen]?\\
     has Conny {} {} read book.\textsc{def}\\\danish
\glt `Has Conny read the book?'
\ex
\gll Hat$_i$ Conny das Buch [gelesen \_$_i$]?\\
     has Conny the book \spacebr{}read\\ \german
\glt `Has Conny read the book?'
\zl
In (\mex{0}a) the finite verb is connected to a trace in initial position of the VP and in
(\mex{0}b) it is connected to a verb in final position in a verbal complex.


\subsection{Verb second}
\label{sce-verb-second}

Even languages with rather rigid constituent order sometimes allow the fronting of elements or
positioning of elements at the far right, that is, their extraposition. (\mex{1}) shows English examples of fronting:
\eal
\ex I read this book yesterday.
\ex This book, I read yesterday.
\ex Yesterday, I read this book.
\zl
The object \emph{this book} and the adjunct \emph{yesterday} are fronted in (\mex{0}b) and
(\mex{0}c), respectively.

The Germanic languages (with the exception of English) place one constituent in front of the finite
verb. As the German examples in (\mex{1}) show, the fronted constituent can be of any grammatical function:
\eal
\ex 
\gll Ich habe das Buch gestern gelesen.\\
     I have the book yesterday read\\\german
\glt `I have read the book yesterday.'
\ex 
\gll Das Buch habe ich gestern gelesen.\\
     the book have I yesterday read\\
\ex 
\gll Gestern habe ich das Buch gelesen.\\
     yesterday have I the book read\\
\ex 
\gll Gelesen habe ich das Buch gestern, gekauft hatte ich es aber schon vor einem Monat.\\
     read have I the book yesterday bought had I it but yet before a month\\
\glt `I read the book yesterday, but I bought it last month already.'
\ex 
\gll Das Buch gelesen habe ich gestern.\\
     the book read    have I yesterday\\
\zl
Such frontings are not clause"=bounded, that is, the fronting may cross one or several clause boundaries
and also boundaries of other constituents. (\mex{1}) shows English examples in which the object of
\emph{saw} is extracted across one and two clause boundaries:
\eal
\ex\label{ex-chris-we-saw} Chris, Sandy saw.
\ex\label{ex-chris-we-think-that-sandy-saw} Chris, we think that Sandy saw.
\ex Chris, we think Anna claims that Sandy saw.
\zl
In German such extractions can be found as well:
\eal
\label{ex-fernabhaengigkeit-one}
\ex
\label{ex-wen-glaubst-du-dass}
\gll Wen$_i$ glaubst du, daß ich \_$_i$ gesehen habe.\footnotemark\\
     who believes you that I {} seen have\\\german
\footnotetext{
    \citew[\page84]{Scherpenisse86a}.
    }
\glt `Who do you believe that I have seen?'
\ex 
\gll "`Wer$_i$, glaubt\iw{glauben} er, daß er \_$_i$ ist?"' erregte sich ein Politiker vom Nil.\footnotemark\\
     \hphantom{"`}who believes he that he {} is was.upset \textsc{refl} a politician from.the Nile\\
\footnotetext{
        Spiegel, 8/1999, p.\,18.
}
\glt `A politician from the Nile was upset: ``Who does he believe he is?''.'
\zl
It is generally said that they are more common in Southern German varieties, but there are other
examples that show that nonlocal dependencies are involved. In (\ref{bsp-um-zwei-millionen}) the
prepositional object \emph{um zwei Millionen Mark} `around two million Deutsche Marks' depends on
\emph{betrügen} `to cheat'. It does not
depend on any of the verbs in the matrix clause. The phrase \emph{eine Versicherung zu betrügen} `an
insurance to betray' is extraposed that is it is positioned to the right of the verbal bracket in the so-called \nf. The
position of \emph{um zwei Millionen Mark} cannot be accounted for by local reordering. Similarly,
\emph{gegen ihn} `against him' depends on \emph{Angriffe} `attacks', which is part of the phrase \emph{Angriffe zu
  lancieren} `attacks to launch'. Again an analysis based on local reordering of dependents of a head is impossible.
\eal
\label{bsp-Fernabhaengigkeit}
\ex\label{bsp-um-zwei-millionen}
\gll {}[Um zwei Millionen Mark]$_i$ soll er versucht haben, [eine Versicherung \_$_i$ zu betrügen].\footnotemark\\
       \spacebr{}around two million Deutsche.Marks should he tried have \spacebr{}an insurance.company {} to deceive\\\german
\footnotetext{
         taz, 04.05.2001, p.\,20.
}
\glt `He apparently tried to cheat an insurance company out of two million Deutsche Marks.'
\ex
\gll {}[Gegen ihn]$_i$ falle es den Republikanern hingegen schwerer, [~[~Angriffe \_$_i$] zu lancieren].\footnotemark\\
	 {}\spacebr{}against him fall it the Republicans however more.difficult \hphantom{[~[~}attacks {} to launch\\
\footnotetext{
  taz, 08.02.2008, p.\,9.
}
\glt `It is, however, more difficult for the Republicans to launch attacks against him.'
\zl



\section{The analysis}
\label{sec-analysis-verb-mevement}

This section deals with verb"=first sentences in Subsection~\ref{sec-verb-first} and then covers verb
second sentences in Subsection~\ref{sec-verb-second}. I explain two different mechanisms for the
respective analyses: the double slash notation for dependencies involving a head (so-called ``head-movement'') and the slash notation
for nonlocal-dependencies (so-called ``constituent-movement'').\footnote{
  Note that HPSG differs from theories like Government \& Binding (GB, \citealt{Chomsky81a}) and
  Minimalism \citep{Chomsky95a-u} in that nothing is
  actually moved. While GB assumes that there are two or more structures related to each other by
  movement of constituents, HPSG assumes just one structure with an empty element and sharing of
  information. This is an important difference as far as psycho-linguistic plausibility of theories
  is concerned. See \citew[Chapter~15]{MuellerGT-Eng4} and \citew[Section~3.2]{Wasow2021a} for discussion.
}
The analysis of ``head"=movement'' goes back to \citet{Jacobson87}, who suggested such an analysis
in the framework of Categorial Grammar for English. \citet{Borsley89} adapted this analysis for HPSG
and \citet{KW91a,Kiss95a,Kiss95b} suggested a verb-movement analysis for German in HPSG. The analysis of
constituent-movement is actually older than the one of head-movement. It goes back to work in
Generalized Phrase Structure Grammar \citep{Gazdar81a} and was adapted to HPSG by \citet[Section~3.4]{ps} and \citet[Chapter~4]{ps2}. 

\subsection{Verb first}
\label{sec-verb-first}

The analysis uses a special mechanism that passes up information in a tree. The verbal trace contains the
information that a verb is missing locally. This information about the missing verb is passed up to
the node that dominates the verbal trace. It is represented using a device that is called ``double
slash'' and is written as $/\!/$\is{\dslmath}. The
respective information is head-information and therefore it is passed up the head-path along with
other information, such as part of speech. Figure~\vref{fig-analysis-German-verb-first}
illustrates. The verbal trace is missing a V, the V$'$ is missing a V, and the S as well.
\begin{figure}
\centering
\begin{forest}
sm edges
[S
  [{V \sliste{ S$/\!/$V }} 
    [V [liest$_j$;reads] ] ]
       [{S$/\!/$V}
           [NP [Conny;Conny] ]
           [{V$'$$\!/\!/$V}
             [NP [das Buch;the book, roof] ]
             [{\mybox[v1]{V}$\!/\!/$\mybox[v2]{V}} [\_$_j$] ] ] ] ] ]
%\draw[semithick,<->,color=green] (v1.south)--(v2.south);
%% \draw[semithick,<->,color=green] (3.1,-3.9) ..controls +(south east:.5) and +(south west:.5)..(2.7,-3.9);
%% \draw[semithick,<->,color=green] (3.5,-3.7) ..controls +(east:.5) and +(east:.5)..(2.8,-2.5);
%% \draw[semithick,<->,color=green] (2.8,-2.3) ..controls +(east:.5) and +(east:.5)..(1.7,-1.1);
%% \draw[semithick,<->,color=green] (1.5,-0.9) ..controls +(north:.5) and +(north:.5)..(-0.8,-0.9);
%% \draw[semithick,<->,color=green] (-0.7,-1.1) ..controls +(south east:.2) and +(north
       %% east:.5)..(-1.0,-2.4);
\end{forest}
\caption{\label{fig-analysis-German-verb-first}\label{fig-liest-jens-das-buch}Analysis of verb position in German}
\end{figure}
The initial verb selects for a sentence that is lacking a V \sliste{ S$/\!/$V }. The lexical item for
the verb in initial position is licensed by a lexical rule that relates a verb to a verb that
selects for a sentence that is lacking the input verb. Since the selectional requirement of this
verb (S$/\!/$V) is identified with the sentence lacking a V (Conny das Buch \_$_j$), the information
about the original verb \emph{liest} is identified with the V in S$/\!/$V. Since the double slash
information is head information, it percolates down along the head path to the verbal trace. The
information about the initial V is identified with the syntactic and semantic information of the
verbal trace in final position, and hence this verbal trace behaves exactly like the verb in initial
position that was input to the lexical rule.

Various researchers have argued that the finite verb in initial position behaves like a complementizer in
subordinated clauses \citep{Hoehle97a,Weiss2005a-u,Weiss2018a-u}. This is captured by the analysis. Compare
Figure~\ref{fig-analysis-German-verb-first} with Figure~\vref{fig-analysis-German-verb-last}.
\begin{figure}
\centering
\begin{forest}
sm edges
[CP
  [{C \sliste{ S }} [dass;that] ]
  [S
           [NP [Conny;Conny] ]
           [V$'$
             [NP [das Buch;the book, roof] ]
             [V [liest;reads] ] ] ] ]
\end{forest}
\caption{\label{fig-analysis-German-verb-last}Analysis of a verb"=final clause with complementizer in German}
\end{figure}
The complementizer \emph{dass} `that' selects for a complete sentence, that is, a sentence that does
not have a missing verb, and the initial verb \emph{liest} `reads' in Figure~\ref{fig-analysis-German-verb-first} selects
for a sentence that is missing \emph{liest}. So apart from the overt or covert verb, the structures
are identical. This fact is important when it comes to the analysis of the scope facts. 
\begin{figure}
\centering
\begin{forest}
sm edges
[S
        [{V \sliste{ S$/\!/$V }} 
          [V [lacht$_j$;laughs] ] ]
        [{S$/\!/$V}
           [NP [er;he] ]
           [{V$'$$\!/\!/$V}
             [Adv [nicht;not] ]
             [{V$'$$\!/\!/$V}
               [Adv [absichtlich;deliberately] ]
               [{V$\!/\!/$V} [\_$_j$] ] ] ] ] ]
\end{forest}
\caption{\label{fig-analysis-German-verb-initial-scope}Analysis of sentences with adverbials in German}
\end{figure}
Since the structure is completely parallel to the one we have in verb"=final sentences, the scope
facts follow immediately: the trace behaves like the verb in initial position, \emph{absichtlich}
`deliberately' modifies the trace, and the resulting semantics is passed up in the tree (see Figure~\vref{fig-analysis-German-verb-initial-scope}). The next
step is the modification by \emph{nicht} `not'. Again the resulting semantics is passed
up. \emph{lacht} `laughs' combines with the clause and takes its semantics over. Since \emph{lacht}
is the head the semantics is passed on from there.

The analysis of Danish is completely parallel to the analysis of German. The only difference between
Figure~\ref{fig-analysis-German-verb-first} and Figure~\vref{fig-analysis-verb-first-Danish} is the
position of the verbal trace relative to the object: the trace follows the object in German, but
it precedes it in Danish.
\begin{figure}
\centering
\begin{forest}
sm edges
[S
        [{V \sliste{ S$/\!/$V }} 
          [V [læser$_j$;reads] ] ]
        [{S$/\!/$V}
           [NP [Conny;Conny] ]
           [{VP$\!/\!/$V}
             [{V$\!/\!/$V} [\_$_j$] ] 
             [NP [bogen;book.\textsc{def}] ] ] ] ]
\end{forest}
\caption{\label{fig-analysis-verb-first-Danish}Analysis of verb position in Danish}
\end{figure}%


The last thing that is explained in this chapter is the analysis of negation and verb fronting in
Danish. Figure~\vref{fig-analysis-verb-fronting-negation-Danish} shows that the negation attaches to
the VP as in verb"=final clauses and the verb is fronted so that it appears to the left of the negation.
\begin{figure}
\centering
\begin{forest}
sm edges
[S
        [{V \sliste{ S$/\!/$V }} 
          [V [læser$_j$;reads] ] ]
        [{S$/\!/$V}
           [NP [Conny;Conny] ]
           [{VP$\!/\!/$V}
             [Adv [ikke;not] ]
             [{VP$\!/\!/$V}
               [{V$\!/\!/$V} [\_$_j$] ] 
               [NP [bogen;book.\textsc{def}] ] ] ] ] ]
\end{forest}
\caption{\label{fig-analysis-verb-fronting-negation-Danish}The analysis of verb fronting and
  negation in Danish}
\end{figure}
The next section explains the extraction of constituents, and it will then be possible to provide the
full structure for sentences like (\mex{1}a). It will also become clear why the order
of negation and verb differs in embedded and main clauses:
\eal
\ex 
\gll Conny læser ikke bogen.\\
     Conny reads not  book.\defsc\\
\glt `Conny does not read the book.'
\ex 
\gll at Conny ikke læser bogen\\
     that Conny not reads book.\defsc\\
\glt `that Conny does not read the book'
\zl 

\subsection{Verb second}
\label{sec-verb-second}


The technique that is used for the analysis of nonlocal dependencies is the same that was employed
for the analysis of the reorderings of verbs: an empty element takes the position of the fronted
constituent, and the information about the missing constituent (the so-called gap) is passed up in
the tree until it is finally bound off by the fronted element, the so-called
filler. Figure~\vref{fig-chris-sandy-saw} illustrates the analysis of (\ref{ex-chris-we-saw}).

\begin{figure}
\begin{forest}
sm edges
[S
  [NP [Chris] ]
  [S/NP 
    [NP [Sandy] ] 
    [VP/NP  
      [V [saw] ]
      [NP/NP [\trace] ] ] ] ]
%% \draw[connect] (NP/NP.north east) [bend right] to (VP/NP.south east);
%% \draw[connect] (VP/NP.north east) [bend right] to (S/NP.south east);
%% \draw[connect] (S/NP.north east) [bend right] to (NP);
\end{forest}
\caption{\label{fig-chris-sandy-saw}The analysis of extraction in English}
\end{figure}
The category following the slash (`/') stands for the object that is missing locally in the position
of the trace. Traces are like jokers in card games: they can fill (almost) any position. They
pretend to be of the category that is required locally (the NP in the accusative in the example at
hand), but the information that this category is missing locally is passed up (from NP/NP to VP/NP
to S/NP). When a matching filler is combined with a slashed constituent, the information about the missing element is not
passed up any further. The nonlocal dependency is said to be bound off at this point. In
Figure~\ref{fig-chris-sandy-saw}, S/NP is combined with the filler NP and hence the mother node is an
unslashed S. The verb has all its arguments and no slashed element is missing in the sentence: the
sentence is complete.

Figure~\vref{fig-chris-we-think-that-sandy-saw} shows the analysis of example
(\ref{ex-chris-we-think-that-sandy-saw}), which really requires a nonlocal dependency. As is shown
in the figure, the information about the missing object is passed up to the sentence level (S/NP),
to the CP level (CP/NP) and up to the next higher S. There it is bound off by the filler \emph{Chris}.
\begin{figure}
\begin{forest}
sm edges
[S
  [NP [Chris] ]
  [S/NP
    [NP [we] ] 
    [VP/NP  
       [V [think] ]
       [CP/NP
         [C [that] ]
         [S/NP
            [NP [Sandy] ] 
            [VP/NP  
               [V [saw] ]
               [NP/NP [\trace ] ] ] ] ] ] ] ]
%% \draw[connect] (NP/NP.north east)  [bend right] to (VP/NP.south east);
%% \draw[connect] (VP/NP.north east)  [bend right] to (S/NP.south east);
%% \draw[connect] (S/NP.north east)   [bend right] to (CP/NP.south east);
%% \draw[connect] (CP/NP.north east)  [bend right] to (VP/NP1.south east);
%% \draw[connect] (VP/NP1.north east) [bend right] to (S/NP1.south east);
%% \draw[connect] (S/NP1.north east)  [bend right] to (NP);
\end{forest}
\caption{\label{fig-chris-we-think-that-sandy-saw}Extraction crossing the clause boundary}
\end{figure}
The binding off of the missing element is licensed by a special schema, which is called the
Filler"=Head Schema. Figure~\vref{fig-filler-head} provides a sketch of this schema.
\begin{figure}
\begin{forest}
[{S[\type{fin}]}
  [\ibox{1}]
  [{S[\type{fin}]}/\ibox{1}]]
\end{forest}
\caption{\label{fig-filler-head}Sketch of the Head-Filler Schema}
\end{figure}



English is the only non-V2 language among the Germanic languages. In what follows I show how German
(V2+SOV) and Danish (V2+SVO) can be analyzed with the techniques that were introduced so far.
Figure~\vref{fig-das-buch-liest-jens} shows the analysis of (\mex{1}):
\ea
\gll Das Buch liest Conny.\\
     the book reads Conny\\
\glt `Conny reads the book.'
\z
\begin{figure}
\begin{forest}
sm edges
[S
  [NP$_i$ [das Buch;the book, roof] ]
  [S/NP
     [V \sliste{ S$/\!/$V } 
        [V [liest$_j$;reads] ] ]
     [S$/\!/$V\!/NP
        [NP/NP [\trace$_i$] ]
        [V$'$$\!/\!/$V
           [NP [Conny;Conny] ]
           [V$\!/\!/$V [\_$_j$] ] ] ] ] ] ]
%% \draw[connect] (NP/NP) [bend right] to (S//V\!/NP.south east);
%% \draw[connect] (S//V\!/NP.north east) [bend right] to (S/NP.east);
%% \draw[connect] (S/NP.north east) [bend right] to (NP);
\end{forest}
\caption{\label{fig-das-buch-liest-jens}Analysis of V2 in German (SOV)}
\end{figure}
The analysis of the German example is more complicated than the English one since verb movement is
involved. The verb is fronted as was explained with reference to
Figure~\ref{fig-liest-jens-das-buch}. In addition, the object is realized by a trace and then filled
by the filler \emph{das Buch} `the book', which is realized preverbally. 

I follow \citet{Fanselow2003d} and \citet{Frey2004a}, who assume that the position of the object is
initial in the \mf. Since German allows for both nominative--accusative and accusative--nominative
order, the position of the trace for the extracted object could be initial or final as in (\mex{1}a)
and (\mex{1}b), respectively:
\eal
\ex 
\gll {}[Das Buch]$_i$ liest$_j$ \_$_i$ Conny \_$_j$.\\
       \spacebr{}the book reads {} Conny\\
\ex 
\gll {}[Das Buch]$_i$ liest$_j$ Conny \_$_i$ \_$_j$.\\
       \spacebr{}the book reads Conny\\
\zl
Fanselow and Frey refer to information structural properties that elements in the initial position
have and argue that fronted elements like \emph{das Buch} `the book' have information structural properties
that correspond to those of non-fronted elements in the initial \mf position:
\ea
\gll Liest das Buch Conny?\\
     reads the book Conny\\
\glt `Does Conny read the book.'
\z
They argue that (\mex{0}) patterns with (\mex{-1}a) rather than with (\mex{-1}b).

The complete discussion will not be repeated here, since this would take us too far away, but the
interested reader may consult the discussion in Section~\ref{sec-discussion-scope}.

The analysis of the parallel Danish V2 example in (\mex{1}) is similar. 
\ea
\gll Bogen             læser Conny.\\
     book.\textsc{def} reads Conny\\
\glt `Conny reads the book.'
\z
The analysis consists of two parts: firstly, there is the analysis of verb-initial position that involves the
double slash mechanism and secondly, there is the fronting of the object using the slash
mechanism. Figure~\vref{fig-bogen-laeser-jens} illustrates.
\begin{figure}
\begin{forest}
sm edges
[S
   [NP$_i$ [bogen;book.\textsc{def}] ]
      [S/NP
         [V \sliste{ S$/\!/$V }
           [V [læser$_j$;reads] ] ]
           [S$/\!/$V\!/NP
             [NP [Conny;Conny] ]
             [VP$\!/\!/$V\!/NP
               [V$\!/\!/$V  [\_$_j$] ]
               [NP/NP [\trace$_i$ ] ] ] ] ] ] ] 
%% \draw[connect] (NP/NP.north east) [bend right] to (V/V.south east);
%% \draw[connect] (V/V.north east) [bend right] to (S//V\!/NP.south east);
%% \draw[connect] (S//V\!/NP.north east) [bend right] to (S/NP.east);
%% \draw[connect] (S/NP.north east) [bend right] to (NP);
\end{forest}
\caption{\label{fig-bogen-laeser-jens}Analysis of V2 in Danish (SVO)}
\end{figure}

The careful reader will ask why we use two different mechanisms to analyze verb movement and
extraction. The answer is that these movement types are different in nature: verb movement is
clause"=bounded, while the movement of other constituents may cross clause boundaries. This is captured by the
fact that the double slash information is passed up together with other head features, such as the
part of speech information, and the slash information is passed up separately. 

Before we deal with passive in the next chapter, we can compare the three sentences in (\mex{1}):
\eal
\ex Conny reads a book.
\ex Conny læser en bog.
\ex Conny liest ein Buch.
\zl
Again the order of the elements is the same in all three languages. However, English is an SVO
non-V2 language, Danish is an SVO+V2 language, and German is an SOV+V2 language. The analyses in
bracket notation are given in (\mex{1}), the tree structures are depicted in Figure~\vref{fig-English-Danish-German-declerative-main-clause}:
\eal
\ex {}[\sub{S} Conny [\sub{VP} reads [\sub{NP} a book]]].
\ex {}[\sub{S} Conny$_i$ [\sub{S/NP} læser$_j$ [\sub{S/NP} \_$_i$ [\sub{VP}  \_$_j$ [\sub{NP} en bog]]]].
\ex {}[\sub{S} Conny$_i$ [\sub{S/NP} liest$_j$ [\sub{S/NP} \_$_i$ [\sub{\vbar} [\sub{NP} ein Buch] \_$_j$]]]].
\zl
\begin{figure}
\scalebox{.8}{%
\begin{forest}
sm edges
[S
  [NP [Conny]]
  [VP
    [V [reads]]
    [NP [a book,baseline,roof]]]]
\end{forest}}
\hfill
\scalebox{.8}{%
\begin{forest}
sm edges
[S
   [NP$_i$ [Conny;Conny] ]
      [S/NP
         [V \sliste{ S$/\!/$V }
           [V [læser$_j$;reads] ] ]
           [S$/\!/$V\!/NP
             [NP/NP [\trace$_i$ ] ]
             [VP$\!/\!/$V
               [V$\!/\!/$V  [\_$_j$] ]
               [NP [en bog;a book,baseline,roof] ] ] ] ] ] 
%% \draw[connect] (NP/NP.north east) [bend right] to (V/V.south east);
%% \draw[connect] (V/V.north east) [bend right] to (S//V\!/NP.south east);
%% \draw[connect] (S//V\!/NP.north east) [bend right] to (S/NP.east);
%% \draw[connect] (S/NP.north east) [bend right] to (NP);
\end{forest}}
\hfill
\scalebox{.8}{%
\begin{forest}
sm edges
[S
   [NP$_i$ [Conny;Conny] ]
      [S/NP
         [V \sliste{ S$/\!/$V }
           [V [liest$_j$;reads] ] ]
           [S$/\!/$V\!/NP
             [NP/NP [\trace$_i$ ] ]
             [V$'$$\!/\!/$V
               [NP [ein Buch;a book,baseline,roof] ] 
               [V$\!/\!/$V  [\_$_j$] ] ] ] ] ] 
\end{forest}}
\caption{\label{fig-English-Danish-German-declerative-main-clause}Declarative main clauses with
  subject in initial position in English, Danish and German: despite similar appearance, the
  syntactic structure is different}
\end{figure}
It may be surprising that these three sentences get such radically different analyses although the
order of elements are the same. The difference in structures is the result of the assumption that
all declarative main clauses in the Germanic V2 languages follow the same pattern, namely that the
finite verb is fronted and then another constituent is fronted. This particular construction is
connected to the clause type, that is, to the meaning of the utterance (imperative, question,
assertion). The sentences in (\ref{ex-fernabhaengigkeit-one}) and (\ref{bsp-Fernabhaengigkeit}) show
that V2 involves a nonlocal dependency. Therefore the analysis
of (\mex{0}b) is more complex than (\mex{1}) and involves the fronting of the finite verb to initial
position with a successive fronting of the subject:
\ea
{}[\sub{S} Conny [\sub{VP} læser [\sub{NP} en bog]]].
\z
The reason is that now all declarative main clauses are subsumed under the same structure, namely
(\mex{-1}b). A declarative main clause in all Germanic V2 languages is the combination of an extracted
phrase with a verb"=initial phrase in which the extracted element is missing. Fronting of the finite
verb is a way to mark the clause type: if just the finite verb is fronted, the result is a yes/no question (\mex{1}a)
or an imperative sentence (\mex{1}b).\footnote{
  Verb initial clauses may also be declarative clauses if so-called \emph{topic drop} \citep{Fries88b} is involved:
  \ea
  \gll Was macht Peter? Gibt ihm ein Buch.\\
       what does Peter  gives him a book\\\german
  \glt `What does Peter do? He gives him a book.'
  \z
  The subject of \emph{gibt} `gives' is dropped. The complete sentence would be a V2 sentence:
  \emph{Er gibt ihm ein Buch}.
}
\eal
\ex 
\gll Gibt er ihm das Buch?\\
     gives he him the book\\\german
\glt `Does he give him the book?'
\ex 
\gll Gib mir das Buch!\\
     give me the book\\
\zl
If another constituent is fronted, a question with question word (\mex{1}a), an imperative
(\mex{1}b) or a declarative clause (\mex{1}c) results.
\eal
\ex 
\gll Wem gibt er das Buch?\\
     who gives he the book\\\german
\glt `Whom does he give the book to?'
\ex 
\gll Jetzt gib ihm das Buch!\\
     now give me the book\\
\glt `Give me the book now!'
\ex 
\gll Jetzt gibt er ihm das Buch.\\
     now gives he him the book\\
\glt `He gives him the book now.'
\zl
The analysis of the semantics of clause types cannot be given here but the interested reader is
referred to \citew{MuellerSatztypen,MuellerGS}.


%\if0
\section{Alternatives}

As with the sections about alternatives in previous chapters, this section is for advanced readers
only. It is not necessary to read it in order to understand the rest of the book.

In the preceding section I suggested an analysis in which the basic SVO order is just that: a
subject followed by the verb and a verb followed by the objects. The verb"=final sentences of SOV
languages are analyzed as a verb that is preceded by its arguments. The position of the finite verb
is accounted for by fronting it via the double slash mechanism.

There are alternative proposals to SVO and SOV order and also to the placement of the finite
verb. The proposal by \citet{Kayne94a-u} suggests that all languages have an underlying
specifier-head-complement order. The orders we see in the Germanic SOV languages would then be
derived by movement. The counterproposal by \citet{Haider2000a,Haider2020a} does not suggest that all languages are
like English or Romance but instead claims that the VO languages are derived from an underlying OV
order. These two approaches are discussed in the following two subsections~\ref{sec-ov-derived-from-vo} and~\ref{sec-vo-derived-from-ov}. As will be shown,
Kayne's proposal makes wrong predictions and Haider's proposal is not without problems either. For
both proposals it would be unclear how they should be acquired by learners of the respective
languages without the assumption of a rich Universal Grammar.

The third class of proposals to be discussed in Section~\ref{sec-aux-flat} does not assume verb movement at
all. Rather than assuming a structure with layered VPs and some sort of movement that reorders the
finite verb, authors like \citet*{GKPS85a} and \citet{Sag2020a} assume that there are alternative
linearizations for finite verbs and their subjects. The pros and cons of such analyses are the topic
of Section~\ref{sec-aux-flat}.

The CP/TP/VP model was discussed in Section~\ref{sec-cp-tp-vp-scrambling} on scrambling already. 
There are also arguments against this approach when it comes to verb movement. They are discussed in
Section~\ref{sec-cp-tp-vp-and-verb-last-non-movement}.  

\subsection{OV derived from VO: \citet{Kayne94a-u}}
\label{sec-ov-derived-from-vo}



\citet{Kayne94a-u} stipulates that all sentences in all languages have a Specifier–Head–Complement
order. Languages with orders that are not SVO are assumed to be derived by movement from SVO. We
already discussed \citegen[\page 224]{Laenzlinger2004a} analysis of a German sentence in
Section~\ref{sec-autonomy-of-syntax}. Figure~\ref{fig-Kayne-for-German} shows his analysis without
the functional nodes for adjuncts and without the fronting of the object to TopP.


\begin{figure}
\oneline{%
\begin{forest}
%where n children=0{delay=with unaligned translation}{}
sm edges
[CP
	[C$^0$[weil;because, tier=word]]
%	\[TopP
%		[DP$_j$ [diese Sonate;this sonata]]
		[SubjP
			[DP$_i$ [der Mann;the man,roof]]
%			\[ModP
%				[AdvP [wahrscheinlich;probably,l=20\baselineskip]]
				[ObjP
					[DP$_j$ [diese Sonate;this sonata,roof]]
%					\[NegP
%						[AdvP [nicht;not,tier=word]]
%						\[AspP
%							[AdvP [oft;often,tier=word]]
%							\[MannP
%								[AdvP [gut;well,tier=word]]
								[AuxP
									[VP$_k$ [gespielt;played,tier=word]]
									[Aux+
										[Aux [hat;has,tier=word]]
										[vP
											[DP$_i$ [,phantom]]
											[VP$_k$
												[V [,phantom]]
												[DP$_j$
                                                                                                  [,phantom]]]]]]]]]
%]]]]
\end{forest}%
}
\caption{\label{fig-Kayne-for-German}Abbreviated analysis of sentence structure with leftward remnant movement
  and functional heads following \citet[\page 224]{Laenzlinger2004a}}
\end{figure}%

The figure shows a vP to the right of the auxiliary. So the underlying structure without adjuncts is assumed to be
(\mex{1}a) and the derived one in (\mex{1}b):
\eal
\ex[*]{ 
\gll weil hat der Mann gespielt die Sonate\\
     because has the man played the sonata\\
}
\ex[]{
\gll weil der Mann die Sonate gespielt hat\\
     because the man the sonata played has\\\german
}
\zl

\noindent
There is a very simple argument against such derivations: it comes from the Poverty of the
Stimulus, an old argument by \citet[\page 34]{Chomsky80b-u}, and I would like to call the argument I
am using here the Inverse Poverty of the Stimulus Argument, since I am using the same argument in a
different direction. Chomsky argues that knowledge that cannot be learned from the
input and can nevertheless be shown to be acquired must be innate. Since we know by now that it is
highly unlikely that elaborated language specific knowledge is part of our genome
(\citealt*{HCF2002a}; \cites{Bishop2002a}[Section~6.4.2.2]{Dabrowska2004a}{FM2005a}), it follows that
the machinery assumed in linguistics cannot be such that it would not be acquirable. Since the
underlying structure assumed by \citeauthor{Laenzlinger2004a} is not connected to observable
material in any way, there would not be a way to learn the transformations to derive German
sentences. It follows that the complete machinery would have to be part of Universal Grammar, but
since it is unclear how it should be gotten their and why one should assume it in the first place,
we have to conclude that \citeauthor{Kayne94a-u}'s proposal is wrong.

\citet{Haider2000a} shows in detail why one should not assume OV to be derived from VO. I will not
repeat the discussion here. Haider argues that one should see VO as derived from OV instead. I think
that this is not a good idea either and that VO and OV are just different and not derived from
eachother. I deal with Haider's suggestion and my alternative proposal in the next subsection.



\subsection{VO derived from OV: \citet{Haider2020a}}
\label{sec-vo-derived-from-ov}


As was shown in the previous section, SVO approaches to SOV languages are not acquirable in a
surface-oriented way, require a large part of innate language specific knowledge and are hence
incompatible with everything we know about language acquisition. Now, Hubert
\citet{Haider2000a,Haider2010a,Haider2020a} argues with respect to psycholinguistics that a VO language like
English has a structure that is basically the structure of OV languages like German with the head
moved to an initial position.
\citet[\page 15]{Haider2010a} compares head-initial and head-final approaches:
\eal
\ex {}[[[h$^0$ A$_1$] A$_2$] A$_3$]
\ex {}[A$_3$ [A$_2$ [A$_1$ h$^0$]]]
\zl
He argues that (\mex{0}a), which is the mirror image of (\mex{0}b), does not exist crosslinguistically. He argues that the argument structure
in VO and OV languages is the same and that VO languages have the same order of arguments as the OV
languages.
He concludes on p.\,28 that the clause structure involving an English three place verb is as in (\mex{1}):
\ea
\label{ex-Haider-English-head-movmenet}
{}[XP [h$^0$ [YP [h$^0$ ZP]]]]
\z
I argued instead in the previous chapters that the order in which the arguments are combined with their verbs is
free. Hence we can combine the verb with A$_3$ first even if the verb is head-initial:
\ea
{}[[[h$^0$ A$_3$] A$_2$] A$_1$]
\z
For English, we get a clause structure as in (\mex{1}):
\ea
{}[XP [[h$^0$ YP] ZP]]
\z
The difference between Haider's structure in (\ref{ex-Haider-English-head-movmenet}) and (\mex{0})
is that Haider assumes that there is head movement in simple English SVO structures. The head starts out to the left of ZP
where it selects the ZP to the right and then moves up to the top of YP. This is shown in the left
figure in Figure~\ref{fig-Haider-English-German}.
\begin{figure}
\hfill
\begin{forest}
[vP
  [XP]
  [v$'_{\nliste{x}}$
     [v$_i^0$,name=v2]
     [VP$_{\nliste{x}}$
       [YP,name=YP]
       [V$'_{\nliste{x,y}}$
          [V$_{i\nliste{x,y,z}}^0$,name=v1]
          [ZP]]]]]
\draw[->](v2) to (YP);
\draw[->](YP) to (v1);
\end{forest}
\hfill
\begin{forest}
[VP
  [XP,name=XP]
  [V$'_{\nliste{x}}$,name=V2
     [YP,name=YP]
     [V$'_{\nliste{x,y}}$,name=V1
        [ZP,name=ZP]
        [V$_{i\nliste{x,y,z}}^0$,name=V0]]]]
% \draw[->](V0.west) to (ZP.east);
% \draw[->](V1.west) to (YP.east);
% \draw[->](V2.west) to (XP.east);
\draw[->](V0) to (ZP);
\draw[->](V1) to (YP);
\draw[->](V2) to (XP);
\end{forest}
\hfill\mbox{}
\caption{English vs. German according to \citet[\page 29]{Haider2010a}}\label{fig-Haider-English-German}

\end{figure}

Haider argues that the structures for English are determined by UG and that there would be too much
structure with too many brackets if there would not be any head movement to make the structure
plausible from a processing perspective.

The question is: What is a psycholinguistically plausible story for the analysis of English
sentences? We know for sure now that human language processing is incremental \citep{Marslen-Wilson75a,TSKES96a,SW2011a,Wasow2021a}. We use information
from all available sources as soon as we have it: phonology, syntax, semantics, pragmatics,
gestures, world knowledge. When we hear an NP, we entertain hypothesis how
the utterance may proceed. One possible continuation is as a sentence. So, we expect a VP following
the NP as in Figure~\ref{fig-NP-S}.
\begin{figure}
\begin{forest}
[S
  [NP
    [Kim,tier=word]]
  [VP,edge=dashed
    [V,edge=dashed [,no edge,tier=word]]]]
\end{forest}
\caption{A VP is expected to follow the NP to form a sentence}\label{fig-NP-S}
\end{figure} 
After hearing an NP, we expect a VP containing a verb somewhere, but the next upcoming word could be
an adverb or two as in (\mex{1}a):
\itdopt{Cite \citet{Jurafsky96a-u}}
\eal
\ex their willingness [[usually [strongly [depends on this]]]]\footnote{
  ENCOW, doc\#40288,www.psy.gla.ac.uk
}
\ex Kim [[promised and gave] Robin a book].
\zl
It could also be a verb being part of a coordination of two or more verbs. And combinations are
possible of course. The fact that the structure is underdetermined is indicated by a dashed line.
When we hear the next word, we can form a more concrete hypothesis though. In
the case of a strictly transitive verb we could have it as part of a coordination or, more likely, as a part
of a VP. The next NP is predicted as in Figure~\ref{fig-NP-S-V}.
\begin{figure}
\begin{forest}
[S
  [NP
    [Kim,tier=word]]
  [VP,edge=dashed
    [V
      [devoured,tier=word]]
    [NP]]]
\end{forest}
\caption{An NP is expected to follow the NP + verb to form a sentence with a strictily transitive verb}\label{fig-NP-S-V}
\end{figure} 
If we hear a ditransitive verb instead of a strictly transitive one, the structure in Figure~\ref{fig-NP-S-V-ditrans} is
predicted instead.\footnote{
  Note that from a psycholinguistic point of view there is no difference between this binary
  branching VP structure and a flat one. In a flat structure approach, one would assume that one
  started a VP and would check one daughter after another in a flat structure, while one has more
  elaborate structure in a binary branching approach checking the NP daughters in seperate subtrees
  one after the other.
}
\begin{figure}
\begin{forest}
[S
  [NP
    [Kim,tier=word]]
  [VP,edge=dashed
    [V$'$
      [V
        [gave,tier=word]]
      [NP]]
     [NP]]]
\end{forest}
\caption{Two NPs are expected to follow the NP + verb to form a sentence with a ditransitive verb}\label{fig-NP-S-V-ditrans}
\end{figure} 
The object NPs have to be filled in as in Figure~\ref{fig-NP-S-V-ditrans-NP-NP}, but further adjuncts may be added to the right of the VP. So
there has to be room for this. So all we know for sure until the end of the sentence is that the S
node dominates a VP. There may be more than one VP node.
\begin{figure}
\begin{forest}
sm edges
[S
  [NP
    [Kim]]
  [VP,edge=dashed
    [V$'$
      [V
        [gave]]
      [NP [Robin]]]
     [NP [the book,roof]]]]
\end{forest}
\caption{NP + ditransitive verb + two objects forms a sentence to which adjuncts can be added.}\label{fig-NP-S-V-ditrans-NP-NP}
\end{figure} 

\begin{figure}
\begin{forest}
sm edges
[S
  [NP
    [Kim]]
  [VP
    [VP
      [V$'$
        [V
          [gave]]
        [NP [Robin]]]
      [NP [the book,roof]]]
    [Adv [yesterday]]]]
\end{forest}
\caption{Sentence with VP adjunct to the right.}\label{fig-NP-S-V-ditrans-adverb}
\end{figure} 

So, what all this shows us is that sentences may be internally complex and several adjuncts may be
attached to a VP. Nevertheless the human sentence processor can cope with it and the standardly
assumed nesting of adverbials and VPs. It follows that it is unnecessary to assume a
complicated head movement approach for English verbal projections.

Shravan Vasishth reminds me the ``absence of evidence is not evidence of absence''. This means that
the fact that we cannot find any psycholinguistic reflexes of the structures assumed by Haider does
not mean that they are not there. This is true but I would like to argue that evidence of absence
together with Occam's razor is an argument against certain structures: if a structure is complex and
unnecessary and there is no psycholinguistic evidence \emph{for} it, it should not be assumed.  

% \subsection{Binding and branching}

% The previous section discussed \citeauthor{Haider2010a}'s approach, which assumes that VO languages
% are derived from OV languages. The structures he assumes for sentences with ditransitive verbs are
% similar to structures assumed elsewhere in GB/""Minimalism. For example, \citet{Adger2003a} assumes
% the right-most structure in Figure~\ref{fig-ditransitives-options}.
% \begin{figure}
% \begin{forest}
% baseline
% [\vbar
%  [\textit{show}]
%  [\textit{himself}]
%  [\textit{Benjamin}]]
% \end{forest}
% \hfill
% \begin{forest}
% baseline
% [\vbar
%    [\vbar
%      [\textit{show}]
%      [\textit{himself}] ]
%  [\textit{Benjamin}]]
% \end{forest}
% \hfill\hfill
% \begin{forest}
% baseline
% [\littlevbar
%  [\textit{show}]
%  [VP
%    [\textit{himself}]
%    [\vbar
%     [V]
%     [\textit{Benjamin}]]]]
% \end{forest}
% \caption{\label{fig-ditransitives-options}Three possible analyses of ditransitives}
% \end{figure}
% The motivation for this structure is Binding Theory. Binding Theory deals, among other things, with
% the distribution of personal pronouns and reflexives. It was an important part of Chomsky's book on
% Government \& Binding \citep{Chomsky81a}. English reflexives have to match their antecedent in
% gender:
% \eal
% \ex Peter saw himself.
% \ex Mary saw herself.
% \zl
% Reflexives must be bound in a certain local domain, while personal pronouns have to be free. For
% example, the pronouns in (\mex{1}) cannot refer to Peter and Mary, but they may refer to John and
% Helen, respectively.
% \eal
% \ex John claims that Peter saw him.
% \ex Helen claims that Mary saw her.
% \zl
% For the definition of binding domains the notion of c-command plays an important role. c-command is
% defined with respect to trees. (\mex{1}) gives the definition that is used by \citet[\page 117]{Adger2003a}:
% \ea
% A node A c-commands B if, and only if A's sister either:\\
% \begin{tabular}[t]{@{}l@{~}l@{}}
% a. & is B, or\\
% b. & contains B
% \end{tabular}
% \z

% The following two sentences show that \emph{himself} cannot be bound by Benjamin. The first example
% is simply ungrammatical since \emph{Emily} is not a suited candidate for binding because of its
% gender and the second example is grammatical but only with a binding of \emph{himself} by Peter.
% \eal
% \ex[*]{
% Emily showed himself Benjamin in the mirror.
% }
% \ex[]{
% \label{ex-Peter-showed-himself-Benjamin}
% Peter showed himself Benjamin in the mirror.
% }
% \zl
% In order to be able to explain these binding facts via c-command, a structure like the left one or
% the one in the middle of Figure~\ref{fig-ditransitives-options} would not be
% suited. \citeauthor{Adger2003a} uses a special functional head \littlev and the 
% verb \emph{show} moves from V to \littlev \citep{Larson88a}. \Littlev is assumed to contribute a causative meaning
% component \citep[\page 70]{HK93a-u}.

% The theory presented here assumes the structure in the middle. It is a binary branching structure,
% but there is no head movement in simple SVO structures. The verb is just combined with one object
% after the other.\footnote{
%   This kind of branching is also assumed in Categorial Grammar. See for example \citew[\page 392]{Steedman2019a-u}.
% }
% Now, there seems to be something missing. How is binding accounted for? c-command cannot be used for
% this. There is an interesting solution that does not require trees that are structured in the
% GB/""Minimalism way: \textcites{PS92a,ps2} developed a Binding Theory that is based on the order of
% elements in the \argstl. The \argstl basically also determines local trees containing the subject
% and objects. Since the elements in the \argstl are ordered in a certain way, one can use this list
% for establishing or ruling out binding relations. Trees are not needed for binding. This means that
% HPSG trees can be formed according to the adjacent material that is actually combined. Movement and
% additional structure is not necessary. For further details on Binding Theory in HPSG and a more
% recent approach see \citew{Branco2002a}. An overview of HPSG's Binding Theory can be found in
% \citew{Mueller2021a}.


\subsection{Analyses of verb-initial sentences in SVO languages without verb-movement}
\label{sec-aux-flat}

\if0

\citet*{GKPS85a}, \citet{Sag2020a}

\fi

\subsection{CP/TP/VP}
\label{sec-cp-tp-vp-and-verb-last-non-movement}

Section~\ref{sec-cp-tp-vp} was devoted to the CP/TP/VP analysis and
scrambling. Here, I discuss double particle verbs (Section~\ref{sec-cp-tp-vp-double-particles}) and landing sites for extraposition (Section~\ref{sec-cp-tp-vp-extraposition}).

\subsubsection{Double particle verbs}
\label{sec-cp-tp-vp-double-particles}

% Haider 1993: 62; Vikner 2001; Fortmann 2007)

Figure~\ref{fig-cp-tp-vp} on p.\,\pageref{fig-cp-tp-vp} showed that the verb stem is assumed to move from V to T to
pick up an ending and check inflectional features there. \citet[\page 63]{Haider93a},
\citet{Vikner2001a} and \citet[\page 59--60]{Haider2010a} found an
argument against such proposals: German has certain verbs with two particles. \emph{vorankündigen} `preannounce'
is an example. This particle verb consists of the combination of \emph{an} `on' and \emph{kündigen} `announce' with
the additional addition of another prefix \emph{vor} `pre'. Example
(\ref{ex-dass-sie-es-vorankündigt}) was already discussed on
p.\,\pageref{ex-dass-sie-es-vorankündigt}, but it is repeated here as (\mex{1}) for convenience:
\ea 
\label{ex-dass-sie-es-vorankündigt-two}
\gll dass sie es vor-an-kündigt\\
     that she it pre-on-announces\\
\glt `that she preannounces it'
\z
Now, the interesting thing about these verbs is that they cannot be fronted. The verb stem has to be
adjacent to the particle:
\eal
\ex[*]{
\gll Sie kündigt$_i$ es vor-an \_$_i$.\\
     she announces   it pre-on\\
}
\ex[*]{
\gll Sie an-kündigt$_i$ es vor \_$_i$.\\
     she on-announced     it pre\\
}
\ex[*]{
\gll Sie vor-an-kündigt$_i$ es  \_$_i$.\\
     she pre-on-announces it\\
}
\zl
Haider pointed out that such verbs are predicted to not have finite forms under approaches assuming
that the verb stem moves from V to T to check agreement features.

The TP-based analysis would have to assume that the verb stem \stem{kündig} moves from V to T as in (\mex{1}), but
since this kind of movement is ruled out for double particle verbs as (\mex{0}) shows, finite forms
of double particle verbs should not exist, not even in clause final position.
\ea 
\gll dass sie es vor-an \_$_i$ kündig$_i$ -t\\
     that she it pre-on {}    announce -s\\
\glt `that she preannounces it'
\z
But as (\ref{ex-dass-sie-es-vorankündigt}) shows, such sentences do exist. So, as Haider pointed
out, a CP/VP model seems to be more appropriate. Verbs do not move to higher functional projections
like T to check their agreement features. They just do it in the position they are. In V. The
assumption of a T projection is unnecessary, in fact, it is incompatible with the observable data.

\subsubsection{VPs as landing sites}
\label{sec-cp-tp-vp-extraposition}

% Hubert Haider noted early on that sentences like (\mex{1}) pose a problem for transformational
% theories:
% \eal
% \ex[]{
% Einen Hund füttern, der Hunger hat, wird man müssen.
% }
% \ex[*]{
% Wird man einen Hund füttern, der Hunger hat, müssen?
% }
% \ex[]{
% Wird man einen Hund füttern müssen, der Hunger hat?
% }
% \zl

\citet[\page 62--64]{Haider2010a} examined PP/adverb placement data in relation to the V to T movement
hypothesis. He noted that prepositional adverbials can be placed in between the verbs of a verbal
complex only marginally and full PPs are ungrammatical. Both PPs and pronominal adverbs are completely
unacceptable between auxiliaries and modals:
\eal
\ex[?/*]{
\gll dass er viel gelernt \emph{dafür} haben muss\\
     that he much learnt it.for have must\\
}
\ex[*]{
\gll dass er viel gelernt haben \emph{dafür} muss\\
     that he much learnt  have it.for must\\
}
\ex[?/*]{
\gll ohne    viel gelernt \emph{dafür} haben zu müssen\\
     without much learnt  it.for have to have.to\\
}
\ex[*]{
\gll ohne    viel gelernt haben \emph{dafür} zu müssen\\
     without much learnt  have  it.for to have.to\\
}
\ex[*]{
\gll dass er viel gelernt \emph{für} \emph{das} \emph{Examen} hat\\
     that he much learnt  for the exam   has\\
}
\ex[*]{
\gll ohne    viel gelernt \emph{für} \emph{das} \emph{Examen} zu haben\\
     without much learnt  for the exam   to have\\
}
\ex[]{
\gll [\sub{VP} Gelernt haben \emph{dafür} / \emph{für} \emph{das} \emph{Examen}] muss er viel.\\
     {}        learnt  have  it.for {} for the exam must he much\\
}
\zl 
However, as (\mex{0}g) shows, VPs are a legitimate landing site for PP extraposition and for the
  extraposition of pronominal adverbs. \emph{Gelernt haben} `learnt have' forms the right sentence
  bracket and the PP material is placed in the \nf of the fronted VP.
Haider continues with the examples in (\mex{1}):
\eal
\ex[]{
\gll [\sub{VP} Angefangen damit]$_i$ hat bloß einer \trace$_i$\\
     {}        on.caught  it.with    has just one\\
\glt `Only one has started with it'
}
\ex[*]{
\gll weil bloß einer an.\trace$_i$ damit fing$_i$\\
     because just one on it.with caught\\
}
\ex[]{
\gll weil bloß einer anfing damit\\
     because just one on.caught it.with\\
}
\zl
(\mex{0}a) shows that \emph{damit} can be placed to the right of a particle verb. If there is a VP
embedded under a TP, one would expect this VP also to be a possible landing cite 
for extraposition as it is in (\mex{0}a). One would expect that the \emph{damit} can be placed next
to the verbal particle \emph{an} as in (\mex{0}b), but if the pronominal adverb is extraposed, it
has to go to the right of the verb as in (\mex{0}c).\footnote{
Note that the situation is not as simple as one might expect. Particles can be placed in the
Mittelfeld in German. But these particle placements are different from movement of the verb to the
right. Such movements have to be ruled out in any case.%
} There may be ways to explain the problematic data away in a
VP/TP system, but the most straightforward explanation is of course not to assume a TP in the first
place. If there is a verb \emph{anfing} as the head of the VP, it does not move to a higher head in
verb-last sentences and hence, no material has to be blocked from intervening between VP and T.


\clearpage


\questions{

\begin{enumerate}
\item How are clause types determined in the Germanic languages?

\end{enumerate}


}

\exercises{


\begin{enumerate}
\item Classify the Germanic languages according to their basic constituent order (SVO, SOV, VSO,
  \ldots) and V2 assuming that you know that one of the following patterns exist in the language:
\eal
\ex NP[acc] V-Aux NP[nom] V NP[dat]  % V2 SVO cannot be English since English does not have a dative
\ex NP[acc] V-Aux NP[nom] NP[dat] V  % V2 SOV
\ex NP[acc] NP[nom] V NP[acc]        % -V2 SVO
\ex NP[acc] NP[nom] V-Aux V NP[acc]  % -V2 SVO
\ex NP[acc] V-Aux NP[nom] V PP       % kann man nicht sagen
\zl
Every sentence should be paired with ±V2 and one of the six permutations of S, O, and V.

If you cannot determine the order unambiguously, please say so. If you think that this pattern does
not exist in any of the Germanic languages say so. Please keep in mind that English is a so-called
residual V2 language, which means that there are some traces of V2 left in the grammar. Think about
question formation in English.

\item Sketch the analysis for the following examples. Use the abbreviations used in this chapter,
  that is, do not go into the details regarding \spr and \compsvs but use S, VP, and V$'$. Verb movement should be indicated with the `//' symbol.
\eal
\ex
\gll Arbejder Bjarne ihærdigt  på bogen?\\
     works    Bjarne seriously at book.\textsc{def}\\\danish
\glt `Does Bjarne work seriously on the book?'
\ex
\gll Arbeitet Bjarne ernsthaft an dem Buch?\\
     works    Bjarne seriously at the book\\\german
\glt `Does Bjarne work seriously on the book?'
\ex
\gll Wird sie darüber    nachdenken?\\
     will she there.upon think\\\german
\glt `Will she think about this?'
\zl

\item Sketch the analysis for the following examples. Use the valence features \spr and \comps
  rather than the abbreviations S, VP, and V$'$. Since the value of \spr in German is always the
  empty list, you may omit it in the German examples. NPs and PPs can be abbreviated as NP and PP,
  respectively. Verb movement should be indicated with the `//' symbol.
\eal
\ex 
\gll dass sie darüber nachdenkt\\
     that she there.upon \particle.thinks\\\german
\glt `that she laughs about this'
\ex 
\gll dass sie darüber nachdenken wird\\
     that she there.upon \particle.think will\\
\glt `that she will think about this'
\ex
\gll Wird sie darüber nachdenken?\\
     will she there.upon think\\
\glt `Will she think about this?'
\zl

\eal
\ex
\gll Arbejder Bjarne ihærdigt  på bogen?\\
     works    Bjarne seriously at book.\textsc{def}\\\danish
\glt `Does Bjarne work seriously on the book?'
\ex
\gll Arbeitet Bjarne ernsthaft an dem Buch?\\
     works    Bjarne seriously at the book\\\german
\glt `Does Bjarne work seriously on the book?'
\zl

\item Sketch the analysis of the following examples. NPs may be abbreviated. Valence features should
  not be given, but node labels like V, V$'$, VP and S should be used instead. If non-local
  dependencies are involved indicate them using the `/' symbol.
\eal
\ex Such books, I like.
\ex 
\gll Solche Bücher mag ich.\\
     such   books  like I\\\german
\glt `I like such books.'
\ex
\gll Boger som det elsker jeg.\\
     books like this like I\\\danish
\glt `I like such books.'
\zl

\end{enumerate}

}


\furtherreading{
The analysis of nonlocal dependencies was developed by \citet{Gazdar81a}. \citet{Sag2010b} deals
with further constraints necessary in a theory of nonlocal dependencies in English and how they can
be represented in HPSG. The HPSG Handbook also contains a chapter on nonlocal dependencies \citep{BC2021a}.


}

%      <!-- Local IspellDict: en_US-w_accents -->
