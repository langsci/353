%% -*- coding:utf-8 -*-
\chapter{Valence, argument order and adjunct placement}
\label{chap-valence}


%\if0
This chapter deals with the representation of valence information and sketches the basic structures that
are assumed for SVO and SOV languages. I provide an account for scrambling in those languages that
allow for it and discuss the fixed vs.\ free position of adjuncts.

\section{Valence representations}
\label{sec-valence}

The word sequences in (\mex{1}) were already discussed in footnote~\ref{fn-ex-das-kind-erwartet} on
page~\pageref{fn-ex-das-kind-erwartet}.
\eal
\ex[*]{
\gll der        Delphin erwartet\\
     the.\NOM{} dolphin expects\\
}
\ex[*]{
\gll des        Kindes       der       Delphin den        Ball dem        Kind  gibt\\
     the.\GEN{} child.\GEN{} the.\NOM{} dolphin the.\ACC{} ball the.\DAT{} child gives\\
}
\zl
The problem is that there are too few (\mex{0}a) or too many NPs (\mex{0}b) present. The concept
that is needed here is valence: like in chemistry it is assumed that heads have a certain potential
to enter into stable relations with other material \citep[\page 239]{Tesniere2015a-u}. For example, the verb \emph{erwarten} `to
expect' requires an NP in the nominative and one in the accusative. \emph{geben} `to give' is the
prototypical ditransitive verb: it can be combined with an NP in the nominative, an NP in the dative
and an NP in the accusative, but as (\mex{0}b) shows, a genitive object could not be integrated into
a sentence. 

The NPs in the examples in (\mex{1}) are arguments of the respective verbs:
\eal
\ex[]{
\gll {}[dass] der        Delphin den        Menschen erwartet\\
     \that{}  the.\NOM{} dolphin den.\ACC{} human   expects\\
\glt `that the dolphin expects the human'
}
\ex[]{
\gll {}[dass] der        Delphin dem        Kind  den        Ball gibt\\
     \that{}  the.\NOM{} dolphin the.\DAT{} child the.\ACC{} ball gives\\
\glt `that the dolphin gives the child the ball'
}
\zl
Most syntactic arguments also fill a so-called \term{semantic role} in the semantic representation of the
head. For example, the dolphin is the giver, the child is the recipient, and the ball is the item
given. \citet[Chapter~48]{Tesniere2015a-u} suggested using the analogy of dramas for the explanation of valence: if we
imagine the scene of giving, what has to happen on stage to call an event that is acted out a giving
event? There have to be the three participants, a giver, a recipient and something that is
given. Without these participants, we do not have a proper giving event. 

In addition to elements like the NPs in the examples above, which are called arguments\is{argument},
there are also so-called adjuncts. \emph{schnell} `quickly' and \emph{quickly} are examples for adjuncts:
\eal
\ex
\gll \dass{} der        Delphin dem        Kind  schnell den        Ball gibt\\
     \that{} the.\NOM{} dolphin the.\DAT{} child quickly the.\ACC{} ball gives\\
\glt `that the dolphin gives the child the ball quickly'
\ex that the dolphin gives the child the ball quickly
\zl
The adverbials provide additional information about the giving event, but they do not fill a
semantic role.

To make things complicated not all arguments have to be realized in a sentence. The ditransitive verb \emph{geben} can be
realized with any subset of its arguments, provided the context fills in the missing information.
\eal
\ex 
\gll Sie gibt Geld.\\
     she gives money\\
\glt `She gives money.'
\ex 
\gll Sie gibt den Armen.\\
     she gives the poor\\
\glt `She gives to the poor.'
\ex\label{ex-sie-gibt} 
\gll Sie gibt.\\
     she gives\\
\ex 
\gll Gib!\\
     give\\
\zl
In the case of (\mex{0}a), a certain charity setting could have been established and one can either
donate food or money or contribute some voluntary work. In such a situation, (\mex{0}a) is perfectly
fine. The transferred object in (\mex{0}b) is probably money. A possible context for (\mex{0}c) and
(\mex{0}d) is the card game skat where the person who deals out is rotating among the
players. (\mex{0}d) is an imperative. Even subjects can be dropped in imperatives since the referent
of the subject is obvious: it is the addressee of the utterance. 

The examples in (\mex{0}) show that the arguments of \emph{geben} `to give' may be omitted. This is
not the case for the accusative object of \emph{erwarten} `to expect': it is obligatory. So
arguments may be optional or obligatory, but adjuncts are always optional. While the number of
argument is limited (by the number of available slots), the number of adjuncts is not: there can be
arbitrarily many adjuncts in a phrase. (\mex{1}) shows an example with two adjuncts:
\ea
\gll \dass{} der        Delphin jetzt dem        Kind  schnell den        Ball gibt\\
     \that{} the.\NOM{} dolphin now   the.\DAT{} child quickly the.\ACC{} ball gives\\
\glt `that the dolphin now gives the child the ball quickly'
\z

The analogy to chemistry and drama may be confusing since H$_2$O is a very nice and stable molecule
and it is helpful to imagine the parallel combination of a verb with its two
arguments. Figure~\vref{fig-chemistry-valence} shows H$_2$O and the parallel combination of a verb
with its arguments corresponding to (\mex{1}a). The problem is that a single H and an O do not form
a stable combination, while (\mex{1}b) is fine: 
\eal
\ex Kirby helps Sandy.
\ex Kirby helps.
\zl
\begin{figure}
\centering
\begin{forest}
[O
  [H] 
  [H] ]
\end{forest}
\hspace{5em}
\begin{forest}
[helps
 [Kirby]
 [Sandy] ]
\end{forest}
\caption{\label{fig-chemistry-valence}Combination of hydrogen and oxygen and the combination
of a verb with its arguments}
\end{figure}%
Of course one can simply assume that there is a version of \emph{helps} that has a valence different
from the two-place valence usually assumed. Here is where the parallel breaks down since we do not
have an oxygen atom with just one open slot for the hydrogen atom. The drama analogy adds to the
confusion since the helping event described in (\mex{0}b) of course involves somebody who is
helped. The solution to this problem is to distinguish between syntactic and semantic valence \citep[Section~3]{Jacobs2003a-u}. The
drama analogy helps us to find the semantic valence, the chemistry analogy is more about syntactic valence.

Given that chemistry and drama have their problems, we may go for
another\label{page-shopping-analogy} analogy: food. Let's assume you want to prepare a meal with pasta, tofu and a
tomato sauce. For the tomato sauce you also need some onions. You put all the ingredients onto a
shopping list and go to the shop. Once in the shop you realize that they ran out of tofu. Your meal
will work without tofu. Tofu is optional. Fortunately, the shop has plenty of pasta. You may choose
between the different types and select the pasta type and brand you prefer. Some onions, tomatoes
and you are done. Wait, next to the cashier there are these gummy bears. OK, you take some of these as
well although you did not want to and they have nothing to do with your meal and your shopping
list. The gummy bears are the adjuncts.


Back to linguistics: there are two ways of ensuring that arguments are realized together with their heads. The first one
uses techniques that were introduced in Chapter~\ref{chap-psg-xbar}. If one uses flat phrase
structure rules, one can make sure that certain arguments appear together with certain heads. A
schema similar to the one in (\mex{1}) was discussed as (\ref{ditrans-schema}) on page~\pageref{ditrans-schema}.

\ea
\label{ditrans-schema-two}
\begin{tabular}[t]{@{}l@{ }l@{ }l}
S  & $\to$ & NP[\type{nom}] NP[\type{dat}] NP[\type{acc}] V[\type{ditransitive}]\\
\end{tabular}
\z

Such schemata were used in Generalized Phrase Structure Grammar \citep*{GKPS85a,Uszkoreit87a}, but they were
abandoned later in favor of lexicalist models, that is, models assuming that information about
arguments of a head is encoded in the lexical description of the head rather than in phrase
structure rules \parencites{Jacobson87b}[Section~5.5]{MuellerGT-Eng1}{MWArgSt}. Reasons for abandoning the phrasal
approach of GPSG were problems with so-called partial verb phrase frontings \citep{Nerbonne86a,Johnson86a} and with accounting for
interactions with morphology \citep[Section~5.5.1]{MuellerGT-Eng1}.\footnote{%
  Starting with influential work by Adele \citet{Goldberg95a} in the framework of \isi{Construction
  Grammar}, the phrasal approaches had a revival \citep{GJ2004a}. Phrasal approaches are wide-spread
  and also assumed in other frameworks 
\citep{Haugereid2007a,Haugereid2009a,CJ2005a,Alsina96a,Christie2010a,% resultative constructions as phrasal constructions and 
ADT2008a,ADT2013a}.            % argue for a phrasal analysis of the (Swedish) caused motion construction. 
The problems that led to the abandonment of GPSG are
  ignored in the literature and newly introduced ones are not properly addressed. See
  \cites{Mueller2006d,MuellerPersian,MuellerUnifying,MWArgSt,MWArgStReply,MuellerFCG,MuellerLFGphrasal,MuellerPotentialStructure,MuellerGT-Eng4,MuellerCxG}
  for some discussion. Note that there are also lexical variants of Construction
  Grammar. \citet*{SBK2012a}, introducing \sbcg, explicitly argue for a
  lexical view citing some of the references just given. The framework underlying the proposals
  sketched in this book is Constructional HPSG \citep{Sag97a}.
} 
 
In lexical approaches, the valence of a head is represented in its lexical entry in the form of a list with descriptions of
the elements that belong to the head's valence. (\mex{1}) provides some prototypical examples:
\ea
\label{valence-specifications-German}
\begin{tabular}[t]{@{}l@{~}l@{~}l}
a. & \emph{schläft} `sleeps':        & \sliste{ NP[\type{nom}] }\\
b. & \emph{kennt} `knows':           & \sliste{ NP[\type{nom}], NP[\type{acc}] }\\
%b. & \emph{unterstützt} `supports':  & \sliste{ NP[\type{nom}], NP[\type{acc}] }\\
c. & \emph{hilft} `helps':           & \sliste{ NP[\type{nom}], NP[\type{dat}] }\\
d. & \emph{gibt} `gives':            & \sliste{ NP[\type{nom}], NP[\type{dat}], NP[\type{acc}] }\\
e. & \emph{wartet} `waits':          & \sliste{ NP[\type{nom}], PP[\type{auf}] }\\
\end{tabular}
\z
The elements in such lists come with a fixed order. The order corresponds to the order of the
elements in English and to the so-called unmarked order in German, that is, for ditransitive verbs
the order is usually nom, dat, acc (see \citew{Hoehle82a} for comments on the unmarked order). This
fixed order is needed for establishing the link between syntax and
semantics\is{semantics}.\is{linking} This will be briefly discussed in Section~\ref{sec-linking}.

Given such a valence representation for a verb like \emph{kennen} `know', one can assume a schema
that combines an element from the valence list with the respective head and passes all unsaturated
elements on to the result of the combination. Alternatively, one could assume a flat structure in
which all arguments are combined with a head in one go \parencites[34]{GSag2000a-u}[Section~3]{MuellerOrder}. I do not assume such flat structures since
this would make the account of adjuncts (see Section~\ref{sec-adjuncts}) more difficult \citep[377--378]{MuellerOrder}.
The first step of the analysis of (\mex{1}) is
provided in Figure~\vref{fig-ihn-kennt}.\footnote{%
  Note that this sounds as if there were an order in which things have to be combined. This is not
  the case. HPSG grammars are sets of constraints that can be applied in any order. It is for
  explanatory purposes only that analyses are explained in a bottom up fashion throughout the book.%
}
\ea
\label{ex-dass-niemand-ihn-kennt}
\gll  {}[dass] niemand ihn kennt\\
      \spacebr{}that nobody.\NOM{} him.\ACC{} knows\\ 
\glt `that nobody knows him'
\z
\begin{figure}
\centerfit{%
\begin{forest}
sm edges
[{V \sliste{ NP[\type{nom}] } }
  [{NP[\type{acc}]} [ihn;him] ]
  [{V \sliste{ NP[\type{nom}], NP[\type{acc}] }} [kennt;knows]] ] ]
\end{forest}}
\caption{\label{fig-ihn-kennt}Analysis of \emph{ihn kennt} `him knows', valence information is represented in a list}
\end{figure}
The lexical item for \emph{kennt} `knows' has a valence description containing two NPs. In a first
step \emph{kennt} is combined with its accusative object. The resulting phrase \emph{ihn kennt} `him
knows' is something whose most important constituent is a verb. Therefore it has V as its category
label. Certain important properties of linguistic objects are called \emph{head features}\is{head feature}. Part of speech
is one of these properties. It is assumed that all head features are passed up from the head in the tree
automatically.

The element that is not yet combined with \emph{kennt} `knows' is the \npnom. It is still
represented in the valence list of \emph{ihn kennt} `him knows'. Figure~\vref{fig-niemand-ihn-kennt}
shows the next step combining \emph{ihn kennt} with the subject \emph{niemand} `nobody'.
\begin{figure}
\centerfit{%
\begin{forest}
sm edges
[V \eliste
  [{NP[\type{nom}]} [niemand;nobody] ]
  [{V \sliste{ NP[\type{nom}] } }
    [{NP[\type{acc}]} [ihn;him] ]
    [{V \sliste{ NP[\type{nom}], NP[\type{acc}] }} [kennt;knows]] ] ]
\end{forest}}
\caption{\label{fig-niemand-ihn-kennt}Analysis of (\emph{dass}) \emph{niemand ihn kennt} `that nobody
  knows him'}
\end{figure}
The result is a linguistic object of category verb with the empty list as valence representation.

As will be shown shortly, the schema that licenses structures like the V \eliste{} and V \sliste{ NP[\type{nom}] }  in
Figure~\ref{fig-niemand-ihn-kennt} is a more abstract version of the rule in (\ref{psg-binaer}) on page~\pageref{psg-binaer}.

% This can be depicted as in
% Figure~\vref{fig-valence-German}, which is an example analysis of (\mex{1}).
% \ea
% \label{ex-dass-niemand-ihn-kennt}
% \gll  {}[dass] niemand ihn kennt\\
%       \spacebr{}that nobody.\NOM{} him.\ACC{} knows\\ 
% \glt `that nobody knows him'
% \z
% \begin{figure}
% \centerfit{%
% \begin{forest}
% sm edges
% [{S \eliste}
%   [{NP[\type{nom}]} [niemand;nobody] ]
%   [{V$'$\sliste{ NP[\type{nom}] } }
%     [{NP[\type{acc}]} [ihn;him] ]
%     [{V \sliste{ NP[\type{nom}], NP[\type{acc}]}} [kennt;knows]] ] ]
% \end{forest}}
% \caption{\label{fig-valence-German}Analysis of (\emph{dass}) \emph{niemand ihn kennt} `that nobody
%   knows him', valence information is represented in a list}
% \end{figure}

% The lexical item for \emph{kennt} `knows' has a valence description containing two NPs. In a first
% step \emph{kennt} is combined with its accusative object. The resulting phrase \emph{ihn kennt} `him
% knows' is something whose most importent constituent is a verb. Therefore it has a V in its category
% label. Since \emph{ihn kennt} is not a sentence but something intermediate, it gets the label
% V$'$.\footnote{%
%   These labels are abbreviations for complex categories. Their internal makeup is given in
%   Table~\ref{tab-abbreviations-v-vbar-s} on p.\,\pageref{tab-abbreviations-v-vbar-s}. The labels are
%   similar to what is known from \xbart, but -- as was already mentioned in the previous chapter --
%   the theory developed here is not following all the tenets of \xbart. For example, simple nouns
%   like \emph{house} are N$'$ and there is no \nnull in the analysis of NPs like \emph{the house}.

%   Section~\ref{sec-intro-spr-comps} explains why the abbreviation for \emph{ihn kennt} `him knows' is V$'$ rather than VP.
% }
%
% The nodes for V$'$ and S are licensed by a schema that combined a head with one element of its
% valence list. The full schema will be given in Chapter~\ref{chap-HPSG-light}, but we will discuss a
% simplified version of it in Section~\ref{sec-intro-schemata}.

It is probably helpful to return to our meal-shopping analogy. Assume we are using an app to
organize our shopping lists. For our current meal we need pasta and tomatoes. They are listed in the
app in a certain order (tomatoes, pasta) and there are little images attached to the products. Once
we found something matching the pasta, we remove the pasta from the list and the remaining list
contains an icon reminding us of the tomatoes. Once we have those, we remove them from the list and
since nothing is left on the list, we pay. Linguistic structures are similar: we start
with a verb selecting two NPs, we combine it with one NP and then with the second one. The result is
a complete structure, something with an empty valence list.

There are various ways to deal with optional arguments. The simplest is to assume further
lexical items selecting fewer arguments. For the example in (\ref{ex-sie-gibt}) one would assume the valence
representation in (\mex{1}a) and for sentences with \emph{warten} `to wait' without prepositional
object, one would assume (\mex{1}b) in addition to the representations in (\ref{valence-specifications-German}):
\ea
\begin{tabular}[t]{@{}l@{~}l@{~}l}
a. & \emph{gibt} `gives':            & \sliste{ NP[\type{nom}] }\\
b. & \emph{wartet} `waits':          & \sliste{ NP[\type{nom}] }\\
\end{tabular}
\z


\section{Scrambling}
\label{sec-scrambling}

As we already saw in the data discussion in Section~\ref{sec-phenomenon-scrambling}, some languages allow for
scrambling of arguments. For those languages one can assume that heads can combine with any of its
arguments not necessarily beginning with the last one as was the case in the analysis in Figure~\ref{fig-niemand-ihn-kennt}.
Figure~\vref{fig-scrambling-German} shows the analysis of (\mex{1}).
\ea
\gll {}[dass] ihn niemand kennt\\
     \spacebr{}that him.\ACC{} nobody.\NOM{} knows\\
\glt `that nobody knows him'
\z
\begin{figure}
\centerfit{%
\begin{forest}
sm edges
[{V \eliste}
   [{NP[\type{acc}]} [ihn;him] ]
   [{V \sliste{ NP[\type{acc}] } }
      [{NP[\type{nom}]} [niemand;nobody] ]
      [{V \sliste{ NP[\type{nom}], NP[\type{acc}] }} [kennt;knows] ] ] ]
\end{forest}}
\caption{\label{fig-scrambling-German}Analysis of (\emph{dass}) \emph{ihn niemand kennt} `that nobody
  knows him', languages that allow for scrambling permit the saturation of arguments in any order}
\end{figure}
Rather than combining the verb with the accusative argument (the object) first, it is combined with
the nominative (the subject) and the accusative (the object) is added in a later step.


\section{SVO: Languages with fixed SV order and valence features}
\label{sec-intro-schemata}
\label{sec-intro-spr-comps}

The last section demonstrated how verb-final sentences in German can be analyzed. Of course it is
easy to imagine how this extends to VSO languages: The head is initial and combines with the first
element in the valence list first and then with all the other elements. However, nothing has been
said about the SVO languages so far. In languages like Danish, English, and so on, all objects are
realized after the verb, as in (\mex{1}); it is just the subject that precedes the verb.
\ea
Kim gave Sandy the book.
\z
The verb together with its objects forms a unit in a certain sense: it can be fronted (\mex{1}a). It can be
selected by dominating verbs (\mex{1}b), it can be coordinated (\mex{1}c), and it is the place where adjuncts attach to (\mex{1}d--e).
\eal
\ex John promised to read the book and [read the book], he will.
\ex He will [read the book].
\ex Kim [[sold the car] and [bought a bicycle]]. 
\ex He often [reads the book].
%\ex He [reads the book] often.
\ex \ldots{} [often [read the book] slowly], he will.
\zl
This can be modeled adequately by assuming two valence lists: one for the complements (\comps short for \textsc{complements}\isfeat{comps}) and
one for the subject. The list for the subject is called \textsc{specifier} list
(\spr).\isfeat{spr}\footnote{%
  There are various versions of HPSG: \citet[Chapter~3.2]{ps} assumed that all arguments of a head are
  represented in one list. This list was called \subcatl. \citet{Borsley87a} argued that one should
  use several valence features (\subj, \spr, and \comps) and this was adopted in
  \citew[Chapter~9]{ps2}: subjects of verbs were selected via \subj and determiners via
  \spr. \citet*[Chapter~4.3]{SWB2003a} assume that both subjects and determiners are selected via \spr, which is
  what is assumed in the grammars developed here too. \citet[Section~3.3]{Sag2012a} presents a version of HPSG
  called Sign-Based Construction Grammar (SBCG) which assumes one valence list for all arguments as
  was common in 1987. This return to an abandoned approach came without any argumentation. Hence, I do not adopt this
  variant of HPSG but stick to the separation of subjects and other arguments to \spr and \comps. I
  will not use \subj as a valence feature, but it will be introduced in the analysis of verbal
  complexes in Chapter~\ref{chap-verbal-complex}.
} The specifier list plays a role both in the analysis of sentences and in the analysis of noun phrases. Nouns
select their determiner via \spr and all their other arguments via \comps. Figure~\vref{fig-svo}
shows the analysis of the sentence (\mex{1}) using the features \spr and \comps.
\ea
Nobody knows him.
\z
\begin{figure}
\centerfit{%
\begin{forest}
sm edges
[{V[\spr \eliste, \comps \eliste]}, name=S
   [{NP[\type{nom}]} [nobody] ]
   [V\feattab{
      \spr \sliste{ NP[\type{nom}] }, \comps \sliste{}}, name=VP
     [V\feattab{
         \spr \sliste{ NP[\type{nom}] },\\
         \comps \sliste{ NP[\type{acc}] }} [knows] ]
        [{NP[\type{acc}]} [him] ] ] ]
\node [right=4cm] at (S)
    {
        = S
    };
\node [right=4cm] at (VP)
    {
        = VP
    };
\end{forest}}
\caption{\label{fig-svo}Analysis of the SVO order with two separate valence features}
\end{figure}
The \compsl of \emph{knows} contains a description of the accusative object and the accusative
\emph{him} is combined in a first step with \emph{knows}. In addition to the accusative object,
\emph{knows} selects for a subject. This selection is passed on to the mother node, the VP. Hence,
the \sprv of \emph{knows him} is identical to the \sprv of \emph{knows}. The VP \emph{knows him}
selects for a nominative NP. This NP is realized as \emph{nobody} in Figure~\ref{fig-svo}. The
result of the combination of \emph{knows him} with \emph{nobody} is \emph{nobody knows him}, which
is complete: It has both an empty \sprl and an empty \compsl. The two rules that are responsible for
the combinations in Figure~\ref{fig-svo} are called the Specifier-Head Schema and the
Head-Complement Schema. I use VP as abbreviation for something with a verbal head and an empty \compsl and at least
one element in the \sprl, and S as abbreviation for something with a verbal head and empty lists for
both the \spr and the \compsv.

In Section~\ref{sec-scrambling}, it was explained how scrambling can be accounted for: the rules that
combine heads with their arguments can take the arguments from the list in any order. For languages
with stricter constituent order requirements, the rules are stricter: the arguments have to be taken
off the list consistently from the beginning or from the end. So for English and Danish, one starts
at the beginning of the list, and for head-final languages without scrambling, one starts at the end
of the list. Figure~\ref{fig-svo-ditrans} shows the analysis of a sentence with a ditransitive verb.
\begin{figure}
\centerfit{%
\begin{forest}
sm edges
[{V[\spr \eliste, \comps \eliste]}
   [{NP[\type{nom}]} [Kim] ]
   [V\feattab{
      \spr \sliste{ NP[\type{nom}] }, \comps \sliste{}}
     [V\feattab{
         \spr \sliste{ NP[\type{nom}] },\\
         \comps \sliste{ PP[\type{to}] }}
       [V\feattab{
           \spr \sliste{ NP[\type{nom}] },\\
           \comps \sliste{ NP[\type{acc}], PP[\type{to}] }} [gave] ]
         [{NP[\type{acc}]} [a book, roof] ] ]
       [{PP[\type{to}]} [to Sandy, roof] ] ] ]
\end{forest}}
\caption{\label{fig-svo-ditrans}Analysis of the SVO order with two separate valence features and two
  elements in \comps}
\end{figure}
The accusative object is the first element in the \compsl and it is combined with the verb
first. The result of the combination is a verbal projection that has the PP[\type{to}] as the sole
element in the \compsl. It is combined with an appropriate PP in the next step resulting in a verbal
projection that has an empty \compsl (a VP).


The analysis of our first German example in Figure~\ref{fig-niemand-ihn-kennt} did not use a name
for the valence list. So the question is: How does the analysis of German relate to the analysis of
English using \spr and \comps? A lot of researchers from various frameworks have argued that it is not
useful to distinguish the subjects of finite verbs from other arguments in grammars of German. All the tests that have
been used to show that subjects in English differ from complements do not apply to the arguments of
finite verbs in German. For example, it is argued that subjects – in contrast to objects – are extraction \isi{islands}, that is, nothing can be
fronted out of a subject. But \citet[\page 173]{Haider93a} discussing the examples in (\mex{1}) shows that this is not true for
German.
\eal
\ex 
\gll [Über           Strauß]$_i$ hat [ein        Witz \_$_i$] die Runde gemacht.\\
     \spacebr{}about Strauß      has \spacebr{}a.\NOM{} joke {}      the round made\\\hfill(German)
\glt `A joke about Strauß went round.'

\ex 
\gll [Zu drastischeren Maßnahmen]$_i$ hat ihm [der Mut \_$_i$] gefehlt.\\
     \spacebr{}to more.drastic measures has him \spacebr{}the.\NOM{} courage {} lacked\\
\glt `He has lacked the courage for more drastic measures.'
\ex 
\gll [Zu diesem Problem]$_i$ haben uns noch [einige Briefe \_$_i$] erreicht.\footnotemark\\
     \spacebr{}to this problem have us still \spacebr{}some.\NOM{} letters {} reached\\
\footnotetext{\citew[\page 79]{Oppenrieder91a}
}
\glt `Some letters concerning this problem reached us afterwards.'
\zl

\noindent
The \_$_i$ indicates the place where the fronted element, the element in the \vf, belongs to. It is
within the subject NP in all three examples. Since no subject-object asymmetries exist in German,
researchers like \citet[\page 295]{Pollard90a}, \citet[Section~6.3.2]{Haider93a}, \citet[\page
376]{Eisenberg94b}, and \citet[\page 57, 78]{Kiss95a} argued for so-called ``subject as complement'' analyses. 
Figure~\vref{fig-spr-german} shows the adapted analysis of
(\ref{ex-dass-niemand-ihn-kennt}) -- repeated here as
(\ref{ex-dass-niemand-ihn-kennt-two}):
\ea
\label{ex-dass-niemand-ihn-kennt-two}
\gll  {}[dass] niemand ihn kennt\\
      \spacebr{}that nobody.\NOM{} him.\ACC{} knows\\ 
\glt `that nobody knows him'
\z
\begin{figure}
\centerfit{%
\begin{forest}
sm edges
[{V[\spr \eliste, \comps \eliste]}, name=S
        [{NP[\type{nom}]} [niemand;nobody] ]
        [{V\feattab{
              \spr \eliste, \comps \sliste{ NP[\type{nom}] } }}, name = Vs
          [{NP[\type{acc}]} [ihn;him] ] 
          [V\feattab{
              \spr \eliste,\\
              \comps \sliste{ NP[\type{nom}], NP[\type{acc}]}} [kennt;knows] ]
] ]
\node [right=4cm] at (S)
    {
        = S
    };
\node [right=4cm] at (Vs)
    {
        = V$'$
    };
\end{forest}}
\caption{\label{fig-spr-german}The analysis of a German sentence with \spr and \compsl}
\end{figure}
The difference between German and English is that German contains all arguments in the \compsl of
the finite verb and no arguments in the \sprl. Since the elements in the \compsl can be combined
with the head in any order, it is explained why all permutations of arguments are
possible. Specifiers are realized to the left of their head. This is the same for German and
English. For German this is not relevant in the verbal domain, but the Specifier-Head Schema, which
will be introduced shortly, is used in the analysis of noun phrases.

Throughout the remainder of this book, I use the abbreviations in Table~\vref{tab-abbreviations-v-vbar-s}.
\begin{table}
 \begin{tabular}[t]{@{}l@{ = }l}\lsptoprule
             S  & V[\spr \eliste, \comps \eliste]\\
             VP & V[\spr \sliste{ NP[\type{nom}] }, \comps \sliste{}]\\
             V$'$ & all other V projections apart from verbal complexes\\[2pt]
             NP & N[\spr \eliste, \comps \eliste]\\
             N$'$ & N[\spr \sliste{ Det }, \comps \sliste{}]\\\lspbottomrule
             \end{tabular}
\caption{\label{tab-abbreviations-v-vbar-s}Abbreviations for S, VP, and V$'$ and NP, N$'$}
\end{table}

\section{Immediate dominance schemata}

In Section~\ref{sec-valence}, I already mentioned that the non-terminal nodes in a tree, that is, the
nodes that are not the leaves of the tree, are licensed by schemata similar to those introduced in
Chapter~\ref{sec-PSG-Merkmale} and~\ref{sec-xbar}. In fact, the schemata are even more abstract than
\xbar schemata since they do not make any statements about linear order of the daughters. The
two schemata discussed in this section are sketched here as (\mex{1}):
\ea\label{schema-head-spr-and-head-comps-preliminary}
Specifier-Head Schema and Head-Complement Schema (preliminary)
\begin{tabular}[t]{@{}l@{ }l@{}}
H[\spr \ibox{1}]   & $\to$ H[\spr \ibox{1} $\oplus$ \sliste{ \ibox{2} }, \comps \eliste]\hspace{1em}\ibox{2}  \\
H[\comps \ibox{1}] & $\to$ H[\comps \sliste{ \ibox{2} } $\oplus$ \ibox{1}]\hspace{1em}\ibox{2} \\
\end{tabular}
\z


Syntactic rules as used here are usually called schemata since they are rather abstract: they do not
mention specific categories but instead identify certain information in the mother and the daughter descriptions.
%The details about such schemata will be given in Chapter~\ref{chap-HPSG-light}, 
The details about such schemata are given in more formal HPSG literature like
\citew[Chapter~4]{MuellerLehrbuch3} or \citew{Abeille:Borsley2021a}, 
but Figure~\vref{fig-spr-head} and
Figure~\vref{fig-head-comp} provide the respective tree representations.
\begin{figure}
\begin{forest}
[H\feattab{\spr \ibox{1}}%,\\
           %\comps \eliste}
  [\ibox{2}]
  [H\feattab{\spr \ibox{1} $\oplus$ \sliste{ \ibox{2} },\\
             \comps \eliste}
  ]]
\end{forest}
\caption{\label{fig-spr-head}Sketch of the Specifier-Head Schema (preliminary)}
\end{figure}
The H stands for \emph{head}. The term \term{head daughter} is used for the daughter that either is
the head of a phrase or contains the head of the phrase (\eg the verb in a sentence or the noun in a
noun phrase). \emph{append} ($\oplus$) is a relation that concatenates two 
lists. For instance the concatenation of \sliste{ \normalfont a } and \sliste{ \normalfont b } is
\sliste{ \normalfont a, b }. The concatenation of the empty list \eliste{} with another list yields
the latter list. To give some examples that are of relevance in this chapter consider the list
\sliste{ NP[\type{nom}], NP[\type{dat}], NP[\type{acc} ] }. \emph{append} can be used to append two
lists resulting in our list in the following ways:
\eal
\ex \eliste{} $\oplus$ \sliste{ NP[\type{nom}], NP[\type{dat}], NP[\type{acc}] } = \sliste{ NP[\type{nom}], NP[\type{dat}], NP[\type{acc}] }
\ex \sliste{ NP[\type{nom}] } $\oplus$ \sliste{ NP[\type{dat}], NP[\type{acc}] } = \sliste{ NP[\type{nom}], NP[\type{dat}], NP[\type{acc}] }
\ex \sliste{ NP[\type{nom}], NP[\type{dat}] } $\oplus$ \sliste{ NP[\type{acc}] } = \sliste{ NP[\type{nom}], NP[\type{dat}], NP[\type{acc}] }
\ex \sliste{ NP[\type{nom}], NP[\type{dat}], NP[\type{acc}] } $\oplus$ \eliste{} = \sliste{ NP[\type{nom}], NP[\type{dat}], NP[\type{acc}] }
\zl
The schema in Figure~\ref{fig-spr-head} takes a list apart in such a way that a list with a
singleton element ( \sliste{ \ibox{2} } ) and a remaining list \iboxb{1} results. Assuming the
three-element list with nom, dat and acc elements, this would be the case in (\mex{0}c) and \ibox{2} would be NP[\type{acc}] and
\ibox{1} would be \sliste{ NP[\type{nom}], NP[\type{dat}] }. In this book, the \sprl has at most one element.\footnote{
  But see \citew{MOe2013b} for an analysis of \isi{object shift} in \ili{Danish} assuming multiple elements in
  the \sprl.
} It can be an NP[\type{nom}] in the case of verbs in the SVO languages or the determiner, if
the head is a noun. If one splits a list with a singleton element into a list containing one element
and a rest, the rest will always be the empty list. Hence, with the lists at the right-hand side of the equations in (\mex{1}), \ibox{1} will be the
empty list and \ibox{2} will be NP[\type{nom}] and Det, respectively. 
\eal
\ex \eliste{} $\oplus$ \sliste{ NP[\type{nom}] } = \sliste{ NP[\type{nom}] }
\ex \eliste{} $\oplus$ \sliste{ Det } = \sliste{ Det }
\zl

For a schema like the one in Figure~\ref{fig-spr-head} to apply, the descriptions of the
daughters have to match the actual daughters. For instance \emph{sleeps} is compatible with the
right daughter: it has an NP[\type{nom}] in its \sprl. When \emph{sleeps} is realized as a
daughter of the schema in Figure~\ref{fig-spr-head}, \ibox{2} is instantiated as
NP[\type{nom}]. Therefore the left daughter has to be compatible with an NP[\type{nom}]. It can be
realized as a simple pronoun like \emph{she} or a complex NP like \emph{the brown squirrel}. Two
analyses are shown in Figure~\vref{fig-she-sleeps-the-brown-squirrel-sleeps}.
\begin{figure}
\hfill
\begin{forest}
sm edges
[V\feattab{\spr \eliste,\\
           \comps \eliste}
  [{\ibox{1} NP[\type{nom}]} [she]]
  [V\feattab{\spr \sliste{ \ibox{1} NP[\type{nom}] },\\
             \comps \eliste} [sleeps]]]
\end{forest}
\hfill
\begin{forest}
sm edges
[V\feattab{\spr \eliste,\\
           \comps \eliste}
  [{\ibox{1} NP[\type{nom}]} [the brown squirrel,roof]]
  [V\feattab{\spr \sliste{ \ibox{1} NP[\type{nom}] },\\
             \comps \eliste} [sleeps]]]
\end{forest}\hfill\mbox{}
\caption{Head-Specifier phrases with a subject and an intransitive verb}\label{fig-she-sleeps-the-brown-squirrel-sleeps}
\end{figure}
The \ibox{1} in Figure~\ref{fig-she-sleeps-the-brown-squirrel-sleeps} says that whatever is in the
\sprl is identified with whatever is the other element in the tree. I wrote down \npnom following
the \ibox{1} in both the NP node and within the \sprl, but it would have been sufficient to mention
\npnom at one of the two places. The actual number in the box does not matter. What matters is where the same
number appears in the trees or structures. I usually start with \ibox{1} at the top of the tree and
use consecutive number for the following sharings.

Figure~\ref{fig-nobody-gave-the-child-a-book} on p.\,\pageref{fig-nobody-gave-the-child-a-book}
below shows an example analysis with a ditransitive verb also involving the Specifier-Head
Schema. The specification of the \compsv of the head daughter in the Specifier-Head Schema ensures
that the verb is combined with its complements before the specifier is added.

Apart from its use for the analysis of subject--VP combinations in the SVO languages, the Specifier-Head Schema is also
used for the analysis of NPs in all the Germanic languages. Figure~\ref{fig-spr-head-the-squirrel} shows the analysis of the NP \emph{the squirrel}.
\begin{figure}
\begin{forest}
[{N[\spr \eliste, \comps \eliste]}
  [\ibox{1} Det [the]]
  [{N[\spr \sliste{ \ibox{1} }, \comps \eliste]} [squirrel]]]
\end{forest}
\caption{\label{fig-spr-head-the-squirrel}Analysis of the NP \emph{the squirrel}}
\end{figure}
\emph{squirrel} selects for a determiner and the result of combining \emph{squirrel} with a determiner is a
complete nominal projection, that is, an NP. There are also nouns like \emph{picture} that take a
complement:
\ea
a picture of Kim
\z
The combination of \emph{picture} and its complement \emph{of Kim} is parallel to the combination of
a verb with its object in VO languages with fixed constituent order. For such combinations we need a separate schema: the Head-Complement Schema,
which is given in Figure~\ref{fig-head-comp}.
\begin{figure}
\begin{forest}
[{H[\comps \ibox{1}]}
  [{H[\comps  \sliste{ \ibox{2} } $\oplus$ \ibox{1}  ]}]
  [\ibox{2}]]
\end{forest}
\caption{\label{fig-head-comp}Sketch of the Head-Complement Schema (preliminary)}
\end{figure}
The schema splits the \compsl of a head into an initial list with one element \iboxb{2}, which is
realized as the complement daughter to the right.\footnote{%
  In principle daughters are unordered in HPSG as they were in GPSG \citep{GKPS85a}. Special linearization rules are
  used to order a head with respect to its siblings in a local tree. So a schema licensing a tree
  like the one in Figure~\ref{fig-head-comp} would also license a tree with the daughters in a
  different order unless one had linearization rules that rule this out. See
  \citew{MuellerOrder} for an overview of approaches to constituent order in HPSG. Linear precedence
  rules are discussed in more detail in Section~\ref{sec-lp-rules}.
}
This schema licenses both the combination of \emph{gave} and \emph{the child} and the combination of
\emph{gave the child} and \emph{a book} in Figure~\vref{fig-nobody-gave-the-child-a-book}, which shows the
analysis of (\mex{1}).\footnote{%
  English nouns and determiners do not inflect for case. However, case is manifested in pronouns:
  \emph{he} (nominative), \emph{his} (genitive), \emph{him} (accusative). Hence, verbs in double object
  constructions select for two accusatives rather than for dative and accusative as in German. 
}
\ea
\label{ex-nobody-gave-the-child-a-book}
Nobody gave the child a book.
\z
\begin{figure}
\centerfit{%
\begin{forest}
sm edges
[{V[\spr \eliste, \comps \eliste]}
   [{\ibox{1} NP[\type{nom}]} [nobody] ]
   [V\feattab{
      \spr \sliste{ \ibox{1} NP[\type{nom}] }, \comps \sliste{}}
     [V\feattab{
         \spr \sliste{ \ibox{1} NP[\type{nom}] },\\
         \comps \sliste{ \ibox{2} NP[\type{acc}] }} 
        [V\feattab{
           \spr \sliste{ \ibox{1} NP[\type{nom}] },\\
           \comps \sliste{ \ibox{3} NP[\type{acc}], \ibox{2} NP[\type{acc}]}} [gave] ]
        [{\ibox{3} NP[\type{acc}]} [the child,roof] ] ]
     [{\ibox{2} NP[\type{acc}]} [a book,roof ] ] ] ]
\end{forest}}
\caption{\label{fig-nobody-gave-the-child-a-book}Analysis of the sentences with a ditransitive verb}
\end{figure}

To keep things simple, the Specifier-Head Schema did not mention the \compsv of the mother. The
Head-Complement Schema did neither mention the \sprv of the head daughter nor the one of the mother. But the
respective values are important, since something has to be said about these values in structures
that are licensed by these schemata. If the \sprv in the combination of \emph{gave} and \emph{the
  child} would not be constrained by the Head-Complement Schema, an value would be possible. This
includes a \sprl containing two genitive NPs and an accusative NP. Sequences like (\mex{1}) would be
licensed:

\ea[*]{
his his him gave the child a book
}
\z
To avoid such unspecified \sprvs, the \sprv of the head daughter is identified with the \sprv of the
mother node in the schema. This is the \ibox{1} in (\mex{1}b). Similarly, the \compsv of the mother
in Specifier-Head phrases has to be specified to be identical to the \compsv of the head daughter
(\ibox{2} in (\mex{1}a)) and hence the empty list.

\ea\label{schema-head-spr-and-head-comps}
Specifier-Head Schema and Head-Complement Schema (final)
\begin{tabular}[t]{@{}l@{~}l@{ }l@{}}
a. & H[\spr \ibox{1}, \comps \ibox{2}] & $\to$ H[\spr \ibox{1} $\oplus$ \sliste{ \ibox{3} }, \comps \ibox{2} \eliste]\hspace{1em}\ibox{3}  \\
b. & H[\spr \ibox{1}, \comps \ibox{2}] & $\to$ H[\spr \ibox{1}, \comps \ibox{2} $\oplus$ \sliste{ \ibox{3} }]\hspace{1em}\ibox{3} \\
\end{tabular}
\z
Figure~\vref{fig-spr-head-head-comps-final} shows the final versions of the two schemata.
\begin{figure}
\hfill
\begin{forest}
[H\feattab{\spr \ibox{1},\\
           \comps \ibox{2}}
  [\ibox{3}]
  [H\feattab{\spr \ibox{1} $\oplus$ \sliste{ \ibox{3} },\\
              \comps \ibox{2} \eliste}]]
\end{forest}
\hfill
\begin{forest}
[H\feattab{\spr \ibox{1},\\
           \comps \ibox{2}}
  [H\feattab{\spr \ibox{1},\\
             \comps  \sliste{ \ibox{3} } $\oplus$ \ibox{2}  ]}]
  [\ibox{3}]]
\end{forest}
\hfill\mbox{}
\caption{\label{fig-spr-head-head-comps-final}Sketch of the Specifier-Head and Head-Complement Schema}
\end{figure}




\section{Scrambling}

Now, in order to analyze languages with free constituent order, a more liberal variant of
the schema in Figure~\ref{fig-head-comp} is needed. Figure~\vref{fig-head-comp-free} splits the \compsl of a
head into three parts: a list \ibox{1}, a list containing exactly one element \sliste{ \ibox{3} }
and a third list \ibox{2}. The element of the second list is realized as the complement of the head.
\begin{figure}
\begin{forest}
[{H[\comps \ibox{1} $\oplus$ \ibox{2}]}
  [\ibox{3}]
  [{H[\comps  \ibox{1} $\oplus$ \sliste{ \ibox{3} } $\oplus$ \ibox{2}  ]}]]
\end{forest}
\caption{\label{fig-head-comp-free}Sketch of the Head-Complement Schema for languages with free
  constituent order}
\end{figure}
The length of the lists \ibox{1} and \ibox{2} is not restricted. For our example list containing a
nom, a dat and an acc element, there are the following possibilities to split the list:

\eal
\ex \eliste{} $\oplus$ \sliste{ NP[\type{nom}] } $\oplus$ \sliste{ NP[\type{dat}], NP[\type{acc}] } 
\ex \sliste{ NP[\type{nom}] } $\oplus$ \sliste{ NP[\type{dat}] } $\oplus$ \sliste{ NP[\type{acc}] } 
\ex \sliste{ NP[\type{nom}], NP[\type{dat}] } $\oplus$ \sliste{ NP[\type{acc}] } $\oplus$ \eliste 
\zl
So \ibox{3} in Figure~\ref{fig-head-comp-free} would be \npnom in (\mex{0}a), \npdat in (\mex{0}b) and \npacc in (\mex{0}c).

If one restricts \ibox{1} to be the
empty list, one gets grammars that saturate complements from the beginning of the list (VO languages
with fixed order like \ili{English}) and if one restricts \ibox{2} to be the empty list, one gets grammars that take the last
element from the \compsl for combination with a head (this would be an OV languages with fixed
order, if such a language would exist). Scrambling languages like German allow any
complement to be combined with its head since there is neither a restriction on \ibox{1} nor one on \ibox{2}.



\section{Linear precedence rules}
\label{sec-lp-rules}

The abstract schemata are similar to the schemata that were gained by abstracting over simple phrase
structure rules in Chapter~\ref{chap-psg}. They are similar to abstract \xbar rules. However, there
is an important difference: the elements at the right-hand side of a rule and the daughters in the
corresponding treelets in the figures visualizing the schemata are not ordered. This means that a
schema like the one in (\mex{1}) can be used to analyze configurations with a preceding b and with
b preceding a.
\ea
m $\to$ a b
\z
As will be shown shortly, this comes handy in situations in which one wants to leave the actual
order underspecified.

For the Head Complement Schema discussed above this means that actually two orders can be analyzed:
head-daughter before complement and complement before head-daughter. Hence the Head-Complement
Schema is general enough to analyze the German and English phrases in (\mex{1}):
\eal
\ex
\gll dem Kind  ein Buch gibt\\
     the child the book gives\\
\ex gives the child the book
\zl
But such a general schema without restrictions would also allow an analysis for (\mex{1}b) and (\mex{1}c):
\eal
\ex[]{
\gll \dass{} niemand dem Kind  ein Buch vorliest\\
     \that{} nobody  the child a book \partic.reads\\
\glt `that nobody reads a book to the child'
}
\ex[*]{ 
\gll \dass{} dem Kind  niemand vorliest ein Buch\\
     \that{} the child nobody  \partic.reads a book\\
}
\ex[*]{
\gll \dass{} niemand vorliest dem Kind   ein Buch\\
     \that{} nobody  \partic.reads the child a book\\
}
\zl
The structures licensed by the Head-Complement Schema without any restrictions are shown in \thefiguresref{fig-dem-kind-niemand-gab-ein-buch-head-comp}{fig-niemand-gab-dem-kind-ein-buch-head-comp}
\begin{figure}
\begin{forest}
sm edges
[{V[\comps \sliste{ }]},s sep+=1em
  [{V[\comps \sliste{ \ibox{1} }]}
    [\ibox{2} \npdat [dem Kind;the child,roof]]
    [{V[\comps \sliste{ \ibox{2}, \ibox{1} }]} [\ibox{3} \npnom [niemand;nobody]]
       [{V[\comps \sliste{ \ibox{3}, \ibox{2}, \ibox{1} }]}  [vorliest;\textsc{part}.reads]]]]
  [\ibox{1} \npacc [ein Buch;a book,roof]]]]
\end{forest}
\caption{\label{fig-dem-kind-niemand-gab-ein-buch-head-comp}Unwanted analysis using the
  Head-Complement Schema without linearization constraints}
\end{figure} 
\begin{figure}
\begin{forest}
sm edges
[{V[\comps \sliste{ }]}
  [{V[\comps \sliste{ \ibox{1} }]},s sep+=1em
    [{V[\comps \sliste{ \ibox{2}, \ibox{1} }]} [\ibox{3} \npnom [niemand;nobody]]
       [{V[\comps \sliste{ \ibox{3}, \ibox{2}, \ibox{1} }]}  [vorliest;\textsc{part}.reads]]]
    [\ibox{2} \npdat [dem Kind;the child,roof]]] 
  [\ibox{1} \npacc [ein Buch;a book,roof]]]]
\end{forest}
\caption{\label{fig-niemand-gab-dem-kind-ein-buch-head-comp}Unwanted analysis using the
  Head-Complement Schema without linearization constraints}
\end{figure} 

Now, this problem is easy to fix: what is needed is a binary feature specifying whether a head is
initial or not. The feature is called \textsc{initial} (abbreviated as \textsc{ini}). All
head-daughters that are \ini{}+ are always serialized to the left of their complement and all those
that are \ini{}$-$ are serialized to the right. The linearization rules are provided in (\mex{1}):
\eal
\label{lp-regeln}
\ex HEAD [\textsc{initial}+] $<$ COMPLEMENT
\ex COMPLEMENT $<$  HEAD [\textsc{initial}$-$]
\zl
German verbs are specified to be \textsc{initial}$-$, while English verbs are
\textsc{initial}$+$. Because of this specification and the linearization rules in (\mex{0}), verbs
are always ordered after their complements in German (and other SOV languages) and before their
complements in English (and other SVO languages). Of course, there are sentences in German in which
the verb is in first or second position and there are sentences in the Germanic SVO languages in
which the object precedes the verb. These sentences will be covered in
Chapter~\ref{chap-verb-position}.


% erlier?
%\cites{Haider2010a}
\citet[\page 342]{Haider2020a} claims that scrambling is only possible in
head-final projections. If this claim is correct (at least for the
languages under consideration here),\footnote{
  \citet[\page 244]{Santorini93a} claims that \ili{Old French}, \ili{Old Spanish}, \ili{Middle English} and \ili{Russian}
  are VO languages allowing for scrambling. More careful examination of the VO status and scrmabling
  properties of these languages is needed. I leave this for further research.
} the Head-Complement Schema in Figure~\ref{fig-spr-head-head-comps-final} has to
be restricted to head-initial projections and the Head-Complement Schema in Figure~\ref{fig-head-comp-free} to
head-final projections. Nouns in all Germanic languages are head-initial, that is, they take their
complements to the right and these are not allowed to scramble. Genitive NPs have to be adjacent to
the noun that governs them:
\eal
\ex[]{ 
\gll das Verlesen des    Entwurfes durch die Vorsitzende\\
     the reading  of.the draft     by    the chair.woman\\\german
}
\ex[*]{ 
\gll das Verlesen durch die Vorsitzende des Entwurfes\\
     the reading  by    the chair.woman of.the draft  \\
}
\zl
Verbs in SVO languages like English and the Scandinavian languages are \initial $+$ and hence form a
VP via the schema in Figure~\ref{fig-spr-head-head-comps-final}, augmented as the left schema in Figure~\ref{fig-head-comps-schemata}. Verbs in SOV languages are \initial
$-$ and combine with their complements via the schema in Figure~\ref{fig-head-comp-free}, augmented
as the right schema in Figure~\ref{fig-head-comps-schemata}.

\begin{figure}
\hfill
\begin{forest}
[H\feattab{\spr \ibox{1},\\
           \comps \ibox{2}}
  [H\feattab{\ini $+$,\\
             \spr \ibox{1},\\
             \comps  \sliste{ \ibox{3} } $\oplus$ \ibox{2}  ]}]
  [\ibox{3}]]
\end{forest}
\hfill
\begin{forest}
[H\feattab{\spr   \ibox{1},\\
           \comps \ibox{2} $\oplus$ \ibox{3}]}
  [\ibox{4}]
  [H\feattab{\ini $-$,\\
             \spr \ibox{1},\\
             \comps  \ibox{2} $\oplus$ \sliste{ \ibox{4} } $\oplus$ \ibox{3}  ]}]]
\end{forest}
\hfill\mbox{}
\caption{Head-Complement Schemata with ordered daughters and instantiated \initial values. Left
  schema for SVO languages and NP structures (no scrambling), right schema for SOV languages (scrambling)}\label{fig-head-comps-schemata}
\end{figure}

Yiddish with its mix of VO and OV structures poses an interesting formal puzzle, which is addressed
in the following section.

\section{Free VO/OV order}

% @ Haider2010a: 161
As already mentioned in Section~\ref{sec-intro-svo}, Yiddish has mixed VO/OV properties and
researchers like \citet[]{dBMvW86a}, \citet[\page 12]{Schallert2007a}, and \textcites[\page 161]{Haider2010a}{Haider2020a} see
this language as belonging to a third type.\footnote{
  \textcites{Santorini93a} also argues for a classification of Yiddish as a mixed VO/OV language
  but assumes that this means that particular sentences may be either VO or OV, but verbs cannot
  govern both to the left and to the right (p.\,240). The solution outlined below
  will not assume this but rather assume that Yiddish verbs do neither have an \initial value of $+$
  (VO) nor $-$ (OV) but a third value and hence can be placed in the middle of their arguments
  without any movement.
} Example (\ref{ex-yiddish-vo-ov}) from \citet[\page
402]{Diesing97a}, which was discussed on p.\,\pageref{ex-yiddish-vo-ov} and is repeated
below as (\mex{1}), shows that Yiddish can have the order usually observed in SVO languages
(\mex{1}a) and the orders observed in SOV languages with scrambling (\mex{1}b, c). But it can also
have the orders in (\mex{1}d) and (\mex{1}e), in which the verb is in the middle and either the
direct object or the indirect object precedes the verb.
\eal
\ex
\gll Maks hot nit gegebn Rifken dos bukh.\\
     Max  has not given  Rifken the book\\\yiddish
\glt `Max has not given Rifken the book.' 
\ex 
\gll Maks hot Rifken dos bukh nit gegebn.\\
     Max  has Rifken the book not given\\
\glt `Max has not given Rifken the book.'
\ex
\gll Maks hot dos bukh Rifken nit gegebn.\\
     Max  has the book Rifken not given\\
\glt `Max has not given Rifken the book.'
\ex
\gll Maks hot Rifken nit gegebn dos bukh.\\
     Max  has Rifken not given  the book\\
\glt `Max has not given Rifken the book.'
\ex
\glt Max hat dos bukh nit gegebn Rifken.\\
     Max has the book not given  Rifken.\\
\glt `Max has not given Rifken the book.'
\zl
\citet[\page 161]{Haider2010a} argues that Yiddish is a mixed VO/OV language and that heads just may
combine with their complements in any order. Haider claims that scrambling is only possible in
head-final projections. So, a variant of (\mex{0}a) with initial verb and scrambled objects is
predicted to be impossible.
 Now,
freedom seems to be easy to achieve as the absence of constraints. Yiddish would be a language in
which the \initial value is just unspecified. But this is not sufficient since if \emph{Rifken} is
combined with \emph{nit gegebn} to form \emph{Rifken nit gegebn}, the result is an \initial $-$
projection. This projection cannot be used as the daughter in the head-initial schema since it is
incompatible with \initial $+$. 

Fortunately, there is a solution to such problems that was developed
for accounts of case syncretism \citep{Daniels2002a}. It makes use of the type system that is part of the formalism for
specifying linguistic constraints \parencites[Section~3]{AB2021a}[Section~2]{Richter2021a}. Values like part of speech are types and types can have
subtypes. For example, there can be a type \type{part-of-speech} with the subtypes \type{noun},
\type{verb}, \type{adj}, \type{prep} and others. There can be a type \type{vform} with the subtypes \type{fin}, \type{bse},
\type{ppp}, \type{inf} for finite verbs, infinitives without \emph{to}, participles, and infinitives
with \emph{to}. For our problem at hand, one would need a type \type{bool} for boolean values ($+$ or
$-$). Normally, the subtypes of \type{bool} would be just $+$ and $-$, but since the requirements
from the schemata have to be allowed to be compatible, the type hierarchy has to be more complex.
It is given in Figure~\ref{fig-extended-bool}.
\begin{figure}
\begin{forest}
type hierarchy
[bool
  [$+$ or flex
    [$+$]
    [flex,name=flex, before drawing tree={x/.option=!uu.x}]]
  [$-$ or flex
    [,identify=flex]
    [$-$]]]
\end{forest}
\caption{Extended hierarchy for boolean types}\label{fig-extended-bool}
\end{figure} 
There are two new types \type{$+$ or flex} and \type{$-$ or flex}. \type{flex} is the \initial value
of Yiddish verbs. The schemata require their head daughters to be of type \type{$+$ or flex} or
\type{$-$ or flex}. SVO languages have verbs with \initial value $+$ and SOV languages have verbs
with \initial value $-$. $+$ and $-$ are compatible with the requirements of the schemata but they
do not allow a switch in the direction of government as is possible in Yiddish. 

\if0
dat V acc

sukhr habn unzri bridr gigebn fil gelt
 
merchants have our brothers given much money
'Merchants gave our brothers much money'
(1692E-VILNA,217.134)

drum hat er dem menshn gebn di turh ... 
'therefore has he the people given the Torah'
(1620E-LEVTOV1,4l.47)


V dat acc

hat gibrakht meyn oybrstn alirley shpetsirey '[who] brought my boss all kinds of spices'
(1665W-COURT,221.246)
(2) mer haben unzer formuner gegeben meinem stieffater tsvay hundert gulden
'our guardians gave my stepfather 200 guilders' (1518W-GOETZ,.137)


\fi


\section{Adjuncts}
\label{sec-adjuncts}

While arguments are selected by their head, adjuncts select the head. The difference between
languages like Dutch and German on the one hand, and Danish and English on the other hand, can be
explained by assuming that adjuncts in the former languages are less picky as far as the element is
concerned with which they combine. Dutch (\mex{1}) and German (\mex{2})
adjuncts can attach to any verbal projection, while Danish (\mex{3}) and English (\mex{4}) require a
VP (see also Section~\ref{sec-phenomena-position-of-adverbials}):

\eal
% todo check
\ex 
\gll [dat] onmiddellijk iedereen het boek leest\\
     \spacebr{}that promptly everybody the book reads\\\dutch
\glt `that everybody reads the book promptly'
\ex
\gll [dat] iedereen onmiddellijk het boek leest\\ 
     \spacebr{}that everybody promptly the book reads\\ 
\ex
\gll [dat] iedereen het boek onmiddellijk leest\\ 
    \spacebr{}that everybody the book promptly reads\\
\zl

\eal
\ex
\label{ex-m-j-b-l} 
\gll {}[dass] sofort jeder das Buch liest\\
     \spacebr{}that promptly everybody the book reads\\\german
\glt `that everybody reads the book promptly'
\ex
\label{ex-j-m-b-l} 
\gll {}[dass] jeder sofort das Buch liest\\
     \spacebr{}that everybody promptly the book reads\\ 
\ex
\label{ex-j-b-m-l}
\gll {}[dass] jeder das Buch sofort liest\\
    \spacebr{}that everybody the book promptly reads\\
\zl


\eal
% todo check
\ex 
\gll at hver læst bogen straks\\
     that everybody reads book.\textsc{def} promptly\\ \danish
\glt `that everybody reads the book promptly'
\ex 
\gll at hver straks læst bogen\\
     that everybody promptly reads book.\textsc{def}\\
\glt `that everybody promptly reads the book'
\zl

\eal
\ex that everybody reads the book promptly
\ex that everybody promptly reads the book
\zl

% \eal
% \ex Kim will have been [promptly [removing the evidence]].\footnote{
%   The examples are due to Stephen Wechsler (p.\,c.\, 2013).}
% %from \citew{Wechsler2015a}. in the draft but now gone.}
% \ex Kim will have been [[removing the evidence] promptly].
% \zl

For the selection of arguments the features \spr and \comps are used. In parallel there is a \modf
that is part of the lexical description of a head of a phrase that can function as an adjunct (\textsc{mod} is
an abbreviation for \emph{modified}). The value of  \textsc{mod} is a description of an appropriate head. 
Head"=adjunct structures are licensed by the schema in
Figure~\vref{fig-head-adj}.
\begin{figure}
\begin{forest}
[{H[\spr \ibox{1}, \comps \ibox{2}]}
  [{[\textsc{mod} \ibox{3}, \spr \eliste, \comps \eliste]}]
  [{\ibox{3} H[\spr \ibox{1}, \comps  \ibox{2}]}]]
\end{forest}
\caption{\label{fig-head-adj}Sketch of the Head-Adjunct Schema}
\end{figure}
For instance, attributive adjectives have \nbar as their \modv, where \nbar is an abbreviation for a
nominal projection that has an empty \compsl and a \sprl that contains a determiner. (\mex{1}) shows
the lexical item for \emph{brown}:
\eas
Lexical item for \emph{brown}:\\
\avm{
[ phon & \phonliste{ brown }\\
  mod  & \nbar\\
  spr  & <>\\
  comps & <> ]
}
\zs
The analysis of the phrase \emph{brown squirrel} is shown in Figure~\vref{fig-brown-squirrel}.
\begin{figure}
\begin{forest}
sm edges
[{\nbar}
  [{Adj[\textsc{mod} \ibox{2}]} [brown]]
  [{\ibox{2} \nbar} [squirrel]]]
\end{forest}
\caption{\label{fig-brown-squirrel}Analysis of the head-adjunct structure \emph{brown squirrel}}
\end{figure}
In languages like German in which the adjective agrees with the noun in gender, number, and
inflection class \parencites[Section~2.2.5]{ps2}[Section~13.2]{MuellerLehrbuch3}, the properties that the noun must have can be specified inside the \modv. For
instance, \emph{kleiner} selects a masculine noun and \emph{kleine} selects a feminine one:
\eal
\ex 
\gll ein kleiner Hund\\
     a   little  dog\\
\ex 
\gll eine kleine Katze\\
     a    little cat\\
\zl


For German adverbials, the value restricts the part of speech of the head to be verb (or rather
verbal since -- as (\mex{1}b) shows -- adjectival participles can be modified as
well) and the value of \textsc{initial} to be $-$. 
\eal
\ex
\gll dass es oft lacht\\
     that it often laughs\\\german
\glt `that he/she laughs often'
\ex 
\gll des oft lachende Kind\\
     the often laughing child\\
\glt `the child who laughs often'
\zl
The specification of the modified element to be \initial$-$ ensures that the adjunct attaches to
verbs in final position only (verb"=initial sentences are discussed in
Chapter~\ref{chap-verb-position}). A linearization rule has to make sure that adverbials are
serialized to the left of the verb, that is, somewhere in the \mf. The \modv of English adverbials
is simply VP. Without any further restrictions, this allows for a pre- and a post-VP attachment of
adjuncts.
\begin{itemize}
\item SOV (Dutch, German, \ldots): \textsc{mod} V[\textsc{ini}$-$]
\item SVO (Danish, English, \ldots): \textsc{mod} VP
\end{itemize}

The analysis of (\ref{ex-m-j-b-l}) is shown in Figure~\vref{fig-m-j-b-l}, the analysis of
(\ref{ex-j-m-b-l}) in Figure~\vref{fig-j-m-b-l}, and the analysis of (\ref{ex-j-b-m-l}) in
Figure~\vref{fig-j-b-m-l}. The only difference between the figures is the respective place of
attachment of the adverb. I marked the parts of the tree that are licensed by the Head-Adjunct
Schema by including them in a box. All other nodes in the tree are licensed by the Head-Complement
Schema in Figure~\ref{fig-head-comp}.


\begin{figure}

\centerfit{
\begin{forest}
sm edges
[{V[\spr \eliste, \comps \eliste]}, schema
        [{Adv[\textsc{mod} \ibox{3} V]} [sofort;promptly] ]
        [{\ibox{3} V[\spr \eliste, \comps \eliste]}
          [{\ibox{1} NP[\type{nom}]} [jeder;everybody] ]
          [V\feattab{
              \spr \sliste{ }, \comps \sliste{ \ibox{1} } }
            [{\ibox{2} NP[\type{acc}]} [das Buch;the book, roof] ] 
            [V\feattab{
              \spr \sliste{  },\\
              \comps \sliste{ \ibox{1}, \ibox{2} }} [liest;reads] ] ]
] ]
\end{forest}}
\caption{\label{fig-m-j-b-l}Analysis of [\emph{dass}] \emph{sofort jeder das Buch liest} `that everybody reads the
  book promptly' with the adjunct attaching above subject and object}
\end{figure}


\begin{figure}
\centerfit{%
\begin{forest}
sm edges
[{V[\spr \eliste, \comps \eliste]},s sep+=1.5em
          [{\ibox{1} NP[\type{nom}]} [jeder;everybody] ]
          [V\feattab{
              \spr \sliste{ }, \comps \sliste{ \ibox{1} } }, schema
            [{Adv[\textsc{mod} \ibox{3} V]} [sofort;promptly] ]
            [\ibox{3} V\feattab{
                \spr \sliste{ }, \comps \sliste{ \ibox{1} } }
              [{\ibox{2} NP[\type{acc}]} [das Buch;the book, roof] ] 
              [V\feattab{
                \spr \sliste{  },\\
                \comps \sliste{ \ibox{1}, \ibox{2} }} [liest;reads] ] ]
] ]
\end{forest}}

\caption{\label{fig-j-m-b-l}Analysis of [\emph{dass}] \emph{jeder sofort das Buch liest} `that everybody reads the
  book promptly' with the adjunct attaching between subject and object}
\end{figure}


\begin{figure}
\centerfit{%
\begin{forest}
sm edges
[{V[\spr \eliste, \comps \eliste]}
    [{\ibox{1} NP[\type{nom}]} [jeder;everybody] ]
      [V\feattab{
         \spr \eliste, \comps \sliste{ \ibox{1} } }, s sep+=1em
         [{\ibox{2} NP[\type{acc}]} [das Buch;the book, roof] ] 
           [V\feattab{
              \spr \eliste,\\
              \comps \sliste{ \ibox{1}, \ibox{2} }}, schema 
             [{Adv[\textsc{mod} \ibox{3} V]} [sofort;promptly] ]
             [\ibox{3} V\feattab{
                 \spr \eliste,\\
                 \comps \sliste{ \ibox{1}, \ibox{2} }} [liest;reads] ] ] ] ]
\end{forest}}
\caption{\label{fig-j-b-m-l}Analysis of [\emph{dass}] \emph{jeder das Buch sofort liest} `that everybody reads the
  book promptly' with the adjunct attaching between object and verb}
\end{figure}
The attentive reader will notice that there is a description following the \ibox{3} in the \modv of
the adverbials, while there is no such description in the \modvs of the English examples that
follow. Of course this is purely notational since the numbered boxes identify all values with the
same numbers, but the convention behind this is to state the description if it differs from what is
given in other places where the box occurs. In the case of German, the \modv of adverbials is just
\type{verb} without any restrictions regarding valence features. The valence features are given at
the modified node (\eg \spr \eliste, \comps \sliste{ \ibox{1}, \ibox{2} } in
Figure~\ref{fig-j-b-m-l}), but not in the \modv. Since English adverbials modify VPs and since the
modified node is a VP, the value of the \modv is not given in detail in the figures below, but is just shared with the
properties of the modified node.


The Figures~\ref{fig-adj-vp} and~\ref{fig-vp-adj} show the analysis of adjunction with the adverb in
pre-VP and post-VP position respectively.
\begin{figure}
\centerfit{%
\begin{forest}
sm edges
[{V[\spr \eliste, \comps \eliste]}, s sep+=1.5em % puts more space between the NP[nom] and the VP,
                                    % otherwise the box would overlap
          [{\ibox{1} NP[\type{nom}]} [Everybody] ]
          [V\feattab{
              \spr \sliste{ \ibox{1} }, \comps \sliste{  } }, schema
            [{Adv[\textsc{mod} \ibox{3}]} [promptly] ]
            [\ibox{3} V\feattab{
                \spr \sliste{ \ibox{1} }, \comps \sliste{  } }
              [V\feattab{
                \spr \sliste{ \ibox{1} },\\
                \comps \sliste{  \ibox{2} }} [reads] ]
              [{\ibox{2} NP[\type{acc}]} [the book,roof] ] ]
] ]
\end{forest}}
\caption{\label{fig-adj-vp}Analysis of adjuncts in SVO languages: the adjunct is realized left-adjacent to the VP.}
\end{figure}
\begin{figure}
\centerfit{%
\begin{forest}
sm edges
[{V[\spr \eliste, \comps \eliste]},s sep+=1em
          [{\ibox{1} NP[\type{nom}]} [Everybody] ]
          [V\feattab{
              \spr \sliste{ \ibox{1} }, \comps \sliste{  } }, schema
            [\ibox{3} V\feattab{
                \spr \sliste{ \ibox{1} }, \comps \sliste{  } }
              [V\feattab{
                \spr \sliste{ \ibox{1} },\\
                \comps \sliste{ \ibox{2} }} [reads] ]
              [{\ibox{2} NP[\type{acc}]} [the book,roof] ] 
               ]
            [{Adv[\textsc{mod} \ibox{3}]} [promptly] ]
] ]
\end{forest}}
\caption{\label{fig-vp-adj}Analysis of adjuncts in SVO languages: the adjunct is realized right-adjacent to the VP.}
\end{figure}

The values of \spr and \comps in the schema in Figure~\ref{fig-head-adj} on
page~\pageref{fig-head-adj} have not been explained so far. First there is the sharing of the \spr
and \compsvs between mother and head-daughter. Whatever element an adjunct attaches to, the valence
requirements of the mother are always identical to the valence requirement of the
head-daughter. Nothing is added, nothing is missing. Adjuncts are additional elements that are not
selected for via valence features, hence nothing has to be discharged. This can be seen by looking
at the German examples in Figures~\ref{fig-m-j-b-l} to~\ref{fig-j-b-m-l}: \emph{sofort} `promptly'
attaches to a node with certain valence requirements and the dominating node has exactly the same
valence requirements. In principle the figures should have little numbered boxes in them indicating
the identity of the valence requirements of mother and head daughter in head-adjunct combinations. I
omitted these so-called structure sharings to keep things simple and readable. 

The adjunct itself has to have empty valence lists, that is, it has to be complete. Without this
requirement, sentences like the one in (\mex{1}) would be licensed:
\ea[*]{
Sandy read the book in.
}
\z
\emph{in} is a preposition that has an \npacc in its \compsl. If the Head-Adjunct Schema would not
specify the \compsl of the adjunct daughter to be empty, a preposition could function as the adjunct
daughter and a structure for ungrammatical sentence like (\mex{0}) would be licensed by the
grammar. 

The specifier specification is as important as the specification of the \compsl. If non-empty \sprls
were allowed, the contrast in (\mex{1}) could not be explained:

\eal
\ex[]{
\gll dass jeder eine Stunde liest\\ 
     that everybody an   hour reads\\\german
\glt `Everybody is reading for an hour.'
}
\ex[*]{
\gll dass jeder Stunde liest\\ 
     that everybody hour reads\\
%\glt `Aicke is reading for an hour.'
}
\zl
The analysis of (\mex{0}a) is shown in Figure~\vref{fig-dass-Aicke-eine-Stunde-liest}. The adjunct
is a full NP. The schema requires the adjunct daughter to be fully complete. If it did not have this
requirement, a noun without determiner like \emph{Stunde} `hour' in (\mex{0}b) could enter the
schema as adjunct daughter and ungrammatical sentences like (\mex{0}b) would be licensed.
\begin{figure}
\begin{forest}
sm edges
[{V[\comps \eliste]}, s sep+=.5em
  [\ibox{1} NP [jeder;everybody]]
  [{V[\comps \sliste{ \ibox{1} }]} ,schema
    [{NP[\textsc{mod} \ibox{2} V]} [eine Stunde;one hour,roof]]
    [{\ibox{2} V[\comps \sliste{ \ibox{1} }]}  [liest;reads]]]]
\end{forest}
\caption{Analysis of an adverbial NP in \emph{dass jeder eine Stunde liest} `that everybody is reading
  for an hour'}\label{fig-dass-Aicke-eine-Stunde-liest}
\end{figure}



\section{Linking between syntax and semantics}
\label{sec-linking}


HPSG assumes that all arguments of a head are contained in a list that is called \textsc{argument
  structure} (\argst, \citealp*{DKW2021a}).\footnote{%
See \citealp[\page 28--29]{ps2}; \citealp{Wechsler95a-u};
\citealp{Davis2001a-u}; \citealp[Section~5.6]{MuellerLehrbuch1} for argument
linking in HPSG. \citew*{DKW2021a} is a handbook article on linking in HPSG.
} This list contains descriptions of the syntactic and semantic properties of
the selected arguments. For instance the \argstl of English \emph{give} and its German, Danish and
Dutch and Icelandic variants is given in (\mex{1}):
\ea
\sliste{ NP, NP, NP }
\z
The case systems of the involved languages vary a bit as will be explained in
Chapter~\ref{chap-case}, but nevertheless the orders of the NPs in the \argstl are the same across these
languages.\footnote{%
  Interestingly, \citet[\page 15]{Haider2010a} also states that the argument structure is the same
  across the Germanic languages, although he makes different assumptions as far as the structure of
  OV and VO clauses is concerned.%
% : while he suggests the same structure for OV languages, he assumes
%  the structure in (i) for SVO langauges.
%  He argues for this structure with reference to processing facts. I think his argumentation is
%  mistaken for all we know about human language processing it is incremental and hence flat
%  structures as in \citew{Sag2020a} or binary branching structures as in (ii) are the most plausible
%  structures.
%  In structures like (ii), all available material is integrated into
} They correspond to nom, dat, acc in German (\mex{1}a) and subject, primary object, secondary object
in English (\mex{1}b):
\eal
\ex 
\gll dass das Kind dem Eichhörnchen die Nuss gibt\\
    that the child  the squirrel    the nut gives\\
\glt `that the child gives the squirrel the nut'
\ex that the child gives the squirrel the nut
\zl
In addition to the syntactic features we have seen so far semantic features are used to describe the
semantic contribution of linguistic objects. (\mex{1}) shows some aspects of the description of the English verb
\emph{gives}:
\ea
lexical item for \emph{gives}:\\*
\ms{
arg-st & \sliste{ NP\ind{1}, NP\ind{2}, NP\ind{3} }\\[2mm]
cont   & \ms[give]{
          agens & \ibox{1}\\
          goal  & \ibox{2}\\
          trans-obj & \ibox{3}\\
        }\\
}
\z
The lowered boxes refer to the referential indices of the NPs. One can imagine these indices as
variables that refer to the object in the real world that the NP is referring to. These indices are
identified to semantic roles of the verb \emph{give}. Finding reasonable role names is not trivial
and some authors just use \argone, \argtwo and \argthree to avoid the problems (see \citealp{Dowty91a}
for discussion).

The representations for the other languages mentioned above is entirely parallel. Therefore it is
possible to capture crosslinguistic generalizations. Nevertheless there are differences between the
Germanic OV and VO languages. As was explained above the VO languages map their subject to \spr and
all other arguments to \comps, while the finite verbs of OV languages have all arguments on \comps. 
(\mex{1}) shows some examples.\footnote{%
 For readability, I just listed the NPs on the respective lists. In
actual analyses, the \argstl is split into two sublists \ibox{4} and \ibox{5} and \ibox{4} is the
\sprl and \ibox{5} the \compsl. The \sprl contains just one element in languages like English and no
element for finite verbs in languages like German.
  See \citew[171]{GSag2000a-u}, \citew[12]{BMS2001a} and \citew[\page 17]{AB2021a} for details on argument realization.
}
\eal
\ex Linking and argument mapping for English finite verb (SVO):\\
\ms{
spr    & \sliste{ NP\ind{1} }\\[2mm]
comps  & \sliste{ NP\ind{2}, NP\ind{3} }\\[2mm]
arg-st & \sliste{ NP\ind{1}, NP\ind{2}, NP\ind{3} }\\[2mm]
cont   & \ms[give]{
          agens     & \ibox{1}\\
          goal      & \ibox{2}\\
          trans-obj & \ibox{3}\\
        }\\
}

\ex Linking and argument mapping for German finite verb (SOV):\\*
\ms{
spr    & \sliste{ }\\[2mm]
comps  & \sliste{ NP\ind{1}, NP\ind{2}, NP\ind{3} }\\[2mm]
arg-st & \sliste{ NP\ind{1}, NP\ind{2}, NP\ind{3} }\\[2mm]
cont   & \ms[geben]{
          agens     & \ibox{1}\\
          goal      & \ibox{2}\\
          trans-obj & \ibox{3}\\
        }\\
}
\zl

This provides a connection between arguments of a head and the semantic roles they fill. While this
is a first step towards semantics, a lot remains to be said. For example, the contribution of
quantifiers like \emph{every} and \emph{a} in (\mex{1}) and the determination of the scope they take is not explained yet. 
\ea
Every squirrel wants to eat a nut.
\z
But this introduction to syntax is not the place to do this. The reader is referred to
\citew{KoenigRichter2021a} for an overview of approaches to semantics in HPSG. The implemented fragments of
German, Danish, English and Yiddish mentioned in the preface assume Minimal Recursion Semantics (MRS; \citealt{CFPS2005a}), which is
also covered by \citet[Section~6.1]{KoenigRichter2021a}.


\section{Alternatives}

This section is for advanced readers. Subsection~\ref{sec-cp-tp-vp} compares the theory developed here with approaches to
German developed in the theory of Government \& Binding (GB)
\citep{Chomsky81a,Chomsky86b}. Subsection~\ref{sec-down-to-earth-syntax} contains a comparison to
certain approaches to syntax in GB and Minimalism \citep{Chomsky95a-u}. I argue for an approach to
syntactic categories and phrases that is normally used rather than the more recent approaches that
include semantic and pragmatic notions into syntactic structures. Like
Subsection~\ref{sec-cp-tp-vp}, Subsection~\ref{sec-down-to-earth-syntax} is optional and it is
possible to understand the rest of the book without reading it. I suggest to read it nevertheless
since it may deepen the understanding of syntax in general.



\subsection{CP/TP/VP models}
\label{sec-cp-tp-vp}\label{sec-cp-tp-vp-scrambling}
\label{sec-discussion-scope}

\citet{Grewendorf88a,Grewendorf93}, \citet{Lohnstein2014a} and many others assume that German
sentences have a constituent structure that is parallel to the structure that is assumed by
\citet{Chomsky86b} for English. As for English, the verb is
assumed to form a phrase with its objects and this VP functions as the argument of a Tense head to
form a maximal projection together with the subject of the verb, which is realized in the specifier
position of the TP.\footnote{
The Tense Phrase roughly corresponds to the Inflection Phrase (IP) in earlier
publications. \citet[\page 397]{Pollock89a-u} assumes further functional projections. This is called the
Split-IP approach. See also Section~\ref{sec-down-to-earth-syntax} on functional projections.
} Figure~\ref{fig-cp-tp-vp} shows the analysis of (\mex{1}) with the respective
VP, TP, and CP layers.
\ea
\gll dass jeder dieses Buch kennt\\
     that everybody this book knows\\
\glt `that everybody knows this book'
\z
\begin{figure}
\centering
\begin{forest}
sm edges
[CP
  [C$'$
    [C [dass;that]]
    [TP
      [NP [jeder;everybody,roof]]
      [T$'$
	[VP
	  [V$'$
	    [NP [dieses Buch;this book, roof]]
	    [V [\trace$_j$]]]]
	[T [kenn-$_j$ -t;know- -s]]]]]]
\end{forest}
\caption{\label{fig-cp-tp-vp}Sentence in the CP/TP/VP model}
\end{figure}%

The problem with such proposals is the claim that the subject is realized in the specifier position
of TP. Therefore there is no way of serializing the accusative
object before the subject unless one assumes that the 
object is moved to a higher position in the tree, \eg adjoined to TP as in Figure~\ref{fig-cp-tp-vp-scrambling}.
\begin{figure}
\centering
\begin{forest}
sm edges
[CP
[C$'$
	[C [dass;that]]
        [TP
          [NP$_i$ [dieses Buch;this book, roof]]
	  [TP
	    [NP [jeder;everybody,roof]]
	    [T$'$
	      [VP
		[V$'$
		  [NP [\trace$_i$]]
		  [V [\trace$_j$]]]]
	      [T [kenn-$_j$ -t;know- -s]]]]]]]
\end{forest}
\caption{\label{fig-cp-tp-vp-scrambling}Scrambling has to be movement in the CP/TP/VP model}
\end{figure}%

While researchers like \citet[\page 185]{Frey93a} argued that quantifier scopings are 
evidence for movement-based approaches, they actually provide evidence against movement"=based approaches. Let us consider Frey's examples. Frey
argues that sentences without movement have only one reading and sentences like (\mex{1}b) in which
-- according to the movement-based theory -- movement is involved have two readings: one corresponding to
the visible order and one to the order before movement, the so-called underlying order. 
\eal
\ex 
\gll Es ist nicht der Fall, daß er mindestens einem Verleger fast jedes Gedicht anbot.\\
     it is not the case that he at.least one publisher almost every poem offered\\
\glt `It is not the case that he offered at least one publisher almost every poem.'
\ex 
\gll Es ist nicht der Fall, daß er fast jedes Gedicht$_i$ mindestens einem Verleger \_$_i$ anbot.\\
	 it is not the case that he almost every poem at.least one publisher {} offered\\
\glt `It is not the case that he offered almost every poem to at least one publisher.'
\zl

\noindent
However, \citet[\page 146]{Kiss2001a} and \citet[Section~2.6]{Fanselow2001a} pointed out that such
approaches have problems with multiple moved constituents. For instance, in an example such as
(\mex{1}), it should be possible to interpret \emph{mindestens einem Verleger} `at least one
publisher' at the position of \_$_i$, which would lead to a reading where \emph{fast jedes Gedicht}
`almost every poem' has scope over \emph{mindestens einem Verleger} `at least one
publisher'. However, this reading does not exist.


\ea
\gll Ich glaube, dass mindestens einem Verleger$_i$ fast jedes Gedicht$_j$ nur dieser Dichter \_$_i$ \_$_j$ angeboten hat.\\
     I believe that at.least one publisher almost every poem only this poet {} {} offered has\\
\glt `I think that only this poet offered almost every poem to at least one publisher.'
\z

This means that one needs some way to determine the deviation with respect to an unmarked order, but
movement is not the solution. See \citew[Section~3.5]{MuellerGT-Eng4} for further discussion and
\citet{Kiss2001a} for an approach to scope within the framework assumed here.

\subsection{Syntax and other levels of description}
\label{sec-down-to-earth-syntax}

Chapter~\ref{chap-psg} relied on constituency tests that are standardly assumed in the syntactic
literature \parencites[24--31]{Borsley91a}[35--36]{Haegeman94a-u}[20--23]{HP2002a-ed}[29--33]{SWB2003a}[19--22]{KS2008a-u}[Chapter~1.3]{MuellerGT-Eng}{MyP2022a}.
It is usually assumed that phrases are assigned categories that correspond to distribution
classes. For example, a complex noun phrase can be replaced by other complex noun phrases or by
pronouns. 
\eal
\label{ex-np-tisch}
\ex 
\gll der Tisch\\
     the.\NOM{} table\\
\ex 
\gll der Tisch aus Japan\\
     the.\NOM{} table from Japan\\
\ex 
\gll der alte Tisch aus Japan\\
     the.\NOM{} old  table from Japan\\
\ex 
\gll er\\
     he\\
\zl
% The phrase \emph{der Tisch} can be combined with \emph{ist schön} to form
% \ea
% \gll  Der Tisch ist schön.\\
%       the table is  beautiful\\
% \glt `The table is beautiful.'
% \z

Features like case, person, and gender are important for the distribution of noun
phrases. An accusative pronoun cannot be replaced by a nominative pronoun. Similarly, person and
number are important for the distribution of noun phrases since they have to match the properties of
the verb. Similarly, gender is important for the distribution of noun phrases and pronouns:
\emph{der Tisch} `the table' is masculine and can be replaced by the pronoun \emph{er} `he' but it
cannot be replaced by \emph{sie} `she', which is feminine. \emph{der Tisch} `the table' and
\emph{die Vase} `the vase' differ in gender but can be exchanged in many contexts, since both of
these phrases are NPs. The phrases in (\mex{1}) are different, since they are lacking a
determiner. We used the category \nbar for such phrases. Again, the phrases in (\mex{1}) can be
replaced by other phrases of this category: wherever we use \emph{Vase aus China}, we can also use
\emph{alte Vase aus China}. This is why the same category is assigned to all these phrases.
\eal
\label{ex-n-bar-vase}
\ex 
\gll Vase\\
     vase\\
\ex 
\gll Vase aus China\\
     vase from China\\
\ex 
\gll alte Vase aus China\\
     old  vase from China\\
\zl
The phrases in (\ref{ex-np-tisch}) and (\ref{ex-n-bar-vase}) differ in terms of completeness and in
their gender. HPSG models this by assuming the categories can be complex. They consist of various
features like gender, number, person, part of speech, valence and so on.

In fact, the reason for these features to be assumed in grammars is that they play a role
in the distribution of words and larger units of words. If we were to discover the structure of an
unknown language, this would be the task: replace certain units in an utterance and see how things
change. Dan Everett did this when studying \ili{Pirahã}.\footnote{
See \url{https://youtu.be/5NyB4fIZHeU?t=868}, 2022-03-31.
} He first pointed at objects to learn the words used for them. Then he let a stick fall down and asked how this is
expressed in Pirahã. Then he let a leaf fall down and asked for the expression. He can then try and
identify the words he learned in other environments and maybe, depending on the language, with different inflections. So, syntax
is about the distribution of words and groups of words. The classes that can be found this way
correspond to parts of speech and morpho-syntactic features like gender, number, and case. The
phrase structure grammar introduced in Chapter~\ref{chap-psg} follows this tradition, which goes back to
\citet{Bloomfield26a}, \citet{Harris46a-u}, and \citet{Wells47a}. To give a simple example of a
traditional phrase structure grammar, consider the grammar in (\ref{bsp-grammatik-psg}) on
p.~\pageref{bsp-grammatik-psg} -- repeated here as (\mex{1}) for convenience:
\ea
\label{bsp-grammatik-psg-two}
\begin{tabular}[t]{@{}l@{ }l}
{NP} & {$\to$ Det N}\\          
{S}  & {$\to$ NP VP}\\
{VP} & {$\to$ V NP}
\end{tabular}\hspace{2cm}%
\begin{tabular}[t]{@{}l@{ }l}
{NP}  & {$\to$ she}\\
{Det} & {$\to$ the}\\
\end{tabular}\hspace{8mm}
\begin{tabular}[t]{@{}l@{ }l}
{N} & {$\to$ child}\\
{N} & {$\to$ book}\\
{V} & {$\to$ reads}\\
\end{tabular}
\z
This little grammar assigns categories to words: \emph{she} is an NP, \emph{the} is a determiner,
\emph{child} and \emph{book} are nouns, \emph{reads} is  a verb. In addition, there are several
phrase structure rules. The NP rule states that an NP may consist of a Det and an N. The S rule says
that an S may consist of an NP and a VP. The VP rule states that a VP may consist of a V and an
NP. While such rewrite rules are independent of the notion of head in principle, heads play an
important role in linguistics. Usually, we talk about NPs because the phrase under consideration
contains an N as the most important element.

Starting with \citet{Larson88a} and \citet{Pollock89a-u} different views entered into Mainstream
Generative Syntax. They culminated in the cartographic work of Cinque and Rizzi, who assume at least
400 so-called functional heads in an analysis of a sentence, most of them invisible \citep[\page
57]{CR2010a}. The following subsections are devoted to such approaches and I want to argue that
syntax should be about distribution classes rather than fixed cascades of functional heads.

\subsubsection{Autonomy of Syntax, resulting problems and syntactification of other descriptive levels}
\label{sec-autonomy-of-syntax}

Chomsky argued for the autonomy of syntax and developed architectures in which there was a syntactic
component that fed information into a phonology module and into a semantics
module. Figure~\ref{Abb-T-Modell} shows the so-called T~model from \citet[\page 5]{Chomsky81a}.
\begin{figure}
\centering
\begin{forest}
for tree = {edge={->},l=4\baselineskip}
[D-structure
     [S-structure,edge label={node[midway,right]{move $\alpha$}} 
            [Deletion rules{,}\\Filter{,} phonol.\ rules
                    [Phonetic\\Form (PF)]]
            [Anaphoric rules{,}\\rules of quantification and control
                    [Logical\\Form (LF)]]]]
    \end{forest}

\caption{\label{Abb-T-Modell}The T~model as described by \citet[\page 5]{Chomsky81a}}
\end{figure}%
The problem with such an architecture is that syntax interacts with phonology, semantics, and information
structure and that this cannot be captured if one deals with syntax alone. As a result of this,
semantic and information structural notions entered certain flavors of syntax in the GB/Minimalism
framework. For example, approaches in the tradition of \citet{CR2010a} assume phrases called Topic
Phrase or Focus Phrase although these phrases are simply clausal projections. So, in ``normal''
syntax they would be verb phrases (VPs) or sentences
(S). Figure~\ref{Abbildung-Remnant-Movement-Satzstruktur} shows an (abbreviated) analysis of a German clause in this
tradition.\footnote{
  Fronted phrases and adverb phrases are assumed to be in specifier positions of respective
  projections. This means that the actual Topic, Aspect, Manner, and Negation heads are missing in the
  figure. The technical details of such approaches are discussed in Section~\ref{sec-funct-proj-and-adverbs}.
} The syntax tree contains a wild mix of categories including a Topic Phrase (TopP),
Subject Phrase (SubjP), a Negation Phrase (NegP), an Auxiliary Phrase (AuxP), Manner Phrase (MannP),
Aspect Phrase (AspP) and the more common Determiner Phrase (DP; our NP) and Verb Phrase (VP). 
%
Categories like Topic Phrase and Focus Phrase are information structural categories. The
topic and focus of a phrase can be some parts of the phrase, for example, fronted constituents. The
information that there is a topic within a phrase concerns the relevant parts and should not be the name of the complete phrase. See
\citew{DeKuthy2021a} for an overview of the analyses of information structure in HPSG. Manner,
aspect, negation are semantic categories and of course this
information is important in grammatical theory and should be represented somewhere in a grammar, but
it should not be the part of speech label (based on the meaning-contribution of a non-head). See Section~\ref{sec-linking} for the place of semantic information in
HPSG and a sketch of the connection between syntax and semantics. Further details about semantics in
HPSG can be found in \citew{KR2021a}.
%
Subject
and object are grammatical functions. 
Frameworks like Lexical Functional Grammar use grammatical
functions as primitives of their theories. They state formally that a certain phrase is the subject or object
of a head, but they do not assume Subject Phrases or Object Phrases. For more on \citeauthor{CR2010a} style analyses and part of speech information see \citew[Section~4.6.1.1]{MuellerGT-Eng4}.
\begin{figure}
\oneline{%
\begin{forest}
where n children=0{delay=with unaligned translation}{}
[CP
	[C$^0$[weil;because, tier=word]]
	[TopP
		[DP$_j$ [diese Sonate;this sonata,tier=below,l=31\baselineskip]]
		[SubjP
			[DP$_i$ [der Mann;the man,tier=word]]
			[ModP
				[AdvP [wahrscheinlich;probably,l=20\baselineskip]]
				[ObjP
					[DP$_j$ [diese Sonate;this sonata,tier=below]]
					[NegP
						[AdvP [nicht;not,tier=word]]
						[AspP
							[AdvP [oft;often,tier=word]]
							[MannP
								[AdvP [gut;well,tier=word]]
								[AuxP
									[VP$_k$ [gespielt;played,tier=word]]
									[Aux+
										[Aux [hat;has,tier=word]]
										[vP
											[DP$_i$]
											[VP$_k$
												[V]
												[DP$_j$
                                                                                                  [,phantom,tier=word]]]]]]]]]]]]]]
\end{forest}%
}
\caption{\label{Abbildung-Remnant-Movement-Satzstruktur}Analysis of sentence structure with functional heads following \citet[\page 224]{Laenzlinger2004a}}
\end{figure}%

In comparison to so-called Cartographic approaches, the model of syntax argued for in this book is very simple: categories stand for linguistic objects
with certain properties that belong to the same distribution class. For example an NP with masculine
gender in the third person singular (\ref{ex-np-tisch}) or an \nbar with feminine gender in the third person
singular as in (\ref{ex-n-bar-vase}). Since the phrases/words in (\ref{ex-np-tisch}) and
(\ref{ex-n-bar-vase}) share the respective properties, they can be exchanged. These categories can be
selected by other heads. As was described in this chapter, a verb may select an NP in the
nominative. This is not possible in theories with ``creative'' categories. I will explain this with
respect to \citegen{Cinque99a-u} adverb analysis in Section~\ref{sec-funct-proj-and-adverbs}, but
before going into adverbs in general, I examine a specific case in Section~\ref{sec-NegP}: negation
analyses with a Negation Phrase. Section~\ref{sec-ConjP} deals with Cartography and coordination in
general and with coordinaton approaches assuming a Conjunction Phrase in particular.

%%%%%%%%%%%%%%%%%%%%%%%%%%%%%%%%%%%%%%%%%%%%%%%%%%%%%%%%%%%%%%%%%%%%%%%%%%%%%%%%%%%%%%%%%%%%%%%%%%%%%%%%%%%%%%%%%%%%%%%%%%%%%%%%
%%%%%%%%%%%%%%%%%%%%%%%%%%%%%%%%%%%%%%%%%%%%%%%%%%%%%%%%%%%%%%%%%%%%%%%%%%%%%%%%%%%%%%%%%%%%%%%%%%%%%%%%%%%%%%%%%%%%%%%%%%%%%%%%

\subsubsection{Negation as adverb or as special projection}
\label{sec-NegP}

As described above, approaches assuming functional heads often assign category labels to phrases
that would correspond to the non-head in more traditional approaches. This subsection has a
closer look on negation and approaches assuming a Negration Phrase. \citet{Ernst92a} examined an analysis of
negation in which the negation element is analyzed as the head. Ernst pointed out that it is
not just verbs that can be negated. Negation can attach to different verbal projections (\mex{1}a,b), to adjectives (\mex{1}c)
and adverbs (\mex{1}d).
\eal
\ex Ken could not have heard the news.
\ex Ken could have not heard the news.
\ex a [not unapproachable] figure
\ex {}[Not always] has she seasoned the meat.
\zl
But the respective phrases have different distributions. They are not just NegPs. We need
information about the part of speech of the head and about the verb form. It may be possible to
solve these problems by assuming extensions for functional projects as the ones to be discussed in
Subsection~\ref{sec-major-minor-category}. Alternatively, it may be possible to use constraints that
can search in trees for information and find out that the NegP \emph{not always} contains an adverb,
that the NegP \emph{not have heard the news} contains a verb in the base form and that \emph{not
  heard the news} contains a verb in the participle perfect form, but any such non-local approach
would be more complicated than the approach taken here and all other non-Minimalist theories (for
explicit discussion see \citealt{Sag2007a}) and also many Minimalist theories (\eg \citealt[\page 
  223]{Abraham2005a}). The approach that was discussed in this chapter just compares a requirement
  of a head with the properties of the complement or specifier daughter. In comparison, \citegen{Laenzlinger2004a}
  approach has to find the verb in \emph{nicht oft gut gespielt hat} `not often well played has'
  somewhere deeply embedded in a cascade of NegP, AspP, MannP, AuxP, Aux+. 
% Liegt daran, dass das abgekürzt ist. Geht sowieso nicht so.
% Note also that it is
%   unclear for an algorithm looking at structures like the one in Figure~\ref{Abbildung-Remnant-Movement-Satzstruktur} which daughter is the
%   head. It is just different material combined to some phrase whose label is determined by the
%   meaning of an adverbial.

Concluding this subsection, I follow all work in GPSG and HPSG and other frameworks in assuming that
negation particles are not heads but adjuncts. See \citew{KS2002a} and \citew[Section~6]{Sag2020a} on negation in English and
\citew{Kim2021b} for an overview of analyses of negation in HPSG in general.

The next subsection deals with Cinque's theory of adverbs in general. It examines similar problems
of category labeling and selection.

%%%%%%%%%%%%%%%%%%%%%%%%%%%%%%%%%%%%%%%%%%%%%%%%%%%%%%%%%%%%%%%%%%%%%%%%%%%%%%%%%%%%%%%%%%%%%%%%%%%%%%%%%%%%%%%%%%%%%%%%%%%%%%%%
%%%%%%%%%%%%%%%%%%%%%%%%%%%%%%%%%%%%%%%%%%%%%%%%%%%%%%%%%%%%%%%%%%%%%%%%%%%%%%%%%%%%%%%%%%%%%%%%%%%%%%%%%%%%%%%%%%%%%%%%%%%%%%%%


\subsubsection{Functional projections hosting an adverb in their specifier position}
\label{sec-funct-proj-and-adverbs}
\label{sec-major-minor-category}

It is claimed by Cinque and Rizzi that the hierarchy of functional projections is the same across
languages and in principle, one could also imagine that this array of functional projections is just
present in all languages and in all sentences even if some adverbs are not realized in a sentence or
do not exist in the language under consideration \citep[\page 55]{CR2010a}. For example, \citet[\page 106]{Cinque99a-u} suggests the following hierarchy of functional heads:
\ea
\label{ex-adjunct-hierarchy}
The universal hierarchy of clausal functional projections \citep[\page 106]{Cinque99a-u}
{}[ \emph{frankly} Mood\sub{speech act} [ \emph{fortunately} Mood\sub{evaluative} [ \emph{alledgedly}
Mood\sub{evidential} [ \emph{probably} Mood\sub{epistemic} [ \emph{once} T(Past) [ \emph{then}
T(Future) [ \emph{perhaps}
Mood\sub{irrealis} [ \emph{necessarily} Mood\sub{necessity} [ \emph{possibly} Mod\sub{possibility} [ \emph{usually}
Asp\sub{habitual} [ \emph{again} Asp\sub{repetitive(I)} [ \emph{often} Asp\sub{frequentative(I)} [ \emph{intentionally}
Mod{volitional} [ \emph{quickly} Asp\sub{celerative(I)} [ \emph{already} T(Anterior) [ \emph{no longer}
Asp\sub{terminative} [ \emph{still} Asp\sub{continuative} [ \emph{always} Asp\sub{perfect(?)} [ \emph{just}
Asp\sub{retrospective} [ \emph{soon} Asp\sub{proximative} [ \emph{briefly} Asp\sub{durative} [
\emph{characteristically}(?) Asp\sub{generic/progressive} [ \emph{almost} Asp\sub{prospective} [ \emph{completely}
Asp\sub{SgCompletive(I)} [ \emph{tutto} Asp\sub{PICompletive} [ \emph{well} Voice [ \emph{fast}/\emph{early}
Asp\sub{frequuentative(II)} [ \emph{completely} Asp\sub{SgCompletive(II)}
\z 
Starting with some of the functional heads from the top of this hierarchy, all sentences in all
languages are claimed to have structures as in Figure~\ref{fig-cinque-top} as part of their clausal structure.
\begin{figure}
\begin{forest}
[Mood\sub{speech act}P
  [AdvP]
  [Mood\sub{speech act}$'$
    [Mood\sub{speech act}]
    [Mood\sub{evaluative}P
      [AdvP]
      [Mood\sub{evaluative}$'$
        [Mood\sub{evaluative}]
        [Mood\sub{evidential}P
          [AdvP]
          [Mood\sub{evidential}$'$
            [Mood\sub{evidential}]
            [\ldots]]]]]]]
\end{forest}
\caption{Top-most part of \citegen{Cinque99a-u} functional hierarchy of adverbial projections}\label{fig-cinque-top}
\end{figure}
The actual adverbs are assumed to be realized in specifier positions of the functional heads
(Mood\sub{speech act}, Mood\sub{evaluative}, Mood\sub{evidential}).
In order to enforce the sequence stated in (\ref{ex-adjunct-hierarchy}), the functional head Mood\sub{speech act} has to select a Mood\sub{evaluative}P and
the Mood\sub{evaluative} head has to select a Mood\sub{evidential}P and so on. 
%
Since it is assumed that all languages have these structures even if there is no language internal
evidence for them, this requires a rather strong conception of Universal
Grammar: it is assumed in Mainstream Generative Grammar that language acquisition is guided by
innate knowledge about language, the so-called Universal Grammar (UG), and \citet{CR2010a} are proponents of a rather extreme position
claiming that at least 400 syntactic categories are part of this genetically specified linguistic
knowledge \citep[\page 57]{CR2010a}. Since there is no evidence for this and since it is rather unclear how and why categories
like case, gender and even nationality \parencites[\page 96, 99, 100]{Cinque94a-u}[114]{Scott2002a-u} should enter the human genome\footnote{
  See \citet{Bishop2002a} and \citet*{EBJKSPP96a} about genetics and linguistic information. See
  also \citet*{HCF2002a} for a statement about a rather general UG.
}, the overall approach seems
dubious. But even if one followed \citeauthor{CR2010a}, the fixed battery of functional projections
approach does not work without further add ons, as the following example by
\citet[\iaddpages]{Haider2022a} based on \citew[§ 8.20, 495]{QGLS85a-u} shows:
\eanoraggedright
The new theory certainly may possibly have indeed been badly formulated.
\z
The point here is that adverbs of the same type appear at various places in the sentence. They are
attached to VPs whose heads differ in their verb forms. These verb forms are selected by the governing
head. Figure~\ref{fig-verbal-selection} shows the structure assumed in this book.
\begin{figure}
\oneline{%
\begin{forest}
sm edges
[{S[\textit{fin}]}
  [NP [the new theory,roof]]
  [{VP[\textit{fin}]}
    [Adv [certainly]]
    [{VP[\textit{fin}]}
       [{V[\textit{fin}]} [may]]
       [{VP[\textit{bse}]}
         [Adv [possibly]]
         [{VP[\textit{bse}]}
           [{V[\textit{bse}]} [have]]
           [{VP[\textit{prf}]}
             [Adv [indeed]]
             [{VP[\textit{prf}]}
               [{V[\textit{prf}]} [been]]
               [{VP[\textit{pas}]}
                 [Adv [badly]]
                 [{VP[\textit{pas}]} [formulated]]]]]]]]]]
\end{forest}}
\caption{Cascade of selections including adverbs as suggested in this book}\label{fig-verbal-selection}
\end{figure}
Figure~\ref{fig-Cinque-AdvP-VP-selection} shows the structure with \citeauthor{Cinque99a-u} style
empty adverbial heads (here F-Adv). The actual adverbs are analyzed as specifiers of these heads.
\begin{figure}
\oneline{%
\begin{forest}
sm edges
[??
  [NP$_i$ [the new theory,roof]]
  [F-AdvP
    [AdvP [certainly,roof ]]
    [F-Adv$'$
      [F-Adv  [\trace]]
      [TP 
      [NP$_i$ [\trace]]
      [T$'$
        [T [may]]
        [F-AdvP
          [AdvP [possibly]]
          [F-Adv$'$
            [F-Adv  [\trace]]
            [{VP[\textit{bse}]}
              [{V[\textit{bse}]} [have]]
              [F-AdvP
                [AdvP [indeed]]
                  [F-Adv$'$
                    [F-Adv  [\trace]]
                    [{VP[\textit{prf}]}
                      [{V[\textit{prf}]} [been]]
                      [F-AdvP
                        [AdvP [badly]]
                        [F-Adv$'$
                          [F-Adv  [\trace]]
                          [{VP[\textit{pas}]}
                            [{V[\textit{pas}]} [formulated]]
                            [NP$_i$ [\trace]]]]]]]]]]]]]]]]
\end{forest}}
\caption{VP cascade including empty functional heads hosting adverbial phrases in specifier positions
  as suggested by \citet{Cinque99a-u}}\label{fig-Cinque-AdvP-VP-selection}
\end{figure}

The analysis also contains some details that can be ignored here. For example, passive is analyzed
as movement: the object of the verb \emph{formulated} moves to the specifier position of a Tense
head. Auxiliaries are usually classified as Tense elements in grammars of English. Since the adverb
\emph{certainly} attaches to T, the NP \emph{the new theory} has to move to a higher position at the
beginning of the clause. The figure shows that we have four projections with the label F-AdvP. Three
of them have the same semantic type (\emph{certainly}, \emph{possibly}, \emph{indeed}) and hence
should have the same category label in a
% probably Mod epistemic, possibly Mod possibility
\citeauthor{CR2010a} system. But they do not have the same distribution. So even in a system with
fixed arrays of functional projections, something would be missing.
Even with different category labels for the respective adverbs, the problem would not be fixed since
\emph{possibly}, \emph{certainly} and \emph{indeed} my all appear above and below the VP nodes in
Figure~\ref{fig-Cinque-AdvP-VP-selection}. The only way to enforce correct distribution via a local
selection mechanism would be to include the verb form into the category label. \citet{Grimshaw2000a}
developing a theory of extended projections suggested something along these lines. See also \citew{Riemsdijk98a}. One could have a
category that is determined by the lexical head (noun, verb, adjective) and a second category
determined by the functional head (determiner, I or T, or any of the \citeauthor{CR2010a}
categories).\footnote{
This idea is actually much older. Klaus Netter suggested an analysis of the DP in German assuming
\textsc{major} and \textsc{minor} head features \citep[Section~9.3.1]{Netter94}.}  


Figure~\ref{fig-the-president-DP} may serve as an example. It shows the analysis of the phrase \emph{the president} in the DP
analysis, in which the determiner is assumed to be the head.
\begin{figure}
\begin{forest}
sm edges
[DP
  [D$'$
     [D [the]]
     [NP
       [N$'$
         [N [president]]]]]] 
\end{forest}
\caption{\emph{the president} in the DP analysis}\label{fig-the-president-DP}
\end{figure}
Instead of just projecting the category of the functional head D, one could project information about the lexical category as
well. This would make information
about verbs and nouns available at the top node of functional projections. Let us assume the feature \textsc{f} for
functional categories and the feature \textsc{l} for lexical
categories. Figure~\ref{fig-the-president-DP-f-l} shows the analysis of the DP \emph{the president}
with such features. 

\begin{figure}
\begin{forest}
sm edges
[\textsc{f}:DP\\\textsc{l}:NP
  [\textsc{f}:D$'$\\\textsc{l}:NP
     [\textsc{f}:D\\\textsc{l}:none [the]]
     [\textsc{f}:none\\\textsc{l}:NP
       [\textsc{f}:none\\\textsc{l}:N$'$
         [\textsc{f}:none\\\textsc{l}:N [president]]]]]] 
\end{forest}
\caption{\emph{the president} in the DP analysis with functional and lexical features}\label{fig-the-president-DP-f-l}
\end{figure}
The phrase \emph{possibly have indeed been badly formulated} would then have the category
\textsc{f}:Mod\sub{possibility},\textsc{l}:VP[\type{bse}]. This makes it possible to select for the
correct part of speech and the correct verb form by looking at the \textsc{l} value.

So, I -- or rather Klaus Netter -- solved the selection problem for Cartographic approaches. But
note two things: firstly, there is a much simpler -- and I would argue more appropriate -- solution to the goals of Cartography (see
Section~\ref{sec-Cinque-solution}) and secondly, the solution does not solve problems with coordination to be discussed in
more detail in Section~\ref{sec-ConjP}.

As a teaser for the section on corrdination consider the example in (\mex{0}), which shows that verb phrases with different adverbials may be coordinated:
\ea
Kim [unfortunately sang a song] and [allegedly ruined the evening].
\z
According to \citet[\page 106]{Cinque99a-u} \emph{fortunately} is Mood\sub{evaluative} and
\emph{allegedly} is Mood\sub{evidential}. Hence the two verb phrases in (\mex{0}) have different
categories in Cinque's analysis. In the analysis suggested here, they are just verb phrases
containing different adverbials. Since both verb phrases are finite in (\mex{0}) the coordination is
an instance of symmetric coordination in the theory defended here. So, we are back at a point where
Chomsky started: syntax is about distribution of words and phrases, not about semantics. And hence
it is clear that examples like \citegen[\page 15]{Chomsky57a} (\mex{1}) are syntactically wellformed although they do not
make sense:
\ea
Colorless green ideas sleep furiously.
\z
Note that HPSG has semantic
information and information concerning information structure within the information that can be
selected by selecting heads \parencites[Section~2.4]{ps2}[\page 74]{BC2010a}. HPSG categories are complex, they contain phonological, morphological,
syntactic, semantic and information structural properties. This means that relations between
linguistic objects can refer to these properties. This is needed for approaches to language that
take all these descriptive levels into account. What is not needed -- and in fact strongly rejected
here -- is semantic or information structural information within part of speech labels.

%This is a special case of what has been discussed above with respect to NegP.



% 31.03.22
% Man könnte F-AdvP-prf und F-Adv-P-bse für die verschiedenen Zwichenpositionen annehmen. Das wäre
% aber wohl offensichtlicher Irrsin.

For more on \citeauthor{CR2010a} style analyses and the locality of selection see \citew[Section~4.6.1.3]{MuellerGT-Eng4}.


%%%%%%%%%%%%%%%%%%%%%%%%%%%%%%%%%%%%%%%%%%%%%%%%%%%%%%%%%%%%%%%%%%%%%%%%%%%%%%%%%%%%%%%%%%%%%%%%%%%%%%%%%%%%%%%%%%%%%%%%%%%%%%%%
%%%%%%%%%%%%%%%%%%%%%%%%%%%%%%%%%%%%%%%%%%%%%%%%%%%%%%%%%%%%%%%%%%%%%%%%%%%%%%%%%%%%%%%%%%%%%%%%%%%%%%%%%%%%%%%%%%%%%%%%%%%%%%%%


\subsubsection{Coordination and the ConjP analysis}
\label{sec-ConjP}

%%%%%%%%%%%%%%%%%%%%%%%%%%%%%%%%%%%%




%%%%%%%%%%%%%%%%%%%%%%%%%%%%%%%%



Figure~\ref{fig-coordination-conjp} shows a sketch of the analysis in which a ConjP is projected
from the coordinating conjunction. This analysis is assumed by
\citet[\page 596]{Larson90a-u}, \citet[\page 89]{Radford93a-u}, \citet[\page
109]{Johannessen98a-u}, \citet[\page 8]{vanKoppen2005a-u}, \citet[\page 474]{Boskovic2009a-u}, \citet[\page
27]{Citko2011a-u}, \citet[\page 9, 19, 20]{Lohnstein2014a}, and others.
%[Chapter~6]{Kayne94a-u}
\begin{figure}
\begin{forest}
[{ConjP}
 [X]
 [Conj$'$
   [Conj]
   [Y]]]
\end{forest}
\caption{\label{fig-coordination-conjp}Analysis of coordination with ConjP}
\end{figure}
The problem is that the coordination of two NPs should be an NP, the coordination of two VPs a VP
and so on. It should not be a ConjP since a ConjP is different from an NP or VP and selecting heads
are requiring NPs or VPs.\footnote{
  \citet[\page 669]{Johannessen96a-u} suggests an analysis in which a coordinate structure has the
  features of the first conjunct. This does not help since a projection can be either an NP or a
  ConjP. Approaches assuming two categories per head, a functional and a lexical, are discussed
  below. They do not work either. Furthermore, the coordination of two singular NPs is a plural NP not a singular NP as would
  be predicted by \citeauthor{Johannessen96a-u}'s account. See \citet{Borsley2005a} for
  details.
}
\eal
\ex {}[\sub{NP} [\sub{NP} Kim] and [\sub{NP} Sandy]] laugh.
\ex
\label{ex-coordination-NegP}
Kim wants to [\sub{VP} [\sub{VP} sing a song], [\sub{VP} dance], and [\sub{VP} not worry about tomorrow]].
\zl



\citet[\page 122]{Grimshaw2000a} states that symmetric coordination could be handled as extended
projection. She states that in coordinations functional and lexical information has to be identified. But
note that this is actually not easy to establish if the conjunction is seen as a head contributing
its part of speech information to a phrase. Consider the following example:
\ea
the president and the general
\z
If the conjunction is treated as a functional head, the
analysis in Figure~\ref{fig-coordination-of-two-DPs-withConjP} results.
\begin{figure}
\begin{forest}
sm edges
[\textsc{f}:ConjP\\\textsc{l}:NP
  [\textsc{f}:DP\\\textsc{l}:NP
     [\textsc{f}:D$'$\\\textsc{l}:NP
       [\textsc{f}:D\\\textsc{l}:none [the]]
       [\textsc{f}:none\\\textsc{l}:NP
         [\textsc{f}:none\\\textsc{l}:N$'$
           [\textsc{f}:none\\\textsc{l}:N [president]]]]]]
  [\textsc{f}:Conj$'$\\\textsc{l}:NP 
    [\textsc{f}:Conj\\\textsc{l}:none [and]]
    [\textsc{f}:DP\\\textsc{l}:NP
     [\textsc{f}:D$'$\\\textsc{l}:NP
       [\textsc{f}:D\\\textsc{l}:none [the]]
       [\textsc{f}:none\\\textsc{l}:NP
         [\textsc{f}:none\\\textsc{l}:N$'$
           [\textsc{f}:none\\\textsc{l}:N [general]]]]]]]] 
\end{forest}
\caption{Functional and lexical categories and coordination of two functional
  projections}\label{fig-coordination-of-two-DPs-withConjP}
\end{figure}
Now, the interesting thing is that the coordination of two bare NPs without a determiner would have
exactly the same category, which is wrong since they do not have the same distribution:
\eal
\ex[]{
the president and general 
%of Mexico
}
\ex[*]{
a the president and general
}
\zl
Figure~\ref{fig-coordination-of-two-NPs-withConjP} shows the coordination of two NPs.
\begin{figure}
\begin{forest}
sm edges
[\textsc{f}:ConjP\\\textsc{l}:NP
  [\textsc{f}:none\\\textsc{l}:NP
     [\textsc{f}:none\\\textsc{l}:N$'$
     [\textsc{f}:none\\\textsc{l}:N [president]]]]
  [\textsc{f}:Conj$'$\\\textsc{l}:NP 
    [\textsc{f}:Conj\\\textsc{l}:none [and]]
    [\textsc{f}:none\\\textsc{l}:NP
      [\textsc{f}:none\\\textsc{l}:N$'$
        [\textsc{f}:none\\\textsc{l}:N [general]]]]]]
\end{forest}
\caption{Functional and lexical categories and coordination of two lexical
  projections}\label{fig-coordination-of-two-NPs-withConjP}
\end{figure}
The point is clear: a coordination has to reflect both the functional category and the lexical
category of the conjunctions (as \citealt[\page 122]{Grimshaw2000a} pointed out). Making this information
available is impossible if information about a conjunction is projected instead.

Note also that the example in (\ref{ex-coordination-NegP}) shows that the assumption that a negated VP is a VP rather
than a NegP provides for a simple analysis of coordination: it is just VPs that are coordinated and
hence (\ref{ex-coordination-NegP}) is an instance of symmetric coordination. In a NegP approach, more would have to be
said about matching categories in the coordination. And while we want the functional information to
be projected in DP coordination, projecting NegP information in VP/NegP coordination is
counter-productive.\footnote{The difference between DP coordination and NegP/VP coordination is of course that specifiers in the DP/NP are part of the
information that signals completeness of a phrase, while adjuncts are not relevant for this.} 
A traditional, clean syntax seems to be highly preferable here.
See \citew{Borsley2005a} and \citew{BM2021a} for more on the ConjP approach.



The points discussed so far in this subsection on coordination have to do with the status of the
conjunction: Is it a head? What does it project? But there are questions concerning the conjuncts as
well. As was pointed out at the beginning of this section, constituency tests determine distribution
classes. Coordination is one of these tests and if we find two conjoinable constituents this
indicates that they should be assigned to the same category.
\citet[\page 106]{Cinque99a-u} suggests the hierarchy of functional heads already given in (\ref{ex-adjunct-hierarchy}).
Now, consider the sentence in (\mex{1}):
\ea
She [probably goes by train] and [possibly changes in Ostkreuz].
\z
In the theory developed in this book, \emph{probably goes by train} and \emph{possibly changes in
  Ostkreuz} are VPs. They can be coordinated with unmodified VPs or with VPs modified with any of
the adjuncts listed in (\ref{ex-adjunct-hierarchy}). But according to \citeauthor{Cinque99a-u} these phrases
are of different categories: the first VP is Mood\sub{epistemic} and the second is
Mood\sub{possibility}. Does this mean that these VPs cannot be coordinated? No it does not since the
category of the second VP is below the category of the first in Cinque's hierarchy and since he
assumes that all these nodes are present in all sentences even if the respective adverbial elements
are not present. This means that \emph{possibly changes in Ostkreuz} is not just
Mood\sub{possibility} but also Mood\sub{necessity}, Mood\sub{irrealis}, T(Future), T(Past),
Mood\sub{epistemic}. Since it is Mood\sub{epistemic}, it can be coordinated with \emph{probably goes
  by train}, which is Mood\sub{epistemic} as well. But apart from these categories, it is also
Mood\sub{speech act}, Mood\sub{evaluative}, Mood\sub{evidential}. In fact both VPs are. All VPs are
combined with all functional projections in Cinque's system. This means that VPs without adjuncts
are combined with (at least) 27 empty heads and are (at least) 27fold ambiguous as far as their part
of speech label is concerned. When two such VPs are
coordinated there are 27 ways of coordinating them without any difference in meaning that could be
assigned to the different structures. Such ambiguities are calls \emph{spurious ambiguities} and
they are generally frowned upon in syntactic research. See also \citew[\page 70--71]{MuellerGT-Eng1} on the
N–\nbar projection in nominal structures without complements. This projection poses the same problem
with coordinations as the one discussed in this subsection: whenever there are unary projections
that do not add semantics, the result is spurious ambiguities in coordinations.

\subsubsection{Reaching Cartographic's goals}
\label{sec-Cinque-solution}

As was shown in the previous subsections, there are a lot of arguments against
\citeauthor{CR2010a} approaches, but a lot of researchers assume such approaches anyway. Why? The
advantage of these approaches is that they can relate linear order and semantics. As Felix Bildhauer
pointed out to me a long time ago: researchers assuming Cinque-style analyses are entertaining some kind of Construction Grammar
approach. There are slots for certain elements at a certain position in the sentence. We have seen
many instances of these analyses above. Some categories are just unnecessary: SubjP and ObjP, for
example. The effects could be done by adjunction, that is, nodes would be doubled as in
Figure~\ref{fig-cp-tp-vp-scrambling}. Alternatively, one could assume what is called ``base
generation'': rather than moving constituents they are licensed in the places where they are visible. See
\citet{Fanselow2001a} for such an approach within the Minimalist Program.
Furthermore, the information about adverbials could be made accessible
within the semantic contribution of linguistic signs. One representation format of semantics in HPSG
is Minimal Recursion Semantics \citep{CFPS2005a}. All semantic contributions are contained in lists of elementary
predications. Such lists can be augmented by a pointer pointing to the elementary predication that
was added by the last adverbial. Similar pointers are used in MRS already. They are called
\textsc{key} or \textsc{altkey} \parencites[Section~3.7]{FBO2003a-u}[\page 299]{CFPS2005a}. So, the contribution added by an adverbial can be singled out by a
feature called \textsc{modkey}. Since a modifier selects the VP it modifies, it can also access the value of \textsc{modkey} and hence a
modifying adverbial can state constraints on the adverbial that is to the right of it within the VP
it modifies. The selection would refer to semantic properties of the selected linguistic
object. This is, of course, also the case in the Cinque–Rizzi system, except that there the
semantics is pushed into syntax.

\subsubsection{Summary}

I have shown that Cartographic approaches are not compatible with traditional models of syntax since
syntactic categories are mixed with all kinds of information that is not part of syntactic
categories as such. Classical distribution tests fail on Cartography constituents. I have shown that
the normal mechanisms are not sufficient for selection and that the general architecture would have
to be extended. I furthermore showed that this would still not be enough because the proposal fails
to interact properly with approaches of coordination. While there may be ways to fix the remaining
problems, it is clear that the traditional approach to syntax does not have such problems and is
simpler and hence has to be preferred on Occamian grounds.


\exercises{

\begin{enumerate}
\item Provide the valence lists for the following words:

\eal
\ex laugh
\ex eat
\ex to douse
\ex 
\gll bezichtigen\\
     accuse\\\german
\ex he
\ex the
\ex 
\gll Ankunft\\
     arrival\\\german
\zl
If you are uncertain as far as case assignment is concerned, you may use the
  Wiktionary\footnote{
\url{https://de.wiktionary.org/}, 2018-07-02.
}.

\item Draw trees for the NPs that were also used in exercise~\ref{exercise-NP-PSG} on page~\pageref{exercise-NP-PSG} in Chapter~\ref{chap-psg}.
\eal
\ex 
\gll eine Stunde vor der Ankunft des Zuges\\
     one  hour   before the arrival of.the train\\
\glt `one  hour   before the arrival of the train'
\ex 
\gll kurz    nach  der Ankunft in Paris\\
     shortly after the arrival in Paris\\
\glt `shortly after the arrival in Paris'
\ex
\gll das ein Lied singende Kind aus dem Allgäu\\
     this a song  singing child from the Allgäu\\
\glt `the child from the Allgäu singing a song'
\zl


\item Draw trees for the following examples. NPs can be abbreviated.
\eal
\ex\longexampleandlanguage{
\gll weil    Aicke dem        Kind  ein      Buch schenkt\\
     because Aicke the.\DAT{} child a.\ACC{} book gives.as.a.present\\}{German}
\glt `because Aicke gives the child a book as a present'
\ex because Kim gave a book to him
\ex Sandy saw this yesterday.
\ex
\gll at Bjarne læste bogen\\
     that Bjarne read book.\textsc{def}\\\hfill(Danish)
\glt `that Bjarne read the book'
\zl
\end{enumerate}

}




%      <!-- Local IspellDict: en_US-w_accents -->
