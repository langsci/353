%% -*- coding:utf-8 -*-
\chapter{Preface}

This book has two purposes: firstly the comparative analysis of the major syntactic properties of the
\ili{Germanic} languages and secondly the introduction of a specific format for the description and
comparison of languages. The framework in which the analyses are couched is called \emph{HPSG
  light}. It is based on Head-Driven Phrase Structure Grammar (HPSG) \citep{ps,ps2,HPSGHandbook} in the specific version that is described in detail in
\citew{MuellerLehrbuch3}. However HPSG light does not contain any complicated attribute value
matrices (AVMs). If AVMs are used at all, they are reduced to the minimum containing a reduced set
of features like \argst for argument structure, \comps for complements and \spr for specifier. All
other aspects of the analyses are represented in syntactic trees, which are easier to read. The idea
behind the introduction of HPSG light is to provide a tool for linguists who want to provide a
more detailed description of a phenomenon without necessarily being forced to deal with all the
technicalities. The degree of formalization corresponds to what is common in Government and Binding
Theory, Minimalism, and the less formal variants of Construction Grammar. As for the one formal version
of Construction Grammar that is a variant of HPSG, namely Sign-Based Construction Grammar (SBCG, \citealp{Sag2012a}), HPSG
light can be regarded as a light version of SBCG as well, since the differences are neglected in the
abbreviated representations and trees that are used in this book. The work presented here differs
from non-formal work in GB/Minimalism and Construction Grammar in an important way: it is backed up
by implemented grammars that use the full version of HPSG including a semantic analysis in the
framework of Minimal Recursion Semantics (MRS, \citew*{CFPS2005a}). The detailed analyses are described in
conference proceedings, journal articles and books, and the reader is invited to consult these
resources in case she or he is interested in the details. The implemented grammars are distributed
with the Grammix virtual machine \citep{MuellerGrammix} and can be downloaded from the author's
web-page.\footnote{
\url{https://hpsg.hu-berlin.de/Software/Grammix/}, \today.} Grammix contains the grammars for
\ili{German}\footnote{
\url{https://hpsg.hu-berlin.de/Fragments/Berligram/}, \today. The \ili{German} grammar is documented in
\citew{MuellerLehrbuch,MuellerGS} and \citew{MOe2011a,MOe2013a}.
},
\ili{Danish}\footnote{
\url{https://hpsg.hu-berlin.de/Fragments/Danish/}, \today. The \ili{Danish} grammar is documented in
\citew{MOeDanish,MOe2013a,MOe2011a,MOe2013b}. 
}, \ili{English}\footnote{
\url{https://hpsg.hu-berlin.de/Fragments/English/}, \today. The \ili{English} grammar is smaller than the
\ili{German} and \ili{Danish} grammar. It is a proof of concept of a lexicalist analysis of passive, benefactive
constructons, resultative constructions. See \citew{MuellerLFGphrasal} and \citew{MOe2013a} for details.
} and \ili{Yiddish}\footnote{
\url{https://hpsg.hu-berlin.de/Fragments/Yiddish/}, \today. The \ili{Yiddish} grammar is based on
\citew{MOe2011a} and unpublished work by Jong-Bok Kim, Alain Kihm, and me on predicate
topicalization in Korean and \ili{Yiddish} \citep*{MKK2019a}.
} that were developed in the CoreGram project \citep{MuellerCoreGram}. The respective web-pages of the grammars contain a list of test items
that are accepted or rejected by the grammars. Readers are invited to enter these
sentences into the TRALE system \citep{DKMM2004a-u,Penn2004a-u} that comes with Grammix and inspect the complete AVMs.

The book starts with two introductiory chapters: the first chapter introduces the \ili{Germanic} languages
providing basic facts like number of speakers, areas where they are spoken, and some historical
facts. Chapter two discusses the phenomena that are treated in the rest of the book, \eg scrambling,
placement of adverbials, passive, clause types, nonlocal dependencies. The third chapter is an
introduction to Phrase Structure Grammars, which are the foundations of almost all theories since
\citegen{Chomsky57a} formalization of structuralist ideas
\citep{Bloomfield33a-u}. Chapter~\ref{chap-psg-xbar} introduces not just phrase structure grammars
but also grammars using abstractions over phrase structure rules, ultimately resulting in very
abstract grammars of the type also known from \xbart
\citep{Jackendoff77a}. Chapter~\ref{chap-valence} explains how the concept of valence is combined
with abstract phrase structure rules to make sure that the right number and the right kind of
elements is combined with a certain word. For example, a word like \emph{laugh} needs a subject and
a word like \emph{read} needs a subject and an object. This has to be represented somewhere in a
grammar, and Chapter~\ref{chap-valence} explains how it is done in HPSG (light). The basic differences
in the analyses of SVO and SOV languages are explained. This chapter also
explains how the various orders of subject and objects can be explained in a language like \ili{German}
(so-called \emph{scrambling}) and how one can account for the various placement possibilities in languages like \ili{English}
and the North \ili{Germanic} languages on the one hand, and \ili{German}, \ili{Dutch} and \ili{Afrikaans} on the other hand. Verbal complexes are dealt with in Chapter~\ref{chap-verbal-complex}, verb"=first position (used for question formation) and verb"=second position (for assertions) are explained
in Chapter~\ref{chap-verb-position}. Passive and case assignment in general are treated in
Chapter~\ref{chap-case}. The \ili{Germanic} languages are especially interesting here as \ili{Icelandic}
belongs to this language group and is known for its quirky subjects (subjects in the genitive, dative, or
accusative case, \citealp{ZMT85a}). Chapter~\ref{chap-expletives} deals with expletive pronouns and how they are used
throughout the \ili{Germanic} languages to help mark clause types. For example, expletives are used in
\ili{German} main clauses to fill the initial position so that the clause is an assertion. \ili{Danish} uses
expletives in embedded sentences with subjects as interrogative elements. Again the differences in
general grammatical properties influence the grammar in other parts such as the placements of expletives. 

% The final chapter, Chapter~\ref{chap-HPSG-light}, is for advanced readers. It tries to relate the simplified version
% of HPSG that was used throughout the earlier chapters to HPSG as it is used in full-fledged HPSG publications.

The final chapter, Chapter~\ref{chap-outlook}, is a brief summary of what was done in the book and
points the interested reader to some further literature on HPSG.

\ili{German} slides developed for the course I am teaching with this book are available on GitHub.\footnote{
\url{https://github.com/stefan11/germanic-syntax-slides}, 2021-09-14.
}
Lectures in \ili{German} corresponding to the chapters can also be found on YouTube.\footnote{
\url{https://www.youtube.com/playlist?list=PLXwGGsuPxWRp4AB2LsWH6LKc0II7uc6tg}
}

\section*{On the way this book is published}

Teachers at schools and at many universities are paid by the state, that is by the public (you). Among
their duties is the creation of teaching material. There is no reason whatsoever to leave the
teaching material to profit"=oriented publishers. On the contrary, teaching material should be open
and adaptable to the needs of the teachers who want to use it. 

A study by the American Enterprise Institute shows that the price of college books rose by 812\,\%
from 1978 to 2012 while the general consumer prices rose a mere 250\,\%.\footnote{
\url{https://www.aei.org/carpe-diem/the-college-textbook-bubble-and-how-the-open-educational-resources-movement-is-going-up-against-the-textbook-cartel/}.
2022-12-22.%
} Similar figures exist for scientific books in general and for university textbooks. My favorite example is a thin textbook
on logic \emph{Logik für Linguisten}, which is a translation of the \ili{English} textbook \emph{Logic for
Linguists} \citep{AAD73a}. This book has 112 pages. It was sold for 9,40€ as a paperback by the Max Niemeyer
Verlag. This publisher was bought by De Gruyter and the book is now sold for \$126.00/89,95€ as an
eBook and \$133,00/94,95€ for the hardcover book\footnote{
%  \url{http://www.degruyter.com/isbn/978-3-11-096350-2}. 2014-09-01.
I noticed in 2022 that De Gruyter stopped offering this book. I think this is even worse.
} (see \citealp{MuellerOA} for other examples and a general discussion). Both the eBook and the printed book are unaffordable for students. The way out of this highly
problematic situation is to publish books in open access. The PDF version of this book is free for
everybody and the printed copy is available for a reasonable price since the book is licenced under
a Creative Commons license and hence is not owned by a
profit"=oriented publisher and everybody can choose his or her own print on demand service in case
the default service provided by Language Science Press is more expensive.

\if0
\section*{Gender issues}

\itdopt{M: drop section}

\citet{MB97a} did a study on ten textbooks in syntax published between 1969 and 1994. They showed
that some of the textbooks used examples describing violence against women. Examples like \emph{John
  beats Mary.} or \emph{John hits Mary.} were frequent in the 1970ies and 1980ies. I know of a paper
written by two female authors and one male author containing a \emph{John hits Mary} example. \citet[\page
812]{MB97a} discus some even more extreme cases from the textbooks they examined. Furthermore, females were
depicted in stereotypical situations like teaching children, never in work situations with the
exception of work as secretaries. Some examples were explicitly sexist, referring to women as stupid
and to men as intelligent. 


\citeauthor{MB97a} examined the semantic roles in which women and men appeared, checked for
usage of full NPs, proper names and pronouns. They found that men appear more often than women
in example sentences and they appear more often as agents then as experiencers/patients. While
openly sexist examples disappeared from textbooks in the time since the 80ies, \citet{PCKSDMC2017a}
repeated the studies of \citeauthor{MB97a} and noticed that not much has changed in respect to the
women/men distribution in examples. 

I think that such biased examples contribute to the stereotypes regarding genders that most of us
have since they are deeply entrenched in our societies. Mikaela Wapman and Deborah Belle did a study
asking students of psychology about the following situation: a father and son are in a horrible car
crash and the father gets killed. The son is brought to the hospital and has to undergo an emergency
surgery but the surgeon refuses to do it and say, ``I can’t operate—that boy is my son!''. Question:
How is this possible? People came up with all sorts of explanations like gay fathers, ghost surgeons
and so on, overlooking the possibility of the surgeon being the mother of the child. This shows that
the idea that surgeons are male is deeply entrenched in our societies (the same experiment works in
\ili{German} and it works in the other direction with nurses). The obvious solution to such problems is to
change the employment of women and to change the payment of jobs usually taken by women. This is a
political problem and it would be helpful if women were in at least 50\,\% of the positions in
political systems. Many of the problems that women have can be traced back to a lack of economic
independence. There is a good portray of East \ili{German} women showing the consequences of economic
independence: \url{https://www.mdr.de/zeitreise/schwerpunkte/ostfrauen-106.html}. When the wall came
down and East \ili{German} women and West \ili{German} women united there was much confusion since the issues
they wanted to address were totally different (see also the MDR video documentation). Almost all women in
East Germany worked, child care was available everywhere and so on.

What we seem to have here is an hen--egg problem: society and stereotypes would be different if women
would be in other jobs and better payed, but stereotypes prevent women from even trying because they
feel not welcome in certain jobs. Here is where language plays an important role. \ili{German} marks
gender for professions: one can talk about \emph{Kindergärtner} `male nurse' and
\emph{Kindergärtnerinnen} `female nurse'. Looking at the \ili{German} grammatical system, the plural is unmarked for
gender, but there is the form \emph{Kindergärtnerinnen} that makes gender explicit. Studies show
that recipients consider women less often, if the unmarked form is used. For instance, \citet{SSB2001a}
showed that people come up with more names of male musicians, when asked for their favorite \emph{Musiker} rather than
their favorite \emph{Musiker} `musician' or \emph{Musikerin} `female musician'. One proposal solved this
problem by using abbreviations for fusing both variants: \emph{MusikerIn} `male or female musician'. Further more inclusive
forms have been developed including people with diverse genders: \emph{Musiker*in}. These forms are
standard in the communication of the \ili{German} Research Foundation and some universities (\eg the
Humboldt-Universität zu Berlin, where I am currently working).    

While the so-called \emph{Binnen-I} helps to make women visible in communication this does not help
as far as textbooks are concerned since respective forms are rather rarely used in examples. So, to
make non-male readers feel more welcome, other means have to be employed. One has to work on the
referents that are used in examples. The names used most frequently in examples in the papers/textbooks from which I was learning syntax
(written in the 80ies and 90ies) were \emph{John} and \emph{Mary} or \emph{Karl} and \emph{Maria} in
\ili{German} publications. (Books on HPSG using gender neutral names like \emph{Kim} and \emph{Sandy} were an early exception \citep{ps,ps2}.)
I have never used discriminating examples in my publications and I avoided stereotypes, but I also used
\emph{Karl} and \emph{Maria} and I had \emph{den Mann} `the man' reading books and \emph{der Mann}
`the man' giving books to \emph{der Frau} `the woman' or \emph{dem Kind} (neuter, `the child'). One
reason for having more male referents in \ili{German} examples is that the masculine inflection paradigm
differentiates between all four cases while the feminine one collapses nominative and accusative as
well as genitive and dative. The neuter is somewhere in the middle: nominative and accusative are syncretic. Table~\ref{table-German-case-syncretism} gives an overview of the situation.
\begin{table}
\begin{tabular}{llll}\lsptoprule
           & the man    & the woman & the child\\\midrule
nominative & der Mann   & die Frau  & das Kind\\
genitive   & des Mannes & der Frau  & des Kindes\\
dative     & dem Mann   & der Frau  & dem Kind\\
accusative & den Mann   & die Frau  & das Kind\\
\lspbottomrule
\end{tabular}
\caption{\label{table-German-case-syncretism}Inflectional marking within noun phrases of feminine and masculine gender. Only the masculine
  paradigm is unambiguously marked for case.}
\end{table}  
If one wants to avoid male referents, one could use animals, but depending on the verb one
uses, a dolphin as subject would not make sense. So, one way to solve this problem, is to use proper
names like \emph{Kim}, \emph{Sandy}, and \emph{Chris} that are unisex. This is of course only
possible if case is unimportant since proper names do not inflect for case in almost all of the \ili{Germanic}
languages (\ili{Icelandic} being the exception, \citealp[\page 443]{ZMT85a}). 

As mentioned above, there is some tradition of what names are used in linguistic examples: for
\ili{English} it was \emph{John} and \emph{Mary}, for \ili{German} \emph{Karl} and \emph{Maria}, \ili{Dutch} authors
used \emph{Jan} and \emph{Piet} and \emph{Marie}. Since this book is about \ili{Germanic} languages I
thought I collect names that are typical for the languages under discussion. It is clear that the
names trick does not work globally. I learned that \emph{Gert} can be used both for women and
men in the \ili{Scandinavian} countries, but in Germany this name exclusively refers to
males.\footnote{%
See Bernadette La Hengst's song \emph{Ein Mädchen namens Gerd} `A girl named Gerd', which is a parallel to \emph{A boy
  named Sue} by Johnny Cash: \url{https://www.youtube.com/watch?v=Y6WVcOk4D3c}, 2020-06-26.
} In
addition speaker communities may associate certain names with women and girls since there are more females
than males with this name in a certain community and vice versa. And finally there are differences
on the individual level: you may happen to know three female Connys and no male one but this may
change during the course of your live. So an example about Conny reading a book or feeding a child
could work for or against a certain stereotype, depending on the experiences of the reader. And of course the assignment to a certain gender may change in
time: a name that is given more often to girls may be given more often to boys some decades later. 
So, what I did is not perfect but since I state here that all names are meant to be unisex including
non-binary people, it is a step into the right direction.

In a discussion on twitter, XY pointed out to me that names do not necessarily fit the gender of the
referent. There are famous examples of a male US American lawyer with the name Sue\footnote{
\url{https://en.wikipedia.org/wiki/Sue_K._Hicks}, 2020-07-06.
%\emph{Johnny Cash Is Indebted to a Judge Named Sue}. The New York Times, July 12, 1970, p.\ 66.
} and of women with the name Michael (the columnist Michael Sneed, Chicago Sun-Times, and The
Bangles' bassist Michael Steele).\footnote{
  \url{https://en.wikipedia.org/wiki/Michael_Burnham}, 2020-07-06.
}
So in principle
one would have to make the gender of the referent explicit. But this is exactly what I want to avoid
by using unisex names: I do not want to entrench stereotypes. By not using \emph{Sue}, I do not have to say
that Sue may be a man.

A special case are non-binary people. They may not feel represented in examples with unisex
names like \emph{Kim} and \emph{Sandy}. This is the reason why they sometimes choose names that are
not used by binary people. Kirby Conrod suggested that non-binary people give permission to use their names and allowed me to
use theirs (Kirby's), which I do at some places in the book. 
% https://nonbinary.miraheze.org/wiki/Names

%Kim Gerdes (\ili{German} linguist), Kim Wilde (\ili{British} pop singer).

Table~\ref{table-given-names-used-in-the-book} shows the names I am using throughout the book.

\begin{table}
\begin{tabular}{llp{4cm}p{4cm}}\lsptoprule
language    & name      & female basis         & male basis\\\midrule
\ili{Danish}      & Gert/Gerd & Asgjerd, Hallgerd, Hildegard, Ingjerd        & Gerhard\\
\ili{Dutch}       & Robin     &                      & Robrecht, Robert\\ %https://nl.wikipedia.org/wiki/Robin_(voornaam)
            & Benedikt \\
\ili{German}      & Conny     & Constanze, Cornelia  & Conrad, Constantin\\
            & Aicke     & & Eckehard \\
\ili{English}     & Chris     & Christina, Christine & Christian, Christoph(er)\\
            & Kim       & Kimberly             & Kimberly, Kimball,\newline Joachim, Joakim\\
            & Sandy     & Sandra               & Alexander, Sander, Alasdair\\
non-binary  & Kirby\\
            & Elvan\\% from Wiki
\lspbottomrule
\end{tabular}
\caption{\label{table-given-names-used-in-the-book}This table lists the names used in the book for
  the languages given}
\end{table}
%\end{sideways}

\fi

~\medskip

\noindent
Berlin, 21st April 2023\hfill Stefan Müller
%Berlin, \today\hfill Stefan Müller




%      <!-- Local IspellDict: en_US-w_accents -->
