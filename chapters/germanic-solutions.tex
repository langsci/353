%% -*- coding:utf-8 -*-
\chapter{Solutions}

\section{Phrase structure grammars and \xbart}

\begin{enumerate}
\item Draw trees for the following phrases. You may use the symbol NP for proper names and \nbar for
  nouns not requiring complements (as in Figure~\ref{Abb-das-schoene-Bild-von-Paris-HPSG}).
\eal
\ex 
\gll eine Stunde vor der Ankunft des Zuges\\
     one  hour   before the arrival of.the train\\
\glt `one  hour   before the arrival of the train'
\ex 
\gll kurz    nach  der Ankunft in Paris\\
     shortly after the arrival in Paris\\
\glt `shortly after the arrival in Paris'
\ex
\gll das ein Lied singende Kind aus dem Allgäu\\
     this a song  singing child from the Allgäu\\
\glt `the child from the Allgäu singing a song'
\zl

\begin{figure}
\begin{forest}
sm edges
[PP
  [NP
    [Det   [eine;one]]
    [\nbar [Stunde;hour]]]
  [\pbar
    [P [vor;before] ]
    [NP 
      [Det [der;the]]
      [\nbar
        [N [Ankunft;arrival]]
        [NP 
          [Det [des;the]]
          [N   [Zuges;train]]]]]]]
\end{forest}
\caption{Analysis of \emph{eine Stunde vor der Ankunft des Zuges} `one hour before the arrival of
  the train'}
\end{figure}

\begin{figure}
\begin{forest}
sm edges
[PP
  [AP
    [kurz;shortly]]
  [\pbar
    [P [nach;after] ]
    [NP 
      [Det [der;the]]
      [\nbar
        [\nbar [Ankunft;arrival]]
        [PP
          [\pbar 
            [P [in;in]]
            [NP [Paris;Paris]]]]]]]]
\end{forest}
\caption{Analysis of \emph{kurz nach der Ankunft in Paris} `shortly after the arrival in Paris'}
\end{figure}

\begin{figure}
\begin{forest}
sm edges
[NP
  [Det [das;the]]
  [\nbar
    [\nbar
      [AP
        [\abar
          [NP 
            [Det [ein;a]]
            [\nbar [Lied;song]]]
          [A [singende;singing]]]]
      [\nbar [Kind;child]]]
    [PP
      [\pbar
        [P [aus;from]]
        [NP
          [Det [dem;the]]
          [\nbar [Allgäu;Allgäu]]]]]]]
\end{forest}
\caption{Analysis of \emph{das ein Lied singende Kind aus dem Allgäu} `the child from the Allgäu
  singing a song'}
\end{figure}

\end{enumerate}
\clearpage

\section{Valency, argument order and adjunct placement}

\settowidth\jamwidth{(German)}
\begin{enumerate}
\item Provide the valence lists for the following words:

\ea
\begin{tabular}[t]{@{}l@{~}ll@{\hspace{2em}}r@{}}
a. & laugh      & \spr \sliste{ NP[\type{nom}] }\\
b. & eat        & \spr \sliste{ NP[\type{nom}] }, \comps \sliste{ NP[\type{acc}] }\\
c. & to douse   & \spr \sliste{ NP[\type{nom}] }, \comps \sliste{ NP[\type{acc}] }\\
d. & bezichtigen& \spr \eliste, \comps \sliste{ NP[\type{nom}], NP[\type{gen}] } &(German)\\
   & accuse\\ 
e. & he  & \spr \eliste, \comps \eliste\\
f. & the & \spr \eliste, \comps \eliste\\
g. & Ankunft & \spr \sliste{ Det }, \comps \sliste{ NP } & (German)\\ 
   & arrival & \spr \sliste{ Det }, \comps \eliste\\
\end{tabular}
\z
If you are uncertain as far as case assignment is concerned, you may use the
  Wiktionary: \url{https://de.wiktionary.org/}.

\item Draw trees for the NPs that were also used in Exercise~\ref{exercise-NP-PSG} on page~\pageref{exercise-NP-PSG} in Chapter~\ref{chap-psg}.
\eal
\ex 
\gll eine Stunde vor der Ankunft des Zuges\\
     one  hour   before the arrival of.the train\\
\glt `one  hour   before the arrival of the train'
\ex 
\gll kurz    nach  der Ankunft in Paris\\
     shortly after the arrival in Paris\\
\glt `shortly after the arrival in Paris'
\ex
\gll das ein Lied singende Kind aus dem Allgäu\\
     this a song  singing child from the Allgäu\\
\glt `the child from the Allgäu singing a song'
\zl

\begin{figure}
\oneline{%
\begin{forest}
sm edges
[P\feattab{\spr   \eliste,\\
           \comps \eliste}
  [\ibox{1} N\feattab{\spr   \eliste,\\
                      \comps \eliste}
    [\ibox{2} Det   [eine;one]]
    [N\feattab{\spr   \sliste{ \ibox{2} },\\
               \comps \eliste} [Stunde;hour]]]
  [P\feattab{\spr   \sliste{ \ibox{1} },\\
             \comps \eliste}
    [P\feattab{\spr   \sliste{ \ibox{1} },\\
               \comps \sliste{ \ibox{3} }} [vor;before] ]
    [\ibox{3} N\feattab{\spr   \eliste,\\
                         \comps \eliste}
      [\ibox{4} Det [der;the]]
      [N\feattab{\spr   \sliste{ \ibox{4} },\\
                 \comps \eliste}
        [N\feattab{\spr   \sliste{ \ibox{4} },\\
                   \comps \sliste{ \ibox{5} }} [Ankunft;arrival]]
        [\ibox{5} N\feattab{\spr   \eliste,\\
                            \comps \eliste} 
          [\ibox{6} Det [des;the]]
          [N\feattab{\spr   \sliste{ \ibox{6} },\\
                     \comps \eliste}   [Zuges;train]]]]]]]
\end{forest}}
\caption{Analysis of \emph{eine Stunde vor der Ankunft des Zuges} `one hour before the arrival of
  the train'}
\end{figure}

\begin{figure}
\begin{forest}
sm edges
[P\feattab{\spr   \eliste,\\
           \comps \eliste}
  [\ibox{1} A\feattab{\spr   \eliste,\\
                      \comps \eliste}
    [kurz;shortly]]
  [P\feattab{\spr   \sliste{ \ibox{1} },\\
             \comps \eliste}
    [P\feattab{\spr   \sliste{ \ibox{1} },\\
               \comps \sliste{ \ibox{2} }} [nach;after] ]
    [\ibox{2} N\feattab{\spr   \eliste,\\
                        \comps \eliste}
      [\ibox{3} Det [der;the]]
      [N\feattab{\spr   \sliste{ \ibox{3} },\\
                 \comps \eliste}
        [\ibox{4} N\feattab{\spr   \sliste{ \ibox{3} },\\
                            \comps \eliste} [Ankunft;arrival]]
        [P\feattab{\textsc{mod} \ibox{4},\\
                   \spr   \eliste,\\
                   \comps \eliste}
          [P\feattab{\textsc{mod} \ibox{4},\\
                     \spr   \eliste,\\
                     \comps \sliste{ \ibox{5} }} [in;in]]
          [\ibox{5} N\feattab{\spr   \eliste,\\
                     \comps \eliste} [Paris;Paris]]]]]]]
\end{forest}
\caption{Analysis of \emph{kurz nach der Ankunft in Paris} `shortly after the arrival in Paris'}
\end{figure}

\begin{figure}
\oneline{%
\begin{forest}
sm edges
[N\feattab{\spr   \eliste,\\
           \comps \eliste}
  [\ibox{1} Det [das;the]]
  [N\feattab{\spr   \sliste{ \ibox{1} },\\
             \comps \eliste}
    [\ibox{2} N\feattab{\spr   \sliste{ \ibox{1} },\\
               \comps \eliste}
      [A\feattab{\spr   \eliste,\\
                 \comps \eliste}, tier=3
        [\ibox{4} N\feattab{\spr   \eliste,\\
                   \comps \eliste}
          [\ibox{6} Det [ein;a]]
          [N\feattab{\spr   \sliste{ \ibox{6} },\\
                     \comps \eliste} [Lied;song]]]
        [A\feattab{\spr   \eliste{ },\\
                   \comps \sliste{ \ibox{4} }} [singende;singing]]]
      [N\feattab{\spr   \sliste{ \ibox{1} },\\
                 \comps \eliste}, tier=3 [Kind;child]]]
    [P\feattab{\textsc{mod} \ibox{2},\\
               \spr   \eliste,\\
               \comps \eliste}
      [P\feattab{\textsc{mod} \ibox{2},\\
                 \spr   \eliste,\\
                 \comps \sliste{ \ibox{3} }},tier=3 [aus;from]]
      [\ibox{3} N\feattab{\spr   \eliste,\\
                          \comps \eliste},tier=3
        [\ibox{5} Det [dem;the]]
        [N\feattab{\spr   \sliste{ \ibox{5} },\\
                   \comps \eliste} [Allgäu;Allgäu]]]]]]]
\end{forest}}
\caption{Analysis of \emph{das ein Lied singende Kind aus dem Allgäu} `the child from the Allgäu
  singing a song'}
\end{figure}


\item Draw trees for the following examples. NPs can be abbreviated.

\eal
\ex 
\gll weil    Aicke dem        Kind  ein      Buch schenkt\\
     because Aicke the.\DAT{} child a.\ACC{} book gives.as.a.present\\\hspace{-2cm} \german
\glt `because Aicke gives the child a book as a present'
\ex
\gll weil    dem        Kind  solch ein      Buch niemand       schenkt\\
     because the.\DAT{} child such  a.\ACC{} book nobody.\NOM{} gives.as.a.present\\
\glt `because nobody gives the child such a book as a present'
\ex because Kim gave a book to him
\ex Sandy saw this yesterday.
\ex
\gll at Bjarne læste bogen\\
     that Bjarne read book.\textsc{def}\\\danish
\glt `that Bjarne read the book'
\zl


The trees with the solutions are given in the
following. Figure~\ref{fig-weil-dem-Kind-solch-ein-buch-niemand-schenkt} differs from
Figure~\ref{fig-weil-Aicke-dem-Kind-ein-buch-schenkt} in the way the elements in the \compsl are
numbered, but in each case the order of the elements in the \compsl of \emph{schenkt} `gives as a
present' is \sliste{ \npnom, \npdat, \npacc }. The different numbering is due to the order in which
the elements are combined. If the numbering is done consistently from top to bottom,
Figure~\ref{fig-weil-dem-Kind-solch-ein-buch-niemand-schenkt} is the result. If one is more liberal
in the way the numbers are assigned, the same situation can be depicted as in
Figure~\ref{fig-weil-dem-Kind-solch-ein-buch-niemand-schenkt-different-numbering}. Figure~\ref{fig-weil-dem-Kind-solch-ein-buch-niemand-schenkt-different-numbering}
has the same numbering in the valence list as Figure~\ref{fig-weil-Aicke-dem-Kind-ein-buch-schenkt}
and maybe easier to grasp because of this.

\begin{figure}
\centerfit{%
\begin{forest}
sm edges
[{C[\spr \eliste, \comps \eliste]}
  [{C[\spr \eliste, \comps \sliste{ \ibox{1} }]} [weil;because]]
  [{\ibox{1} V[\spr \eliste, \comps \eliste]}
      [{\ibox{2} NP[\type{nom}]} [Aicke;Aicke] ]
      [V\feattab{
            \spr \eliste, \comps \sliste{ \ibox{2} }}
        [{\ibox{3} NP[\type{dat}]} [dem Kind;the child,roof] ] 
        [V\feattab{
              \spr \eliste, \comps \sliste{ \ibox{2}, \ibox{3} }}
          [{\ibox{4} NP[\type{acc}]} [ein Buch;a book,roof] ] 
          [V\feattab{
              \spr \sliste{  },\\
              \comps \sliste{ \ibox{2}, \ibox{3}, \ibox{4} }} [schenkt;gives] ]]
] ] ]
\end{forest}}
\caption{\label{fig-weil-Aicke-dem-Kind-ein-buch-schenkt}The analysis of \emph{weil Aicke dem Kind
    ein Buch schenkt} `because Aicke gives the child a book as a present'}
\end{figure}

\begin{figure}
\centerfit{%
\begin{forest}
sm edges
[{C[\spr \eliste, \comps \eliste]}
  [{C[\spr \eliste, \comps \sliste{ \ibox{1} }]} [weil;because]]
  [{\ibox{1} V[\spr \eliste, \comps \eliste]}
      [{\ibox{2} NP[\type{acc}]} [solch ein Buch;such a book,roof] ] 
      [V\feattab{
            \spr \eliste, \comps \sliste{ \ibox{2} }}
        [{\ibox{3} NP[\type{dat}]} [dem Kind;the child,roof] ] 
        [V\feattab{
              \spr \eliste, \comps \sliste{ \ibox{3}, \ibox{2} }}
          [{\ibox{4} NP[\type{nom}]} [niemand;nobody] ]
          [V\feattab{
              \spr \sliste{  },\\
              \comps \sliste{ \ibox{4}, \ibox{3}, \ibox{2} }} [schenkt;gives] ]]
] ] ]
\end{forest}}
\caption{\label{fig-weil-dem-Kind-solch-ein-buch-niemand-schenkt}The analysis of \emph{weil dem Kind
    solch ein Buch niemand schenkt} `because nobody gives the child such a book as a present'}
\end{figure}

\begin{figure}
\centerfit{%
\begin{forest}
sm edges
[{C[\spr \eliste, \comps \eliste]}
  [{C[\spr \eliste, \comps \sliste{ \ibox{1} }]} [weil;because]]
  [{\ibox{1} V[\spr \eliste, \comps \eliste]}
      [{\ibox{4} NP[\type{acc}]} [solch ein Buch;such a book,roof] ] 
      [V\feattab{
            \spr \eliste, \comps \sliste{ \ibox{4} }}
        [{\ibox{3} NP[\type{dat}]} [dem Kind;the child,roof] ] 
        [V\feattab{
              \spr \eliste, \comps \sliste{ \ibox{3}, \ibox{4} }}
          [{\ibox{2} NP[\type{nom}]} [niemand;nobody] ]
          [V\feattab{
              \spr \sliste{  },\\
              \comps \sliste{ \ibox{2}, \ibox{3}, \ibox{4} }} [schenkt;gives] ]]
] ] ]
\end{forest}}
\caption{\label{fig-weil-dem-Kind-solch-ein-buch-niemand-schenkt-different-numbering}The analysis of \emph{weil dem Kind
    solch ein Buch niemand schenkt} `because nobody gives the child such a book as a present'}
\end{figure}


\begin{figure}
\centerfit{%
\begin{forest}
sm edges
[{C[\spr \eliste, \comps \eliste]}
  [{C[\spr \eliste, \comps \sliste{ \ibox{1} }]} [because]]
  [{\ibox{1} V[\spr \eliste, \comps \eliste]}
      [{\ibox{2} NP[\type{nom}]} [Kim] ]
      [V\feattab{
            \spr \sliste{ \ibox{2} }, \comps \eliste}
        [V\feattab{
                \spr \sliste{ \ibox{2} }, \comps \sliste{ \ibox{3} }}
           [V\feattab{
              \spr \sliste{ \ibox{2} },\\
              \comps \sliste{ \ibox{4}, \ibox{3} }} [gave] ]
            [{\ibox{4} NP[\type{acc}]} [a book, roof] ] ] 
        [{\ibox{3} PP[\type{to}]} [to him,roof] ] ] ] ]
\end{forest}}
\caption{\label{fig-because-Kim-gave-the-book-to-him}The analysis of \emph{because Kim gave a book to him}}
\end{figure}

\begin{figure}
\centerfit{%
\begin{forest}
sm edges
[{V[\spr \eliste, \comps \eliste]}
   [{\ibox{1} NP[\type{nom}]} [Sandy] ]
   [V\feattab{
      \spr \sliste{ \ibox{1} }, \comps \eliste}
     [\ibox{2} V\feattab{
        \spr \sliste{ \ibox{1} }, \comps \eliste}
       [V\feattab{
           \spr \sliste{ \ibox{1} },\\
           \comps \sliste{ \ibox{3} }} [saw] ]
          [{\ibox{3} NP[\type{acc}]} [this] ] ]
     [{Adv[\textsc{mod} \ibox{2} VP]} [yesterday] ] ] ]
\end{forest}}
\caption{\label{fig-Sandy-saw-this-yesterday}Analysis of \emph{Sandy saw this yesterday.}}
\end{figure}

\begin{figure}
\centerfit{%
\begin{forest}
sm edges
[{C[\spr \eliste, \comps \eliste]}
  [{C[\spr \eliste, \comps \sliste{ \ibox{1} }]} [at;that]]
  [{\ibox{1} V[\spr \eliste, \comps \eliste]}
    [{\ibox{2} NP[\type{nom}]} [Bjarne;Bjarne] ]
    [V\feattab{
       \spr \sliste{ \ibox{2} }, \comps \eliste}
       [V\feattab{
           \spr \sliste{ \ibox{2} },\\
           \comps \sliste{ \ibox{3} }} [læste;read] ]
          [{\ibox{3} NP[\type{acc}]} [bogen;book.\textsc{def}] ] ] ] ]
\end{forest}}
\caption{\label{fig-at-bjarne-laeste-bogen}Analysis of \emph{at Bjarne læste bogen} `that Bjarne
  read the book'}
\end{figure}



\end{enumerate}

\clearpage

\section{The verbal complex}

\begin{enumerate}
\item Sketch the analysis of the verbal complexes in the following examples:
\eal
\ex
\gll dass sie darüber lachen muss\\
     that she there.about laugh must\\\german
\glt `that she has to laugh about it' 
\ex 
\gll dass sie darüber hat lachen müssen\\
     that she there.about has laugh  must\\
\glt `that she had to laugh about it' 
\ex 
\gll dass sie darüber     wird haben lachen müssen\\
     that she there.about will have  laugh  must\\
\glt `that it will be the case that she had to laugh about it'
\zl
You may omit the \spr values, since they are the empty list for all German verbs anyway.


\begin{figure}
\begin{forest}
sm edges
[C\feattab{%\spr \eliste,\\
             \comps \eliste}
  [C\feattab{%\spr \eliste,\\
             \comps \sliste{ \ibox{1} }} [dass;that]]
  [{\ibox{1} V\feattab{%\spr \eliste,\\
                       \comps \eliste}}
     [{\ibox{2} NP[nom]} [sie;she]]
     [V\feattab{%\spr  \eliste,\\
                \comps \sliste{ \ibox{2} }}
       [\ibox{3} PP [darüber;there.about]]
       [V\feattab{%\spr \eliste,\\
                  \comps \sliste{ \ibox{2}, \ibox{3} }}
         [\ibox{4} V\feattab{\subj \sliste{ \ibox{2} },\\
                             %\spr \eliste,\\
                             \comps \sliste{ \ibox{3} }} [lachen;laugh]]
         [V\feattab{%\spr \eliste,\\
                    \comps \sliste{ \ibox{2}, \ibox{3}, \ibox{4} }} [muss;must]]]
]]]
\end{forest}
\caption{Analysis of \emph{dass sie darüber lachen muss} `that she has to laugh about this'}
\end{figure}


\begin{figure}
\oneline{%
\begin{forest}
sm edges
[C\feattab{%\spr \eliste,\\
             \comps \eliste}
  [C\feattab{%\spr \eliste,\\
             \comps \sliste{ \ibox{1} }} [dass;that]]
  [{\ibox{1} V\feattab{%\spr \eliste,\\
                       \comps \eliste}}
     [{\ibox{2} NP[nom]} [sie;she]]
     [V\feattab{%\spr  \eliste,\\
                 \comps \sliste{ \ibox{2} }}
       [\ibox{3} PP [darüber;there.about]]
       [V\feattab{%\spr \eliste,\\
                  \comps \sliste{ \ibox{2}, \ibox{3} }}
         [V\feattab{%\spr \eliste,\\
                    \comps \sliste{ \ibox{2}, \ibox{3}, \ibox{4} }} [hat;has]]
           [\ibox{4} V\feattab{\subj \sliste{ \ibox{2} },\\
                               %\spr \eliste,\\
                               \comps \sliste{ \ibox{3} }}
             [\ibox{5} V\feattab{\subj \sliste{ \ibox{2} },\\
                                 %\spr \eliste,\\
                                 \comps \sliste{ \ibox{3} }} [lachen;laugh]]
             [V\feattab{\subj \sliste{ \ibox{2} },\\
                        %\spr \eliste,\\
                        \comps \sliste{ \ibox{3}, \ibox{5} }} [müssen;must]]]]]]]
\end{forest}}
\caption{Analysis of \emph{dass sie darüber hat lachen müssen} `that she had to laugh about this'}
\end{figure}


\begin{figure}
\oneline{%
\begin{forest}
sm edges
[C\feattab{%\spr \eliste,\\
             \comps \eliste}
  [C\feattab{%\spr \eliste,\\
             \comps \sliste{ \ibox{1} }} [dass;that]]
  [{\ibox{1} V\feattab{%\spr \eliste,\\
                       \comps \eliste}}
     [{\ibox{2} NP[nom]} [sie;she]]
     [V\feattab{%\spr  \eliste,\\
                 \comps \sliste{ \ibox{2} }}
       [\ibox{3} PP [darüber;there.about]]
       [V\feattab{%\spr \eliste,\\
                  \comps \sliste{ \ibox{2}, \ibox{3} }}
         [V\feattab{%\spr \eliste,\\
                    \comps \sliste{ \ibox{2}, \ibox{3}, \ibox{4} }} [wird;will]]
           [\ibox{4} V\feattab{\subj \sliste{ \ibox{2} },\\
                              %\spr \eliste,\\
                               \comps \sliste{ \ibox{3} }}
             [V\feattab{\subj \sliste{ \ibox{2} },\\
                       %\spr \eliste,\\
                        \comps \sliste{ \ibox{3}, \ibox{5} }} [haben;have]]
             [\ibox{5} V\feattab{\subj \sliste{ \ibox{2} },\\
                                 %\spr \eliste,\\
                                 \comps \sliste{ \ibox{3} }}
               [\ibox{6} V\feattab{\subj \sliste{ \ibox{2} },\\
                                   %\spr \eliste,\\
                                   \comps \sliste{ \ibox{3} }} [lachen;laugh]]
               [V\feattab{\subj \sliste{ \ibox{2} },\\
                         %\spr \eliste,\\
                          \comps \sliste{ \ibox{3}, \ibox{6} }} [müssen;must]]]]]]]]
\end{forest}}
\caption{Analysis of \emph{dass sie darüber wird haben lachen müssen} `that it will be the case that
  she had to laugh about this'}
\end{figure}



\end{enumerate}
\clearpage


\section{Verb position: Verb-first and verb-second}



\begin{enumerate}
\item Classify the Germanic languages according to their basic constituent order (SVO, SOV, VSO,
  \ldots) and V2 assuming that you know that one of the following patterns exists in the language:
\eal
\label{ex-v2-task-solution}
\ex 
\label{ex-acc-aux-nom-v-dat}
NP[acc] V-Aux NP[nom] V NP[dat]   \hfill  V2 SVO 
\ex
\label{ex-acc-aux-nom-dat-v} 
NP[acc] V-Aux NP[nom] NP[dat] V   \hfill  V2 SOV
\ex 
\label{ex-acc-nom-v-acc}
NP[acc] NP[nom] V NP[acc]         \hfill $-$V2 SVO
\ex 
\label{ex-acc-nom-aux-v-acc}
NP[acc] NP[nom] V-Aux V NP[acc]   \hfill $-$V2 SVO
\ex 
\label{ex-acc-aux-nom-v-pp}
NP[acc] V-Aux NP[nom] V PP        \hfill not classifiable
\zl

The pattern in (\ref{ex-acc-aux-nom-v-dat}) cannot be English, since English does not have a dative. Hence it is a V2
language. The dative object follows the verb, so it must be an SVO language. An example would be Icelandic:
\ea
\gll Bókina          hafa ég       gefið honum.\\
     book.the.\ACC{} have I.\NOM{} given he.\DAT\\\icelandic
\glt `I gave him the book.'
\z

(\ref{ex-acc-aux-nom-dat-v}) has an auxiliary and two NPs followed by a verb. Since the dative object would follow the verb
in an SVO language, it must be a SOV language. Since all Germanic SOV languages are also V2
languages, (\ref{ex-acc-aux-nom-dat-v}) must be a V2 language. German and Dutch would be examples.
\ea
\gll Den Roman hat jemand dem Kind gegeben.\\
     the.\ACC{} novel has somebody.\NOM{} the.\DAT{} child given\\
\glt `Somebody has given the child the novel.'
\z

Ignoring multiple frontings in German \citep{Mueller2003b}, (\ref{ex-acc-nom-v-acc}) must be a non-V2 pattern. The language can only be
English:
\ea
This book, Kim gave Sandy.
\z
For the same reason, (\ref{ex-acc-nom-aux-v-acc}) is non-V2 and SVO. The language must be English:
\ea
This book, Kim had given Sandy.
\z
The pattern in (\ref{ex-acc-aux-nom-v-pp}) cannot be unambiguously classified with respect to V2 and
SOV/SVO. Since PPs can be extraposed easily, it could be an SOV langauge with extraposition (\eg
German) or it could be English with question formation (residual V2):
\eal
\ex 
\gll Wen hat Aicke gesehen bei    der Demonstration?\\
     who has Aicke seen    during the rally\\
\glt `Who has Aicke seen during the rally.'
\ex Who did Kim see during the rally?
\zl
\item Sketch the analysis for the following examples. Use the abbreviations used in this chapter;
  that is, do not go into the details regarding \spr and \compsvs but use S, VP, and V$'$.
\eal
\ex
\gll Arbejder Bjarne ihærdigt  på bogen?\\
     works    Bjarne seriously at book.\textsc{def}\\\danish
\glt `Does Bjarne work seriously on the book?'
\ex
\gll Arbeitet Bjarne ernsthaft an dem Buch?\\
     works    Bjarne seriously at the book\\\german
\glt `Does Bjarne work seriously on the book?'
\ex
\gll Wird sie darüber    nachdenken?\\
     will she there.upon \particle.think\\\german
\glt `Will she think about this?'
\zl

\begin{figure}
\begin{forest}
sm edges
[S
  [V\sliste{ S//V } 
    [V [arbejder;works]]]
  [S//V
     [NP [Bjarne;Bjarne]]
     [VP//V
       [Adv [ihærdigt;seriously]]
       [VP//V
         [V//V [\trace]]
         [PP [på bogen;{at book.\textsc{def}},roof]]]]]]
\end{forest}
\caption{Analysis of \emph{Arbejder Bjarne ihærdigt på bogen?} `Does Bjarne work seriously on the book?'}
\end{figure}

\begin{figure}
\begin{forest}
sm edges
[S
  [V\sliste{ S//V } 
    [V [arbeitet;works]]]
  [S//V
     [NP [Bjarne;Bjarne]]
     [V$'$//V
       [Adv [ernsthaft;seriously]]
       [V$'$//V
         [PP [an dem Buch;at the book,roof]]
         [V//V [\trace]]]]]]
\end{forest}
\caption{Analysis of \emph{Arbejder Bjarne ihærdigt på bogen?} `Does Bjarne work seriously on the book?'}
\end{figure}

\begin{figure}
\begin{forest}
sm edges
[S
  [V\sliste{ S//V }
    [V
      [wird;will]]]
  [S//V
     [NP [sie;she]]
     [V$'$//V
       [PP [darüber;there.about]]
       [V//V
         [V [nachdenken;\textsc{part}.think]]
         [V//V [\trace]]]
]]]
\end{forest}
\caption{Analysis of \emph{Wird sie darüber nachdenken?} `Will she think about this?'}
\end{figure}

\clearpage

\item Sketch the analysis for the following examples. Use the valence features \spr and \comps
  rather than the abbreviations S, VP, and V$'$. Since the value of \spr in German is always the
  empty list, you may omit it in the German examples. NPs and PPs can be abbreviated as NP and PP, respectively.
\eal
\ex 
\gll dass sie darüber nachdenkt\\
     that she there.upon \particle.thinks\\\jambox{(German)}
\glt `that she thinks about this'
\ex 
\gll dass sie darüber nachdenken wird\\
     that she there.upon \particle.think will\\
\glt `that she will think about this'
\ex
\gll Wird sie darüber nachdenken?\\
     will she there.upon \particle.think\\
\glt `Will she think about this?'
\zl

\eal
\ex
\gll Arbejder Bjarne ihærdigt  på bogen?\\
     works    Bjarne seriously at book.\textsc{def}\\\jambox{(Danish)}
\glt `Does Bjarne work seriously on the book?'
\ex
\gll Arbeitet Bjarne ernsthaft an dem Buch?\\
     works    Bjarne seriously at the book\\\german
\glt `Does Bjarne work seriously on the book?'
\zl


\begin{figure}
\begin{forest}
sm edges
[C\feattab{%\spr \eliste,\\
             \comps \eliste}
  [C\feattab{%\spr \eliste,\\
             \comps \sliste{ \ibox{1} }} [dass;that]]
  [{\ibox{1} V\feattab{%\spr \eliste,\\
                       \comps \eliste}}
     [{\ibox{2} NP[nom]} [sie;she]]
     [V\feattab{%\spr  \eliste,\\
                 \comps \sliste{ \ibox{2} }}
       [\ibox{3} PP [darüber;there.about]]
       [V\feattab{%\spr \eliste,\\
                  \comps \sliste{ \ibox{2}, \ibox{3} }} [nachdenkt;\textsc{part}.thinks]]]]]
\end{forest}
\caption{Analysis of \emph{dass sie darüber nachdenkt} `that she thinks about this'}
\end{figure}


\begin{figure}
\begin{forest}
sm edges
[C\feattab{%\spr \eliste,\\
             \comps \eliste}
  [C\feattab{%\spr \eliste,\\
             \comps \sliste{ \ibox{1} }} [dass;that]]
  [{\ibox{1} V\feattab{%\spr \eliste,\\
                       \comps \eliste}}
     [{\ibox{2} NP[nom]} [sie;she]]
     [V\feattab{%\spr  \eliste,\\
                 \comps \sliste{ \ibox{2} }}
       [\ibox{3} PP [darüber;there.about]]
       [V\feattab{%\spr \eliste,\\
                  \comps \sliste{ \ibox{2}, \ibox{3} }}
         [\ibox{4} V\feattab{\subj \sliste{ \ibox{2} },\\
                             %\spr \eliste,\\
                             \comps \sliste{ \ibox{3} }} [nachdenken;\textsc{part}.think]]
         [V\feattab{%\spr \eliste,\\
                    \comps \sliste{ \ibox{2}, \ibox{3}, \ibox{4} }} [wird;will]]]
]]]
\end{forest}
\caption{Analysis of \emph{dass sie darüber nachdenken wird} `that she will think about this'}
\end{figure}


\begin{figure}
\begin{forest}
sm edges
[V\feattab{%\spr \eliste,\\
           \comps \eliste}
  [V\feattab{%\spr \eliste,\\
             \comps \sliste{ \ibox{1} }}
    [V
      [wird;will]]]
  [{\ibox{1} V//V\feattab{%\spr \eliste,\\
                       \comps \eliste}}
     [{\ibox{2} NP[nom]} [sie;she]]
     [V//V\feattab{%\spr  \eliste,\\
                 \comps \sliste{ \ibox{2} }}
       [\ibox{3} PP [darüber;there.about]]
       [V//V\feattab{%\spr \eliste,\\
                  \comps \sliste{ \ibox{2}, \ibox{3} }}
         [\ibox{4} V\feattab{\subj \sliste{ \ibox{2} },\\
                             %\spr \eliste,\\
                             \comps \sliste{ \ibox{3} }} [nachdenken;\textsc{part}.think]]
         [V//V\feattab{%\spr \eliste,\\
                    \comps \sliste{ \ibox{2}, \ibox{3}, \ibox{4} }} [\trace]]]
]]]
\end{forest}
\caption{Analysis of \emph{Wird sie darüber nachdenken?} `Will she think about this?'}
\end{figure}



%\gll Arbejder Bjarne ihærdigt  på bogen?\\
%     works    Bjarne seriously at book.\textsc{def}\\\jambox{(Danish)}


\begin{figure}
\begin{forest}
sm edges
[V\feattab{\comps \eliste}
  [V\feattab{\comps \sliste{ \ibox{1} }} 
    [V [arbejder;works]]]
  [\ibox{1} V//V\feattab{\spr   \eliste,\\
                         \comps \eliste}
     [\ibox{2} NP [Bjarne;Bjarne]]
     [V//V\feattab{\spr   \sliste{ \ibox{2} },\\
                    \comps \eliste}
       [{Adv[\textsc{mod} \ibox{3} VP]} [ihærdigt;seriously]]
       [\ibox{3} V//V\feattab{\spr   \sliste{ \ibox{2} },\\
                      \comps \eliste}
         [V//V\feattab{\spr \sliste{ \ibox{2} },\\
                       \comps \sliste{ \ibox{4} }} [\trace]]
         [\ibox{4} PP [på bogen;{at book.\textsc{def}},roof]]]]]]
\end{forest}
\caption{Analysis of \emph{Arbejder Bjarne ihærdigt  på bogen?} `Does Bjarne work seriously on the book?'}
\end{figure}

\begin{figure}
\begin{forest}
sm edges
[V\feattab{\comps \eliste}
  [V\feattab{\comps \sliste{ \ibox{1} }} 
    [V [arbeitet;works]]]
  [\ibox{1} V//V\feattab{%\spr   \eliste,\\
                         \comps \eliste}
     [\ibox{2} NP [Bjarne;Bjarne]]
     [V//V\feattab{%\spr   \eliste,\\
                    \comps \sliste{ \ibox{2} }}
       [{Adv[\textsc{mod} \ibox{3} V[\textsc{ini}$-$]]} [ernsthaft;seriously]]
       [\ibox{3} V//V\feattab{%\spr   \sliste{  },\\
                      \comps \sliste{ \ibox{2} }}
         [\ibox{4} PP [an dem Buch;at the book,roof]]
         [V//V\feattab{%\spr \sliste{  },\\
                       \comps \sliste{ \ibox{2}, \ibox{4} }} [\trace]]]]]]
\end{forest}
\caption{Analysis of \emph{Arbeitet Bjarne ernsthaft an dem Buch?} `Does Bjarne work seriously on the book?'}
\end{figure}


\item Sketch the analysis of the following examples. NPs may be abbreviated. Valence features should
  not be given, but node labels like V, V$'$, VP and S should be used instead. If non-local
  dependencies are involved indicate them using the `/' symbol.
\eal
\ex Such books, I like.
\ex 
\gll Solche Bücher mag ich.\\
     such   books  like I\\\german
\glt `I like such books.'
\ex
\gll Boger som det elsker jeg.\\
     books like this like I\\\danish
\glt `I like such books.'
\zl

\begin{figure}
\begin{forest}
sm edges
[S
  [NP$_i$ [such books,roof] ]
  [S/NP 
    [NP [I] ] 
    [VP/NP  
      [V [like] ]
      [NP/NP [\trace$_i$] ] ] ] ]
\end{forest}
\caption{Analysis of \emph{Such books, I like.}}
\end{figure}

\begin{figure}
\begin{forest}
sm edges
[S
  [NP$_i$ [solche Bücher;such books, roof] ]
  [S/NP
     [V \sliste{ S$/\!/$V } 
        [V [mag$_j$;like] ] ]
     [S$/\!/$V/NP
        [NP/NP [\trace$_i$] ]
        [V$'$$\!/\!/$V
           [NP [ich;I] ]
           [V$\!/\!/$V [\_$_j$] ] ] ] ] ] ]
\end{forest}
\caption{Analysis of the German example \emph{Solche Bücher mag ich.} `I like such books.'}
\end{figure}


\begin{figure}
\begin{forest}
sm edges
[S
   [NP$_i$ [boger som det;books like this,roof] ]
      [S/NP
         [V \sliste{ S$/\!/$V }
           [V [elsker$_j$;like] ] ]
           [S$/\!/$V/NP
             [NP [jeg;I] ]
             [VP$\!/\!/$V/NP
               [V$\!/\!/$V  [\_$_j$] ]
               [NP/NP [\trace$_i$ ] ] ] ] ] ] ] 
\end{forest}
\caption{\label{fig-boger-som-et-elsker-jeg}Analysis of the Danish example \emph{Boger som det elsker jeg.} `I like such books.'}
\end{figure}


\end{enumerate}

\clearpage

\section{Passive}

\begin{enumerate}
\item Which NPs in (\mex{1}) do have structural and which lexical case?
\eal
\ex 
\gll Der        Junge lacht.\\
     the.\NOM{} boy   laughs\\
\glt `The boy laughs.'
\ex 
\gll Mich friert.\\
     I.\ACC{} freeze\\
\glt `I am cold.'
\ex 
\gll Er zerstört die Umwelt.\\
     he.\NOM{} destroys the.\ACC{} environment\\
\glt `He destroys the environment.'
\ex 
\gll Das dauert ein ganzes Jahr.\\
     this.\NOM{} takes  a.\ACC{} whole year\\
\glt `This takes a whole year.'
\ex 
\gll Er hat nur einen Tag dafür gebraucht.\\
     he.\NOM{} has just one.\ACC{}  day there.for needed\\
\glt `He needed a day for this.'
\ex 
\gll Er denkt an den morgigen Tag.\\
     he.\NOM{} thinks at the.\ACC{} tomorrow day\\
\glt `He thinks about tomorrow.'
\zl

All nominatives in (\mex{0}) are structural cases. The accusatives in (\mex{0}b, d, f) are lexical,
the ones in (\mex{0}c, e) are structural.

\item Give \argst lists for the following verbs. Provide the \argstl with the maximum amount of arguments. 
\eal
\ex show, eat, meet \english
\ex zeigen `show', essen `eat', begegnen `meet', treffen `meet' \german
%\ex vise `show', spise `eat', møde `meet' \danish
%\ex sýna `show', eta `eat', mæta `meet', hitta `meet' \icelandic
\zl

\eal
\ex \emph{show}: \sliste{ NP[\str], NP[\str], NP[\type{lacc}] }
\ex \emph{eat}: \sliste{ NP[\str], NP[\str] }
\ex \emph{meet}: \sliste{ NP[\str], NP[\str] }
\zl
\eal
\ex \emph{zeigen}: \sliste{ NP[\str], NP[\type{ldat}], NP[\str] }
\ex \emph{essen}: \sliste{ NP[\str], NP[\str] }
\ex \emph{begegnen}: \sliste{ NP[\str], NP[\type{ldat}] }
\ex \emph{treffen}: \sliste{ NP[\str], NP[\str] }
\zl
If you are uncertain as far as case is concerned, you may use the
  Wiktionary: \url{https://de.wiktionary.org/}.

\item Draw the analysis tree for the following clause:
\ea
that the box was opened
\z
Please provide valence features (\spr and \comps) and part of speech information. You may abbreviate
the NP using a triangle.


\begin{figure}
\begin{forest}
sm edges
[{C[\comps \eliste]}
  [{C[\comps \sliste{ \ibox{1} }]} [that]]
  [{\ibox{1} V\feattab{\spr \eliste,\\
                       \comps \eliste}}
     [{\ibox{2} NP[nom]} [the box, roof]]
     [V\feattab{\spr  \sliste{ \ibox{2} },\\
                 \comps \eliste}
       [V\feattab{\spr  \sliste{ \ibox{2} },\\
                  \comps \sliste{ \ibox{3} }} [was]]
       [{\ibox{3} V\feattab{\spr \sliste{ \ibox{2} },\\
                            \comps \eliste}} [opened]]]]]
\end{forest}
\caption{Analysis of the passive clause \emph{that the box was opened}}
\end{figure}
The transitive verb \emph{open} takes a subject and an object. The \argst list contains two NPs with
structural case. The passive lexical rule removes one argument. For the passive participle this
leaves us with one element on the \argstl. This element gets mapped to the \sprl of
\emph{opened}. The passive auxiliary takes a VP in passive form and takes over its element from
\spr. After combination of auxiliary and passive VP, we have the VP \emph{was opened} still
selecting for a specifier. The NP \emph{the box} functions as the specifier and the combination of
\emph{the box} and \emph{was opened} is a complete sentence.

\end{enumerate}

%% \begin{figure}
%% \begin{forest}
%% sm edges
%% [CP
%%   [C [dass;that]]
%%   [V
%%      [\ibox{1} {NP[nom]} [der Kasten;the box, roof]]
%%      [V
%%        [{\ibox{2} V[\comps \ibox{3} \sliste{ \ibox{1} }]} [geöffnet;opened]]
%%        [{V[\comps \ibox{3} $\oplus$ \sliste{ \ibox{2} }] [wurde;was]]]]]
%% \end{forest}
%% \end{figure}

\section{Clause types and expletives}

\begin{enumerate}
\item Analyze the interrogative clauses in (\mex{1}):
\eal
\ex
\gll Ich weiß, wen Kim kennt.\\
     I   know  who Kim knows\\\german
\glt `I know who Kim knows.'
\ex 
\gll Jeg ved, hvem det kende Kim\\
     I know who  \expl{} knows Kim\\\danish
\glt `I know who knows Kim.'
\zl 


\begin{figure}
\centerline{\begin{forest}
sm edges
[S
       [{NP[\sacc]} [wen;who] ]
       [{S/NP[\sacc]} 
         [{NP[\sacc]/NP[\sacc]}  [\trace] ]
         [V$'$
           [{NP[\snom]} [Kim;Kim] ]
           [V [kennt;knows] ] ] ] ]
\end{forest}}
\caption{Analysis of \emph{wen Kim kennt} `who Kim knows'}
\end{figure}


\begin{figure}
\centerline{\begin{forest}
sm edges
[S
       [{NP[\snom]} [hvem;who] ]
       [{S/NP[\snom]}
         [{NP[\lnom]} [det;\textsc{expl}] ]
         [{VP/NP[\snom]}
           [{V$'$/NP[\snom]}
             [V [kende;knows] ]
             [{NP[\snom]/NP[\snom]} [\trace] ] ]
           [{NP[\sacc]} [Kim;Kim ] ] ] ] ]
\end{forest}}
\caption{Analsis of \emph{hvem det knows Kim} `who knows Kim'}
%\caption{Analysis of interrogative clauses in Danish with subject extraction}\label{fig-interrogative-Danish-subject-extraction}
\end{figure}


\item Analyze the clause in (\mex{1}). Use triangles for the NP and the PP.
\ea
\gll Es schwammen zwei Delphine neben dem Boot.\\
     \expl{} swam two dolphins  next.to the boat\\
\glt `Two dolphins were swimming next to the boat.'
\z

\begin{figure}

\begin{forest}
sm edges
[S
  [{NP[\lnom]} [es;\textsc{expl}]]
  [{S/NP[\lnom]}
     [V\sliste{ S$/\!/$V } [schwammen;swam]]
     [{S$/\!/$V/NP[\lnom]}
       [{NP[\lnom]/NP[\lnom]}  [\trace]]
       [V$'$$/\!/$V
         [{NP[\snom]} [zwei Delphine;two dolphins,roof]]
         [V$'$$/\!/$V
           [PP [neben dem Boot;next.to the boat,roof]]
           [V$/\!/$V [\trace]]]]]]]
\end{forest}
\caption{Analysis of \emph{Es schwammen zwei Delphine neben dem Boot.} `Two dolphins were swimming next to the boat.'}
\end{figure}

\end{enumerate}



%      <!-- Local IspellDict: en_US-w_accents -->
